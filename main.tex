\documentclass[11pt,partial,draft,doublespace]{aucklandthesis}
% \documentclass[11pt,partial]{aucklandthesis}
%
% This is a template for University of Auckland theses.
%
% Written by Alistair Kwan, June 2016
% 
%
% Options:
%	10pt, 11pt, 12pt: size of main text
% 	examcopy: asserts confidentiality for examination copies
%	partial: thesis partial fulfils degree requirements
%	singlespace, onehalfspace, doublespace: line spacing
%	oneside: format for single-sided printing
%	draft: adds 'draft' and date to footer
%

%%%% Packages introduced by me, not included with the template
\usepackage[normalem]{ulem}
\usepackage{amsmath}
\usepackage[binary-units]{siunitx}
\usepackage{fixme}
% \fxsetup{targetlayout=changebar}
\fxusetheme{color}
%
% Add, delete or un-comment packages below as required.
%

\usepackage[utf8]{inputenc}
\usepackage[T1]{fontenc}

% Readability options
%
\usepackage{booktabs} % for table rules
\usepackage{microtype} % for improved justification

\usepackage{graphicx} % for inserting graphics files
\usepackage{appendix} % for appendices

% Typeface options — choose one if desired
% or choose a different typeface to accommmodate character sets
% as needed for East Asian and other languages.
%
% Consider compiling using the XeLaTeX engine if you have more extreme
% typeface needs, e.g. for multiple languages, or a need for symbols particular
% to a typeface.
%
% See also the LaTeX Symbols List at
% https://http://www.ctan.org/pkg/comprehensive
%
% \usepackage{mathptmx} % Times New Roman, including mathematics
% \usepackage{mathpazo} % Palatino with mathematics support
% \usepackage{fourier} % Utopia, a serif typeface with Fourier mathematics
% \usepackage{gentium} % a contemporary serif typeface
% \usepackage{libertine} % a softer-feeling serif typeface; also installs sans-serif font Biolinum
% \usepackage{fouriernc} % Century Schoolbook with Fourier maths
% \usepackage{mathpple} % Palatino with Fourier maths


% To set the sans serif font (for \sffamily):
%\usepackage[scaled]{helvet} % Nimbus, like Helvetica
%\usepackage{universalis} % Universalis
%\usepackage{avant} % URW Gothic, like Avant Garde
%\usepackage{PTSansNarrow}
%\usepackage{AlegreyaSans} % Alegreya Sans

% To set the mathematics font:
%\usepackage{eulervm} % Euler, based on a Zapf design

% To set the (usually monospaced) typewriter font:
%\usepackage[ttdefault=true]{AnonymousPro}
%\usepackage[scaled]{beramono}
%\usepackage{inconsolata}
%\usepackage{sourcecodepro}

%\usepackage{cjk} % for Chinese, Japanese, Korean

\usepackage{tabularx} % For easier table formatting.

% \usepackage[nottoc]{tocbibind} % Controls the table of contents
%   nottoc: don't list table of contents inside itself
%   section: go as far as section-level headings

% Automated bibliography
%
\usepackage[
	bibstyle=ieee, 
	citestyle=numeric-comp,
	backend=biber,
	sorting=nyt,
	sortcites,
	backref,
	dashed=false,
	urldate=long,
	]
	{biblatex}
	\addbibresource{bib/stereo.bib}
	\addbibresource{bib/compmodels.bib}
	\addbibresource{bib/psystems.bib}
	\addbibresource{bib/languages.bib}
% bibliography{bibliography1.bib, bibliography2.bib} % Specify bibliography files 
% bibliography{bib/stereo.bib, Mendeley.bib}

% The below was copied verbatim from https://tex.stackexchange.com/a/154875 on 9 December 2019, and then modified to reverse the ordering of the URL and DOI fields.  The purpose of it is to suppress the use of the URL field if the DOI field is defined - thus avoiding the many instances where the URL is printed, but is just a link to the DOI (besides which, in the PDF the DOI becomes a link to the DOI site).  It seems to work for my purposes.
\DeclareSourcemap{
  \maps[datatype=bibtex]{
    \map{
      \step[fieldsource=doi,final]
      \step[fieldset=url,null]
    }  
  }
}

\usepackage{hyperref} % for formatting web addresses and other URLs
\urlstyle{same} % try also tt, sf if this option doesn't produce clear enough output

%%% All the glossaries stuff was put here, since the Glossaries documentation says it should be used after hyperref.

\usepackage[acronym,noredefwarn,toc]{glossaries}
\setacronymstyle{long-short}
\makeglossaries

% The below was taken from https://tex.stackexchange.com/a/562932.  It is used in the glossary.tex file, but placed here because I'm not sure putting it in the glossary file won't b0rk things.  It was originally called 'newdefineabbreviation', but I changed it because that was irritatingly long.
% #1 - reference e.g. api
% #2 - Short e.g. API
% #3 - Full name e.g. Application Programming Interface
% #4 - Description
\newcommand{\newdefacr}[4]
{
    % Glossary entry
    \newglossaryentry{#1-glossary}
    {
        text={#2},
        long={#3},
        name={\glsentrylong{#1-glossary} (\glsentrytext{#1-glossary})},
        description={#4}
    }

    % Acronym
    \newglossaryentry{#1}
    {
        type=\acronymtype,
        name={\glsentrytext{#1-glossary}}, % Short
        description={\glsentrylong{#1-glossary}}, % Full name
        first={\glsentryname{#1-glossary}\glsadd{#1-glossary}},
        see=[Glossary:]{#1-glossary} % Reference to corresponding glossary entry
    }
}

\loadglsentries{glossary}

\begin{document}

% \newcommand{\cml}{Concurrent~ML}
% \newcommand{\csp}{Communicating Sequential Processes}
\newcommand{\fsharp}{F\nolinebreak\hspace{-.05em}\raisebox{.3ex}{\tiny{\textbf{\#}}}}
\newcommand{\cps}{cP~systems}

% ====================================================
%
% FRONTMATTER
%
% Arabic pagination, starting with the title page
% which is counted but not numbered
%
% ====================================================

% Specify the title page content
% \title{Message Passing-Based Stereo Matching Algorithms in cP~systems and Concurrent~ML}
% \subtitle{Using functional programming to make life simpler and promote efficiency}
\title{Neighbourhood Message Passing Computation on a Grid}
\subtitle{With Application to Stereo Matching}
\author{James Cooper}
\degreesought{Doctor of Philosophy} 
\degreediscipline{Computer Science}
\degreecompletionyear{2021}

% Print the title page
\maketitle

% Abstract, up to 350 words
% % \begin{abstract}
% Something about how there's a group of stereo matching algorithms that can be regarded as `message passing-based', and how they can be represented in cP~systems, as well as with Concurrent~ML.

% This dissertation presents both cP~systems representations and Concurrent~ML-based implementations of four stereo matching algorithms.
% \end{abstract}

\chapter*{Abstract}
Something about how there's a group of stereo matching algorithms that can be regarded as `message passing-based', and how they can be represented in cP~systems, as well as with Concurrent~ML.

This dissertation presents both cP~systems representations and Concurrent~ML-based implementations of message-passing-based stereo matching algorithms.

Could also look at doing something where I `fake' message passing on a shared-memory system using a backing array or two.  As in, while the program is written with a message passing basis, where each disparity map pixel does its own thing and then sends out messages to its neighbours, but in reality all that is happening is that there is a big array behind it all, and each message passing step is just transformed behind-the-scenes into array updates.  Thus, one could program with a message passing style, but still retain the relatively efficient array-update style in the background. % it's in a separate file

% Dedication (optional)
%\thesisdedication{Dedicated to grandma, and to grammar.}
% \thesisdedication{This work is dedicated to the interested reader and all those who make use of material presented within.}

% Preface and/or acknowledgements (optional)
% \chapter*{Acknowledgements}

Acknowledgement must be made, firstly, of my supervisors, Doctor Radu Nicolescu and Associate-Professor Patrice Delmas.  Without their support and guidance, I certainly would have never made it to this point, and Doctor Nicolescu's influence in particular can be found throughout this dissertation.  So too must acknowledgement be made of Professor Emeritus Georgy Gimel'farb and Doctor Wannes van der Mark, both of whom have provided invaluable support at times.  Also, Doctor José Calderón Trilla.

Furthermore, thanks must be given to the professional staff from: the School of Computer Science; the School of Graduate Studies; Libraries and Learning Services (particularly those who served as Computer Science subject librarian); the Centre for eResearch; and the wider University of Auckland.  Particular mention must be made of Kaleigh Fennah, Emma Gavenda, Sue Skelly, Sarah Sneyd and Robyn Young.

I also would like to mention several people who were students contemporaneously with me (and in a few cases were also co-authors with me), including: Arabella Anderson; Mihailo Azhar; Daniel Britten; Doctor Trevor Gee; Alec Henderson; Yan Kolezhitskiy; Yezhou Liu; William Hsu; and Eirian Perkins.  Each of these people in some way has improved my time and experience during the PhD process.

\vspace{1cm}
\hfill James Cooper

\hfill Auckland, New Zealand

\hfill September 2021

 % it's in a separate file

% Contents, lists of tables and figures
\settocdepth{subsection} % choose chapter, section, subsection 
\cleardoublepage\tableofcontents
%\cleardoublepage\listoffigures
%\cleardoublepage\listoftables
\cleardoublepage\listoffixmes

% Glossary (optional)
% % \makeglossaries

% Acronyms (sorted in alphabetical order by their label, rather than their full definition)

%% A-M
\newacronym{bp}{BP}{belief~propagation}
\newacronym{cham}{CHAM}{chemical abstract machine}
\newacronym{cp}{CP}{concurrent~propagation}
\newacronym{csp}{CSP}{communicating sequential processes}
\newacronym{dpsm}{DPSM}{dynamic~programming \gls{sm}}
\newacronym{dsp}{DSP}{digital signal processor}
\newacronym{enps}{ENPS}{enzymatic numerical P~systems}
\newacronym{fifo}{FIFO}{first-in-first-out}
\newacronym{fpga}{FPGA}{field-programmable gate array}
\newacronym{fps}{FPS}{frames per second}
\newacronym{gpu}{GPU}{graphics processing unit}
\newacronym{gpgpu}{GPGPU}{general-purpose \glsxtrshort{gpu}}
\newacronym{lbp}{LBP}{loopy \gls{bp}}
\newacronym{mpbsm}{MPBSM}{message passing-based \gls{sm}}
\newacronym{mrf}{MRF}{Markov random field}
\newacronym{mwt}{MWT}{moving window transform}

%% N-Z
\newacronym{ndcsm}{NDCSM}{noise-driven concurrent \gls{sm}}
\newacronym{nm}{NM}{neighbourhood messaging}
\newacronym{nmp}{NMP}{neighbourhood message passing}
\newacronym{oq}{OQ}{oracle query}
\newacronym{pe}{proxel}{processing element}
\newacronym{pram}{PRAM}{parallel random access machine}
\newacronym{sad}{SAD}{sum of absolute differences}
\newacronym{skps}{SKPS}{simple kernel P~systems}
\newacronym{ssd}{SSD}{sum of squared differences}

% Combo glossary & acronyms

% A-M
\newdefacr{cml}{CML}{Concurrent~ML}{Concurrent~ML}{A programming style introduced by John Reppy as an extension of the pre-existing programming language Standard~ML.  Conceptually, Concurrent~ML has a lot in common with \glsxtrlong{csp} \cite{Hoare1985}.  Both involve every logical process mostly advancing on its own, but sometimes synchronising with others over channels (see also \vref{fig:back:cml_exchange}).  Concurrent~ML is described in detail in \cite{Reppy2007}.}
\newdefacr{gcp}{GCP}{graph colouring problem}{Graph Colouring Problem}{The problem of finding a way to colour the nodes in a graph, such that no adjacent nodes share a colour.  This problem is typically further constrained to using a maximum number of colours for the entire graph, or finding the minimum number of colours required to perform the colouring.}
\newdefacr{hcp}{HCP}{Hamiltonian cycle problem}{Hamiltonian Cycle Problem}{Closely related to the \glsdesc{hpp}.  In this problem, the Hamiltonian Path must also be a cycle, \ie{} it ends back at the origin node.}
\newdefacr{hpp}{HPP}{Hamiltonian path problem}{Hamiltonian Path Problem}{The problem of determining whether, for a given graph, there exists a traversal of the graph in which each node is visited exactly once.}
\newdefacr{lhs}{LHS}{left-hand side}{Left-hand Side}{A part of the definition of a \gls{cps} rule.  It is the list of terms which \emph{must} be present and as-yet unused in a rule at the current step, in order for the current rule to be applicable.  All terms appearing here are deleted from the containing cell at the end of the step.}

% N - Z
\newdefacr{rhs}{RHS}{right-hand side}{Right-hand Side}{A part of the definition of a \gls{cps} rule.  It is the list of terms which are created by the rule's application, and appear in the relevant cell at the end of the step.}
%
%
\newdefacr{tsp}{TSP}{travelling salesman problem}{Travelling Salesman Problem}{An extension of the \glsdesc{hcp}.  Here, the edges between nodes have weights, and the goal of the problem is to find a minimum weight Hamiltonian cycle on the graph.}

%% A-M
\newglossaryentry{actor}{text={actor},name={Actor},description={A model of message-passing-based concurrent programming originally developed  chiefly by Carl Hewitt, and further developed for practical programming by Gul Agha and collaborators.  Its defining characteristics are that it uses asynchronous messaging, whereby the sender and receiver do not need to coordinate or synchronise at all; and that instead of using channels or similar as a go-between, Actors send messages directly to each other, which necessitates ``knowing'' (\ie{} holding a reference to) the intended recipient.  Notable examples of implementations of Actors are found in the programming languages Erlang and Pony, and the Scala library Akka.}}
%
\newglossaryentry{clps}{text={cell-like P~systems},name={Cell-like P~Systems},plural={cell-like P~system},description={The original \gls{ps} variant first proposed by Gheorghe Păun, based on the operation of chemicals inside biological cells.  The main characteristic of cell-like P~systems is that the chemicals are represented by an unlimited variety of atomic objects, and are contained within one or more membranes.  Every cell-like P~system has an outermost membrane, the \emph{skin membrane} which separates the cell from the \emph{environment}, and may have more membranes inside itself.  These latter membranes are nested in such a fashion that they resemble a graphical tree.  Each \gls{compartment} delimited by a membrane has a separate \gls{ruleset}, though the \glspl{ruleset} are not necessarily unique.  Rules may create and remove chemicals based on the presence (or absence) of other chemicals inside a membrane, and also send chemicals to an inner \emph{child} membrane, or to an outer \emph{parent} membrane.  Perhaps due to it being the original variant (and thus pre-dating the term ``cell-like P systems''), some forms of P~systems which could be regarded as cell-like are not necessarily labelled as such, \eg{} P~systems with active membranes.}}
%
% \newglossaryentry{compartment}{text={compartment},name={Compartment},description={A generic umbrella term for the processing units, which also act as the containers of the system's objects, of a given \gls{ps} type.  For example, these are: the membranes of \gls{clps}; the cells of \gls{tlps}; the neurons of \gls{snps}; and the \glspl{tlc} of \gls{cps} --- though in the case of \gls{cps} the data-only complex objects are sometimes also described as sub-compartments or micro-compartments, because they still contain objects.}}
\newglossaryentry{compartment}{text={compartment},name={Compartment},description={A generic umbrella term for the combined processing units and object containers of a given \gls{ps} type.  For example, these are: the membranes of \gls{clps}; the cells of \gls{tlps}; the neurons of \gls{snps}; and the \glspl{tlc} of \gls{cps} --- though in the case of \gls{cps} the data-only complex objects are sometimes also described as sub-compartments or micro-compartments, because they still contain objects.}}
%
\newglossaryentry{cps}{text={cP~systems},name={cP~Systems},description={A variant of \gls{ps} created by Radu Nicolescu and developed further by Nicolescu and collaborators.  The main point of difference between cP~systems and other variants is that, in general, cP~systems is `higher-level'.  That is, instead of specifying (semi-)uniform families of rules customised to each specific problem of a given type, cP~systems \glspl{ruleset} typically cover all possible problem specifications without customisation ahead-of-time through the use of variables.  The cP~systems `runtime' performs unification over the variables.  Furthermore, only the \glspl{tlc} have associated rules, and all other terms contained within are merely inert objects.  See also \cref{chap:cpsystems}.},plural={cP~system}}
%
\newglossaryentry{disparity}{text={disparity},name={Disparity},description={The shift, measured as a number of pixels, of a point in one stereo image to its location in the other image.  When combined with information about the cameras used to capture the images which is derived from calibration, the measured disparity is used to estimate the distance from the cameras to the point in the scene.  Disparity is often also referred to as `parallax'.}}
%
\newglossaryentry{disparitymap}{text={disparity map},name={Disparity Map},description={An output array/image (often just using a single channel of 8- or 16-bit integers) which encodes the final estimated disparities for every pixel in the input images.  The range of values in a disparity map itself is restricted to range of possible disparities used in the \gls{sm} algorithm.  It is common, however, to adjust the disparity map in some fashion to make differences in disparity estimates more visible through techniques such as histogram equalisation.  Most published figures of disparity maps will have had such an adjustment applied to make them more useful.  Disparity maps are sometimes also referred to as `digital parallax maps.'}}
%
\newglossaryentry{fne}{description={The `standard' neighbourhood arrangement in situations such as \glsxtrlong{lbp}.  When used in reference to a grid, it typically means the other locations in the grid immediately to the left, right, above and below.},text={four-neighbourhood},name={Four-neighbourhood}}
%
\newglossaryentry{functor}{description={In the context of \gls{cps}, a functor is the label applied to the delimiters of a complex term.  \Eg{} for \(\cpfunc{a}{\cpfunc{b}{c}}\) both \(a\) and \(b\) are functors, because they are the outer label for their respective inner terms.  ``Functor'' is also sometimes used as a shorthand to refer to the complex term as a whole.  Strictly speaking, this is incorrect but convenient nevertheless, as writing \enquote{the complex term denoted by the functor \(a\)} is usually overly verbose.},text={functor},name={Functor}}
%
\newglossaryentry{gs}{description={In the context of \glsxtrlong{nmp}, this refers to an approach to messaging whereby the entire lattice waits until every \glsxtrshort{pe} in the lattice has finished its messaging for a given generation before proceeding to the next generation.},text={globally-synchronous},name={Globally-synchronous}}
%
\newglossaryentry{inhibitor}{name={Inhibitor},text={inhibitor},description={In \gls{cps}, an inhibitor on a rule denotes an object which \emph{must not} be present in the relevant \gls{tlc} for the rule to be applicable.}}
%
\newglossaryentry{ls}{description={In the context of \glsxtrlong{nmp}, this refers to an approach to messaging whereby every \glsxtrshort{pe} in the lattice waits until it has received \emph{all} messages pertaining to one generation before accepting any messages pertaining to a subsequent generation.},text={locally-synchronous},name={Locally-synchronous}}
%
\newglossaryentry{mc}{text={membrane computing},name={Membrane Computing},description={A computational model inspired by the functioning of biological systems, specifically the interactions of chemicals inside the membranes of a biological cell.  The terms Membrane Computing and P systems are normally used interchangeably.}}
%
\newglossaryentry{mecosim}{name={MeCoSim},description={Software used to simulate arbitrary P~systems, from a multitude of variants.  MeCoSim was created and is maintained by researchers from Universidad de Sevilla (University of Seville) in Spain (see further \cite{MeCoSim}).  Generally, rules for a given system are specified with a \gls{plingua} file.  These rules can then be run on a specific set of starting objects, with MeCoSim reporting the outcome from executing the rules at the end of each step.}}
%
\newglossaryentry{medianfilter}{text={median filter},name={Median Filter},description={An image processing technique often used to remove `salt \& pepper' noise from images.  Every pixel's value is updated to the median value of it and its neighbours, in a given neighbourhood.  It is an embarassingly-parallel problem in that each pixel in the image can compute its median without relying on other pixels' results, so long as it can observe their current value.  The necessity for performing selection of the median across the neighbourhood, however, means that it is non-trivial to implement a fast median filter.  It does not easily reduce to basic arithmetic operations only.}}
%
\newglossaryentry{ms}{text={micro-surgery},name={Micro-surgery},description={An alternative operation that can be performed on \gls{cps}' compound terms.  Instead of each rule application seizing control of the entire term, it merely takes hold of a subset of the term's contents.  A major benefit of this is to enable in a single step arithmetic on numeric multisets that would instead require multiple iterations to complete.},plural={micro-surgeries}}

%% N-Z
\newglossaryentry{promoter}{name={Promoter},text={promoter},description={In \gls{cps}, a promoter on a rule denotes an object which \emph{must} be present in the relevant \gls{tlc} for the rule to be applicable, but which is \emph{not} consumed or deleted by the rule.}}
%
\newglossaryentry{ps}{name={P~Systems},see={mc},text={P~systems},description={An alternative name for \gls{mc}.},plural={P~system},}
%
\newglossaryentry{plingua}{name={P-Lingua}, description={A plain-text markup language used to represent \gls{ps} specifications in computer files, readable by both humans and computers.}}
%
\newglossaryentry{ruleset}{text={ruleset},name={Ruleset},description={The rules associated with a particular \gls{compartment} in a given \gls{ps} variant.}}
\newglossaryentry{sm}{text={stereo~matching},name={Stereo~Matching},description={A family of methods to match points from different images of the same scene to estimate the distance from the capturing cameras to objects in the scene.}}
\newglossaryentry{snps}{text={spiking neural P~systems},name={Spiking Neural P~Systems},description={Spiking neural P~systems is one of the three main \gls{ps} variants in common use today, and was introduced most recently.  Whereas \gls{clps} is based on intra-cellular chemical interactions, and \gls{tlps} on inter-cellular interactions, spiking neural P~systems is explicitly based on the idea of neurons in the brain interacting via synapses.  In principle, the communication-based nature of spiking neural P~systems makes it closer to \gls{tlps} than \gls{clps}, but unlike those two it uses only a single object in its alphabet, the spike.  This difference can make solutions to some problems more complex, but has the advantage that it makes it (comparatively) easy to implement those solutions with linear algebra operations.}}
\newglossaryentry{tlc}{text={top-level cell},name={Top-level Cell},description={The outermost membranes/\glspl{compartment} and processing units of \gls{cps}, loosely equivalent to the skin membrane of \gls{clps}.  These are the only objects in the system which have rules and are `active'.  Generally, all other objects present in a system except channels are found inside a top-level cell.}}
\newglossaryentry{tlps}{text={tissue-like P~systems},name={Tissue-like P~Systems},description={Tissue-like P~systems is one of the oldest \gls{ps} variants, especially of those still in common use, and is based on the transmission of chemicals between cells via channels in biological tissue (rather than movement of chemicals through membranes within a cell like \gls{clps}).  A notable aspect of tissue-like P~systems is that they often include no ability for the cells to process their own contents.  Instead, all computation arises as a consequence of the communication.},plural={tissue-like P~system}}

% ====================================================
%
% MAINMATTER
%
% Include external chapter files here using
% the \input{} command
%
% If you run out of memory during compilation,
% switch some or all chapters to \include{} instead of \input{}, 
% but watch out for pagination problems.
%
% ====================================================

% \chapter{Introduction}
Some wordiness will go here.

\section{My first section}
Post-introductory wordiness

Main goal:  To achieve at least one of the below
\begin{itemize}
\item run more efficiently (faster and/or use less memory)
\item Scalability.  If one only scales well up to, say 4-8 cores, but the other works up to a much larger number of cores, then that would suggest that the latter approach is much more future-proof, since best guesses still suggest that manycore systems are the way of the future.
\item `cleaner'/nicer code (by some reasonably objective, quantifiable measure)
\item `easier to program/easier to read' -- lower complexity (how does one measure that though?)
\end{itemize}

Why is message passing better than using an algorithmic skeleton?  Or how much can the two ideas be unified?  (e.g. a \gls{cml} inspired skeleton for operations on regular grids -- perhaps using specific neighbourhoods).

Some of the stereo matching algorithms can be described in terms of message passing, but also can be given fairly efficient GPU implementations.  How do the two connect?

From \url{https://abacus.bates.edu/~ganderso/biology/resources/writing/HTWsections.html}:  ``the Introduction must answer the questions, "What was I studying? Why was it an important question? What did we know about it before I did this study? How will this study advance our knowledge?"''

\section{Research Questions}
\begin{enumerate}
    \item Can real-time message passing-based stereo matching be implemented using a message passing programming approach?
    \item How do the results compare with more traditional implementations?
    \item 
\end{enumerate}

\subsection{Research Question One}
Here real-time means achieving a speed of at least 1 frame per second, i.e. at least 1 Hz.

\subsection{Research Question Two}
Comparisons can be (will be?) made based on runtime efficiency, both with `wall time' taken and peak memory use, as well as based on code-quality measures in some fashion.  Also, if possible, check how well each approach scales.

\subsection{Research Question Three}

\section{Hypotheses}

\subsection{Hypothesis One}
On larger/more powerful systems, yes.  On smaller systems (e.g. Nvidia Jetson Nano) no.

\subsection{Hypothesis Two}
\gls{mpbsm} will be \emph{slower} than traditional methods, and use more memory.  The program will be longer, but it will be more clearly logically structured.  It will scale better than traditional approaches though, when there is shared memory but distinct processors.

\subsection{Hypothesis Three}

\section{Outline}
Where I summarise the upcoming dissertation (this section probably needs a different/better name).  Where in the intro do I explain the novelty?
\chapter{Background and Related Work}
\fxerror{Some of this probably should be chopped now...}{This review of related past work} focuses on the main areas of Computer Science that are relevant to this dissertation:  Formal Models of Computation (especially \gls{ps}) and \glsentrylong{cml-glossary} --- a big part of the \gls{ps} stuff is \gls{cps}, but that seemed to be big enough to merit its own chapter.  In none of these cases does the review come close to comprehensively covering the entire span of the given area.  It merely tries to cover as much of the relevant recent and classic work as possible, while giving an extremely brief introduction to these areas.  The interested reader who is unfamiliar with any of these topics is strongly advised to refer to the materials cited.

% See \url{http://forneylab.org/} for something seemingly quite similar to what I'm doing...;  fglib \url{https://github.com/danbar/fglib};  IGraph library \url{https://igraph.org/};  LGNpy \url{https://github.com/ostwalprasad/LGNpy};  

\section{Formal Models of Concurrent Computation}
Perhaps the earliest (or at least, the earliest that is still widely known) formalised model of concurrent computation is the Petri Net \cite{Dennis2011}, first conceived of by Carl Petri, initially for the purpose of describing chemical processes \cite{Petri2008}.

\begin{anfxerror}{Why not Petri Nets?}
Why are Petri Nets not more popular?  Why do we choose \gls{csp} or \glspl{actor} or \gls{ps} over them?
\end{anfxerror}

\cite{Varela2013}

This work 

\subsection{\glsentrylong{csp} \& Pi~Calculus}
\gls{csp} is a `process algebra' and abstract model of concurrent computation put forward by Hoare \cite{Hoare1985,Roscoe2011}.  A typical sequential computation is represented by a `process'.  Processes' ``behaviour is described in terms of the occurrence and availability of abstract entities called \textit{events}'' \cite[p.~478]{Roscoe2011}.  Should more than one event be available simultaneously for a given process, then one will be chosen non-deterministically.  This choice is internal to the process, and not influenced by or visible to any other process.  %E.g. for a situation where there is one possible event \(a\) at a given point in time, the process \(P\) will choose that event and then proceed according to the result, written as: \[ a \rightarrow P(a) \]

Concurrency is introduced by the existence of multiple processes.  In general, the processes evolve independently, responding to events as they come.  Should a particular event appear in the alphabet of multiple processes, however, then all processes \emph{must} choose to participate in that event at the same time.  Should all processes involved make such a choice, they engage in a synchronous multi-way atomic synchronisation (hence `communicating').  \gls{csp} has provided significant inspiration for concurrency design in a number of programming languages, notably including Ada \cite{Defense1983,Taft2013}, Occam \cite{Elizabeth1987}, Google's Go \cite{Meyerson2014} (not to be confused with the earlier language Go! \cite{Clark2004}, which itself was explicitly designed for concurrency) and \gls{cml} \cite{Reppy2011}. 

Milner appreciated \gls{csp}, which advanced concurrency models by explicitly incorporating \emph{synchronised} interaction, something Milner's earlier Calculus for Communicating Systems \cite{Milner1980} had lacked  \cite{Milner1993}.  Milner still regarded \gls{csp} as incomplete, however, in that it had no support for the concept of `mobility' -- i.e. the ability of the system to reconfigure itself during operation.  Pi Calculus was created as an attempt to build upon those earlier systems but present a complete calculus of concurrent computation in much the same way that Lambda Calculus \cite{Barendregt1984} is a complete calculus for sequential computation.\footnote{Milner also pointed out that sequential computation is, in fact, a special case of concurrent computation.}

\subsection{\label{subsec:actors}\texorpdfstring{\Glspl{actor}}{Actors}}
The \gls{actor} \cite{Agha1986} model was introduced by Hewitt \cite{Hewitt1973}.  Much like \gls{csp} \& its cousins, the \gls{actor} model is based around the concept of separated, sequential but communicating processes which exchange messages.  Again, the processes make decisions and proceed based on their communications.  A key difference, however, is that in the \gls{actor} model the message exchanges are \emph{a}synchronous.  Each \gls{actor} has its own `mailbox', and may send messages to other \glspl{actor} so long as it knows their name (which is equivalent for this purpose to a concept of an address for the \gls{actor}), \emph{but does not wait at all for a response before proceeding}.

The \gls{actor} model is a popular one for concurrent programming, possibly owing to its intuitive concept.  The fact that communication is asynchronous makes \glspl{actor} much more suitable for modelling distributed systems without shared memory than \gls{csp} or similar -- \glspl{actor} can send messages and proceed without (necessarily) needing to wait for a response, instead continuing to process based on the messages they themselves have received.  By contrast, a system with synchronous communication would have prohibitive time costs, given the relative slow speed of typical links between distributed computers as compared to their capacity for local processing.  Many \gls{actor} systems have been implemented for different programming languages (e.g. \cite{Varela2001,Srinivasan2008,Charousset2016,Bernstein2016} \fxnote[inline,nomargin]{[need more refs?]}), and in fact it is a core component, and perhaps largely responsible for the success, of Erlang/OTP \cite{Armstrong2010,Armstrong2013,Vinoski2012}.  A relatively new language, Pony \cite{Clebsch2015,Clebsch2017}, takes this even further.
\begin{anfxnote}{Positive example for actors?}
In Concurrency in .NET, Terrell described using an actor to control access to a shared resource and stated that it was an extremely effective solution.
\end{anfxnote}

\Glspl{actor}, when used for non-trivial real-world software, have been criticised at times \cite{Welsh2013,Stucchio2013}.  While some of the criticisms described are implementation-specific (relating to Akka, a Scala \gls{actor} library), a common thread is that \glspl{actor} do not compose well.  This has the negative consequence that it is difficult to combine an \gls{actor} with anything else to create a new abstraction, and can require extensive modifications in source code to make relatively simple changes.

\subsection{Join Calculus}

\begin{anfxwarning}{Join Calc refs}
Chemical abstract machine, joinads, Scala communicating objects (IIRC, that was join calculus based), reagents
\end{anfxwarning}

\subsection{Others}
Mobility calculus.  Others?

% \subsection{Criticisms}

% Gorlatch \cite{Gorlatch2004} argued against basic message passing, decrying it as an unnecessary and unhelpful complication and favouring \fxerror*{What does `collective operations' mean?}{`collective' operations} instead.  This criticism focused upon \gls{mpi} as it was at the time, however, and made no reference to either Actors or \gls{csp}. \fxerror{Does subsection this truly belong here?}

\section{\glsentrytext{ps}/\glsentrytext{mc}}
% \Gls{ps}, also known as \Gls{mc} (the two terms are generally used interchangeably), is a bio-inspired model of computing created by Gheorghe Păun in the late 1990s \cite{tPaun98a,Paun2000}.  It was originally conceived of by considering the process of chemical reactions and exchanges that occur inside living biological cells \& the membranes within, and regarding this process as a form of computation.  \Gls{ps} was identified in 2016 by the National Research Council of Canada as ``a rigorous and comprehensive framework that provides a parallel distributed framework with flexible evolution rules.'' \cite[p. 17]{Wiseman2016}

% \Gls{ps}, also known as \Gls{mc} (the two terms are generally used interchangeably), is a bio-inspired model of computing created by Gheorghe Păun in the late 1990s \cite{tPaun98a,Paun2000}.  It was originally conceived of by considering the process of chemical reactions and exchanges that occur inside living biological cells \& the membranes within, and regarding this process as a form of computation.  \Gls{ps} was identified in 2016 by the National Research Council of Canada as \enquote{a rigorous and comprehensive framework that provides a parallel distributed framework with flexible evolution rules.} \cite[p. 17]{Wiseman2016}

\emph{\Gls{ps}}, also known as \emph{\gls{mc}} (the two terms are generally used interchangeably), is a bio-inspired model of computing created by Gheorghe Păun in the late 1990s \cite{tPaun98a,Paun2000}.  It was originally conceived of by considering the process of chemical reactions and exchanges that occur inside living biological cells \& the membranes within, and regarding this process as a form of computation.  \Gls{ps} was identified in 2016 by the National Research Council of Canada as \textcquote[][p. 17]{Wiseman2016}{a rigorous and comprehensive framework that provides a parallel distributed framework with flexible evolution rules.}

% Păun describes \gls{mc} \cite[p.~VII]{Paun2002} as:
% \begin{quote}
% Membrane computing is a branch of natural computing which abstracts from
% the structure and the functioning of living cells. In the basic model, the membrane
% systems - also called P systems - are distributed parallel computing
% devices, processing multisets of objects, synchronously, in the compartments
% delimited by a membrane structure. The objects, which correspond to chemicals
% evolving in the compartments of a cell, can also pass through membranes.
% The membranes form a hierarchical structure - they can be dissolved, divided,
% created, and their permeability can be modified. A sequence of transitions between
% configurations of a system forms a computation. The result of a halting
% computation is the number of objects present at the end of the computation
% in a specified membrane, called the output membrane. The objects can also
% have a structure of their own that can be described by strings over a given
% alphabet of basic molecules - then the result of a computation is a set of
% strings.
% \end{quote}

Păun describes \gls{mc} as:
\blockcquote[][p.~VII]{Paun2002}{Membrane computing is a branch of natural computing which abstracts from the structure and the functioning of living cells. In the basic model, the membrane systems -- also called P systems -- are distributed parallel computing devices, processing multisets of objects, synchronously, in the compartments delimited by a membrane structure. The objects, which correspond to chemicals evolving in the compartments of a cell, can also pass through membranes. The membranes form a hierarchical structure --- they can be dissolved, divided, created, and their permeability can be modified. A sequence of transitions between configurations of a system forms a computation. The result of a halting computation is the number of objects present at the end of the computation in a specified membrane, called the output membrane. The objects can also have a structure of their own that can be described by strings over a given alphabet of basic molecules - then the result of a computation is a set of strings.}

\Gls{ps} works analogously to a typical modern electronic computer, in that the system stores data and processes \& updates those data based on a predefined program, with a view to arriving at a computable answer based on the starting state and any inputs to the system \cite{Paun2002,Paun2010b}.  In the case of \gls{ps}, the data are multisets of symbols, representing various chemicals and their quantities.  These are found inside one or more \emph{cells},\footnote{Loosely based on real biological cells.} which (to a certain extent at least) form a hybrid between main memory and the processing units of a computer.  The instructions of the program itself are provided by \emph{rules}, which specify transformations of objects and interactions with the surrounding environment and other \fxerror{Need to define membranes}{membranes} or cells.

There are now, broadly, three main families of \gls{ps} variants:  \gls{clps} \cite{Paun2001,Paun2002}, \gls{tlps} \cite{tMaPaPaRo01a,Martin-Vide2003} and \gls{snps} \cite{Ionescu2006}.\footnote{Several other variants have been created, but most are used infrequently, if ever.  This work addresses only Numerical \gls{ps}, \gls{skps} and \gls{cps}, in \cref{subsec:numpsys}, \cref{chap:gcol} and \cref{sec:lr:cpsystems} respectively.}  \Gls{clps} is the original, and sees objects compartmented into \emph{membranes}, which are arranged in a graphical tree structure with the outermost membrane -- which separates the cell from its environment -- as the root of the tree.  In most variants, objects can evolve inside a membrane, but also be communicated between membranes (and the environment).  Furthermore, membranes can \emph{divide} or \emph{dissolve} themselves, and may have one or more special properties, such as \emph{polarization} \cite{Paun1999a}.

On the other hand, \gls{tlps} and \gls{snps} both arrange their computing compartments -- named cells or \emph{neurons}, respectively -- as nodes in arbitrary digraphs, with the edges between them representing connecting channels.  Whereas \gls{clps} emphasise the evolution of multisets of objects inside membranes of a given cell, \gls{tlps} and \gls{snps} emphasise communication between separate cells/neurons, and may not include any capacity for internal evolution inside cells --- if new objects are required, they are imported via communication with the environment, which is considered to possess an unlimited number of all objects but has no rules of its own.

% While \gls{tlps} have arbitrary alphabets, only one object is used in \gls{snps}, the `spike'.  This means that \gls{tlps} are frequently much like \gls{clps} in that they have custom objects for each purpose, with the key difference (usually) being in the arrangement of the compartments/cells (\glspl{pe}) relative to each other and the choice between the two motivated primarily by which one seems like a better fit to the computation to be modelled.

While \gls{tlps} have arbitrary alphabets, only one object is used in \gls{snps}, the \emph{spike}.  This means that \gls{tlps} are frequently much like \gls{clps} in that they have custom objects for each purpose, with the key difference (usually) being in the arrangement of the compartments/cells relative to each other and the choice between the two motivated primarily by which one seems like a better fit to the computation to be modelled.

Conversely, \gls{snps} represent everything through the use of differing quantities of the spike, kept in different neurons.  This means that it can be more complex to model certain problems, but also arguably means that \gls{snps} are, \textit{prima facie}, closer to Lambda Calculus \cite{Barendregt1984} and Church Numerals (see e.g. \cite{Koopman2014,Hinze2005}), as well as Register Machines (see e.g. \cite{Korec1996}) (and indeed Register Machines have been simulated with \gls{snps} \cite{Pan2010}).  All three types of \gls{ps}, \fxnote{If there is some sort of requirement to bulk out the number of references, a whole bunch more could be included here -- perhaps even half of the P systems papers ever published include some form of universality result}{in some form}, have been proven Turing-universal though \cite{Bernardini2005,Chen2008,Freund2005}, so all three should be capable of expressing the same computations in different forms.  Furthermore, because \gls{snps} can be easily represented numerically, they lend themselves well to vector/matrix representations \cite{Zeng2010,Martinez-del-Amor2021,Gheorghe2021,Hu2016}.  This means that, potentially, \gls{snps} implementations can take advantage of high-performance techniques such as directly using \gls{blas} and/or \glspl{gpu} \cite{Aboy2019}.

Arguably, the most noteworthy and important aspects of \gls{ps} models are that they:  i) Generally have no space limit.  That is, they contain an unbounded number of cells, objects and membranes;  ii) Across all cells and membranes, all rules that can be applied are applied, as many times as possible given the current number of objects available.  These two features mean that \gls{ps} have unbounded space and processing capacity, which can be used to solve traditionally computationally difficult problems relatively quickly \cite{Sosik2003,Jimenez2003,Paun1999a,Henderson2020}.  Most of these solutions, however, rely on trading time complexity for space complexity.  While this works in the theoretical framework, electronic simulations of the systems do not have access to unlimited instantaneous memory space, meaning many of the fast solutions are impractical with current real-world computers, e.g. \cite{Cooper2019,Cooper2019a} \fxnote[inline]{[refs]} (see further \cref{sec:psystemsuses}).

\subsection{\label{subsec:numpsys}Numerical \glsentrytext{ps}}
Numerical P systemsss
\fxerror{Expand/explain}{\cite{Florea2017,Paun2006a,Yuan2019,Leporati2014,Maeda2014,Pavel2010,Pang2018,Raghavan2020}}

\subsection{\label{sec:psystemsuses}Operative Implementations and Practical Uses of \texorpdfstring{\gls{ps}}{P systems}}
\Gls{mc} is not just a theoretical model with limited practical use.  Besides Image Processing \& Computer Vision (see \cref{subsec:imgprocpsys}), \gls{ps} variants have been applied to a range of fields, from power grid management to robotic control systems \cite{Zhang2017}.

\fxwarning{Expand/explain}{\cite{Zhang2020,Colomer2010,Gheorghe2010,Liu2016,Huang2016,Florea2017,Perez-Hurtado2018,Perez-Hurtado2010,Verlan2012,Syropoulos2004,Liu2020,Lefticaru2011,Oltean2008}}
[P-Lingua and simulation systems, e.g. MeCoSim]

\subsection{\label{subsec:imgprocpsys}Image Processing and Computer Vision in \texorpdfstring{\gls{ps}}{P systems}}
\fxerror*{Expand/explain}{\cite{Zhang2012,Yuan2019}}

Perhaps owing to the potentially unbounded space and parallelism capacity of \gls{ps}, combined with the embarrassingly parallel nature of many tasks in Image Processing \& Computer Vision, the latter have proved to be fertile ground for the former, although not every publication puts its model to the test with a computerised simulation, or if it does, the authors may only provide scant details \cite{Diaz-Pernil2019}.

\citeauthor{Christinal2011} \cite{Christinal2011} described a family of \gls{tlps} to perform region-based segmentation of both 2D and 3D images.  Despite their family of systems requiring only two cells, it also needed custom rule sets based on the size of the images as well as the number of colours present, with a number of rules per set proportional to the same measurements.  The paper showed the results of simulating the system, but provides no details on performance.

% \citeauthor{Diaz-Pernil2013} \cite{Diaz-Pernil2013} commented that \enquote{... commonly [a] parallel algorithm needs to be re-designed with only slight references to the [sequential original].  ... the design of a new parallel implementation not inspired by the sequential one allows ... the proposal of new creative solutions.}  They then demonstrated this fact by designing a new edge detection and segmentation algorithm named `A Graphical P (AGP) segmentator', inspired by the Sobel operator (see e.g. \cite{Nixon2012}) and using the segmentation method from \cite{Christinal2011}, which they modelled in \gls{tlps}.  The authors implemented their new algorithm on a \gls{gpu} and compared it with an implementation of the 3x3 and 5x5 Sobel operators, finding that theirs had near-identical runtimes but superior edge detection capabilities.

\citeauthor{Diaz-Pernil2013} \cite{Diaz-Pernil2013} commented that \enquote{\textelp{} commonly \textins{a} parallel algorithm needs to be re-designed with only slight references to the \textins{sequential original}.  \textelp{} the design of a new parallel implementation not inspired by the sequential one allows \textelp{} the proposal of new creative solutions.}  They then demonstrated this fact by designing a new edge detection and segmentation algorithm named `A Graphical P (AGP) segmentator', inspired by the Sobel operator (see e.g. \cite{Nixon2012}) and using the segmentation method from \cite{Christinal2011}, which they modelled in \gls{tlps}.  The authors implemented their new algorithm on a \gls{gpu} and compared it with an implementation of the 3x3 and 5x5 Sobel operators, finding that theirs had near-identical runtimes but superior edge detection capabilities.

\citeauthor{Diaz-Pernil2013a} \cite{Diaz-Pernil2013a} further explored modelling classic image processing techniques by implementing Guo \& Hall's binary image skeletonisation technique \cite{Guo1989} with \gls{snps}.  The overall system's rules templates are reasonably simple, but include references to a set \(DEL\) (used as a lookup to determine whether a cell should turn white or stay black) which does not appear to be modelled inside the system, meaning that it is not self-contained.  The authors simulated this system on a \gls{gpu}, but found that their implementation was upwards of twice as slow as another pre-existing implementation.  Confusingly, however, they state that one of the reasons for this is \enquote{that the use of an alphabet with only one object, the spike \(a\), does not fit in the GPU architecture}.  This statement is perplexing, given that spikes can easily be represented as simple integers.  The authors also commented that the synchronous nature of the model is unrealistic, and imposing a global clock upon the system can be problematic.

\citeauthor{Nicolescu2014} \cite{Nicolescu2014} alternatively applied \gls{cps} to image skeletonisation based on Guo \& Hall's technique \cite{Guo1989}, presenting three forms of a solution: Synchronous versions that use multiple or a single cell (essentially the latter replicates the former via the use of sub-membranes), and an asynchronous multi-cell version.  The asynchronous version no longer assumes that all messages are passed between cells simultaneously and instantaneously, compensating for this by increasing the number of messages used.  This form, while arguably more realistic to modern computers, requires a greater message complexity. A simplified \Gls{actor}-model-based (see \cref{subsec:lr:actors}) implementation using \fsharp{}'s \texttt{MailboxProcessor} \cite[ch.~11]{Syme2015a} was presented, but no results from running it were reported.

\citeauthor{Nicolescu2015a} \cite{Nicolescu2015a} further applied \gls{cps} to seeded region growing of greyscale images.  The described system used a two-level approach, based on the `Structured Grid Dwarf' of the 13 Berkeley Dwarves \cite{Asanovic2006}, where the image was divided into rectangular blocks of multiple pixels.  Each block was modelled with a single cell, inter-block processing was carried out via message passing, and intra-block processing was performed by typical object evolution.  It was again suggested that this would fit well to the \Gls{actor} model.

\citeauthor{Diaz-Pernil2016} \cite{Diaz-Pernil2016} built upon the AGP segmentator algorithm to create a version that works with RGB images rather than greyscale and applied it to a common medical Computer Vision task, isolating the `optic disc' in images of the inner eye.  With this they used the skeletonisation algorithm from \cite{Diaz-Pernil2013a} and a number of other steps not based on \gls{ps} to produce a complete imaging pipeline.  The authors implemented this on a \gls{gpu}, and found that their system was both more accurate and faster than previous systems.

\fxerror{Expand this a lot.}{Most directly relevant to the current work} are \cite{GimelFarb2013a,Gimelfarb2011,Nicolescu2014b}, which model Dynamic Programming \gls{sm} in \gls{ps}, and indeed saw the genesis of \gls{cp}.

\subsubsection{\glsentrytext{ps} on \glsentrylongpl{gpu}}
In many instances, a \gls{ps} model for a problem involves many independent small elements processing their data separately, and perhaps updating each other's state at the end of a step.  Given that this sounds remarkably close to the Single-Instruction Multiple-Thread \cite[Ch. 4.4.1]{Hennessy2012} nature of modern \gls{gpgpu}, it is no surprise that there has been much work put into simulating \gls{ps} on \glspl{gpu}.

\fxerror*{Expand/explain all these}{\cite{Cecilia2010,Cecilia2010a,Cecilia2013,Macias-Ramos2015,Martinez-Del-Amor2015,Martinez-Del-Amor2013a,Maroosi2014,Maroosi2014a}}
% \section{\glsentrylong{mpbsm}}

% \subsection{Preliminaries}
% What do I actually need to put in here?

% \subsubsection{Stereo Matching}

\subsection{\label{subsec:smgeneral}\glsentrytext{sm} in General}
Szeliski defines \gls{sm} \cite[p. 469]{Szeliski2011} as ``the process of taking two or more images and estimating a 3D model of the scene by finding matching pixels in the images and converting their 2D positions into 3D depths.''  It is essentially an attempt to replicate one of the techniques the brain uses to provide depth perception,\footnote{The brain has others, such as exploiting familiarity with everyday objects to estimate their actual size, and thus their rough distance from the eyes.} namely correlating the images received from each eye to estimate the distances to objects within view, but using digital images and a computer.

Indeed, \gls{sm} is not the only method for computer depth perception in use \cite{Sinha2020}.  Other approaches include, for example, Structured Light \cite{Giancola2018}, Time-of-Flight \cite{Hansard2013} and LiDAR \cite{Dong2017}.  \Gls{sm} has some advantages over those other techniques, though.  It is the only one which is entirely passive, i.e. it takes in data from the environment without interacting with it in some way, whereas the other three all involve projecting some form of light into the environment.  It is also arguably quicker to perform the necessary observations of the environment than the other methods, in that only a single pair of images need be captured, which can be accomplished in the time it takes for the pixel values to be read from the sensor planes into storage.  The choice of which method is most appropriate depends heavily upon the intended use of the computerised depth perception.  Or, if sufficient processing power is available, two or more of them can be used in combination to offset each other's weaknesses \cite{Zanuttigh2016}.

The `canonical' stereo camera arrangement is two cameras arranged in parallel, with a small horizontal offset between them.  This configuration leads to a general expectation that changes in the points in the scene will be shifted along one image's x-axis as compared to the other image's.  The identified distance of a shift is termed the `\gls{disparity}', and is used with other information about the cameras to compute an estimate to the matched points in the scene.  Generally, it is \fxnote*{name of assumption?}{assumed} that points in the image from the left camera will appear further to the right in said image as compared the same point in the right camera's image, and vice versa.

\begin{figure}
    \centering
    \includegraphics[width=1.0\textwidth]{chapters/litreview/images/stereo_matching-eps-converted-to.pdf}
    \caption[Diagram of the basic process of \gls{sm}]{Diagram of the basic process of \gls{sm}. In this instance, for each pixel in the left image, a horizontal range of the pixels in the right image are searched, to find the one on the right that matches most closely to the one on the left. This has the effect that the \gls{disparitymap} is from the perspective of the left camera. The red dots represent the compared pixels. The brown dashed arrow and vertical bars represent the range of pixels in the right image to compute matching scores against. The length of the range is the lesser of the number of pixels before reaching the left border of the image, or the maximum \gls{disparity}, which is a parameter set by the user and represented here by \(\Lambda\).  Image from \cite{bsmpcvpic}.}
    \label{fig:stereomatchingbasic}
\end{figure}

% While precise approaches vary, the vast majority of \gls{sm} algorithms utilise some form of pixelwise comparison between images.  The basic process for this is shown in \autoref{fig:stereomatchingbasic}.    This comparison function be as simple as taking the absolute difference of the pixels under comparison.

In the general case \gls{sm} is impossible to perform perfectly because it is an ill-posed problem \cite{Gimelfarb1998}.  Going from a three-dimensional scene to a two-dimensional image necessarily involves a loss of information.  For any given image there are potentially an unbounded number of possible real-world scenes that could produce said image.  Using multiple images -- the more the better -- for \gls{sm} permits some recovery of information, but the process inevitably suffers from various sources of noise (where `noise' is defined broadly).

Liu \textit{et al.} \cite{Liu2005} describe four types of noise:  Signal, geometric, electronic and optical.  Signal noise arises from the normal operation of digital cameras, and electronic from differences between the internal operation of cameras used to capture images from different perspectives of the same scene.  Optical noise mainly stems from differences in the intensity and colour of light seen by the cameras at different perspectives when capturing images of the scene, caused by differences in the interactions between the objects of the scene and the available light sources at different points.  Lastly, geometric noise is a natural consequence of the fact that different perspectives must be used, and can be caused by issues as simple as the fact that points visible in one scene may not be visible in the other -- so-called `occlusions', caused by one part of the scene obscuring another part.

A key consequence of the last source of noise is the fact that, even in ideal circumstances with multiple `perfect' cameras, flawless lighting throughout the scene and \fxnote*{Provide reference for Lambertian surfaces}{objects which do not reflect light differently at different parts of their surfaces}, occlusions mean that for an arbitrary scene it is impossible to be certain a given algorithm has achieved a perfect reconstruction of the depths of the scene \cite{Gimelfarb1998}.  Strictly speaking, it \emph{is} possible that \gls{sm} produces an entirely accurate \gls{disparitymap}, but there would be no way to know without the use of additional information.

\fxwarning[inline,nomargin]{Include some example images to show the idea of stereo matching?}

\subsubsection{Image Rectification}
To reduce the computational complexity involved in performing \gls{sm}, many, perhaps most, algorithms only directly compare pixels along a single line in each image, typically the same horizontal line \cite[Ch. 11]{Szeliski2011}.  If the lines in the two images do not correspond to roughly the same part of the scene, then the matching process will likely fare poorly.  Such a discordance can occur when the cameras were not adequately aligned in terms of their spatial positions and angles relative to each other at the time of mutual image capture.

To overcome the challenges caused by mismatched lines, stereo image pairs are usually `rectified' (see e.g. \cite[Ch. 1.5.1]{Wohler2013}), wherein the captured images are adjusted so that they were effectively \fxerror*{Rectification could be described more precisely}{taken by properly aligned cameras}.  If rectification is performed well, the lines in the image should be properly aligned.  The parameters used in rectification in turn are deduced via camera calibration (see e.g. \cite[Ch. 1.4]{Wohler2013}), though neither topic is discussed further here.  For current purposes, all stereo image sets used are assumed to have been appropriately rectified already.

\begin{anfxnote}{}
    Discuss epipolar geometry?
\end{anfxnote}

\subsubsection{Middlebury}
\fxnote[inline,nomargin]{Talk about the Middlebury benchmarks, resources and website here.}

\subsubsection{Local vs Global}
\fxerror*{Expand/explain}{\cite{Scharstein2002}  (similar terms were in use earlier \cite{Gimelfarb1998})}

Perhaps the simplest and most obvious ways to perform \gls{sm} involve simply comparing the values of pixels in one image to the values of pixels in the other.

\subsubsection{\glsentrylong{mrf} \& Bayesian Inference}
\fxerror*{Expand/explain}{\cite{Kolmogorov2015,Blake2011}}

Not all global \gls{sm} algorithms utilise message passing, e.g. Graph Cuts \cite{Kolmogorov2001,Tappen2003}

Geman \& Geman \cite{Geman1984} showed that \glspl{mrf} are equivalent to Gibbs Distributions and that the two could be applied usefully to image tasks \cite{Gimelfarb1999}.

\begin{anfxwarning}{Pixel similarity measures}
    Move the below discussion about pixel similarity measures further up, probably into the \gls{sm} in general section?
\end{anfxwarning}

Frequently, in global methods the function used for the data cost is quite simplistic.  Most common is the use of simple absolute difference between the intensities of the pixels compared.  Other popular methods include \gls{sad} and \gls{ssd} \fxerror[inline]{[ref]}, adaptive window methods \cite{Yoon2005,Yoon2006}, and Birchfield \& Tomasi's Pixel Dissimilarity Measure \cite{Birchfield1998}.

In general, most \gls{mrf} approaches to \gls{sm} tend to use a truncated linear function to estimate the discontinuity/smoothness cost.  Such a function typically takes a form such as \[ E_{discontinuity} = \alpha \times min(| d_p - d_q |, \beta) \] where, for the purposes of this equation, \(E_{discontinuity}\) represents the total estimated cost of the assignment; \(\alpha\) is a scaling coefficient that may or may not be used; \(\beta\) is a constant that provides the upper limit to the cost estimate; and \(d_p\) and \(d_q\) are the proposed \fxwarning*{labels?}{labels} of the current pixel and its neighbour currently under consideration.  While a simple absolute difference function is perhaps the most common applied to the labels, it is important to note that it is not the only one that could be used.  

For example, Ha and Jeong \cite{Ha2016} use a two-step \fxnote*{Define Potts model}{Potts model}, with different penalties for a difference of 1 compared to a difference of 2 or greater. Conversely, Tan \textit{et al.} \cite{Tan2017} comment that a typical Potts function can be viewed as a special case of the absolute difference truncated linear function, where the truncation value (\(\beta\) in the equation above) is 1, while the coefficient is the value of the Potts penalty parameter.

The choice of the truncated linear function is motivated by the assumption that most surfaces in images either are planar, or smoothly vary in disparities, and thus larger jumps should be penalised more heavily, but very large jumps are almost certainly indicative of an object boundary where a large difference in disparities is warranted.  Therefore, at a certain point, the penalty to assign significantly different values should stop growing, so as not to reduce the likelihood of correctly assigning large differences in disparities at object edges.

\subsection{Dynamic~Programming}

\gls{dpsm} was first introduced by Gimel'farb, Marchenko and Rybak in 1972 \cite{Gimelfarb1972}.\footnote{There is a popular misconception that \gls{dpsm} was introduced in the 1980s with \cite{Ohta1985}.  This is plainly false, given that \gls{dpsm} was first described years earlier.  The confusion is unsurprising, however, because \cite{Ohta1985} was likely the first description of \gls{dpsm} many in the English-speaking world saw, as a consequence of the Cold War.  For example, the authors \cite{Salmen2009} of seem to have this misunderstanding.}  

% \begin{anfxnote}{DP's streaking}
%     Be sure to mention the streaking commonly seen with \gls{dpsm} -- if nothing else it will be important for explaining why \gls{sgm} was created.
% \end{anfxnote}

Anecdotally, variants of \gls{dpsm} are still popular in practical applications of \gls{sm} because it tends to give acceptable results \fxerror[inline]{[ref?]} and is amenable to fast implementations with low-powered devices \fxerror[inline]{[ref?]}.

\begin{anfxnote}{Why mention DP?}
    Partly because it is a basis for other algorithms, but at least as much because the process of propagating information up and down the scanlines that it entials is very much reflective of message passing.  ``The main difference between DP and 1D BP is the word `message' '' -- Georgy (at my provisional).  Something similar was mentioned in appendix B to \cite{Szeliski2011}.
\end{anfxnote}

\subsubsection{Symmetric Dynamic Programming Stereo}
Motivated by observations of the physical reality of the image capture process and propounded by Gimel'farb \fxerror*{Expand/explain}{\cite{Gimelfarb1979,Gimelfarb2001} + \cite{Nguyen2013,VanMeerbergen2001}.  \cite{Khan2016}}

\subsection{\glsentrylong{bp}}
\gls{bp} was introduced by Pearl \cite{Pearl1982} for use with inference engines, in the context of Bayesian Statistics \fxerror[inline]{[ref]} and Gibbs Distributions \fxerror[inline]{[ref]}.  \gls{bp} was first applied to \gls{sm} in \cite{Sun2003} where it demonstrated excellent matching performance compared to many contemporary matching algorithms, but the `breakthrough' paper was arguably \cite{Felzenszwalb2006}, where a near-real time implementation was presented which still had extremely good results.  Szeliski commented in c. 2011 that \gls{lbp} was still used at that time in some of the best-performing \gls{sm} algorithms \cite[p. 163]{Szeliski2011}.

\begin{figure}
    \centering
    \includegraphics[width=1.0\textwidth]{chapters/litreview/images/bp_diagram_recoloured.pdf}
    \caption[Pictorial representation of the concept behind \acrlong{lbp} for \gls{sm}]{Pictorial representation of the concept behind Loopy Belief Propagation for Stereo Matching. The symbols in the central cell refer to sum-product Belief Propagation, one of the earliest and mostly widely discussed forms of Belief Propagation. Each cell sends a new message to its neighbouring cells at each iteration after in turn having received and processed new messages from the neighbours in the previous iteration. The outgoing message to a given neighbour is computed from the information received from the other neighbours previously, represented here by the three thin and one fat arrow.  Image from \cite{lbpmpsmpic}.}
    \label{fig:bpdiagram}
\end{figure}

Yang \textit{et al.} \cite{Yang2006a} built upon \fxerror*{Explain hierarchical BP}{hierarchical \gls{bp}}, adding in extra steps before and after the \gls{bp} process.  They combine information derived from using the mean shift algorithm \cite{Comaniciu2002}; a colour-weighted correlation method based on Yoon \& Kweon's \cite{Yoon2006} applied to both the left and right images; a left-right consistency check to detect occluded pixels; a plane-fitting process based on Tao \& Sawhney's \cite{Tao2000}; as well as \gls{bp} itself.  While combining these various techniques leads to a highly-accurate disparity map,\footnote{This algorithm achieved the top ranking on Middlebury when it was first introduced.} this approach is \emph{extremely} slow.

Typical \gls{bp} uses the four-connected neighbourhood to define the neighbours of each given node in the grid.  This means that each node passes messages to and from it's immediate neighbours up, down and to the left and right of it in the grid.  Other neighbourhood arrangements are possible though, depending on the underlying model one wants to use.  For example, Tan \textit{et al.} \cite{Tan2017} describe an approach to \gls{bp} where every pixel is considered to be a neighbour of every other pixel.  Messages are weighted according to the distance across the grid between the neighbours, with nearer neighbours having a greater impact upon a pixel's final beliefs.  The major advantage of this approach is that it almost eliminates the need for repeated iterations of message passing.

Ha and Jeong \cite{Ha2016} suggested a different approach for scheduling the messages.  Instead of each pixel repeatedly exchanging messages with its neighbours until a reasonable amount of the grid has been spanned, they start in one of the corners in the image, and sequentially pass messages along two directions until reaching the other corner, repeating this process once for each corner.  The great advantage of this is that in principle one only needs to perform message exchanges in each direction once.  Their implementation still required roughly \SI{3.5}{\second} to complete, however, without returning a significantly more accurate disparity map.\footnote{The authors claimed that their method was \numrange{300}{600} times faster than `standard` \gls{bp}, but they did not specify their stopping condition.  Based on their reported results it appears that they used well over 300 iterations on an image with their standard comparison -- many more than would be reasonable for image sizes likely to be targeted for real-time \gls{sm}.}

Balossino \textit{et al.} \cite{Balossino2007} suggested an alternative formulation to the traditional grid of \gls{lbp}.  Instead, they built a forest of trees, each of which was rooted at the given pixel under consideration, and which has a handful of neighbouring pixels as children.  The attraction of this approach is that it restores the properties of optimality and convergence described in \cite{Pearl1982} for one round of messages up and down each tree.  This advantage is tempered, however, by the necessity of combining results from different trees.  The final accuracy appeared to be worse than with \cite{Felzenszwalb2006}, and there was no reporting of the running time, though it seems unlikely that this approach was fast.

It should be noted that \gls{bp} is \emph{not} regarded as the current top-performing \gls{sm} algorithm.  Tippets \textit{et al.} found c. 2012 that, of algorithms implemented on the CPU, SADL from \cite{VanDerMark2006} was the fastest accurate-enough method, and ADCensus from \cite{Mei2011} was the most accurate, and in fact was described as ``Pareto-optimal'' by Tippets \textit{et al.}  \fxnote{Move this paragraph?}  In terms of \gls{gpu} implementations, the algorithm from \cite{Zhao2011} was the fastest by far (though it only properly worked when observing scenes with motion).  The authors do not state an overall most-accurate \gls{gpu}-based algorithm, but based on Fig. 7 in \cite{Tippetts2016}, it appears that the \gls{gpu} implementation of the ADCensus algorithm from \cite{Mei2011} was again the best.\footnote{Other, faster, algorithms were also discussed, but those required specialist hardware such as \glspl{fpga} or \glspl{dsp}.}  \Gls{bp} \emph{is} amenable to parallelisation (unlike its traditional rival, Graph Cuts \cite{Tappen2003}) and \gls{gpu} implementations, but the main point of interest for it in this work is the fact that it is explicitly built around the concept of independent processing elements exchanging messages.

\begin{anfxwarning}{Top algos on Middlebury}
    Thinking about it, I probably should investigate the current top-performing algorithms on Middlebury, at least the ones which have accompanying publications.
\end{anfxwarning}

\subsubsection{Real-time/resource-constrained \glsentrylong{bp}}
One of the major drawbacks of \gls{bp} as compared to a number of other approaches to \gls{sm} is that a simple naïve implementation is both quite slow, and very memory-intensive.  Slow because of the requirement to perform many iterations, and memory-intensive because \emph{at least} one copy of the data costs and the message estimates for each neighbour must be stored in memory, with the result that a number of values on the order of at least \(O(5XYD)\) are kept in memory, where X and Y are the width and height of the stereo images, and D is the size of the disparity range.

Seeking to derive the comparative benefits of a global stereo algorithm without compromising resource and time requirements too much, there have been a number of attempts at a real-time \gls{bp} algorithm \cite{Liang2011,Perez2010}.

Felzenszwalb \& Huttenlocher \cite{Felzenszwalb2006} made three significant improvements:  i) They demonstrated a way to reduce the complexity of the message update process from \(O(|D|^2)\) to \(O(|D|)\) (where \(|D|\) is the total number of potential disparity labels).  ii) They showed that, because each pixel relies entirely upon the messages received from its neighbours at the previous iteration, only half of the pixels in fact need to be updated in a given iteration, without affecting the final results.  This both halves the number of message computations required for each iteration, but moreover means that only a single copy of the messages need be kept while ensuring that messages computed earlier in an iteration have no impact upon messages computed later.  iii)  They introduced a hierarchical approach, where the first iterations were performed over a much smaller grid, representing an amalgamation of the actual grid, but later iterations would operate over larger and larger grids until reaching the full size.  This had the benefit of propagating information across the grid in a much faster fashion, with relatively little loss in accuracy.  Almost every claimed real-time \gls{bp} algorithm since uses the hierarchical approach.

Notably, Tippetts \textit{et al.} characterised the final algorithm implemented in \cite{Felzenszwalb2006} as Pareto-optimal against almost all other CPU-based \gls{sm} algorithms that they examined, suggesting that there were only five others which provided either better accuracy \emph{or} faster runtimes.  Of course, in the meantime there likely have been improvements in both metrics by newer algorithms.

Yang \textit{et al.} \cite{Yang2006} claimed that they had devised a new approach that would provide a 45x speedup, and boasted that their system could achieve a frame rate of 16 \gls{fps} on a 320 x 240 image with 16 disparity levels.  This claim was largely based, however, in the fact that they used a \gls{gpu} to implement it -- but later stated that they had not yet implemented their method on a \gls{gpu}.  Furthermore, they did not present anything conceptual that had not already been described in \cite{Felzenszwalb2006}. %by Felzenszwalb \& Huttenlocher.

Yu \textit{et al.} \cite{Yu2007} presented a proposed approach for compressing the messages, thus reducing total memory occupied, but it has not proven popular.  This may be because it is not amenable to parallelisation, thus significantly reducing its practicality \cite{Yang2010}.

Yang, Wang \& Ahuja \cite{Yang2010} proposed an approach which they claim needs only constant memory space, regardless of the number of disparities involved, while still returning results that are almost as accurate.  For example, they claim that for an image with 800 x 600 pixels and 300 disparity levels, their algorithm requires only around \SI{9}{\mebi\byte} of memory --- though it is not clear though whether they include storing the computed data costs in that amount or not.  The main element of their approach is that as they move from the coarser levels of the hierarchy, they proportionally reduce the number of disparity labels considered at each level, keeping the total memory required constant.  %This leads to an issue in that, should the true disparity not be selected for inclusion at a reduction, that pixel will never see the correct disparity label assigned to it.  To work around this, they 

Gupta \& Cho \cite{Gupta2012} used 3x3 tiles in their hierarchical method, rather than the usual 2x2.  This meant that their process was somewhat faster overall, and means that at the more coarse levels they need less memory.  The other main differences between their method and previous ones are that they use an `alternative schedule method' borrowed from \cite{Tappen2003}; and they use a different disparity refinement operation as final step.  The results, in terms of accuracy and speed, do not appear to be any better than earlier papers, though.

Xiang \textit{et al.} \cite{Xiang2012} also boasted of a new technique that enabled faster speeds, but again their implementation largely merely borrowed concepts from \cite{Felzenszwalb2006} and used a \gls{gpu}.  They did improve accuracy results, however, by incorporating Yoon \& Kweon's \cite{Yoon2005} adaptive support-weight approach as a post-processing step, with minimal extra computational requirements.

Tan \textit{et al.} \cite{Tan2017} claim that their fully-connected \gls{bp} method is highly-amenable to parallelisation, suggesting it could be implemented to run in real-time, but they do not appear to have done so themselves.

% \subsection{\glsentrylong{sgm-glossary}}
% This won't really be touched upon in this work anymore, but it might be a good idea to mention/describe it (and perhaps \gls{cp} too), if just so I can mention it again as an obvious future work target.

% \subsection{Noise-Driven Concurrent Stereo Matching}


% \subsection{\label{subsec:concprop}\glsentrylong{cp}}

% \cite{Gong2015,Gong2013a}

% \subsection{Message Passing \glsentrytext{sm} -- other?}
% Look at, e.g.:
% Tan, X. et al. (2017) ‘Efficient Message Passing Methods With Fully Connected Models for Early Vision’, IEEE Transactions on Image Processing, 26(12), pp. 5994–6005. doi: 10.1109/TIP.2017.2750406.
% Ružic, T., Pižurica, A. and Philips, W. (2011) ‘Neighbourhood-consensus message passing and its potentials in image processing applications’, in Astola, J. T. and Egiazarian, K. O. (eds) Image Processing: Algorithms and Systems IX. San Francisco: Society of Photo-Optical Instrumentation Engineers, p. 78700Z. doi: 10.1117/12.872464.
% Ružić, T., Pižurica, A. and Philips, W. (2012) ‘Neighborhood-consensus message passing as a framework for generalized iterated conditional expectations’, Pattern Recognition Letters, 33(3), pp. 309–318. doi: 10.1016/j.patrec.2011.10.014.
% Szeliski, R. et al. (2008) ‘A Comparative Study of Energy Minimization Methods for Markov Random Fields with Smoothness-Based Priors’, IEEE Transactions on Pattern Analysis and Machine Intelligence, 30(6), pp. 1068–1080. doi: 10.1109/TPAMI.2007.70844.
% Thomas, D. et al. (2019) ‘Revisiting Depth Image Fusion with Variational Message Passing’, in 2019 International Conference on 3D Vision (3DV). IEEE, pp. 328–337. doi: 10.1109/3DV.2019.00044.


\section{\glsentrylong{cml-glossary} \& related}

\begin{anfxerror}{\gls{cml} no longer relevant?}
    This entire section is now not nearly so relevant.  Not sure what to replace it with/modify it to, however.  Maybe something about reported results of implementations of the models of concurrent computation?  (In which case, I could probably just fold it into that section)
\end{anfxerror}

\Glsentryfirst{cml} \cite{Reppy1991,Panangaden1997} is an approach to concurrent programming originally developed by Reppy (based on his earlier `Pegasus Meta-Language' \cite{Reppy1988}).  It was created originally as a library in Standard ML of New Jersey, but its concepts have subsequently appeared elsewhere.   \Gls{cml} is designed to avoid many of the problems with concurrency that arise in traditional sequential programming, where the use of locks, mutexes and semaphores etc. are frequently required, and often lead to the potential introduction of problems such live/deadlocks, data races and extreme resource contention.  This is achieved by changing the approach to concurrent programming to one of logically separate, internally-sequential, processing elements that share data as required by `passing messages'\footnote{This is the logical concept, but there is not strictly any specific required software implementation.} between themselves.  In CML, these logical processing elements are referred to as threads, and they exchange messages over channels synchronously, i.e. there is a temporal overlap between one thread offering to send, and another to receive, over the same channel, and the first to offer blocks until the second makes its offer.%  Hereafter in this paper, unqualified use of the word `thread' refers to CML's threads, as opposed to other meanings of the term.

Reppy describes a concurrent program as one that supports multiple sequential sub-programs conceptually executing in parallel separately, but interacting through shared resources to achieve a common goal.  CML is concerned with the scenario where said interactions are explicit, and in order to facilitate that ``CML takes the unique approach of supporting \emph{higher-order concurrent programming},'' (emphasis Reppy's) whereby communication and synchronisation are made into first-class members of the language, in a similar way to how functional programming languages made functions into first-class members of themselves \cite[Preface]{Reppy2007}.


% \chapter{Choice of model, programming language and library}

\section{Selection of theoretical model}
\gls{csp} vs actors vs join calculus vs pi calculus etc.  Why I think one of them is the right choice

Message passing as discussed here is different to the message passing of the Message Passing Interface (how?)

\begin{itemize}
    \item Synchronous vs asynchronous message passing?
    \item Actors
    \item Join-calculus
    \item Pi calculus
    \item Communicating Sequential Processes
    \item Shared memory only
    \item Reactive approach?
\end{itemize}

\subsection{Why not Actors?}
Of the many theoretical models for concurrency, perhaps the two most rooted in concepts of individual \glspl{prox} passing messages between themselves are \gls{csp} and Actors.  On paper, the key difference between them is that \gls{csp} involves \glspl{prox} \emph{synchronously} exchanging messages through channels, which need not be associated with a particular \gls{prox}, whereas Actors \emph{asynchronously} exchange messages which are placed directly in their own mailbox.

% In general, asynchronicity is considered preferable, because it permits the \glspl{prox} to process at their own pace, without having to wait for another 

In practice, \gls{cml} has a major advantage in terms of resource requirements.  Actor implementations usually can support perhaps \num{100 000} or so individual actors per \si{\gibi\byte} of memory, but \gls{cml} implementations can stretch into the millions \cite{Butcher2014}.  This arises because there is no need to provide a mailbox to and store an unbounded number of messages for each \gls{prox}.  Channels and events do take a small amount of memory, but this has been found to be on the order of tens of bytes \cite{Reppy1991}.  Furthermore, it has been shown that, to a certain extent, the two can be done away with when operating in a parallel fashion \cite{Reppy2007a}.

Move the above into the introduction perhaps?

\subsection{Beyond Go}
Go is the obvious shared-memory message-passing language at the time of writing, but it doesn't have the full complement of \gls{cml} primitives (instead, it rougly corresponds to \gls{csp}).  For this reason, this work needs to consider far beyond simply Go (or indeed, many other equivalent options).

\section{Requirements}
% Requirements:
% \begin{itemize}
%     \item Must support \gls{csp}/Pi Calculus/etc. approach.  In particular, needs to enable multi-threaded, parallel message passing.  Should have some concept of channels or rendezvous, but needs to support more than just that.  I.e. needs to support the actual \gls{csp} approach/\gls{cml} features.  And needs to be highly scalable.
%     \item Needs to support tail call optimisation/tail recursion.  Or at least has a strong and clear path to trampolining in order to simulate TCO.
%     \item The message passing must be \emph{extremely} lightweight.
%     \item \emph{Must} have good linear algebra capabilities somehow
%     \item Comes with some ability to work with common image formats -- this is more a convenience than strictly necessary, since under normal circumstances these images will be coming through just as numeric arrays or similar, but it'll make testing things out a lot easier.
%     \item On the above note, not only should it be able to do IO, but it would be good if the library has things like histogram equalisation built in.  Not actually necessary though, since I could always write a quick command line tool to do that for me.
%     \item Preferably, has some degree of FFI/interop with C
%     \item Ideally includes some capacity to support GPU programming with CUDA, OpenCL, OpenGL etc. or at least has good C/C++ interop so that one can call libraries written in those languages (OpenCL 1.2 support is strongly to be preferred, as that's the best that the RPi can support, with the VC4CL library -- this may not be true later on, with the release of the RPi 4 Model B, but certainly VC4CL won't be there yet)
%     \item On the same line as above, the language really needs to make some form of vectorization available (better if it can do some automatically, but I probably will want the ability to control it manually also)
%     \item Probably doesn't necessarily need terribly strong concepts of typing to be honest, since almost all values that will be used will be numeric -- much more important would be memory safety
%     \item Still somewhat under active development/being supported
%     \item \emph{Preferably} not just someone's toy research language
%     \item If I'm (roughly) going to be treating every pixel as a separate computing unit, then the individual units must be extremely lightweight also.  Instantiation time isn't too big a deal, as that can be considered to be amortised, but it needs to be possible to manage a boatload of them.  Probably an argument against actors.
%     \item Must be able to be used across Linux implementations, as the embedded systems pretty much all use some form of Linux.
%     \item Needs to be able to produce native, \emph{dependency-free and standalone} executables \emph{for both x86-64 and ARM} - anything using a big VM will probably be too heavyweight (not definite though)
%     \item It would be better if there are already high quality implementations of stereo matching algorithms available for the given language, though that is not necessarily the be-all and end-all.
%     \item Also needs to have some degree of viable support for optimisation techniques such as gradient descent/Levenberg-Marquardt/RANSAC
% \end{itemize}

\subsection{Strict Requirements}
If a possible option does not meet even one of the below, it will not be considered further.

\begin{itemize}
    \item Suitable support for \gls{csp}/\gls{cml} style concurrency.  At a minimum, not only must there be capacity to send and receive over channels, but furthermore selecting over channels is \emph{mandatory}.  Simply having channels and the ability to select over them is the bare minimum required, but it does not make something a full \gls{cml} implementation.  The other combinators (e.g. \texttt{wrap} \& \texttt{guard}), and the ability to compose them, must be provided for it to be a full \gls{cml} implementation.  The message passing mechanics themselves must be lightweight, in terms of memory requirements and the number of instructions that need to be executed.
    \item Support of last-call optimisation, or an equivalent capability (this somewhat is implied by the above).
    \item Currently supported, or at least receiving regular maintenance
    \item Almost goes without saying, but includes some sort of green threads/fibers system, so that all the concurrent processes can be scheduled onto the processors efficiently.  This system must operate across multiple cores when they are available.
    \item Support for data-parallel programming, such as CPU vector instructions and/or OpenCL, in some capacity (either explicit or compiler-done).  The ability to use other frameworks and standards such as OpenMP, OpenACC or CUDA would be a bonus.
    \item Available for `standard' Linux distributions, such as those derived from Debian (e.g. Ubuntu, Mint) or Red Hat Enterprise Linux (e.g. Fedora, CentOS).  \emph{Most} languages target/support Linux in general, and thus this shouldn't be an issue typically.
\end{itemize}

\subsection{Desirable Qualities}
The absence of one or more of the below requirements will not necessarily disqualify an option, but having more of them will be a benefit.

\begin{itemize}
    \item Support for working with common image formats.  If need be, it should be relatively trivial to open an image elsewhere and convert it to something which I could parse relatively simply, like a .ppm/.pgm file, or some sort of binary file that is just all the pixel values jammed together.
    \item Support for common standard image processing routines.  Again, not necessarily needed, but it would save me the effort of implementing them if they turn out to be required.
    \item Support for compilation to native executable binaries, rather than relying on a virtual machine.
    \item High-quality implementations of stereo matching algorithms already available.  This is preferred simply so that it provides an easy comparison between what is implemented for this work against what someone else has created (probably using more of a traditional imperative style).  Using the same language for the comparisons helps to remove uncontrolled variables that could be confounding factors.
    \item Easy inter-operation with C, or equivalent (e.g. a Foreign Function Interface to C).
    \item An integrated or otherwise easy-to-use benchmarking system.
\end{itemize}

\section{Possible options}
Languages and their relevant libraries

% See also \url{https://github.com/kevin-chalmers/cpa-lang-shootout} and \url{www.teigfam.net/oyvind/home/technology/135-towards-a-taxonomy-of-csp-based-systems/} and \url{https://arrayfire.com/}\footnote{On top of ArrayFire, there's also Thrust \url{https://thrust.github.io/}  and ConcurrencyKit \url{http://concurrencykit.org/} and LibCDS \url{https://github.com/khizmax/libcds}}.  ArrayFire is the only one that appears to have bindings for any other languages besides C/C++, however.  Also, Furthark \url{https://futhark-lang.org/}

% \begin{itemize}
% \item \gls{cml}/\gls{csp}/Pi calculus
% \begin{itemize}
    % \item Standard ML -- it seems that MLTon is still under development, and it provides a recent implementation of Standard ML and \gls{cml}.\footnote{see \url{http://mlton.org/ConcurrentML}}
    % \item \sout{Concurrent C?  Appears to be loooong dead.}
    % \item Concurrent Haskell -- It's not entirely clear what the status of \gls{cml}/\gls{csp} style stuff is in Haskell.  Most up-to-date part seems to be Communicating Haskell Processes,\footnote{\url{https://hackage.haskell.org/package/chp}} though that appears to have been unmaintained for a few years.  See also Transactional Events.\footnote{\url{https://www.cs.rit.edu/~mtf/research/tx-events/}}
    % \item \sout{Occam?  Looks like Occam-Pi is the only version that's even close to current, and it doesn't look like that's under active development}\footnote{\url{https://github.com/concurrency/kroc}}
    % \item \sout{F\# with Hopac (Hopac seems to be fairly well no longer under development) and CoreRT.\footnote{For useful CoreRT references, see \url{https://github.com/dotnet/corert}, \url{https://github.com/dotnet/corert/tree/master/samples/HelloWorld} and \url{https://github.com/FoggyFinder/FSharpCoreRtTest}}}  Could possibly use System.Threading.Channels myself, it's apparently super-fast.\footnote{See e.g. \url{https://ndportmann.com/system-threading-channels/}}  Could perhaps alternatively use the TPL Dataflow library (F\# utility wrappers here:  \url{https://github.com/TheAngryByrd/FSharp.TPLDataflow})
    % \item Fibers in Guile Scheme
    % \item OCaml with Events module
    % \item SML/NJ is still going apparently\footnote{\url{https://www.smlnj.org}} although I'm not sure if \gls{cml} is up to date enough to work with it.  Bigger problem is that I \emph{think} it doesn't support parallelism at all.  There's also PolyML (\url{https://www.polyml.org/}) but it doesn't appear to include \gls{cml} at all.
    % \item Clojure with core.async using an AOT compiler or lightweight JVM.  GraalVM\footnote{\url{https://www.graalvm.org/}} appears to offer the former, while OpenJ9\footnote{\url{https://www.eclipse.org/openj9/index.html}} is apparently the latter, but as of writing it appears that neither of them supports targeting ARM (looks like Graal does kinda now).  See \url{https://neanderthal.uncomplicate.org/} and \url{} apparently for fast linear algebra operations.  There's also \url{http://docs.paralleluniverse.co/quasar/}, which apparently basically does Go fibers in Java - but it hasn't been updated in a while, and it looks like the company behind it might have gone out of business.
    % \item Rust\footnote{see \url{https://doc.rust-lang.org/book/ch16-02-message-passing.html}} with Crossbeam and its subcrates -- see also Tokio and Actix, as well as Desync\footnote{\url{https://github.com/logicalshift/desync/}}, Grease\footnote{\url{https://github.com/cambridgeconsultants/grease}} and Threadpool\footnote{\url{https://github.com/rust-threadpool/rust-threadpool} and see also \url{https://gsquire.github.io/static/post/a-rusty-go-at-channels/}} plus \url{https://gluon-lang.org/} (Gluon is a Haskell-y version of Rust, essentially).  Also \url{https://crates.io/crates/single_value_channel/1.2.1}.
    % \item Manticore is kinda sorta a successor to SML/NJ apparently, with a strong focus on \gls{cml}-style parallelism\footnote{\url{http://manticore.cs.uchicago.edu/}}
    % \item Go?  \sout{Apparently Crystal has a very similar \gls{csp}-derived built-in basis for Concurrency.}\footnote{While Crystal has similar built-in support for channels like Go, it apparently doesn't yet do parallelism - see near the start of \url{https://crystal-lang.org/reference/guides/concurrency.html}}  \url{https://www.gonum.org/}  Go's support for SIMD seems to be kinda clunky:  \url{https://medium.com/@c_bata_/optimizing-go-by-avx2-using-auto-vectorization-in-llvm-118f7b366969}, \url{https://www.reddit.com/r/golang/comments/ep4vyp/question_how_do_you_implement_simd_in_go/}, \url{https://goroutines.com/asm}, \url{https://golang.org/doc/asm}, \url{https://github.com/minio/simdjson-go}, \url{https://github.com/golang/crypto/tree/master/blake2b}, \url{https://www.reddit.com/r/golang/comments/42zppi/is_there_any_way_to_use_simd_intrinsics_in_gccgo/}.  Could an events library over the top of what's in Go be viable?
    % \item Is there anything in C++?  (Couldn't find anything when I looked the other day)\footnote{but see \url{https://docs.microsoft.com/en-us/cpp/parallel/concrt/concurrency-runtime?view=vs-2017} which is Windows only I think, also \url{https://www.cs.kent.ac.uk/projects/ofa/c++csp/} and \url{https://stackoverflow.com/questions/218786/concurrent-programming-c}}.  Also, there are a number of Boost libraries which come close, but none of which do precisely what I need:  \url{https://www.boost.org/doc/libs/?view=category_concurrent}.  MicroC++ (\url{https://plg.uwaterloo.ca/~usystem/uC++.html}) is similar.  Also, Stackless Coroutines introduced channels, but it looks like that is a dead experiment:  \url{https://github.com/jbandela/stackless_coroutine/tree/channel_dev} \& \url{https://github.com/CppCon/CppCon2016/blob/master/Presentations/Channels%20-%20An%20Alternative%20to%20Callbacks%20and%20Futures/Channels%20-%20An%20Alternative%20to%20Callbacks%20and%20Futures%20-%20John%20Bandela%20-%20CppCon%202016.pdf}  Actually, Boost:Fiber \emph{might} provide what I need.  See its channels, and \texttt{when_any} construct.
    % \item \sout{Smalltalk -> Objective-C \& Parallax (all of these have been out of active development for too long, but they probably will make good references)}\footnote{see \UrlBreaks{https://stackoverflow.com/questions/6145421/what-other-programming-languages-have-a-smalltalk-like-message-passing-syntax}}
    % \item Pharo(?)\footnote{\url{http://www.pharo.org/web}} -- a modern version of Smalltalk, apparently.  Also, Squeak.\footnote{\url{https://squeak.org}}
    % \item Ada (something about rendezvous).  Supposedly has a proper real-time focus, might be worth looking at.
    % \item Kotlin.  Has channels, and apparently an (experimental) implementation of selection:  \url{https://kotlinlang.org/docs/reference/coroutines/channels.html} ... ``Only single-threaded code (JS-style) on Kotlin/Native is currently supported.'' (\url{https://github.com/Kotlin/kotlinx.coroutines})
% \end{itemize}
% \item Join Calculus
% \begin{itemize}
%     \item F\# with Joinads
%     \item JOCaml
%     \item Scala with Chymist
%     \item \sout{Polyphonic C\#/C\(\omega\)}
% \end{itemize}
% \item Actors
% \begin{itemize}
%     \item Erlang / Elixir
%     \item Halide with message-passing?  There's \url{https://doi.org/10.1016/j.sysarc.2017.10.005}, \url{https://doi.org/10.1007/s11265-017-1283-1}, and Aaron Epstein's Masters Thesis at MIT, ``A Distributed Backend for Halide'', but with regards to message passing they all seem to focus on MPI.%  I couldn't find anything on \gls{cml} style in Halide.
%     \item \sout{Don't forget Pony...  (almost certainly not stable enough at this point)}
%     \item Rust with Actix
% \end{itemize}
% \item Julia?\footnote{\url{https://docs.julialang.org/en/v1/manual/parallel-computing/index.html} \& \url{https://docs.julialang.org/en/v1/manual/control-flow/\#man-tasks-1}} -- it is supposed to be very close to mathematical notation, so that's a big plus.  Can't find anything suggesting that there's a \gls{cml}/\gls{csp}/Pi calculus type thingy out there yet, though they do have lightweight threads and channels.  Plus, parallelism is apparently not really a big thing in it yet.
% \item What of the LMAX Disruptor approach?\footnote{\url{https://github.com/LMAX-Exchange/disruptor}} -- looks like that would have me working on the JVM, so probably Java/Scala/Kotlin/Ceylon/Clojure/probably something else is out there too.  There is a .NET port of it too.\footnote{\url{https://github.com/disruptor-net/Disruptor-net}}  Definitely looks like it would be worth investigating, but it doesn't follow the \gls{csp} model, so it is out-of-scope for this particular work.  Also \url{https://github.com/lthibault/turbine}.
% \item Scala with Chymist \sout{(there was also JCSP (\url{https://www.cs.kent.ac.uk/projects/ofa/jcsp/}), which seems to be long dead now, but see Communicating Scala Objects, though I can't find an actual implementation of that available anywhere)}
% \item \sout{SCOOP (Betrand Meyer) \& Eiffel (too OO, not really close enough to \gls{csp} it looks like)}
% \item \sout{Does ATS \cite{Shi2013} include anything? (couldn't find anything)}  Or D?\footnote{relevant: \url{http://www.informit.com/articles/article.aspx?p=1609144} and \url{https://wiki.dlang.org/Go_to_D}}  Or Nim?  \sout{Idris?  C++?}\footnote{There's C++CSP, but despite the most recent paper apparently being published in 2016, I couldn't find anywhere that actually hosted a usable version.  Could only find \url{https://www.cs.kent.ac.uk/projects/ofa/c++csp/doc/index.html} \& \url{https://github.com/olahol/cpp-csp}, neither of which are complete or recently-updated.}  Looks like I should check C++ Concurrency in Action, 2nd Edition -- it'll probably cover this if anything does (update: it doesn't).
% \item Further C++ \gls{csp} libraries:  SObjectizer\footnote{\url{https://stiffstream.com/en/products/sobjectizer.html}}, libmill\footnote{\url{http://libmill.org/index.html}} \& libdill\footnote{\url{http://libdill.org/index.html}}, LibProxC++\footnote{\url{https://github.com/edvardsp/libproxcplusplus} (also LibProxC \url{https://github.com/edvardsp/libproxc})}.  High Performance ParallelX\footnote{\url{http://stellar-group.org/libraries/hpx/}} claims to be a replacement to/improvement over \gls{csp} (see \url{https://stellar-group.github.io/hpx/docs/sphinx/tags/1.4.0/html/why_hpx.html#what-is-hpx}).
% \item \sout{How can strong typing be used beneficially? (if at all?)}
% \item \sout{Probably don't need transactional memory (?)}
% \item \sout{Lock-free is better than locking, but how to achieve?  Is that a relevant consideration here?  In theory at least, should be able to leave that up to the implementation I'm using.}
% \item \sout{Can F*, Adga, Coq or similar be of any use here? (not clear how)}
% \item \sout{Reppy has recently been working on Diderot, ``a Parallel Domain-specific Language for image analysis and visualization'', but it seems like that probably isn't what is needed here.}  Diderot doesn't seem to be ready for others to play with yet.
% \item Ferret\footnote{\url{https://ferret-lang.org}} appears to be a lot of what is needed, but it doesn't have its own channel/message-passing implementation, and I don't think it interoperates with Clojure.  Not entirely clear how one uses multi-threading with it, except possibly just calling out to C++.
% \item Single-Assignment C.\footnote{\url{http://www.sac-home.org/} \& \url{https://github.com/SacBase}}  It looks like development on it has more-or-less stopped in the past couple of years, but it \emph{might} still be suitable for my purposes.  It appears to come with some built-in support for image processing (e.g. parts of the standard lib directed towards using .pgm files), but I'm struggling to see anything on concurrency/parallelism -- it might be the case that all of that is done implicitly, e.g. there's \url{https://github.com/SacBase/NASParallelBenchmarks}, but I couldn't see any explicit handling of parallel constructs in it.
% \item Checked Dart, but it seems to be focused on making apps with responsive UIs.  It has a concept called `isolates' which seem to be moderately similar to fibers, but they seemingly are asynchronous only (Wikipedia explicitly compares it to Erlang).
% \item Checked Io, which is another Smalltalk-esque language, but it apparently only does asynchronous.  Not clear that it does parallelism across multiple cores, either.
% \item Checked X10, which doesn't seem to provide any sort of \gls{csp} style support, and appears to be focused primarily at HPC.
% \item Even the latest version of OpenMP doesn't seem to have anything \gls{csp}-ish.
% \item Looks like there \emph{might} be some chance to use C# for some of this stuff:  \url{https://github.com/DragonSpit/HPCsharp}, \url{https://github.com/chrisa23/fibrous}, possibly \url{https://github.com/domn1995/Marathon}.  \url{https://linksplatform.github.io/Hardware.Cpu/}, \url{https://github.com/jackmott/LinqFaster}, \url{https://gitlab.com/pomma89/object-pool}.
% \item Other ones that could be mentioned include Mozart \url{https://github.com/mozart/mozart2}, Red \url{https://www.red-lang.org/}, P \url{https://github.com/p-org/P}, OForth \url{https://www.oforth.com/}, Esterel \url{http://www.esterel.org/}.
% \end{itemize}

% Other Rust resources:  Bastion \url{https://github.com/bastion-rs/bastion} (see also it's sub-libraries Bastion Executors and LightProc); Actix \url{https://docs.rs/actix/0.9.0/actix/} (the underlying Actix system, \emph{not} Actix-Web); Greenie \url{https://github.com/playXE/greenie}, but it appears to have only just started; Rust-Executors \url{https://github.com/Bathtor/rust-executors}, though I can't work out precisely what it does.  It basically seems to say that it lets you choose between some threading runtimes; Par-Array-Init \url{https://crates.io/crates/par-array-init/0.0.5} looks like it was set up and then abandoned, but \emph{might} still be useful for my purposes; RustaCUDA \url{https://github.com/bheisler/RustaCUDA} \& Accel \url{https://gitlab.com/termoshtt/accel};  Also, Testbench \url{https://github.com/HadrienG2/testbench} and Criterion \url{https://github.com/bheisler/criterion.rs}.  Emu for OpenCL \url{https://calebwin.github.io/emu/} and Ocl \url{https://github.com/cogciprocate/ocl/tree/master}, but the latter is looking fairly abandoned.

% Other Nim resources: Memo \url{https://github.com/andreaferretti/memo}; Loop-fusion \url{https://github.com/numforge/loop-fusion}; Nim-Schedules \url{https://github.com/soasme/nim-schedules}; Shared \url{https://github.com/genotrance/shared}; Nim-Chronos \url{https://github.com/status-im/nim-chronos} (not clear that this is relevant);  Stew \url{https://github.com/status-im/nim-stew}; Stones \url{https://github.com/binhonglee/stones} (not really clear precisely what it provides); Nim-CLBlast \url{https://github.com/numforge/nim-clblast}; Nim-Optionsutils \url{https://github.com/PMunch/nim-optionsutils}; Nimterop \url{https://github.com/nimterop/nimterop}; Neo \url{https://github.com/unicredit/neo}; Nim-GLM \url{https://github.com/stavenko/nim-glm}; Memviews \url{https://github.com/ReneSac/memviews}; Nim-Curry \url{https://github.com/zer0-star/nim-curry};

%For OCaml, be sure to check \url{https://ocaml.xyz/}  (also take a look at \url{https://www.eff-lang.org/} and \url{https://github.com/kayceesrk/effects-examples})

\subsection{Ranking of of possible options}
% The ones chosen to assess \emph{must} meet all the criteria below.  If they fail even one, then they are excluded from further consideration.

% \begin{itemize}
%     \item Support for \gls{csp}/\gls{cml} style concurrency
%     \item Tail/Last call optimisation, or something equivalent
%     \item \emph{Extremely lightweight} message passing and representations of processing elements
%     \item Linear Algebra support
%     \item Support for vectorization and/or OpenCL
%     \item Maintained or under development
%     \item Usable on most Linux distributions
%     \item Can produce executable files for both x86\_64 and ARM
% \end{itemize}

The ones (that probably will be) finally chosen to assess are among:
\begin{itemize}
\item C++ with Boost:Fiber or one of the \gls{csp} libraries
% \item Clojure with core.async -- core.async seems to be a channels + selection library only
% \item Go -- base Go is channels + selection only, and there don't seem to be any full \gls{cml} libraries out there for it.
\item Guile Scheme with Fibers library -- Fibers is a full \gls{cml} library
% \item Kotlin -- Not 100\% clear.  It looks like Kotlin's coroutines go beyond just channels + selection, but if it goes to full \gls{cml}, the combinators go by different names.
% \item Manticore -- has a full \gls{cml} library
\item MLTon -- has a full \gls{cml} library
% \item Nim -- Unfortunately, Nim does not appear to have any selection over channels capability
\item OCaml with Events module -- Events is a full \gls{cml} library, but OCaml is (currently) single-core only
% \item Pharo
\item Rust with Crossbeam crate -- Crossbeam seems to be channels + selection only
\item Racket with its Sync library
\item Felix-lang (?) (\url{https://github.com/felix-lang/felix})
\end{itemize}

Big table assessing what features they in fact have goes here (?).  That can be used as the basis of choosing, say, the top 3-5 languages that seem like they would be best suited to what I want to do, and those ones can be further assessed.



Further distinguishing criteria are examined below.  These are not as critical, so if an option lacks one but has others strongly, it may still be the best choice.

\subsubsection{Support for working with common image formats}

\paragraph{C++}

\paragraph{Clojure}
Almost certainly, though I couldn't find a current library in a quick search.  JavaFX (\url{https://openjfx.io/}) provides something it seems.  There seem to be some basic classes in Java.AWT and Javax.ImageIO.  Seems like it is also possible to interact with OpenCV via Java bindings (best as I can tell).

\paragraph{Go}
Yes - built in.

\paragraph{Guile}
Yes -- at the very least Guile-CV \url{https://www.gnu.org/software/guile-cv/}

\paragraph{Racket}

\paragraph{Kotlin}
JavaFX (\url{https://openjfx.io/}) provides something it seems.  There seem to be some basic classes in Java.AWT and Javax.ImageIO.  Seems like it is also possible to interact with OpenCV via Java bindings (best as I can tell).  Also see JavaCV \url{https://github.com/bytedeco/javacv}, BoofCV \url{https://github.com/lessthanoptimal/BoofCV} and AlgART \url{https://algart.net/java/AlgART/}

\paragraph{Manticore}
Doubtful.

\paragraph{MLTon}
Unclear.  There are many old libraries around, but none of the ones found dealt with images.  Matthew Fluet has stated (via email) that he is unaware of any libraries, and has suggested using the C FFI to work with one of the C libraries that performs the task.

\paragraph{Nim}
Yes.  At least, partial (see e.g. \url{https://github.com/nim-lang/needed-libraries/issues/77}).  Looks like NiGui \url{https://github.com/trustable-code/NiGui/blob/master/examples/example_11_image_processing_cli.nim} includes some capability for interacting with images...  Also, Flippy \url{https://github.com/treeform/flippy}; 

\paragraph{OCaml}
Yes, e.g. Bimage \url{https://github.com/zshipko/ocaml-bimage} or CamlImages \url{https://bitbucket.org/camlspotter/camlimages/src/default/}.

\paragraph{Rust}
Yes.

\subsubsection{Support for common standard image processing routines}

\paragraph{C++}

\paragraph{Clojure}
Almost certainly, though I couldn't find a current library in a quick search.

\paragraph{Go}
Most likely - haven't found it, but given Go's popularity it seems highly likely.

\paragraph{Guile}
Yes.

\paragraph{Kotlin}
At least as much as Clojure it looks like.

\paragraph{Manticore}
Couldn't find any.

\paragraph{MLTon}
Couldn't find any.

\paragraph{Nim}
Yes: \url{https://github.com/numforge/laser}

\paragraph{OCaml}
Yes (some, at least).


\paragraph{Rust}
Yes.


\subsubsection{Inter-operation with C/C++}

\paragraph{C++}
N/A

\paragraph{Clojure}
It is likely possible, but not sure that it is \emph{easy}.

\paragraph{Go}
Yes, with something called `cgo' \url{https://blog.golang.org/c-go-cgo}

\paragraph{Guile}
Yes, definitely, although perhaps not in quite the same way as other languages -- one of the goals of Guile is to permit the incorporation of Guile elements into a C program.

\paragraph{Kotlin}
Yes. \url{https://kotlinlang.org/docs/reference/native/c_interop.html}

\paragraph{Manticore}
Not entirely clear, but almost certainly can be done.

\paragraph{MLTon}
Yes.

\paragraph{Nim}
Yes.  Nim actually transpiles to C, which is then compiled as normal.

\paragraph{OCaml}
Yes.

\paragraph{Rust}
Yes.


\subsubsection{Ahead-of-time compilation to native executables}

\paragraph{C++}
Yes

\paragraph{Clojure}
Kinda with OpenJ9.  Yes with GraalVM.

\paragraph{Go}
Yes.

\paragraph{Guile}
Yes, by `embedding' the Guile program into a C program.

\paragraph{Kotlin}
Yes - though it appears that only 32-bit ARM is supported for Linux right now: \url{https://kotlinlang.org/docs/reference/native-overview.html}.

\paragraph{Manticore}
Yes - though it's not clear what backend is used by default.

\paragraph{MLTon}
Has it - through GCC and LLVM

\paragraph{Nim}
Yes (in fact, they brag they're particularly good at it)

\paragraph{OCaml}
Yes.

\paragraph{Rust}
Yes.

\subsubsection{High-quality implementations of stereo matching algorithms already available}

\paragraph{C++}
Er... OpenCV?

\paragraph{Clojure}
Didn't find anything.  It seems unlikely.  I did see somewhere that there is apparently a Clojure wrapper of OpenCV, though.

\paragraph{Go}
Not sure, but it's likely.

\paragraph{Guile}
Couldn't find any.  There's presumably some in C, however.

\paragraph{Kotlin}
OpenCV if nothing else (and I didn't find anything else).

\paragraph{Manticore}
Couldn't find any.

\paragraph{MLTon}
Couldn't find any.

\paragraph{Nim}
There are wrappers over OpenCV, at least.

\paragraph{OCaml}
Couldn't find any.

\paragraph{Rust}
Didn't find any, but could well be out there.

\subsubsection{Works on x86* \emph{and} ARM}

\paragraph{C++}
Yes - GCC definitely supports ARM targets, I'm pretty sure LLVM does too.

\paragraph{Clojure}
Not 100\% clear.  It basically comes down to whether a JVM implementation supports targeting more architectures than x84-64.  It \emph{looks like} OpenJ9 doesn't support ARM, and neither does GraalVM.

\paragraph{Go}
Yes.

\paragraph{Guile}
Looks like it does x86* and ARMv7 specifically.

\paragraph{Kotlin}
It has a native compilation feature, which supports x86-64, and one of ARM32 or ARM64 I think (I have seen references to one or the other, but not both at once).

\paragraph{Manticore}
No.  x86-64 only.

\paragraph{MLTon}
Potentially, yes, if you fiddle with the GCC settings it uses.

\paragraph{Nim}
Yes (I believe it should work for whatever architectures GCC and Clang/LLVM support).

\paragraph{OCaml}
Yes.

\paragraph{Rust}
Yes.

\subsubsection{More than one individual's toy/research language, and still under active development/support}

\paragraph{C++}
Yes.

\paragraph{Clojure}
Yes.

\paragraph{Go}
Yes.

\paragraph{Guile}
Doesn't have a huge team or wealthy company behind it, but it is a central part of GNU these days, and has been in development for a long time.

\paragraph{Kotlin}
Yes.

\paragraph{Manticore}
Not really, it is a research language with a small core group, most of whom are inactive.

\paragraph{MLTon}
Yes.  It appears that development is still in progress to some degree.

\paragraph{Nim}
Yes.

\paragraph{OCaml}
Yes.

\paragraph{Rust}
Yes.

% \subsubsection{Support for optimisation algorithms}

% \subsection{Excluded options}
% The following options were investigated but excluded from consideration due to a variety of issues.  These include that they emulate a related model such as Join Calculus or Actors, but not \gls{csp}.  

% \subsubsection{Actors}
% \begin{itemize}
%     \item Erlang / Elixir
%     \item Halide with message-passing?  There's \url{https://doi.org/10.1016/j.sysarc.2017.10.005}, \url{https://doi.org/10.1007/s11265-017-1283-1}, and Aaron Epstein's Masters Thesis at MIT, ``A Distributed Backend for Halide'', but with regards to message passing they all seem to focus on MPI.%  I couldn't find anything on \gls{cml} style in Halide.
%     \item \sout{Don't forget Pony...  (almost certainly not stable enough at this point)}
%     \item Rust with Actix
%     \item \gls{mpi}
% \end{itemize}

% \subsubsection{Join Calculus}
% \begin{itemize}
%     \item \fsharp{} with Joinads
%     \item JOCaml
%     \item Scala with Chymist\footnote{\url{https://github.com/Chymyst}}
%     \item \sout{Polyphonic C\#/C\(\omega\)}
% \end{itemize}


\section{Assessment of chosen candidates}
Sample applications, and results of profiling them.  Comparison with other pre-existing implementations.

Probably want to do one or more stress tests on each system, to see how they cope, as well as see how easy it is to program with them reasonably effectively.  E.g. implementations of the median filter ala my IVCNZ 2018 paper.

\subsection{Criteria}
How to assess?
\begin{itemize}
    \item Mean runtime (minimum is suggested to be better as it more accurately reflects ONLY the process in question, but mean is probably going to happen more often in practice - particularly relevant with garbage collections)
    \item Peak \& average memory use
    \item `Code quality' measures
\end{itemize}

\subsection{Tests}

\subsection{Test results}
What applications?...  Preferably something similar to what I actually expect to be doing.  Running these on the sample input images, perhaps set up to be a stream of images (even if it is the same one over and over again - though could that advantage some algorithms or specific implementations?), while measuring relevant metrics.  %If possible, try to deploy one of them into the field in some fashion.

% \section{Heterogeneous computing}
% I.e. effective combined use of the CPU and GPU?

\section{Final language choice}
And the winner is...


\section{Appendix}
List of languages that were investigated but fell at the first hurdle, namely not meeting one of the essential requirements above.  This is not a complete list.

\begin{itemize}
    \item \texttt{Standard ML of New Jersey} with \texttt{\gls{cml}}:  The original \gls{cml} implementation was in fact intended only for concurrent programming, and not parallel.  Thus, it does not support modern multiprocessors well.  Instead, MLTon or Manticore can perhaps be used.  (MLton is single-threaded-only, it turns out)
    \item \texttt{Julia}:  Does not have support for parallelism in its channels operations.  ``The current version of Julia multiplexes all tasks onto a single OS thread.'' (from \url{https://docs.julialang.org/en/v1/manual/parallel-computing/#Coroutines-1}, accessed on 2 February 2020).  --- It \emph{looks like} the situation has changed, and Julia now supports running tasks across multiple cores.  Need to investigate further.  Seems pretty likely that it doesn't have Events, but \emph{maybe} they could be implemented with relatively little difficulty.
    \item \texttt{Concurrent C}:  Long dead.
    \item \texttt{Crystal}:  At present, Crystal does have fibers, but not support for parallelism.  ``At the moment of this writing, Crystal has concurrency support but not parallelism: several tasks can be executed, and a bit of time will be spent on each of these, but two code paths are never executed at the same exact time.'' (from \url{https://crystal-lang.org/reference/guides/concurrency.html}, accessed on 2 February 2020)
    \item \texttt{\fsharp{}} with \texttt{Hopac}:  More-or-less abandoned at this point, it seems.  Apparently (according to gossip) the original creator doesn't even work in \fsharp{} anymore, and the `current maintainer' doesn't seem to have much interest.  Also looked at .NET with \texttt{System.Threading.Channels}, but it doesn't seem to have any ability to select over channels.
    \item \texttt{Concurrent Haskell}:  There have been a number of \gls{csp}-inspired libraries implemented in Haskell.  The most recent known one, however is Communicating Haskell Processes, which has not been updated since 2014 (see \url{https://hackage.haskell.org/package/chp}).  Matthew Fluet suggested his Haskell `Transactional Events' (see \url{https://www.cs.rit.edu/~mtf/research/tx-events/} and \url{https://hackage.haskell.org/package/transactional-events})
    \item \texttt{Eiffel} with \texttt{SCOOP}:  SCOOP implements synchronous rendezvous, but does not seem to support actual message-passing.  It should be considered for future efforts, but is not close enough to \gls{cml} to be appropriate here.
    \item \texttt{Occam/Occam-\(\pi\)}:  The latest version of an Occam compiler, KRoC,\footnote{https://github.com/concurrency/kroc/} has not been updated since April 2017.
    \item \texttt{Haxe}:  It claims to be fast, and run cross-platform.  Unfortunately, no message-passing system could be found for it, in the standard library or user-created libraries.
    \item \texttt{Single Assignment C}:  No longer being maintained it seems, plus there doesn't seem to be any capacity for explicit concurrency mechanisms.
    \item \texttt{Python} with \texttt{PyCSP}:\footnote{\url{https://github.com/runefriborg/pycsp}}  Actually implements a lot of what I would need, but is unlikely to be all that performant.  Plus, it hasn't been updated in nearly four years.
    \item \texttt{Stackless Python}:\footnote{\url{http://www.stackless.com/}} This appears to have some of the basics, but no provision for selective choice.  Same story with \texttt{PyPy}\footnote{\url{https://www.pypy.org/}}.
    \item \texttt{Ada}:  In most ways it would be an excellent fit for this, but this work focuses specifically on channel-based implementations, and Ada doesn't come with channels by default (it has a similar-but-different mechanism).  One was implemented \cite{Atiya2005}, but it doesn't appear to have been incorporated into the language or made widely available.
    \item \texttt{D}:  The D language has nearly everything desired, but does not appear to have support for selection over channels.  No trace of it could be found in the standard library.  Mention of it is made in \url{https://wiki.dlang.org/Go_to_D}, but the library referred to there\footnote{\url{https://github.com/nin-jin/go.d}} seems to have been an unfinished prototype, and has not been updated in years.
    \item \texttt{Scala} with \texttt{Scala Communicating Objects}:\footnote{\url{https://www.cs.ox.ac.uk/people/bernard.sufrin/personal/CSO/}}  While it seems like it's probably a good library, it appears to have been one person's research experiment, which has no support and has not been updated for a while.
    \item \texttt{Pharo} or \texttt{Squeak}:  In most ways they would be suitable, but they don't \emph{quite} follow the \gls{cml} explicit channels approach.  They're more akin to Ada.
\end{itemize}
% \chapter{Belief Propagation}

Fundamentally most of the pixel-based stereo matching algorithms (at least the ones I'm familiar with) boil down to solving an equation like

\[ E(\vec{f}) = \sum_{p \in \Omega} \Bigg[ E_{data}(p, f_p) + \sum_{q \in A(p)} E_{smooth}(f_p, f_q)\Bigg] \] with the goal of finding
\[ \arg\min_{\vec{f}} E(\vec{f})\] where \(f_p \in L\) is a label from the set of possible disparities \(L = 0, 1, \ldots, D_{max}\), \(p\) is a given pixel in the disparity map, \(\Omega\) is the set of all pixels in the disparity map, \(A(p)\) is the set of all pixels that are considered `adjacent' to \(p\) for some meaning of adjacency, and \(E\) is the total error over the whole image and \(\vec{f}\) is the vector of all disparity labels assigned to pixels in the disparity map, i.e. \( \forall_{f_p}~f_p\) is in \(\vec{f}\), and \(|\vec{f}| = |\Omega|\).

(alternative notation can be seen in Felzenswalb and Zabih 2011)

They're all essentially some form of global optimisation (though in cases where there is no smoothness function used, they degenerate to the local optimisation case because the overall lowest cost assignment becomes the lowest cost assignment at each particular location \(p\)).  The biggest differences between each of them are the functions used for the data and smoothness costs, the meaning of adjacency and the scheduling of the computations (plus whether there are iterations, etc).

Come to think of it, why can't these parts all be separated out?

All the known adjacency forms are \emph{symmetric}, meaning that \(x A y \implies y A x\), where A is the binary adjacency relation.  Thus, in every single rendezvous of processing elements, \emph{both} of them will want to both send to and receive from the other, and thus two-way exchanges are to be preferred, rather than the usual one-way of base Concurrent~ML.

A \emph{possible} optimisation might be to apply some sort of pre-processing step on each input stereo image, where all contiguous pixels of the exact same intensity are grouped together into one super-pixel (which I believe is, in fact, different from regular super-pixel approaches -- those could be trialled also), and then those super-pixels used as the message-exchanging entities.  Potential downsides to this approach, beyond the obvious that the pre-computation will take some time, are that the regular grid layout of the images, which is kinda relied upon in the adjacency/neighbourhood relations as well as in determining which pixels are along a given epipolar line.  This approach would likely reduce the number of message-passing entities, but at the cost of a greater number of messages sent by each, as it seems plausible that each one of them could end up being considered to be adjacent to many more super-pixels than the typical 4.  There have apparently been some attempts to do similar, but simply using rectangular blocks, without a great deal of success (these are discussed a bit in sect. 2.2 of Rui Gong's 2011 Masters thesis).

Introduce \gls{bp} here

Note to self.  In all instances, the final produced programs should really include some way to update each pixel/task/communicating process with its new intensity value from the new image.  So, the ideal operation is that at startup the program initialises its processing elements, then somehow supplies to them their intensity value, and then once computations are performed the PEs then send their final disparity decisions back to somewhere to be compiled into the final product.  So, timing testing will need to be done over more than just one iteration of an image.  Preferably many more.

\section{Operation of \glsentrylong{bp}}
Description of \gls{bp} goes here.  (A programmer's intro to \gls{bp}?)

\section{cP~systems}

% Will probably need to give relevant background on cP systems here also.

Look at incorporating use of PROMELA too?

\section{\glsentrylong{cml} implementation}

Perhaps this section should be moved into a `methods' chapter.  Wherever I write it, I should really explain which particular style of \gls{bp} I chose to emulate, and why.

\section{Experiments}

Perhaps this section should be moved into a `results' chapter.
% \chapter{\glsentrylong{cp}}

\section{Operation of \glsentrylong{cp}}

\section{\glsentrytext{cps}}

\section{\glsentrylong{cml-glossary} implementation}

Perhaps this section should be moved into a `methods' chapter.

\section{Experiments}

Perhaps this section should be moved into a `results' chapter.
% \chapter{Discussion}

Where I discuss the good \& bad about this work

% \section{Comparison of \glsentrylong{mpbsm} algorithms}

% What am I comparing them to?  Algos coded by myself?  The best I can find out there of others'?

% \subsection{\glsentrylong{bp}}

% \subsection{\glsentrylong{cp}}

\section{Limitations}

\subsection{Threats to Validity}

\section{Alternatives for Investigation}

\subsection{Other languages \& libraries}
Implement my own version of \gls{cml} in Rust or Julia, etc?  Looks like Felix has most if not all of the necessary components for it.

\subsection{Other Computational Models}
Actors, Join Calculus, Pi Calculus

\subsection{Potentially Useful Hardware}

\subsubsection{CPU Additions}
Intel's Transactional Synchronisation Extensions (TSX) -- \url{https://en.wikipedia.org/wiki/Transactional_Synchronization_Extensions}

AMD's Advanced Synchronization Facility (ASF) -- \url{https://en.wikipedia.org/wiki/Advanced_Synchronization_Facility}

mEDA (modified Extended Dataflow Actor) \& full/empty memory tagging (F/E bits?) e.g. \url{https://link.springer.com/chapter/10.1007/BFb0057916} \& \url{https://people.kth.se/~vladv/abstracts/TRITA-IT-0004.pdf}

\subsubsection{Is Hyper-Threading Helpful?}

\subsubsection{GPUs?}

% \chapter{Conclusion}

I did some stuff!

\section{Future Directions}

\begin{anfxwarning}{Where should this go?}
    Similar material is also covered in the discussion chapter.  Should much of the below be shifted into there?  Alternatively, should much of the stuff from the discussion chapter move to here?
\end{anfxwarning}

% The use of Full/Empty bits looks highly promising for making message passing more efficient.  It doesn't seem to have any real support in mainstream/commodity hardware, however.  There were Intel's TGX instructions, but they have been taken out of processors again.

% Extended Dataflow Actors?  Or is it Extended Dataflow Architecture?

% \fxerror*{Not entirely sure BSP is relevant here}{See also the Bulk Synchronous Parallel approach.}

% https://scholar.google.co.nz/scholar?q=related:Z8GZl-HQcSkJ:scholar.google.com/&scioq=A+Static+Mapping+System+for+Logically+Shared+Memory+Parallel+Programs&hl=en&as_sdt=0,5&inst=15360723290749679499

% http://citeseerx.ist.psu.edu/viewdoc/download?doi=10.1.1.50.8739&rep=rep1&type=pdf

% https://link.springer.com/chapter/10.1007/BFb0057916

% https://link.springer.com/chapter/10.1007/3-540-63697-8_82

None of the below were investigated further, due to a lack of time, but they are obvious next steps to look at.

% \subsection{\glsentrytext{cml}}
% \Gls{cml} seems like the obvious approach to take from here.\footnote{In fact, considerable time and effort was spent on attempting to identify a suitable \gls{cml} implementation to use for just this purpose.  Unfortunately, in the end, it was found that there was not currently a usable \gls{cml} implementation with appropriate performance available at this time.}

% \subsection{`Faked' message passing in shared memory}
% Roughly, most of the message passing involved here is largely, in effect, just handing around pointers to memory locations.  It would seem that the message passing itself places some overhead in the way of that.  Could there be some way to fake the message passing, so that to the program's writer it looks like an actual message passing implementation, but in reality it is just doing normal updates on mutable memory?  This \emph{might} enable the best of both worlds -- programming the algorithms according to their theory, but running in a highly efficient fashion `under-the-hood'.

% It is not too clear how to achieve this (if it is indeed possible), but Rust would appear to be a good language to target for it.  Rust is fairly high-performance by default and enables quite a lot of low-level memory manipulation.  Moreover, its move semantics pretty much fit exactly to the concepts used here, and, on the face of it, its macro system would appear to be a convenient way to abstract over many of the details and provide a message passing façade, while actually doing efficient operations behind that.

% It looks like the C++ \texttt{mess} library is intended to be exactly this sort of thing:  \url{https://github.com/LouisCharlesC/mess}.  The developer seems to say that the user gets to write their program in a message-passing fashion, but mess does some clever meta-programming so that there ends up being zero overhead in the end.  Also, absolutely \emph{must} address Halide, and explain why it wasn't pursued here.  Otherwise, that'll be the elephant in the room.

% \subsection{Other hardware}
% This work focused on CPU-based systems, largely excluding other hardware.  If used well, all three of \glspl{gpu}, \glspl{fpga} and \glspl{dsp} have potential to perform vastly more computations per second, suggesting that a high-performance message passing-based system would do well to use them.  In each case, however, they do not work in quite the same way as CPUs, meaning that programming them is not necessarily straightforward, especially when trying to achieve a programming style that differs to their default.  As described in \fxwarning*{insert the appropriate cross-reference}{the literature review chapter}, fast \gls{gpu} implementations of \gls{bp} have been created, suggesting that message passing on a \gls{gpu} is entirely plausible.

% It would appear worthwhile to investigate how one might be able to implement some sort of message passing system atop these hardware alternatives, due to the potential for many more computations per second.  Even better would be to create a heterogeneous system which can take full advantage of the strengths of each hardware type, while overcoming its weaknesses.  Precisely how to achieve efficient implementations on them is unknown.  The fact that OpenCL \fxerror[inline]{[ref]} (and, to some extent at least, OpenACC \fxerror[inline]{[ref]}) can be compiled from the same base code to different devices makes it an obvious starting point.   There has already been at least one publication on implementing \glspl{actor} in OpenCL \fxerror[inline]{[ref]}, suggesting it is possible -- though how well synchronous message passing will work as compared to asynchronous remains to be seen.

% See also \url{https://hastlayer.com/} -- they seem to say that they do relevant stuff on FPGAs.  Interestingly, they also do unums/posits, apparently.

\subsection{Universal Numbers}
The \gls{cps} work kinda ignored non-integer numbers, which in real practice is quite a glaring omission.  Investigating modelling IEEE-754 floating point (and/or the fixed-point version), as well as Gufstafson's unums/posits, would be another good step to take.

%%%%%%%%%%%%%%%%%%%%%%%%%%%%%%%%%%%%%%%%%%%%%%%%%%%%%%%%%%%%%%%%%%%%%%%%%%%%%%%%%

\subsection{\nameref{chap:tsp}}
One-way multiset unification occurs frequently in \gls{cps}, with unification being used in every rule presented above.  \fxerror*{Isn't this what Yezhou's paper \cite{Liu2021} covers?}{An efficient algorithm to perform this task would be highly beneficial for creating useful simulations of systems (we are not aware of an efficient algorithm in the case of multisets).}  For example, our simulations of the \gls{tsp} algorithm written in functional programming languages (see \cref{sec:tsp:simulation}) regularly simply iterate over all relevant objects in the system, even though frequently most will be of little use in a given function call, and so the simulations could benefit from improved unification in practice.

We would like to further develop the capacity to simulate \gls{cps}, in particular developing more advanced techniques for translating \gls{cps} rules to efficient parallel code.  Work down this avenue has not begun as yet, however.

%%%%%%%%%%%%%%%%%%%%%%%%%%%%%%%%%%%%%%%%%%%%%%%%%%%%%%%%%%%%%%%%%%%%%%%%%%%%%%%%%

\subsection{\nameref{chap:median}}

The median filter algorithm investigated was pretty much the most basic one possible.  The experiment should be extended to cover more complicated algorithms that perform better at reconstructing the image.  These may be better suited to a message-passing approach compared to a na\"ive one than the basic method.

Future work should explore \gls{cml} as applied to other computer vision algorithms, especially those which can naturally be characterised in terms of message passing, or which use more complex data types.  Other \gls{cml} implementations beyond the one used here should be tested too, as it is not clear how much the problems experienced by this \gls{cml} program may stem from the implementation of the \gls{cml} library used.

%%%%%%%%%%%%%%%%%%%%%%%%%%%%%%%%%%%%%%%%%%%%%%%%%%%%%%%%%%%%%%%%%%%%%%%%%%%%%%%%%

\subsection{\nameref{chap:nmp}}
The next step in this work is to adapt the system to the purpose of Belief Propagation Stereo Matching, using \gls{nmp} precepts.  This will be presented in Part Two.  Furthermore, work has begun on implementing a close approximation of the asynchronous system as a framework in a standard programming languages, to explore the effectiveness of this approach in modern computer systems.  We plan to implement Belief Propagation Stereo Matching (see e.g. \cite{Blake2011,Felzenszwalb2011,JianSun2003}) atop this as a proof-of-concept.

One aspect the systems presented above lack is that both the size and shape of the grid involved, as well as the communication topology between neighbours, are permanently fixed at the time of system initialisation.  In most cases, this is unneeded, but the greater flexibility could be of use when implementing certain algorithms.

Furthermore, at present, it is implicitly assumed that every \gls{pe} remains active throughout the entirety of the system's evolution until it has sent and received all of its scheduled messages.  Within the context of \gls{cps} this is largely irrelevant, but permitting \glspl{pe} to deactivate at appropriate points could save processing power in other circumstances with bounded parallelism.  Complicating this is ensuring that those \glspl{pe} which do remain active can continue messaging as needed despite one or more neighbours deactivating.

We also have yet to examine the systems with respect to communication complexity measures such as those found in \cite{Juayong2020}.  The precise results presented there are not directly applicable to this work, given the use of different P~systems models, but the underlying concepts appear directly relevant.

%%%%%%%%%%%%%%%%%%%%%%%%%%%%%%%%%%%%%%%%%%%%%%%%%%%%%%%%%%%%%%%%%%%%%%%%%%%%%%%%%
%\input{tex/chapter1} % I hope that you have better titles than this
%\input{chapter2} 
%\input{chapter3} 

% ====================================================
%
% ENDMATTER
%
% Appendices and bibliography 
% Pagination arabic, re-starts at 1
%
% ====================================================

\printbibliography[title={Works Cited}, heading=bibintoc]
% \chapter*{Glossary}

\printglossary
 
\printglossary[type=\acronymtype]

\cleardoublepage % start afresh on a new page
\setcounter{page}{1} % re-sets the page counter
\appendixpage* % makes a page to mark beginning of appendices
% \input{appendix1} 


\end{document}