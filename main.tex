\documentclass[11pt,partial,draft,doublespace]{aucklandthesis}
%
% This is a template for University of Auckland theses.
%
% Written by Alistair Kwan, June 2016
% 
%
% Options:
%	10pt, 11pt, 12pt: size of main text
% 	examcopy: asserts confidentiality for examination copies
%	partial: thesis partial fulfils degree requirements
%	singlespace, onehalfspace, doublespace: line spacing
%	oneside: format for single-sided printing
%	draft: adds 'draft' and date to footer
%

%%%% Packages introduced by me, not included with the template
\usepackage[normalem]{ulem}
\usepackage{amsmath}
\usepackage[binary-units]{siunitx}
\usepackage[acronym,noredefwarn]{glossaries}
\setacronymstyle{long-short}
\makeglossaries

%
% Add, delete or un-comment packages below as required.
%

\usepackage[utf8]{inputenc}
\usepackage[T1]{fontenc}

\usepackage{graphicx} % for inserting graphics files
\usepackage{appendix} % for appendices

\usepackage{hyperref} % for formatting web addresses and other URLs
\urlstyle{same} % try also tt, sf if this option doesn't produce clear enough output

% Readability options
%
\usepackage{booktabs} % for table rules
\usepackage{microtype} % for improved justification

% Typeface options — choose one if desired
% or choose a different typeface to accommmodate character sets
% as needed for East Asian and other languages.
%
% Consider compiling using the XeLaTeX engine if you have more extreme
% typeface needs, e.g. for multiple languages, or a need for symbols particular
% to a typeface.
%
% See also the LaTeX Symbols List at
% https://http://www.ctan.org/pkg/comprehensive
%
% \usepackage{mathptmx} % Times New Roman, including mathematics
% \usepackage{mathpazo} % Palatino with mathematics support
% \usepackage{fourier} % Utopia, a serif typeface with Fourier mathematics
% \usepackage{gentium} % a contemporary serif typeface
% \usepackage{libertine} % a softer-feeling serif typeface; also installs sans-serif font Biolinum
% \usepackage{fouriernc} % Century Schoolbook with Fourier maths
% \usepackage{mathpple} % Palatino with Fourier maths


% To set the sans serif font (for \sffamily):
%\usepackage[scaled]{helvet} % Nimbus, like Helvetica
%\usepackage{universalis} % Universalis
%\usepackage{avant} % URW Gothic, like Avant Garde
%\usepackage{PTSansNarrow}
%\usepackage{AlegreyaSans} % Alegreya Sans

% To set the mathematics font:
%\usepackage{eulervm} % Euler, based on a Zapf design

% To set the (usually monospaced) typewriter font:
%\usepackage[ttdefault=true]{AnonymousPro}
%\usepackage[scaled]{beramono}
%\usepackage{inconsolata}
%\usepackage{sourcecodepro}

%\usepackage{cjk} % for Chinese, Japanese, Korean

\usepackage{tabularx} % For easier table formatting.

% \usepackage[nottoc]{tocbibind} % Controls the table of contents
%   nottoc: don't list table of contents inside itself
%   section: go as far as section-level headings

% Automated bibliography
%
\usepackage[
	style=numeric-comp, 
% 	citestyle=authortitle,
	backend=biber,
	sorting=nyt,
	sortcites=true,
	]
	{biblatex}
	\addbibresource{bib/stereo.bib}
	\addbibresource{bib/compmodels.bib}
	\addbibresource{bib/psystems.bib}
	\addbibresource{bib/languages.bib}
% bibliography{bibliography1.bib, bibliography2.bib} % Specify bibliography files 
% bibliography{bib/stereo.bib, Mendeley.bib}

% The below was copied verbatim from https://tex.stackexchange.com/a/154875 on 9 December 2019, and then modified to reverse the ordering of the URL and DOI fields.  The purpose of it is to suppress the use of the URL field if the DOI field is defined - thus avoiding the many instances where the URL is printed, but is just a link to the DOI (besides which, in the PDF the DOI becomes a link to the DOI site).  It seems to work for my purposes.
\DeclareSourcemap{
  \maps[datatype=bibtex]{
    \map{
      \step[fieldsource=doi,final]
      \step[fieldset=url,null]
    }  
  }
}

\begin{document}

% \newcommand{\cml}{Concurrent~ML}
% \newcommand{\csp}{Communicating Sequential Processes}
\newcommand{\fsharp}{F\nolinebreak\hspace{-.05em}\raisebox{.3ex}{\tiny{\textbf{\#}}}}

% ====================================================
%
% FRONTMATTER
%
% Arabic pagination, starting with the title page
% which is counted but not numbered
%
% ====================================================

% Specify the title page content
\title{Message passing-based stereo matching algorithms in cP~systems and Concurrent~ML}
%\subtitle{Using functional programming to make life simpler and promote efficiency}
\author{James Cooper}
\degreesought{Doctor of Philosophy} 
\degreediscipline{Computer Science}
\degreecompletionyear{2020}

% Print the title page
\maketitle

% Abstract, up to 350 words
\begin{abstract}
    
% \Gls{mc}, also known as \gls{ps}, is a field of theoretical computer science initially inspired by the principles of the interactions of chemicals within, and their movements through, the membranes of biological cells.  An important property of most types of \gls{ps} is that they are inherently concurrent, with the contents of every membrane evolving simultaneously, and that they have an unbounded space capacity, empowering them to solve traditional computationally challenging problems quickly.  \gls{cps} is a type of \gls{ps} that takes a high-level approach to modelling problems in \gls{mc}, allowing for more straightforward solutions while retaining the advantages for time complexity.

\Gls{mc}, also known as \gls{ps}, is a field of theoretical computer science initially inspired by the principles of the interactions of chemicals within, and their movements through, the membranes of biological cells.  Important properties of most types of \gls{ps} are that they
\begin{inparaenum}[a)]
\item are inherently concurrent, with the contents of every membrane evolving simultaneously; and,
\item that they have an unbounded space capacity, empowering them to solve traditional computationally challenging problems quickly.
\end{inparaenum}
\gls{cps} is a type of \gls{ps} that takes a high-level approach to modelling problems in \gls{mc}, allowing for more straightforward solutions while retaining the advantages for time complexity.
    
This dissertation applies \gls{cps}' capacity for large-scale concurrency to established problems in computer science, including the \glsentrylong{tsp-glossary}, the \glsentrylong{gcp-glossary}, and image \gls{medianfilter}ing.  It also explores from a new angle the pre-existing concept termed here ``\glsxtrlong{nmp}'', which involves separate logical \glsxtrlongpl{pe} communicating with their neighbours via messaging.  A critical difference, explored in some depth in this dissertation, between \glsxtrlong{nmp} and similar concepts is the requirement for a \glsxtrlong{pe} to compute outgoing messages to neighbours based on the messages received from every neighbour \emph{except} the intended recipient.
    
Novel \gls{cps} solutions to the \glsentrylong{tsp-glossary} and \glsentrylong{gcp-glossary}, with time complexity linear to the number of nodes in the graph, along with a constant-time solution to median filtering, are presented.  Experiments with computer implementations of solutions validate those solutions and explore the efficacy of different implementation methods.

Three variants of \glsxtrlong{nmp} are also explored.  Firstly, the traditional \gls{gs} style used in \glsxtrlong{bp}, then an asynchronous approach which is closer to the underlying concept but seemingly so far unexplored, and finally an intermediate \gls{ls} form that naturally arises from studying the previous two.  Analysis of and experiments with these variants show that the \gls{gs} and \gls{ls} styles compute identical results, but the \gls{ls} one is approximately 5-13\% faster in general practice, while the asynchronous variant is typically around another 10\% faster again yet computes almost-identical results.
    
\end{abstract}
 % it's in a separate file

% Dedication (optional)
%\thesisdedication{Dedicated to grandma, and to grammar.}
\thesisdedication{This work is dedicated to the interested reader and all those who make use of material presented within.}

% Preface and/or acknowledgements (optional)
\chapter*{Preface}

Significant parts of this dissertation are based on papers published previously.  In particular: \Cref{chap:tsp}, \nameref{chap:tsp}, is based on \cite{Cooper2019}.  \Cref{chap:gcol}, \nameref{chap:gcol}, is based on \cite{Cooper2019a}.  \Cref{chap:median}, \nameref{chap:median}, presents work from \cite{Cooper2018} and \cite{Cooper2021}.  \Cref{chap:nmp}, \nameref{chap:nmp}, presents work which appeared in an early form in \cite{Cooper2020}, and which has been significantly revised and is under review with the Journal of Membrane Computing as at the time of submission of this dissertation.

\section*{Acknowledgements}

Acknowledgement must be made, firstly, of my supervisors, Doctor Radu Nicolescu and Associate Professor Patrice Delmas.  Without their support and guidance, I certainly would have never made it to this point.  So too must acknowledgement be made of Professor Emeritus Georgy Gimel'farb and Doctor Wannes van der Mark, both of whom have provided invaluable support at times.%  The two Heads of School during my time as a PhD candidate, Professor Robert Amor and Associate Professor Giovanni Russello, have also been supportive of me in their own ways.

Furthermore, thanks must be given to the professional staff from: the School of Computer Science; the School of Graduate Studies; Libraries and Learning Services (particularly those who served as Computer Science subject librarian); the Centre for eResearch; and the wider University of Auckland.  Particular mention must be made of Kaleigh Fennah, Emma Gavenda, Sue Skelly, Sarah Sneyd and Robyn Young.

I also would like to mention several people who were students contemporaneously with me (and in a few cases were also co-authors with me), including: Arabella Anderson; Mihailo Azhar; Daniel Britten; Doctor Trevor Gee; Alec Henderson; Yan Kolezhitskiy; Yezhou Liu; William Hsu; and Eirian Perkins.  Each of these people in some way has improved my time and experience during the PhD process.

I'm fairly sure there are further people who should be mentioned, but I'm afraid that I can't recall them specifically at the time of writing.

% Everyone who assisted me with the use of the languages in [choiceoflang chapter], especially those people who responded to my requests for assistance on the MLton and Guile mailing lists, and in the Racket Slack workspace.  Special mention must be made of: Yawar Raza from the Rochester Institute of Technology, who routinely answered my questions about MLton very quickly and helpfully; and [soegaard2 from Racket slack]

Lastly, and above all else, mention must be made of my mum.  Thanks mum!

\vspace{1cm}
\hfill James Cooper

\hfill Auckland, New Zealand

\hfill September 2021

 % it's in a separate file

% Contents, lists of tables and figures
\settocdepth{subsection} % choose chapter, section, subsection 
\cleardoublepage\tableofcontents
%\cleardoublepage\listoffigures
%\cleardoublepage\listoftables

% Glossary (optional)
% \makeglossaries

% Acronyms (sorted in alphabetical order by their label, rather than their full definition)

%% A-M
\newacronym{bp}{BP}{belief~propagation}
\newacronym{cp}{CP}{concurrent~propagation}
\newacronym{csp}{CSP}{communicating sequential processes}
\newacronym{dpsm}{DPSM}{dynamic~programming \gls{sm}}
\newacronym{dsp}{DSP}{digital signal processor}
\newacronym{enps}{ENPS}{enzymatic numerical P~systems}
\newacronym{fifo}{FIFO}{first-in-first-out}
\newacronym{fpga}{FPGA}{field-programmable gate array}
\newacronym{fps}{FPS}{frames per second}
\newacronym{gpu}{GPU}{graphics processing unit}
\newacronym{gpgpu}{GPGPU}{general-purpose \glsxtrshort{gpu}}
\newacronym{lbp}{LBP}{loopy belief~propagation}
\newacronym{mpbsm}{MPBSM}{message passing-based \gls{sm}}
% \newacronym{mpi}{MPI}{Message Passing Interface}
\newacronym{mrf}{MRF}{Markov random field}
\newacronym{mwt}{MWT}{moving window transform}

%% N-Z
\newacronym{ndcsm}{NDCSM}{noise-driven concurrent \gls{sm}}
\newacronym{nm}{NM}{neighbourhood messaging}
\newacronym{nmp}{NMP}{neighbourhood message passing}
\newacronym{oq}{OQ}{oracle query}
\newacronym{pe}{proxel}{processing element}
\newacronym{pram}{PRAM}{parallel random access machine}
\newacronym{sad}{SAD}{sum of absolute differences}
\newacronym{skps}{SKPS}{simple kernel P~systems}
\newacronym{ssd}{SSD}{sum of squared differences}

% Combo glossary & acronyms

% A-M
\newdefacr{blas}{BLAS}{basic linear algebra subprograms}{Basic Linear Algebra Subprograms}{A collection of low-level functions, typically written in Fortran, to perform core linear algebra processes such as vector and matrix addition and multiplication.  The netlib implementation of BLAS is considered the ``reference'' implementation (see \url{https://www.netlib.org/blas/}), but other implementations are available --- in particular, some hardware vendors may produce their own version optimised for their particular hardware, such as Nvidia's cuBLAS.  Many, perhaps most, high-performance mathematical libraries in higher-level programming languages, \eg{} Python's Numpy, make use of a BLAS implementation internally.}
\newdefacr{cml}{CML}{Concurrent~ML}{Concurrent~ML}{A programming style introduced by John Reppy as an extension of the pre-existing programming language Standard~ML;  specifically, the Standard~ML of New Jersey implementation.  Conceptually, Concurrent~ML has a lot in common with \glsxtrlong{csp}.  Both involve each logical process mostly advancing on its own, but sometimes synchronising with others over channels (see also \vref{fig:back:cml_exchange}).  Concurrent~ML is described in detail in \cite{Reppy2007}.}
\newdefacr{gcp}{GCP}{graph colouring problem}{Graph Colouring Problem}{The problem of finding a way to colour the nodes in a graph, such that no adjacent nodes share a colour.  This problem is typically further constrained to using a maximum number of colours for the entire graph, or finding the minimum number of colours required to perform the colouring.}
\newdefacr{hcp}{HCP}{Hamiltonian cycle problem}{Hamiltonian Cycle Problem}{Closely related to the \glsdesc{hpp}.  In this problem, the Hamiltonian Path must also be a cycle, \ie{} it ends back at the origin node.}
\newdefacr{hpp}{HPP}{Hamiltonian path problem}{Hamiltonian Path Problem}{The problem of determining whether, for a given graph, there exists a traversal of the graph in which each node is visited exactly once.}
\newdefacr{lapack}{LAPACK}{linear algebra package}{Linear Algebra Package}{A (relatively) low-level software library for performing more advanced linear algebra processes, such as solving systems of linear equations, or matrix factorizations \eg{} LU, QR and Cholesky Decompositions.  LAPACK is typically built atop \glstext{blas}, meaning that it is highly portable and efficient across different hardware architectures, so long as a good quality \glstext{blas} implementation is available.  There reference implementation of LAPACK can be found at \url{https://www.netlib.org/lapack/} but others are available too.}
\newdefacr{lhs}{LHS}{left-hand side}{Left-hand Side}{A part of the definition of a \gls{cps} rule.  It is the list of terms which \emph{must} be present and as-yet unused in a rule at the current step, in order for the current rule to be applicable.  All terms appearing here are deleted from the containing cell at the end of the step.}

% N - Z
\newdefacr{rhs}{RHS}{right-hand side}{Right-hand Side}{A part of the definition of a \gls{cps} rule.  It is the list of terms which are created by the rule's application, and appear in the relevant cell at the end of the step.}
%
% \newdefacr{sgm}{SGM}{semi-global~matching}{Semi-global~Matching}{Semi-global~matching is a \gls{sm} algorithm introduced by Hirschmüller that, as the name suggests, sits somewhere between traditional global and local \gls{sm} algorithms.  This provides advantages in that retains much of the speed advantage of local algorithms, while also receiving much of the benefit derived from taking into consideration a larger part of the total input images.}
%
\newdefacr{tsp}{TSP}{travelling salesman problem}{Travelling Salesman Problem}{An extension of the \glsdesc{hcp}.  Here, the edges between nodes have weights, and the goal of the problem is to find a minimum weight Hamiltonian cycle on the graph.}

% Glossary Entries
% \newglossaryentry{}{name={},description={}}

%% A-M
\newglossaryentry{actor}{text={actor},name={Actor},description={A model of message-passing-based concurrent programming originally developed  chiefly by Carl Hewitt.  Its defining characteristics are that it uses asynchronous messaging, whereby the sender and receiver do not need to coordinate or synchronise at all; and that instead of using channels or similar as a go-between, Actors send messages directly to each other, which necessitates ``knowing'' (\ie{} holding a reference to) the intended recipient.  Notable examples of implementations of Actors are found in the programming languages Erlang and Pony, and the Scala library Akka.}}
%
\newglossaryentry{clps}{text={cell-like P~systems},name={Cell-like P~Systems},plural={cell-like P~system},description={The original \gls{ps} variant first proposed by Gheorghe Păun, based on the operation of chemicals inside biological cells.  The main characteristic of cell-like P~systems is that the chemicals are represented by a potentially-infinite variety of atomic objects, and are contained within one or more membranes.  Every cell-like P~system has an outermost membrane, the \emph{skin membrane} which separates the cell from the \emph{environment}, and may have more membranes inside itself.  These latter membranes are nested in such a fashion that they resemble a graphical tree.  Each membrane has a separate \gls{ruleset}, though the \glspl{ruleset} are not necessarily unique.  Rules may create and remove chemicals based on the presence (or absence) or other chemicals inside a membrane, and also send chemicals to an inner \emph{child} membrane, or to an outer \emph{parent} membrane.  Perhaps due to it being the original variant (and thus pre-dating the term ``cell-like P systems''), some forms of P~systems which could be regarded as cell-like are not necessarily labelled as such, \eg{} P~systems with active membranes.}}
%
\newglossaryentry{compartment}{text={compartment},name={Compartment},description={A generic umbrella term for the processing units, which also act as the containers of the system's objects, of a given \gls{ps} type.  For example, these are: the membranes of \gls{clps}; the cells of \gls{tlps}; the neurons of \gls{snps}; and the \glspl{tlc} of \gls{cps} --- though in the case of \gls{cps} the data-only complex objects are sometimes also described as sub-compartments or micro-compartments, because they still contain objects.}}
%
\newglossaryentry{cps}{text={cP~systems},name={cP~Systems},description={A variant of \gls{ps} created by Radu Nicolescu.  The main point of difference between cP~systems and other variants is that, in general, cP~systems is `higher-level'.  That is, instead of specifying (semi-)uniform families of rules customised to each specific problem of a given type, cP~systems \glspl{ruleset} typically cover all possible problem specifications without customisation ahead-of-time through the use of variables.  The cP~systems `runtime' performs unification over the variables.  Furthermore, only the \glspl{tlc} have associated rules, and all other terms contained within are merely inert objects.  See also \cref{chap:cpsystems}.},plural={cP~system}}
%
\newglossaryentry{disparity}{text={disparity},name={Disparity},description={The shift, measured as a number of pixels, of a point in one stereo image to its location in the other image.  When combined with information about the cameras used to capture the images which is derived from calibration, the measured disparity is used to estimate the distance from the cameras to the point in the scene.  Disparity is often also referred to as `parallax'.}}
%
\newglossaryentry{disparitymap}{text={disparity map},name={Disparity Map},description={An output array/image (often just using a single channel of 8- or 16-bit integers) which encodes the final estimated disparities for every pixel in the input images.  The range of values in a disparity map itself is restricted to range of possible disparities used in the \gls{sm} algorithm.  It is common, however, to adjust the disparity map in some fashion to make differences in disparity estimates more visible through techniques such as histogram equalisation.  Most published figures of disparity maps will have had such an adjustment applied to make them more useful.  Disparity maps are sometimes also referred to as `digital parallax maps.'}}
%
\newglossaryentry{fne}{description={The `standard' neighbourhood arrangement in situations such as \glsxtrlong{bp}.  When used in reference to a grid, it typically means the other locations in the grid immediately to the left, right, above and below.},text={four-neighbourhood},name={Four-neighbourhood}}
%
\newglossaryentry{functor}{description={In the context of \gls{cps}, a functor is the label applied to the delimiters of a complex term.  \Eg{} for \(\cpfunc{a}{\cpfunc{b}{c}}\) both \(a\) and \(b\) are functors, because they are the outer label for their respective inner terms.  ``Functor'' is also sometimes used as a shorthand to refer to the complex term as whole.  Strictly speaking, this is incorrect, but used for convenience nevertheless, as writing \enquote{the complex term denoted by the functor \(a\)} is usually overly verbose.},text={functor},name={Functor}}
%
\newglossaryentry{gs}{description={In the context of \glsxtrlong{nmp}, this refers to an approach to messaging whereby the entire lattice waits until every \glsxtrshort{pe} in the lattice has finished its messaging for a given generation before proceeding to the next generation.},text={globally-synchronous},name={Globally-synchronous}}
%
\newglossaryentry{inhibitor}{name={Inhibitor},text={inhibitor},description={In \gls{cps}, an inhibitor on a rule denotes an object which \emph{must not} be present in the relevant \gls{tlc} for the rule to be applicable.}}
%
\newglossaryentry{ls}{description={In the context of \glsxtrlong{nmp}, this refers to an approach to messaging whereby every \glsxtrshort{pe} in the lattice waits until it has received \emph{all} messages pertaining to one generation before accepting any messages pertaining to a subsequent generation.},text={locally-synchronous},name={Locally-synchronous}}
%
\newglossaryentry{mc}{text={membrane computing},name={Membrane Computing},description={A computational model inspired by the functioning of biological systems, specifically the interactions of chemicals inside the membranes of a biological cell.  The terms Membrane Computing and P systems are normally used interchangeably.}}
%
\newglossaryentry{mecosim}{name={MeCoSim},description={Software used to simulate arbitrary P~systems, from a multitude of variants.  MeCoSim was created and is maintained by researchers from Universidad de Sevilla (University of Seville) in Spain (see further \cite{MeCoSim}).  Generally, rules for a given system are specified with a \gls{plingua} file.  These rules can then be run on a specific set of starting objects, with MeCoSim reporting the outcome from executing the rules at the end of each step.}}
%
\newglossaryentry{medianfilter}{text={median filter},name={Median Filter},description={An image processing technique often used to remove `salt \& pepper' noise from images.  Every pixel's value is updated to the median value of it and its neighbours, in a given neighbourhood.  It is an embarassingly-parallel problem in that each pixel in the image can compute its median without relying on other pixels' results, so long as it can observe their current value.  The necessity for performing selection, however, means that it is non-trivial to implement a fast median filter.  It does not easily reduce to basic arithmetic operations only.}}
%
\newglossaryentry{ms}{text={micro-surgery},name={Micro-surgery},description={An alternative operation that can be performed on \gls{cps}' compound terms.  Instead of each rule application seizing control of the entire term, it merely takes hold of a subset of the terms contents.  A major benefit of this is to enable in a single step arithmetic on numeric multisets that would instead require iterations to complete.},plural={micro-surgeries}}

%% N-Z
\newglossaryentry{promoter}{name={Promoter},text={promoter},description={In \gls{cps}, a promoter on a rule denotes an object which \emph{must} be present in the relevant \gls{tlc} for the rule to be applicable, but which is \emph{not} consumed or deleted by the rule.}}
%
\newglossaryentry{ps}{name={P~Systems},see={mc},text={P~systems},description={An alternative name for \gls{mc}.},plural={P~system},}
%
\newglossaryentry{plingua}{name={P-Lingua}, description={A plain-text markup language used to represent \gls{ps} specifications in computer files, readable by both humans and computers.}}
%
\newglossaryentry{ruleset}{text={ruleset},name={Ruleset},description={The rules associated with a particular \gls{compartment} in a given \gls{ps} variant.}}
\newglossaryentry{sm}{text={stereo~matching},name={Stereo~Matching},description={A family of methods to match points from different images of the same scene to estimate the distance from the capturing cameras to objects in the scene.}}
\newglossaryentry{snps}{text={spiking neural P~systems},name={Spiking Neural P~Systems},description={Spiking neural P~systems is one of the three main \gls{ps} variants in common use today, and was introduced most recently.  Whereas \gls{clps} is based on intra-cellular chemical interactions, and \gls{tlps} on inter-cellular interactions, spiking neural P~systems is explicitly based on the idea of neurons in the brain interacting via synapses.  In principle, the communication-based nature of spiking neural P~systems makes it closer to \gls{tlps} than \gls{clps}, but unlike those two it uses only a single object in its alphabet, the spike.  This difference can make solutions to some problems more complex, but has the advantage that it makes it (comparatively) easy to implement those solutions with linear algebra operations.}}
\newglossaryentry{tlc}{text={top-level cell},name={Top-level Cell},description={The processing units of \gls{cps}, loosely equivalent to the skin membrane of other P~systems variants.  These are the only objects in the system which have rules and are `active'.  Generally, all other objects present in a system except channels are found inside a top-level cell.}}
\newglossaryentry{tlps}{text={tissue-like P~systems},name={Tissue-like P~Systems},description={Tissue-like P~systems is one of the oldest \gls{ps} variants, especially of those still in common use, and is based on the transmission of chemicals between cells via channels in biological tissue (rather than movement of chemicals through membranes within a cell like \gls{clps}).  A notable aspect of tissue-like P~systems is that they often include no ability for the cells to process their own contents.  Instead, all computation arises as a consequence of the communication.},plural={tissue-like P~system}}

% ====================================================
%
% MAINMATTER
%
% Include external chapter files here using
% the \input{} command
%
% If you run out of memory during compilation,
% switch some or all chapters to \include{} instead of \input{}, 
% but watch out for pagination problems.
%
% ====================================================

\chapter{Introduction}

There are many problems in computer science that, from the standpoint of finding a solution quicker, appear amenable to the use of concurrency in their solution.  There are many more where opportunities for concurrency are less apparent and where applying concurrent processing is much less straightforward, but which nevertheless can benefit from it.  For example, many problems that arise in image processing, or which can be modelled graphically, fall into one or the other of those categories.  Modelling these problems with a computational model that embraces concurrency at its core can prove beneficial, both in terms of exposing new ways to structure computations on existing computer implementations, \eg{} \cite{GimelFarb2013a,Nicolescu2014b}, and (potentially) in terms of providing blueprints for the efficient use of new hardware implementations in the future.\footnote{One need only consider the example and future importance of such computing luminaries as Babbage, Lovelace and Boole, all working before a complete working electronic or mechanical computer was created, to see the truth of this.}

This dissertation takes precisely that approach.  Pre-existing problems drawn from the history of computer science are modelled in an inherently concurrent theory of computation, \gls{cps}, seeking the lowest possible theoretical time complexity for each.  Those highly concurrent models are then translated into operative computer programs on current widely available computer hardware.  The theoretical computational model used in this dissertation is \gls{cps} \cite{Nicolescu2018}, one of the branches of the broader field of \gls{mc} (also known as \gls{ps}) \cite{Paun2010b,Paun2002}.  Different hardware, programming languages and techniques have been used in the course of this work, varying based on availability, apparent fit with the relevant \gls{cps} solution, and lessons learned from earlier development.

\Gls{mc} was chosen over other computational models with an emphasis on concurrency because it appears to be a highly promising developing area, especially \gls{cps}.  While \gls{mc} is bio-inspired \cite{Paun2000}, its utility extends far beyond simulating biological systems.  In the original concept (still in use), each \gls{compartment} delimited by a membrane is considered a separate processor with unbounded capacity.  Furthermore, the membranes are frequently considered to have the capability of dividing and creating new membranes and thus processors.  This ability for membrane creation, combined with unbounded space capacity, has long been known to enable polynomial-time solutions to problems that have traditionally been considered intractable by trading space for time \cite{Paun1999a,Sosik2003}.
Other particularly notable calculi are briefly summarised, but the focus of this work is very much on \gls{ps}, and specifically \gls{cps}.

All forms of \gls{ps} operate based on sets of rules associated with delimited \glspl{compartment}, which typically contain multisets of objects. The main advantage of \gls{cps} over other \gls{ps} is its powerful concept of unification (see more in \vref{sec:cps:unification}), which permits the rules to adapt themselves automatically at ``runtime'' to the contents of the relevant \gls{compartment}(s).  By contrast, other forms of \gls{ps} typically require some degree of pre-processing to adapt uniform or semi-uniform families of rules to the given circumstances.  This tends to make interpretation of the rules more difficult, but also can mean that the solution to the problem at hand sometimes is partially pre-computed before the \glspl{ps} ever starts working.

The chief aim of this work is to describe new solutions to pre-existing problems, where the new solutions provide superior time complexity characteristics to other known solutions.  That is, applying relatively new and lesser-known tools of computer science to known problems, with the goal of finding novel solutions with a low time complexity, where low is considered linear complexity -- \bigoh{n} -- or better.  
This is achieved by describing these solutions in \gls{cps}, which is high-level and declarative, and should make each problem's overarching operation and core elements evident.
A secondary aim of this work is to try to find novel, useful ways to implement the proposed solutions to the pre-existing problems on current computers, even when the \gls{cps} model does not map clearly to the hardware.  

% \section{Outline}

% This dissertation proceeds as follows: \Cref{chap:back} summarises some alternative models for concurrent computation -- none of which have quite the same properties as \gls{ps} -- that have been explored in the past; provides an overview of a programming approach called \gls{cml} \cite{Reppy2007,Reppy1991}, which appears to be a good fit in many ways to some of the \gls{cps}; and finally surveys the state of \gls{mc} as at the time of writing.  Next, \cref{chap:cpsystems} fairly comprehensively documents \gls{cps} as used in recent publications.  The various elements of \gls{cps} that have been defined and are still in use -- many of which are used later in this work too -- are shown, before demonstrating how well-known data structures may be modelled in \gls{cps}.  Combined with the examples provided in the subsequent chapters, this should permit the interested reader to start modelling problems in \gls{cps} themselves.

% \Crefrange{chap:tsp}{chap:median} each investigate one or more problems exhibiting significant potential for concurrency.  Specifically: \cref{chap:tsp} focuses on the \gls{hpp}, \gls{hcp} and \gls{tsp} \cite{Applegate2006,Cook2012}; \cref{chap:gcol} on the \gls{gcp} \cite{Lewis2016}; and, \cref{chap:median} on the \gls{medianfilter}, a \gls{mwt} operation performed in image processing \cite{Fisher2016,Gimelfarb2018}.  In each of these chapters, the problem has first been modelled using \gls{cps}, with one or more \glspl{ruleset} developed and analysed, providing a highly concurrent solution to the problem.  Then, the \gls{cps} approach has been investigated empirically, using one or more computer programs, using \gls{cml} in the case of \cref{chap:gcol,chap:median}.  These programs validated the correctness of the \gls{cps} \glspl{ruleset}.  They also sometimes highlight interesting aspects of either the problem or the proposed \gls{cps} solution.

% \Cref{chap:nmp} explores the idea of \gls{nmp}.  Originally motivated by \gls{lbp} \cite{Sun2003,Felzenszwalb2006,Felzenszwalb2011} as used in \gls{sm} \cite{Sinha2020,Tippetts2016,Scharstein2002}, this \lcnamecref{chap:nmp} investigates the situation where individual logical \glspl{pe} are arranged in a lattice and exchange messages as part of computing the solution to a given problem, but every \gls{pe} can only send a message to each connected neighbouring \gls{pe} \emph{after} it has received messages from every other one of its neighbours.  There are two main phases of the core computation process for \gls{nmp}:
% \begin{inparaenum}[a)]
% \item inter-\gls{pe} communications between neighbours via message passing; and
% \item intra-\gls{pe} computations to update internal data and produce new messages.
% \end{inparaenum}  The focus in \cref{chap:nmp} is on the first of these only, with the second abstracted over for current purposes using an oracle with which \glspl{pe} communicate to perform the required updates.

% Finally, \cref{chap:conc} concludes this dissertation by summarising the work described here, and suggesting multiple possible avenues for future research.

% All of \crefrange{chap:tsp}{chap:nmp} provide one or more novel \gls{cps} \glspl{ruleset} with an accompanying explanation, at least one worked example of said \gls{ruleset}, some analysis of the proposed \gls{cps} solution including (where applicable) comparisons to related solutions found elsewhere in \gls{ps}, and the results of at least one experimental computer simulation.

% \section{Key Results}

% The \gls{cps} solutions to the \gls{hpp}, \gls{hcp} and \gls{tsp} in \cref{chap:tsp} all have a time complexity that is linear with respect to the number of nodes in the graph -- \ie{} it is \bigoh{n}, where \(n\) is the order of the graph -- and requires a total of four (for the \gls{hpp}) or five (for the \gls{hcp} and \gls{tsp}) fixed rules for \emph{every} graph with a possible solution.  Simulations are implemented in \fsharp{}, Erlang and Prolog, with the Prolog simulation in particular closely matching the \glspl{cps}.

% \Cref{chap:gcol} experiments with using \gls{cml} to implement a pre-existing \gls{ps} solution to the \gls{gcp}, finding that while the implementation is effective, there is less communication involved than initially expected.  Then, it describes a \gls{cps} solution to the \gls{gcp} which has a time complexity of \bigoh{n}, where \(n\) is the number of nodes in the graph, for graphs where a \(k\)-colouring is possible.  For graphs where no such colouring is possible, it has a time complexity of \bigoh{k}.  \Cref{chap:gcol} represents the first known application of \gls{cml} to any type of \gls{ps} (followed by \cref{chap:median} as the second).

% \Gls{medianfilter}ing is structurally more challenging than many similar \gls{mwt} problems in image processing.  It implies a considerable amount of numerical sorting, and the necessity of preserving the image data prevents the use of typical enhancements to other algorithms, which generally perform a simpler piecewise reduction over the data.  The \glspl{cps} presented in \cref{chap:median} demonstrates conclusively that sufficiently large-scale concurrency can solve the \gls{medianfilter} problem in constant time, however.  The experimental results suggest that current CPUs are far from managing this level of concurrency, though, with the \gls{cml} implementation proving to be by far the slowest --- at minimum 20 times slower than the fastest alternative.

% \Cref{chap:nmp} analyses the traditional \gls{gs} approach to \gls{nmp}, where every \gls{pe} waits until every other \gls{pe} has finished exchanging messages before progressing further itself, and an asynchronous alternative where each \gls{pe} sends the next message to each neighbour as soon as it has received the necessary input messages.  From these, an intermediate \gls{ls} approach where each \gls{pe} waits only for all of its own expected messages to arrive before progressing, arises naturally and is explored also.

% The behaviour and comparative running times required for each variant are investigated empirically, and the \gls{ls} method found strictly superior to the \gls{gs} method -- exhibiting the same messaging behaviour but typically running 5-13\% faster -- while the asynchronous method is, in general, roughly 10\% faster again than the \gls{ls} yet computes final results that ordinarily are within roughly 0.1\% of those of the \gls{ls} method.

\section{Outline and Key Results}

This dissertation proceeds as follows: \Cref{chap:back} summarises some alternative models for concurrent computation -- none of which have quite the same properties as \gls{ps} -- that have been explored in the past;  explains the concept of synchronous vs asynchronous communication in distributed systems (relevant later on);  provides an overview of a programming approach called \gls{cml} \cite{Reppy2007,Reppy1991}, which appears to be a good fit to some of the presented \gls{cps} \glspl{ruleset}; and finally surveys the state of \gls{mc} as at the time of writing.  Next, \cref{chap:cpsystems} fairly comprehensively documents \gls{cps} as used recently.  The various elements of \gls{cps} that have been defined and are still in use -- many of which are used later in this work too -- are shown, before demonstrating how well-known data structures may be modelled in \gls{cps}.  Combined with the examples provided in the subsequent chapters, this should permit the interested reader to start modelling problems in \gls{cps} themselves.

\Crefrange{chap:tsp}{chap:median} each investigate one or more established problems exhibiting significant potential for concurrency.  Specifically: \cref{chap:tsp} focuses on the \gls{hpp}, \gls{hcp} and \gls{tsp} \cite{Applegate2006,Cook2012}; \cref{chap:gcol} on the \gls{gcp} \cite{Lewis2016}; and, \cref{chap:median} on the \gls{medianfilter}, a \gls{mwt} operation performed in image processing \cite{Fisher2016,Gimelfarb2018}.

\Cref{chap:nmp} explores the idea of \gls{nmp}.  Motivated initially by \gls{lbp} \cite{Sun2003,Felzenszwalb2006,Felzenszwalb2011} as used in \gls{sm} \cite{Sinha2020,Tippetts2016,Scharstein2002}, this \lcnamecref{chap:nmp} investigates the situation where individual logical \glspl{pe} are arranged in a lattice and exchange messages as part of computing the solution to a given problem, but every \gls{pe} can only send a message to each connected neighbouring \gls{pe} \emph{after} it has received messages from every other one of its neighbours.  There are two main phases of the core computation process for \gls{nmp}:
\begin{inparaenum}[a)]
\item inter-\gls{pe} communications between neighbours via message passing; and
\item intra-\gls{pe} computations to update internal data and produce new messages.
\end{inparaenum}  The focus in \cref{chap:nmp} is on the first of these only, with the second abstracted over for current purposes using an oracle with which \glspl{pe} communicate to perform the required updates.

All of \crefrange{chap:tsp}{chap:nmp} provide one or more novel \gls{cps} \glspl{ruleset} with an accompanying explanation, at least one worked example of said \gls{ruleset}, analysis of the proposed \gls{cps} solution including (where applicable) comparisons to related solutions found elsewhere in \gls{ps}, and the results of at least one experimental computer simulation.

The \gls{cps} solutions to the \gls{hpp}, \gls{hcp} and \gls{tsp} in \cref{chap:tsp} all have a time complexity that is linear with respect to the number of nodes in the graph -- \ie{} it is \bigoh{n}, where \(n\) is the order of the graph -- and requires a total of four (for the \gls{hpp}) or five (for the \gls{hcp} and \gls{tsp}) fixed rules for \emph{every} graph with a possible solution.  Simulations are implemented in \fsharp{}, Erlang and Prolog, with the Prolog simulation in particular closely matching the \glspl{cps}.

\Cref{chap:gcol} experiments with using \gls{cml} to implement a pre-existing \gls{ps} solution to the \gls{gcp}, finding that while the implementation is effective, there is less communication involved than initially expected.  Then, it describes a \gls{cps} solution to the \gls{gcp} which has a time complexity of \bigoh{n}, where \(n\) is the order of the graph, for graphs where a \(k\)-colouring is possible.  For graphs where no such colouring is possible, it has a time complexity of \bigoh{k}.  \Cref{chap:gcol} represents the first known application of \gls{cml} to any type of \gls{ps} (followed by \cref{chap:median} as the second).

\Gls{medianfilter}ing is structurally more challenging than many similar \gls{mwt} problems in image processing.  It implies a considerable amount of numerical sorting, and the necessity of preserving the image data prevents the use of typical enhancements to other algorithms, which generally perform a simpler reduction over the data.  The \glspl{cps} presented in \cref{chap:median} demonstrates conclusively that sufficiently large-scale concurrency can solve the \gls{medianfilter} problem in constant time, however.  The experimental results suggest that current CPUs are far from managing this level of concurrency, though, with the \gls{cml} implementation proving to be by far the slowest --- at minimum 20 times slower than the fastest alternative.

\Cref{chap:nmp} analyses the traditional \gls{gs} approach to \gls{nmp}, where every \gls{pe} waits until every other \gls{pe} has finished exchanging messages before progressing further itself, and an asynchronous alternative where each \gls{pe} sends the following message to each neighbour as soon as it has received the necessary input messages.  From these, an intermediate \gls{ls} approach where each \gls{pe} waits only for all of its own expected messages to arrive before progressing, arises naturally and is also explored.

The behaviour and comparative running times required for each variant are investigated empirically, and the \gls{ls} method found strictly superior to the \gls{gs} method -- exhibiting the same messaging behaviour but typically running 5-13\% faster -- while the asynchronous method is, in general, roughly 10\% faster again than the \gls{ls} yet computes final results that ordinarily are within roughly 0.1\% of those of the \gls{ls} method.

Finally, \cref{chap:conc} concludes this dissertation by summarising the work described here and suggesting multiple possible avenues for future research.
\chapter{Related Work}
This review of related past work focuses on the three main areas of Computer Science that are relevant to this dissertation:  Stereo Matching, Formal Models of Computation (especially P~systems), and Concurrent~ML.  In none of these cases does the review come close to comprehensively covering the entire span of the given area.  It merely tries to cover as much of the relevant recent and classic work as possible, while giving an extremely brief introduction to these areas.  The interested reader who is unfamiliar with any of these topics is strongly advised to refer to the references cited.

\section{\glsentrylong{mpbsm}}

% \subsection{Preliminaries}
% What do I actually need to put in here?

% \subsubsection{Stereo Matching}

\subsection{Stereo Matching}

\subsubsection{Local vs Global}
\cite{Scharstein2002}

\subsubsection{Markov Random Fields \& Bayesian Statistics}
\cite{Kolmogorov2015,Blake2011}

Geman \& Geman \cite{Geman1984} showed, however, that Markov Random Fields are equivalent to Gibbs Distributions and that the two could be applied usefully to image tasks \cite{Gimelfarb1999}.

Frequently, in global methods the function used for the data cost is quite simplistic.  Most common is the use of simple absolute difference between the intensities of the pixels compared.  Other popular methods include \gls{sad} and \gls{ssd}, adaptive window methods \cite{Yoon2005,Yoon2006}, and Birchfield \& Tomasi's Pixel Dissimilarity Measure \cite{Birchfield1998}.

In general, most Markov Random Field approaches to \gls{sm} tend to use a truncated linear function to estimate the discontinuity/smoothness cost.  Such a function typically takes a form such as \[ E_{discontinuity} = \alpha \times min(| d_p - d_q |, \beta) \] where, for the purposes of this equation, \(E_{discontinuity}\) represents the total estimated cost of the assignment; \(\alpha\) is a scaling coefficient that may or may not be used; \(\beta\) is a constant that provides the upper limit to the cost estimate; and \(d_p\) and \(d_q\) are the proposed labels of the current pixel and its neighbour currently under consideration.  While a simple absolute difference function is perhaps the most common applied to the labels, it is important to note that it is not the only one that could be used.  

For example, Ha and Jeong \cite{Ha2016} use a two-step Potts model, with different penalties for a difference of 1 compared to a difference of 2 or greater. Conversely, Tan \textit{et al.} \cite{Tan2017} comment that a typical Potts function can be viewed as a special case of the absolute difference truncated linear function, where the truncation value (\(\beta\) in the equation above) is 1, while the coefficient is the value of the Potts penalty parameter.

The choice of the truncated linear function is motivated by the assumption that most surfaces in images either are planar, or smoothly vary in disparities, and thus larger jumps should be penalised more heavily, but very large jumps are almost certainly indicative of an object boundary where a large difference in disparities is warranted.  Therefore, at a certain point, the penalty to assign significantly different values should stop growing, so as not to reduce the likelihood of correctly assigning large differences in disparities at object edges.

\subsection{Belief Propagation}
\gls{bp} was introduced by Pearl \cite{Pearl1982} for use with inference engines, in the context of Bayesian Statistics and Gibbs Distributions.  \gls{bp} was first applied to \gls{sm} in \cite{Sun2003} where it demonstrated excellent performance compared to many contemporary matching algorithms, but the `breakthrough' paper was arguably \cite{Felzenszwalb2006}, where a near-real time implementation was presented which still had extremely good results.

Yang \textit{et al.} \cite{Yang2006a} built upon hierarchical \gls{bp}, adding in extra steps before and after the \gls{bp} process.  They combine information derived from using the mean shift algorithm \cite{Comaniciu2002}; a colour-weighted correlation method based on Yoon \& Kweon's \cite{Yoon2006} applied to both the left and right images; a left-right consistency check to detect occluded pixels; a plane-fitting process based on Tao \& Sawhney's \cite{Tao2000}; as well \gls{bp} itself.  While combining these various techniques leads to a highly-accurate disparity map,\footnote{This algorithm achieved the top ranking on Middlebury when it was first introduced.} it is \emph{extremely} slow.

Typical \gls{bp} uses the four-connected neighbourhood to define the neighbours of each given node in the grid.  This means that each node passes messages to and from it's immediate neighbours up, down and to the left and right of it in the grid.  Other neighbourhood arrangements are possible though, depending on the underlying model one wants to use.  For example, Tan \textit{et al.} \cite{Tan2017} describe an approach to \gls{bp} where every pixel is considered to be a neighbour of every other pixel.  Messages are weighted according to the distance across the grid between the neighbours, with nearer neighbours having a greater impact upon a pixel's final beliefs.  The major advantage of this approach is that it almost eliminates the need for repeated iterations of message passing.

Ha and Jeong \cite{Ha2016} suggested a different approach for scheduling the messages.  Instead of each pixel repeatedly exchanging messages with its neighbours until a reasonable amount of the grid would have been spanned, they start in one of the corners in the image, and sequentially pass messages along two directions until reaching the other corner, repeating this process once for each corner.  The great advantage of this is that in principle one only needs to perform message exchanges in each direction once.  Their implementation still required roughly \SI{3.5}{\second} to complete, however, without returning a significantly more accurate disparity map.\footnote{The authors claimed that their method was \numrange{300}{600} times faster than `standard` \gls{bp}, but they did not specify their stopping condition.  Based on their reported results it appears that they used well over 300 iterations on an image -- many more than would be reasonable for image sizes likely to be targeted for real-time \gls{sm}.}

Balossino \textit{et al.} \cite{Balossino2007} suggested an alternative formulation to the traditional grid of Loopy \gls{bp}.  Instead, they built a forest of trees, each of which was rooted at the given pixel under consideration, and which has a handful of neighbouring pixels as children.  The attraction of this approach is that it restores the properties of optimality and convergence described in \cite{Pearl1982} for one round of messages up and down each tree.  This advantage is tempered, however, by the necessity of combining results from different trees.  The final accuracy appeared to be worse than with \cite{Felzenszwalb2006}, and there was no reporting of the running time, though it seems unlikely that this approach was fast.

It should be noted that \gls{bp} is \emph{not} regarded as the current top-performing \gls{sm} algorithm.  In 2013 Tippets \textit{et al.} found that, of algorithms implemented on the CPU, blah was the fastest `accurate' method, while blah provided the highest accuracy.  In terms of GPU implementations, blah was the fastest while blah was the most accurate.  \gls{bp} \emph{is} amenable to parallelisation (unlike its traditional rival, Graph Cuts \cite{Tappen2003}) and GPU implementations, but the main point of interest for it in this work is the fact that it is explicitly built around the concept of independent processing elements exchanging messages.

\subsubsection{Real-time/resource-constrained Belief Propagation}
One of the major drawbacks of \gls{bp} as compared to a number of other approaches to \gls{sm} is that a simple naïve implementation is both quite slow, and very memory-intensive.  Slow because of the requirement to perform many iterations, and memory-intensive because \emph{at least} one copy of the data costs and the message estimates for each neighbour must be stored in memory, with the result that a number of values on the order of at least \(O(5XYD)\) are kept in memory, where X and Y are the width and height of the stereo images, and D is the size of the disparity range.

Seeking to derive the comparative benefits of a global stereo algorithm without compromising resource and time requirements too much, there have been a number of attempts at a real-time \gls{bp} algorithm \cite{Xiang2012,Yang2010,Yang2006,Liang2011,Gupta2012,Perez2010,Felzenszwalb2006}.

Felzenszwalb \& Huttenlocher \cite{Felzenszwalb2006} made three significant improvements:  i) They demonstrated a way to reduce the complexity of the message update process from \(O(|D|^2)\) to \(O(|D|)\) (where \(|D|\) is the total number of potential disparity labels).  ii) They showed that, because each pixel relies entirely upon the messages received from its neighbours at the previous iteration, only half of the pixels in fact need to be updated in a given iteration, without affecting the final results.  This both halves the number of message computations required at iteration, but moreover means that only a single copy of the messages need be kept while ensuring that messages computed earlier in an iteration have no impact upon messages computed later.  iii)  They introduced a hierarchical approach, where the first iterations were performed over a much smaller grid, representing an amalgamation of the actual grid, but later iterations would operate over larger and larger grids until reaching the full size.  This had the benefit of propagating information across the grid in a much faster fashion, with relatively little loss in accuracy.  Almost every claimed real-time \gls{bp} algorithm since uses the hierarchical approach.

Yang \textit{et al.} \cite{Yang2006} claimed that they had devised a new approach that would provide a 45x speedup, and boasted that their system could achieve a frame rate of 16 \gls{fps} on a 320 x 240 image with 16 disparity levels.  This claim was largely based, however, in the fact that they used a GPU to implement it -- and then later stated that they had not yet implemented their method on a GPU.  Furthermore, they did not present anything that had not already been described by Felzenszwalb \& Huttenlocher.

Yu \textit{et al.} \cite{Yu2007} presented a proposed approach for compressing the messages, thus reducing total memory occupied, but it has not proven popular.  This may be because it is not amenable to parallelisation, thus significantly reducing its practicality \cite{Yang2010}.

Yang, Wang \& Ahuja \cite{Yang2010} proposed an approach which they claim needs only constant memory space, regardless of the number of disparities involved, while still returning results that are almost as accurate.  For example, they claim that for an image with 800 x 600 pixels and 300 disparity levels, their algorithm requires only around \SI{9}{\mebi\byte} of memory -- it is not clear though whether they include storing the computed data costs in that amount or not.  The main element of their approach is that as they move from the coarser levels of the hierarchy, they proportionally reduce the number of disparity labels considered at each level, keeping the total memory required constant.  %This leads to an issue in that, should the true disparity not be selected for inclusion at a reduction, that pixel will never see the correct disparity label assigned to it.  To work around this, they 

Gupta \& Cho \cite{Gupta2012} used 3x3 tiles in their hierarchical method, rather than the usual 2x2.  This meant that their process was somewhat faster overall, and means that at the more coarse levels they need less memory.  The other main differences between their method and previous ones are that they use an `alternative schedule method' borrowed from \cite{Tappen2003}; and they use a different disparity refinement operation as final step.  The results, in terms of accuracy and speed, do not appear to be any better than earlier papers, though.

Xiang \textit{et al.} \cite{Xiang2012} also boasted of a new technique that enabled faster speeds, but again their implementation largely merely borrowed concepts from \cite{Felzenszwalb2006} and used a GPU.  They did improve accuracy results, however, by incorporating Yoon \& Kweon's \cite{Yoon2005} adaptive support-weight approach as a post-processing step, with minimal extra computational requirements.

Tan \textit{et al.} \cite{Tan2017} claim that their fully-connected \gls{bp} method is highly-amenable to parallelisation, suggesting it could be implemented to run in real-time, but they do not appear to have done so themselves.

% \subsection{Semi-global Matching}

\subsection{Concurrent Propagation}

\cite{Gong2015,Gong2013a}

\section{Formal Models of Concurrent Computation}
Perhaps the earliest (or at least, the earliest that is still widely known) model of concurrent computation is the Petri Net.

\cite{Varela2013}

\subsection{\glsentrylong{csp} \& Pi Calculus}
\gls{csp} is a `process algebra' and abstract model of concurrent computation put forward by Hoare \cite{Hoare1985,Roscoe2011}.  A typical sequential computation is represented by a `process'.  Processes' ``behaviour is described in terms of the occurrence and availability of abstract entities called \textit{events}'' \cite[p.~478]{Roscoe2011}.  Should more than one event be available simultaneously for a given process, then one will be chosen non-deterministically.  This choice is internal to the process, and not influenced by or visible to any other process.  %E.g. for a situation where there is one possible event \(a\) at a given point in time, the process \(P\) will choose that event and then proceed according to the result, written as: \[ a \rightarrow P(a) \]

Concurrency is introduced by the existence of multiple processes.  In general, the processes evolve independently, responding to events as they come.  Should a particular event appear in the alphabet of multiple processes, however, then all processes \emph{must} choose to participate in that event at the same time.  Should all processes involved make such a choice, they engage in a synchronous multi-way atomic synchronisation (hence `communicating').  \gls{csp} has provided significant inspiration for concurrency design in a number of programming languages, including notably Ada [ref], Occam [ref], Google's Go [ref] and \gls{cml} \cite{Reppy2011}.

% Milner had earlier created the Calculus of Communicating Systems [ref] as a relatively early attempt to model concurrent computing.  

Milner appreciated \gls{csp}, which advanced concurrent models by explicitly incorporating \emph{synchronised} interaction, something Milner's earlier Calculus for Communicating Systems [ref] had lacked.  Milner still regarded \gls{csp} as incomplete, however, in that it had no support for the concept of `mobility' -- i.e. the ability of the system to reconfigure itself during operation.  Pi Calculus was created as an attempt to build upon those earlier systems but present a complete calculus of concurrent computation in much the same way that Lambda Calculus [ref] is a complete calculus for sequential computation \cite{Milner1993}.\footnote{Milner also pointed out that sequential computation is in fact a special case of concurrent computation.}

\subsection{\label{subsec:actors}Actors}
The Actor model was introduced by Hewitt \cite{Agha1986}.  Much like \gls{csp} \& its cousins, the Actor model is entirely based around the concept of separated, sequential but communicating processes which exchange messages.  Again, the processes make decisions and proceed based on their communications.  A key difference, however, is that in the Actor model the message exchanges are \emph{a}synchronous.  Each actor has its own `mailbox', and may send messages to other actors so long as it knows their name (which is equivalent for this purpose to a concept of an address for the actor), \emph{but does not wait at all for a response before proceeding}.

The Actor model is a popular one for concurrent programming, possibly owing to its intuitive concept.  The fact that communication is asynchronous makes Actors much more suitable for modelling distributed systems without shared memory than \gls{csp} or similar -- Actors can send messages and proceed without (necessarily) needing to wait for a response, instead continuing to process based on the messages they themselves have received.  By contrast, a system with synchronous communication would have prohibitive time costs, given the relative slow speed of typical links between distributed computers as compared to their capacity for local processing.  Many Actor systems have been implemented for different programming languages [refs], and in fact it is at the very core, and perhaps largely responsible for the success, of Erlang [ref].

\subsection{Join Calculus}


\subsection{Others}

Gorlatch \cite{Gorlatch2004} argued against basic message passing, decrying it as an unnecessary and unhelpful complication and favouring `collective' operations instead.  This criticism focused upon \gls{mpi} as it was at the time, however, and made no reference to either Actors or \gls{csp}.

\subsection{P~systems/Membrane Computing}
P~systems, also known as Membrane Computing (the two terms are generally used interchangeably), is a bio-inspired model of computing created by Păun in the late 1990s \cite{tPaun98a,Paun2000}, originally conceived of by considering the process of chemical reactions and exchanges that occur inside living biological cells and the membranes within and regarding this process as a form of computation.

Păun describes Membrane Computing \cite[p.~VII]{Paun2002} as:
\begin{quote}
Membrane computing is a branch of natural computing which abstracts from
the structure and the functioning of living cells. In the basic model, the membrane
systems - also called P systems - are distributed parallel computing
devices, processing multisets of objects, synchronously, in the compartments
delimited by a membrane structure. The objects, which correspond to chemicals
evolving in the compartments of a cell, can also pass through membranes.
The membranes form a hierarchical structure - they can be dissolved, divided,
created, and their permeability can be modified. A sequence of transitions between
configurations of a system forms a computation. The result of a halting
computation is the number of objects present at the end of the computation
in a specified membrane, called the output membrane. The objects can also
have a structure of their own that can be described by strings over a given
alphabet of basic molecules - then the result of a computation is a set of
strings.
\end{quote}

Membrane Computing works analogously to a typical modern electronic computer, in that the system stores data, and processes and updates those data based on a predefined program, with a view to arriving at a computable answer based on the starting state and any inputs to the system \cite{Paun2002,Paun2010b}.  In the case of P~systems, the data are multisets of symbols, representing various chemicals and their quantities.  These are found inside one or more cells, based on real biological cells, which (to a certain extent at least) form a hybrid between main memory and the processing units of a computer.  The instructions of the program itself are provided by rules, which specify transformations of objects and interactions with the surrounding environment and other membranes or cells.

There are now, broadly, three main families of P~systems variants:  Cell-like, Tissue-like \cite{tMaPaPaRo01a,Martin-Vide2003} and Spiking Neural \cite{Ionescu2006}.\footnote{Other variants have been created, but most are used infrequently, if ever.  Apart from cP~systems, described in \autoref{subsec:cpsys}, this work will not address them.}  Cell-like is the original, and sees objects compartmented into `membranes', which are arranged in a graphical tree structure with the outermost membrane (separating the cell from its environment) as the root of the tree.  In most variants, objects can evolve inside a membrane, but also be communicated between membranes (and the environment).  Furthermore, membranes can divide or dissolve themselves, and may have one or more special properties, such as `polarization'.

Conversely, Tissue-like and Spiking Neural P~systems both arrange their computing compartments, named `cells' or `neurons' respectively, as nodes in arbitrary digraphs, with the edges between them representing connecting channels.  Whereas Cell-like P~systems emphasise the evolution of multisets of objects inside compartments, Tissue-like and Spiking Neural P~systems emphasise communication between separate cells/neurons, and many Tissue-like variants do not include any capacity for internal evolution inside cells -- if new objects are required, they are imported via communication with the environment, which is considered to possess an unlimited number of all objects, but has no rules of its own.

While Tissue-like P~systems have arbitrary alphabets, only one object is used in Spiking Neural P~systems, the `spike'.  This means that Tissue-like systems are frequently much like Cell-like ones in that they have custom objects for each purpose, with the key difference (usually) being in how the \glspl{prox} are arranged relative to each other and the choice between the two motivated primarily by which one seems like a better fit to the computation to be modelled.  Conversely, Spiking Neural P~systems force everything to be represented through the use of differing quantities of the spike, kept in different neurons.  This means that it can be more complex to model certain problems, but also arguably means that Spiking Neural P~systems are, \textit{prima facie}, closer to Lambda Calculus [ref] and Church Numerals [ref], as well as Register Machines[ref] (and indeed Register Machines have been simulated with Spiking Neural P~systems[ref]).  All three approaches have been proven Turing-universal though[ref], so all three should be capable of expressing the same computations in different forms.

Arguably, the most notable and important aspects of P~systems models are that they:  i) Generally have no space limit.  That is, they contain an arbitrary number of cells, objects and membranes;  ii) Across all cells and membranes, all rules that can be applied are applied, as many times as possible given the current number of objects available.  These two features mean that P~systems have unbounded space and processing capacity, which can be used to solve traditionally computationally-difficult problems relatively quickly \cite{Sosik2003,Jimenez2003,Paun1999a}.  Most of these solutions, however, rely on trading time complexity for space complexity.  While this works in the theoretical framework, electronic simulations of the systems do not have access to unlimited instantaneous memory space, meaning many of the fast solutions are impractical with current real-world computers [refs].

Membrane Computing is not just a theoretical model, with limited practical use.  Besides Image Processing \& Computer Vision (see \autoref{subsec:imgprocpsys}), P~systems variants have been applied to a range of fields, from power grid management to robotic control systems \cite{Zhang2017}.  [P-Lingua and simulation systems, e.g. MeCoSim]

\subsubsection{\label{subsec:cpsys}cP~systems}
\cite{Nicolescu2014b,Nicolescu2017}

cP~systems is another variant of P~systems, developed by Nicolescu and collaborators in the early 2010s [ref].  It is largely based on Cell-like P~systems, and can be seen, to some extent at least, as a higher-level abstraction over it \cite{Nicolescu2018}.  It can also incorporate elements of Tissue P~systems, however, in that it includes concepts of channels and message passing between cells \cite{Henderson2019}.  Nicolescu, Ipate \& Wu demonstrated that not only is cP~systems capable of performing the same tasks as other P~systems variants, but also can be used fairly cleanly to model typical computer programs \cite{Nicolescu2014a}.

The major advantages of cP~systems over traditional Cell-like systems is a simplification in the specification of complete systems to solve a given problem.  Cell-like (as well as Tissue-like and Spiking Neural) systems typically require the definition of a family of rulesets customised to the specific instance of the problem at hand, whereas cP~systems usually requires only the definition of a fixed (usually much shorter) set of rules that cover all possible instances.  Only inputs to the system need vary to solve different instances of the problem, e.g. in \cite{Cooper2019} only five fixed rules were needed to solve any instance of the Travelling Salesman Problem, with only customisation of the input objects (in this case, elements describing the nodes and edges of the graph) required.



\subsubsection{\label{subsec:imgprocpsys}Image Processing and Computer Vision in P~systems}
\cite{Zhang2012}

Perhaps owing to the unbounded potential space and parallelism of P~systems, combined with the embarrassingly parallel nature of many tasks in Image Processing \& Computer Vision, the latter has proved to be fertile ground for the former, although not every publication puts its model to the test with a computerised simulation, or if it does, the authors may only provide scant details \cite{Diaz-Pernil2019}.  

Christinal, Díaz-Pernil \& Real \cite{Christinal2011} described a family of Tissue-like P~systems to perform region-based segmentation of both 2D and 3D images.  Despite their family of systems requiring only two cells, it also needed custom rule sets based on the size of the images as well as the number of colours present, with a number of rules per set proportional to the same measurements.  The paper showed the results of simulating the system, but provides no details on performance.

Díaz-Pernil \textit{et al.} \cite{Diaz-Pernil2013} commented that ``... commonly [a] parallel algorithm needs to be re-designed with only slight references to the [sequential original].  ... the design of a new parallel implementation not inspired by the sequential one allows ... the proposal of new creative solutions.''  They then demonstrated this fact by designing a new edge detection and segmentation algorithm named `A Graphical P (AGP) segmentator', inspired by the Sobel operator [ref] and using the segmentation method from \cite{Christinal2011}, which they modelled in Tissue-like P~systems.  The authors implemented their new algorithm on a \gls{gpu} and compared it with an implementation of the 3x3 and 5x5 Sobel operators, finding that theirs had near-identical runtimes but superior edge detection capabilities.

Díaz-Pernil \textit{et al.} \cite{Diaz-Pernil2013a} further explored modelling classic image processing techniques by implementing Guo \& Hall's binary image skeletonisation technique \cite{Guo1989} with Spiking Neural P~systems.  The overall system's rules templates are reasonably simple, but include references to a set \(DEL\) (used as a lookup to determine whether a cell should turn white or stay black) which does not appear to be modelled inside the system, meaning that it is not self-contained.  The authors simulated this system on a \gls{gpu}, but found that their implementation was upwards of twice as slow as another pre-existing implementation.  Confusingly, however, they state that one of the reasons for this is ``that the use of an alphabet with only one object, the spike \(a\), does not fit in the GPU architecture''.  This statement is difficult to understand, given that spikes can easily be represented as simple integers.  The authors also commented that the synchronous nature of the model is unrealistic, and imposing a global clock upon the system can be problematic.

Nicolescu \cite{Nicolescu2014} alternatively applied cP~systems to image skeletonisation based on Guo \& Hall's technique \cite{Guo1989}, presenting three forms of a solution: Synchronous versions that use multiple or a single cell (essentially the latter replicates the former via the use of sub-membranes), and an asynchronous multi-cell version.  The asynchronous version no longer assumes that all messages are passed between cells simultaneously and instantaneously, compensating for this by increasing the number of messages used.  This form, while arguably more realstic to modern computers, requires a greater message complexity. A prospective Actor-model-based (see \autoref{subsec:actors}) simplified implementation using \fsharp{}'s \texttt{MailboxProcessor} \cite[ch.~11]{Syme2015a} was presented also, but no results from running it were reported.

Nicolescu \cite{Nicolescu2015a} further applied cP~systems to seeded region growing of grayscale images.  The described system used a two-level approach, based on the `Structured Grid Dwarf' of the 13 Berkeley Dwarves \cite{Asanovic2006}, where the image was divided into rectangular blocks of multiple pixels.  Each block was modelled with a single cell, inter-block processing was carried out via message passing, and intra-block processing was performed by typical object evolution.  It was again suggested that this would fit well to the Actor model.

Díaz-Pernil \textit{et al.} \cite{Diaz-Pernil2016} built upon their AGP segmentator algorithm to create a version that works with RGB images rather than grayscale and applied it to a common medical Computer Vision task, isolating the `optic disc' in images of the inner eye.  With this they used the skeletonisation algorithm from \cite{Diaz-Pernil2013a} and a number of other steps not based on P~systems to produce a complete imaging pipeline.  The authors implemented this on a \gls{gpu}, and found that their system was both more accurate and faster than previous systems.

Most directly relevant to the current work are \cite{GimelFarb2013a,Gimelfarb2011,Nicolescu2014b}, which model Dynamic Programming \gls{sm} in P~systems, and indeed saw the genesis of \gls{cp}.  

\subsubsection{P~systems on \glsentrylongpl{gpu}}
In many instances, a P~systems model for a problem involves many separate small elements processing their data separately, and perhaps updating each other's state at the end of a step.  Given that this sounds remarkably close to the Single-Instruction Multiple-Thread[ref] nature of modern \gls{gpgpu}, it is no surprise that there has been much work put into simulating P~systems on \glspl{gpu}.

\cite{Cecilia2010,Cecilia2010a,Cecilia2013,Macias-Ramos2015,Martinez-Del-Amor2015,Martinez-Del-Amor2013a,Maroosi2014,Maroosi2014a}

\section{Concurrent ML \& related}

% \section{Measures of `code quality'}
% Measuring code quality is a small subset of a much larger topic, often called, amongst other names, Software Quality Assurance or Software Quality Engineering.  The broader field is largely concerned with ensuring that software meets quality requirements, where quality may be defined in a multitude of ways.  Much of the field is concerned with managing processes for professional developers and organisations which employ them, as opposed to 

% \subsection{Why should one care about code quality in this context?}
% Stereo matching algorithms\footnote{For the purposes of this subsection specifically, unless stated otherwise, `algorithm' shall be used in the sense of referring to particular implementations of the theoretical algorithms} tend to be relatively self-contained -- one can simply treat them as black boxes whereby a pair (or more) of rectified images are passed in, and a disparity map returned.  One might wonder why exactly there is any reason to consider concepts of code quality in this context.

% A number of reasons to consider it are listed below:
% \begin{itemize}
% \item \textbf{Maintainability} -- Inevitably, changes in the environment a particular algorithm is to run in will occur over time.  Eventually, with enough of these changes accumulated, the algorithm will need to be updated in order to ensure that it still continues to run and produce accurate results.  Better quality code can make such changes significantly easier.
% \item \textbf{Changes in maintainers} -- In almost all circumstances outside of academic research, algorithms are not written once and then never looked at again.  Instead, they will be revised at some point as part of necessary maintenance.  It is highly likely that, sooner or later, the person(s) tasked with maintaining the algorithm will change.  While it is plausible that both the old and new maintainers might understand the theoretical underpinnings of an algorithm (though that is by no means guaranteed in general), one inevitably needs time to learn how to work within a new codebase.  Better quality code will make the transition between maintainers significantly easier, and likely lead to the new maintainer becoming productive quicker.
% \item \textbf{Ease of experimentation} -- better code (especially more well factored out code) means that one can probably experiment with different approaches more easily.
% \item \textbf{Upgradability} -- Related to maintainability,\footnote{Both maintainability and upgradability are closely linked to the concept of `technical debt', which is [insert definition here].} it is likely that with new developments in hardware, operating systems, or stereo matching theory, new, more efficient, ways of implementing algorithms may become possible.  An uncoordinated mess of spaghetti code could prove extremely difficult to modify appropriately, with the result that attempts to upgrade the operation of an algorithm may fail, or at least not achieve the success aimed for.  Better quality code, however, will in most cases make such upgrades both quicker to perform, and likely to succeed.
% \end{itemize}

% \subsection{Dimensions of quality}
% See that standard ISO/IEC 25010:2011.  

% In this work, the focus will be entirely upon metrics that can be computed via static analysis.  The are multiple reasons for that.  For one, much of the measurement of software quality is performed over a relatively long timescale, whereby measures such as the number of bugs reported, customer satisfaction, or some other concept of how fit for purpose the final product was, are used.  Secondly, the measurements are typically taken at a much larger scale than that of individual algorithms, which in this case are loosely analogous to individual function calls.  Thirdly, a considerable number the metrics are qualitative in nature, requiring judgement calls on the part of a reviewer to determine the quality of the code.

% Limiting the focus in this work purely to static measures will provide quantitative data that are as close to objective and comparable as can reasonably be achieved, and which can be calculated purely on the basis of the available source code, without needing to run it at all.  Note that even using static metrics may not provide full objectivity, as there will inevitably differences across programming languages that can be taken into account and controlled for entirely.

% \subsection{Metrics chosen}
% \begin{itemize}
% \item SLoC
% \item Cyclomatic complexity
% \item Ease of reading measures?
% \item Nested expressions?
% \item From Alec et al.'s paper on the Santa Claus problem - ratio of uncompressed to compressed code files - the less it can be compressed, the more expressive the current code would appear to be. "The raw/compressed ratio is intuitively a measure of the expressiveness: the lower the ratio, the better: less noise/redundancy and higher information density."
% \item What else?
% \end{itemize}

\chapter{Choice of model, programming language and library}

\section{Selection of theoretical model}
\gls{csp} vs actors vs join calculus vs pi calculus etc.  Why I think one of them is the right choice

Message passing as discussed here is different to the message passing of the Message Passing Interface (how?)

\begin{itemize}
    \item Synchronous vs asynchronous message passing?
    \item Actors
    \item Join-calculus
    \item Pi calculus
    \item Communicating Sequential Processes
    \item Shared memory only
    \item Reactive approach?
\end{itemize}

\subsection{Why not Actors?}
Of the many theoretical models for concurrency, perhaps the two most rooted in concepts of individual \glspl{prox} passing messages between themselves are \gls{csp} and Actors.  On paper, the key difference between them is that \gls{csp} involves \glspl{prox} \emph{synchronously} exchanging messages through channels, which need not be associated with a particular \gls{prox}, whereas Actors \emph{asynchronously} exchange messages which are placed directly in their own mailbox.

% In general, asynchronicity is considered preferable, because it permits the \glspl{prox} to process at their own pace, without having to wait for another 

In practice, \gls{cml} has a major advantage in terms of resource requirements.  Actor implementations usually can support perhaps \num{100 000} or so individual actors per \si{\gibi\byte} of memory, but \gls{cml} implementations can stretch into the millions \cite{Butcher2014}.  This arises because there is no need to provide a mailbox to and store an unbounded number of messages for each \gls{prox}.  Channels and events do take a small amount of memory, but this has been found to be on the order of tens of bytes \cite{Reppy1991}.  Furthermore, it has been shown that, to a certain extent, the two can be done away with when operating in a parallel fashion \cite{Reppy2007a}.

Move the above into the introduction perhaps?

\section{Requirements}
% Requirements:
% \begin{itemize}
%     \item Must support \gls{csp}/Pi Calculus/etc. approach.  In particular, needs to enable multi-threaded, parallel message passing.  Should have some concept of channels or rendezvous, but needs to support more than just that.  I.e. needs to support the actual \gls{csp} approach/\gls{cml} features.  And needs to be highly scalable.
%     \item Needs to support tail call optimisation/tail recursion.  Or at least has a strong and clear path to trampolining in order to simulate TCO.
%     \item The message passing must be \emph{extremely} lightweight.
%     \item \emph{Must} have good linear algebra capabilities somehow
%     \item Comes with some ability to work with common image formats -- this is more a convenience than strictly necessary, since under normal circumstances these images will be coming through just as numeric arrays or similar, but it'll make testing things out a lot easier.
%     \item On the above note, not only should it be able to do IO, but it would be good if the library has things like histogram equalisation built in.  Not actually necessary though, since I could always write a quick command line tool to do that for me.
%     \item Preferably, has some degree of FFI/interop with C
%     \item Ideally includes some capacity to support GPU programming with CUDA, OpenCL, OpenGL etc. or at least has good C/C++ interop so that one can call libraries written in those languages (OpenCL 1.2 support is strongly to be preferred, as that's the best that the RPi can support, with the VC4CL library -- this may not be true later on, with the release of the RPi 4 Model B, but certainly VC4CL won't be there yet)
%     \item On the same line as above, the language really needs to make some form of vectorization available (better if it can do some automatically, but I probably will want the ability to control it manually also)
%     \item Probably doesn't necessarily need terribly strong concepts of typing to be honest, since almost all values that will be used will be numeric -- much more important would be memory safety
%     \item Still somewhat under active development/being supported
%     \item \emph{Preferably} not just someone's toy research language
%     \item If I'm (roughly) going to be treating every pixel as a separate computing unit, then the individual units must be extremely lightweight also.  Instantiation time isn't too big a deal, as that can be considered to be amortised, but it needs to be possible to manage a boatload of them.  Probably an argument against actors.
%     \item Must be able to be used across Linux implementations, as the embedded systems pretty much all use some form of Linux.
%     \item Needs to be able to produce native, \emph{dependency-free and standalone} executables \emph{for both x86-64 and ARM} - anything using a big VM will probably be too heavyweight (not definite though)
%     \item It would be better if there are already high quality implementations of stereo matching algorithms available for the given language, though that is not necessarily the be-all and end-all.
%     \item Also needs to have some degree of viable support for optimisation techniques such as gradient descent/Levenberg-Marquardt/RANSAC
% \end{itemize}

\subsection{Strict Requirements}
If a possible option does not meet even one of the below, it will not be considered further.

\begin{itemize}
    \item Suitable support for \gls{csp}/\gls{cml} style concurrency.  At a minimum, not only must there be capacity to send and receive over channels, but furthermore selecting over channels is \emph{mandatory}.  Simply having channels and the ability to select over them is the bare minimum required, but it does not make something a full \gls{cml} implementation.  The other combinators (e.g. \texttt{wrap} \& \texttt{guard}), and the ability to compose them, must be provided for it to be a full \gls{cml} implementation.  The message passing mechanics themselves must be lightweight, in terms of memory requirements and the number of instructions that need to be executed.
    \item Support of last-call optimisation, or an equivalent capability (this somewhat is implied by the above).
    \item Currently supported, or at least receiving regular maintenance
    \item Almost goes without saying, but includes some sort of green threads/fibers system, so that all the concurrent processes can be scheduled onto the processors efficiently.  This system must operate across multiple cores when they are available.
    \item Support for data-parallel programming, such as CPU vector instructions and/or OpenCL, in some capacity (either explicit or compiler-done).  The ability to use other frameworks and standards such as OpenMP, OpenACC or CUDA would be a bonus.
    \item Available for `standard' Linux distributions, such as those derived from Debian (e.g. Ubuntu, Mint) or Red Hat Enterprise Linux (e.g. Fedora, CentOS).  \emph{Most} languages target/support Linux in general, and thus this shouldn't be an issue typically.
\end{itemize}

\subsection{Desirable Qualities}
The absence of one or more of the below requirements will not necessarily disqualify an option, but having more of them will be a benefit.

\begin{itemize}
    \item Support for working with common image formats.  If need be, it should be relatively trivial to open an image elsewhere and convert it to something which I could parse relatively simply, like a .ppm/.pgm file, or some sort of binary file that is just all the pixel values jammed together.
    \item Support for common standard image processing routines.  Again, not necessarily needed, but it would save me the effort of implementing them if they turn out to be required.
    \item Support for compilation to native executable binaries, rather than relying on a virtual machine.
    \item High-quality implementations of stereo matching algorithms already available.  This is preferred simply so that it provides an easy comparison between what is implemented for this work against what someone else has created (probably using more of a traditional imperative style).  Using the same language for the comparisons helps to remove uncontrolled variables that could be confounding factors.
    \item Easy inter-operation with C, or equivalent (e.g. a Foreign Function Interface to C).
    \item An integrated or otherwise easy-to-use benchmarking system.
\end{itemize}

\section{Possible options}
Languages and their relevant libraries

% See also \url{https://github.com/kevin-chalmers/cpa-lang-shootout} and \url{www.teigfam.net/oyvind/home/technology/135-towards-a-taxonomy-of-csp-based-systems/} and \url{https://arrayfire.com/}\footnote{On top of ArrayFire, there's also Thrust \url{https://thrust.github.io/}  and ConcurrencyKit \url{http://concurrencykit.org/} and LibCDS \url{https://github.com/khizmax/libcds}}.  ArrayFire is the only one that appears to have bindings for any other languages besides C/C++, however.  Also, Furthark \url{https://futhark-lang.org/}

% \begin{itemize}
% \item Concurrent ML/\gls{csp}/Pi calculus
% \begin{itemize}
    % \item Standard ML -- it seems that MLTon is still under development, and it provides a recent implementation of Standard ML and \gls{cml}.\footnote{see \url{http://mlton.org/ConcurrentML}}
    % \item \sout{Concurrent C?  Appears to be loooong dead.}
    % \item Concurrent Haskell -- It's not entirely clear what the status of \gls{cml}/\gls{csp} style stuff is in Haskell.  Most up-to-date part seems to be Communicating Haskell Processes,\footnote{\url{https://hackage.haskell.org/package/chp}} though that appears to have been unmaintained for a few years.
    % \item \sout{Occam?  Looks like Occam-Pi is the only version that's even close to current, and it doesn't look like that's under active development}\footnote{\url{https://github.com/concurrency/kroc}}
    % \item \sout{F\# with Hopac (Hopac seems to be fairly well no longer under development) and CoreRT.\footnote{For useful CoreRT references, see \url{https://github.com/dotnet/corert}, \url{https://github.com/dotnet/corert/tree/master/samples/HelloWorld} and \url{https://github.com/FoggyFinder/FSharpCoreRtTest}}}  Could possibly use System.Threading.Channels myself, it's apparently super-fast.\footnote{See e.g. \url{https://ndportmann.com/system-threading-channels/}}  Could perhaps alternatively use the TPL Dataflow library (F\# utility wrappers here:  \url{https://github.com/TheAngryByrd/FSharp.TPLDataflow})
    % \item Fibers in Guile Scheme
    % \item OCaml with Events module
    % \item SML/NJ is still going apparently\footnote{\url{https://www.smlnj.org}} although I'm not sure if \gls{cml} is up to date enough to work with it.  Bigger problem is that I \emph{think} it doesn't support parallelism at all.  There's also PolyML (\url{https://www.polyml.org/}) but it doesn't appear to include CML at all.
    % \item Clojure with core.async using an AOT compiler or lightweight JVM.  GraalVM\footnote{\url{https://www.graalvm.org/}} appears to offer the former, while OpenJ9\footnote{\url{https://www.eclipse.org/openj9/index.html}} is apparently the latter, but as of writing it appears that neither of them supports targeting ARM.  See \url{https://neanderthal.uncomplicate.org/} and \url{} apparently for fast linear algebra operations.  There's also \url{http://docs.paralleluniverse.co/quasar/}, which apparently basically does Go fibers in Java - but it hasn't been updated in a while, and it looks like the company behind it might have gone out of business.
    % \item Rust\footnote{see \url{https://doc.rust-lang.org/book/ch16-02-message-passing.html}} with Crossbeam and its subcrates -- see also Tokio and Actix, as well as Desync\footnote{\url{https://github.com/logicalshift/desync/}}, Grease\footnote{\url{https://github.com/cambridgeconsultants/grease}} and Threadpool\footnote{\url{https://github.com/rust-threadpool/rust-threadpool} and see also \url{https://gsquire.github.io/static/post/a-rusty-go-at-channels/}} plus \url{https://gluon-lang.org/} (Gluon is a Haskell-y version of Rust, essentially).  Also \url{https://crates.io/crates/single_value_channel/1.2.1}.
    % \item Manticore is kinda sorta a successor to SML/NJ apparently, with a strong focus on \gls{cml}-style parallelism\footnote{\url{http://manticore.cs.uchicago.edu/}}
    % \item Go?  \sout{Apparently Crystal has a very similar \gls{csp}-derived built-in basis for Concurrency.}\footnote{While Crystal has similar built-in support for channels like Go, it apparently doesn't yet do parallelism - see near the start of \url{https://crystal-lang.org/reference/guides/concurrency.html}}  \url{https://www.gonum.org/}  Go's support for SIMD seems to be kinda clunky:  \url{https://medium.com/@c_bata_/optimizing-go-by-avx2-using-auto-vectorization-in-llvm-118f7b366969}, \url{https://www.reddit.com/r/golang/comments/ep4vyp/question_how_do_you_implement_simd_in_go/}, \url{https://goroutines.com/asm}, \url{https://golang.org/doc/asm}, \url{https://github.com/minio/simdjson-go}, \url{https://github.com/golang/crypto/tree/master/blake2b}, \url{https://www.reddit.com/r/golang/comments/42zppi/is_there_any_way_to_use_simd_intrinsics_in_gccgo/}.
    % \item Is there anything in C++?  (Couldn't find anything when I looked the other day)\footnote{but see \url{https://docs.microsoft.com/en-us/cpp/parallel/concrt/concurrency-runtime?view=vs-2017} which is Windows only I think, also \url{https://www.cs.kent.ac.uk/projects/ofa/c++csp/} and \url{https://stackoverflow.com/questions/218786/concurrent-programming-c}}.  Also, there are a number of Boost libraries which come close, but none of which do precisely what I need:  \url{https://www.boost.org/doc/libs/?view=category_concurrent}.  MicroC++ (\url{https://plg.uwaterloo.ca/~usystem/uC++.html}) is similar.  Also, Stackless Coroutines introduced channels, but it looks like that is a dead experiment:  \url{https://github.com/jbandela/stackless_coroutine/tree/channel_dev} \& \url{https://github.com/CppCon/CppCon2016/blob/master/Presentations/Channels%20-%20An%20Alternative%20to%20Callbacks%20and%20Futures/Channels%20-%20An%20Alternative%20to%20Callbacks%20and%20Futures%20-%20John%20Bandela%20-%20CppCon%202016.pdf}  Actually, Boost:Fiber \emph{might} provide what I need.  See its channels, and \texttt{when_any} construct.
    % \item \sout{Smalltalk -> Objective-C \& Parallax (all of these have been out of active development for too long, but they probably will make good references)}\footnote{see \UrlBreaks{https://stackoverflow.com/questions/6145421/what-other-programming-languages-have-a-smalltalk-like-message-passing-syntax}}
    % \item Pharo(?)\footnote{\url{http://www.pharo.org/web}} -- a modern version of Smalltalk, apparently.  Also, Squeak.\footnote{\url{https://squeak.org}}
    % \item Ada (something about rendezvous).  Supposedly has a proper real-time focus, might be worth looking at.
    % \item Kotlin.  Has channels, and apparently an (experimental) implementation of selection:  \url{https://kotlinlang.org/docs/reference/coroutines/channels.html} ... ``Only single-threaded code (JS-style) on Kotlin/Native is currently supported.'' (\url{https://github.com/Kotlin/kotlinx.coroutines})
% \end{itemize}
% \item Join Calculus
% \begin{itemize}
%     \item F\# with Joinads
%     \item JOCaml
%     \item Scala with Chymist
%     \item \sout{Polyphonic C\#/C\(\omega\)}
% \end{itemize}
% \item Actors
% \begin{itemize}
%     \item Erlang / Elixir
%     \item Halide with message-passing?  There's \url{https://doi.org/10.1016/j.sysarc.2017.10.005}, \url{https://doi.org/10.1007/s11265-017-1283-1}, and Aaron Epstein's Masters Thesis at MIT, ``A Distributed Backend for Halide'', but with regards to message passing they all seem to focus on MPI.%  I couldn't find anything on \gls{cml} style in Halide.
%     \item \sout{Don't forget Pony...  (almost certainly not stable enough at this point)}
%     \item Rust with Actix
% \end{itemize}
% \item Julia?\footnote{\url{https://docs.julialang.org/en/v1/manual/parallel-computing/index.html} \& \url{https://docs.julialang.org/en/v1/manual/control-flow/\#man-tasks-1}} -- it is supposed to be very close to mathematical notation, so that's a big plus.  Can't find anything suggesting that there's a Concurrent ML/\gls{csp}/Pi calculus type thingy out there yet, though they do have lightweight threads and channels.  Plus, parallelism is apparently not really a big thing in it yet.
% \item What of the LMAX Disruptor approach?\footnote{\url{https://github.com/LMAX-Exchange/disruptor}} -- looks like that would have me working on the JVM, so probably Java/Scala/Kotlin/Ceylon/Clojure/probably something else is out there too.  There is a .NET port of it too.\footnote{\url{https://github.com/disruptor-net/Disruptor-net}}  Definitely looks like it would be worth investigating, but it doesn't follow the \gls{csp} model, so it is out-of-scope for this particular work.  Also \url{https://github.com/lthibault/turbine}.
% \item Scala with Chymist \sout{(there was also JCSP (\url{https://www.cs.kent.ac.uk/projects/ofa/jcsp/}), which seems to be long dead now, but see Communicating Scala Objects, though I can't find an actual implementation of that available anywhere)}
% \item \sout{SCOOP (Betrand Meyer) \& Eiffel (too OO, not really close enough to \gls{csp} it looks like)}
% \item \sout{Does ATS \cite{Shi2013} include anything? (couldn't find anything)}  Or D?\footnote{relevant: \url{http://www.informit.com/articles/article.aspx?p=1609144} and \url{https://wiki.dlang.org/Go_to_D}}  Or Nim?  \sout{Idris?  C++?}\footnote{There's C++CSP, but despite the most recent paper apparently being published in 2016, I couldn't find anywhere that actually hosted a usable version.  Could only find \url{https://www.cs.kent.ac.uk/projects/ofa/c++csp/doc/index.html} \& \url{https://github.com/olahol/cpp-csp}, neither of which are complete or recently-updated.}  Looks like I should check C++ Concurrency in Action, 2nd Edition -- it'll probably cover this if anything does (update: it doesn't).
% \item Further C++ CSP libraries:  SObjectizer\footnote{\url{https://stiffstream.com/en/products/sobjectizer.html}}, libmill\footnote{\url{http://libmill.org/index.html}} \& libdill\footnote{\url{http://libdill.org/index.html}}, LibProxC++\footnote{\url{https://github.com/edvardsp/libproxcplusplus} (also LibProxC \url{https://github.com/edvardsp/libproxc})}.  High Performance ParallelX\footnote{\url{http://stellar-group.org/libraries/hpx/}} claims to be a replacement to/improvement over \gls{csp} (see \url{https://stellar-group.github.io/hpx/docs/sphinx/tags/1.4.0/html/why_hpx.html#what-is-hpx}).
% \item \sout{How can strong typing be used beneficially? (if at all?)}
% \item \sout{Probably don't need transactional memory (?)}
% \item \sout{Lock-free is better than locking, but how to achieve?  Is that a relevant consideration here?  In theory at least, should be able to leave that up to the implementation I'm using.}
% \item \sout{Can F*, Adga, Coq or similar be of any use here? (not clear how)}
% \item \sout{Reppy has recently been working on Diderot, ``a Parallel Domain-specific Language for image analysis and visualization'', but it seems like that probably isn't what is needed here.}  Diderot doesn't seem to be ready for others to play with yet.
% \item Ferret\footnote{\url{https://ferret-lang.org}} appears to be a lot of what is needed, but it doesn't have its own channel/message-passing implementation, and I don't think it interoperates with Clojure.  Not entirely clear how one uses multi-threading with it, except possibly just calling out to C++.
% \item Single-Assignment C.\footnote{\url{http://www.sac-home.org/} \& \url{https://github.com/SacBase}}  It looks like development on it has more-or-less stopped in the past couple of years, but it \emph{might} still be suitable for my purposes.  It appears to come with some built-in support for image processing (e.g. parts of the standard lib directed towards using .pgm files), but I'm struggling to see anything on concurrency/parallelism -- it might be the case that all of that is done implicitly, e.g. there's \url{https://github.com/SacBase/NASParallelBenchmarks}, but I couldn't see any explicit handling of parallel constructs in it.
% \item Checked Dart, but it seems to be focused on making apps with responsive UIs.  It has a concept called `isolates' which seem to be moderately similar to fibers, but they seemingly are asynchronous only (Wikipedia explicitly compares it to Erlang).
% \item Checked Io, which is another Smalltalk-esque language, but it apparently only does asynchronous.  Not clear that it does parallelism across multiple cores, either.
% \item Checked X10, which doesn't seem to provide any sort of \gls{csp} style support, and appears to be focused primarily at HPC.
% \item Even the latest version of OpenMP doesn't seem to have anything \gls{csp}-ish.
% \item Looks like there \emph{might} be some chance to use C# for some of this stuff:  \url{https://github.com/DragonSpit/HPCsharp}, \url{https://github.com/chrisa23/fibrous}, possibly \url{https://github.com/domn1995/Marathon}.  \url{https://linksplatform.github.io/Hardware.Cpu/}, \url{https://github.com/jackmott/LinqFaster}, \url{https://gitlab.com/pomma89/object-pool}.
% \item Other ones that could be mentioned include Mozart \url{https://github.com/mozart/mozart2}, Red \url{https://www.red-lang.org/}, P \url{https://github.com/p-org/P}, OForth \url{https://www.oforth.com/}, Esterel \url{http://www.esterel.org/}.
% \end{itemize}

% Other Rust resources:  Bastion \url{https://github.com/bastion-rs/bastion} (see also it's sub-libraries Bastion Executors and LightProc); Actix \url{https://docs.rs/actix/0.9.0/actix/} (the underlying Actix system, \emph{not} Actix-Web); Greenie \url{https://github.com/playXE/greenie}, but it appears to have only just started; Rust-Executors \url{https://github.com/Bathtor/rust-executors}, though I can't work out precisely what it does.  It basically seems to say that it lets you choose between some threading runtimes; Par-Array-Init \url{https://crates.io/crates/par-array-init/0.0.5} looks like it was set up and then abandoned, but \emph{might} still be useful for my purposes; RustaCUDA \url{https://github.com/bheisler/RustaCUDA} \& Accel \url{https://gitlab.com/termoshtt/accel};  Also, Testbench \url{https://github.com/HadrienG2/testbench} and Criterion \url{https://github.com/bheisler/criterion.rs}.  Emu for OpenCL \url{https://calebwin.github.io/emu/} and Ocl \url{https://github.com/cogciprocate/ocl/tree/master}, but the latter is looking fairly abandoned.

% Other Nim resources: Memo \url{https://github.com/andreaferretti/memo}; Loop-fusion \url{https://github.com/numforge/loop-fusion}; Nim-Schedules \url{https://github.com/soasme/nim-schedules}; Shared \url{https://github.com/genotrance/shared}; Nim-Chronos \url{https://github.com/status-im/nim-chronos} (not clear that this is relevant);  Stew \url{https://github.com/status-im/nim-stew}; Stones \url{https://github.com/binhonglee/stones} (not really clear precisely what it provides); Nim-CLBlast \url{https://github.com/numforge/nim-clblast}; Nim-Optionsutils \url{https://github.com/PMunch/nim-optionsutils}; Nimterop \url{https://github.com/nimterop/nimterop}; Neo \url{https://github.com/unicredit/neo}; Nim-GLM \url{https://github.com/stavenko/nim-glm}; Memviews \url{https://github.com/ReneSac/memviews}; Nim-Curry \url{https://github.com/zer0-star/nim-curry};

%For OCaml, be sure to check \url{https://ocaml.xyz/}  (also take a look at \url{https://www.eff-lang.org/} and \url{https://github.com/kayceesrk/effects-examples})

\subsection{Ranking of of possible options}
% The ones chosen to assess \emph{must} meet all the criteria below.  If they fail even one, then they are excluded from further consideration.

% \begin{itemize}
%     \item Support for \gls{csp}/\gls{cml} style concurrency
%     \item Tail/Last call optimisation, or something equivalent
%     \item \emph{Extremely lightweight} message passing and representations of processing elements
%     \item Linear Algebra support
%     \item Support for vectorization and/or OpenCL
%     \item Maintained or under development
%     \item Usable on most Linux distributions
%     \item Can produce executable files for both x86\_64 and ARM
% \end{itemize}

The ones (that probably will be) finally chosen to assess are among:
\begin{itemize}
\item C++ with Boost:Fiber or one of the \gls{csp} libraries
% \item Clojure with core.async -- core.async seems to be a channels + selection library only
% \item Go -- base Go is channels + selection only, and there don't seem to be any full \gls{cml} libraries out there for it.
\item Guile Scheme with Fibers library -- Fibers is a full \gls{cml} library
% \item Kotlin -- Not 100\% clear.  It looks like Kotlin's coroutines go beyond just channels + selection, but if it goes to full \gls{cml}, the combinators go by different names.
% \item Manticore -- has a full \gls{cml} library
\item MLTon -- has a full \gls{cml} library
% \item Nim -- Unfortunately, Nim does not appear to have any selection over channels capability
\item OCaml with Events module -- Events is a full \gls{cml} library
% \item Pharo
\item Rust with Crossbeam crate -- Crossbeam seems to be channels + selection only
\item Racket with its Sync library
\end{itemize}

Big table assessing what features they in fact have goes here (?).  That can be used as the basis of choosing, say, the top 3-5 languages that seem like they would be best suited to what I want to do, and those ones can be further assessed.



Further distinguishing criteria are examined below.  These are not as critical, so if an option lacks one but has others strongly, it may still be the best choice.

\subsubsection{Support for working with common image formats}

\paragraph{C++}

\paragraph{Clojure}
Almost certainly, though I couldn't find a current library in a quick search.  JavaFX (\url{https://openjfx.io/}) provides something it seems.  There seem to be some basic classes in Java.AWT and Javax.ImageIO.  Seems like it is also possible to interact with OpenCV via Java bindings (best as I can tell).

\paragraph{Go}
Yes - built in.

\paragraph{Guile}
Yes -- at the very least Guile-CV \url{https://www.gnu.org/software/guile-cv/}

\paragraph{Racket}

\paragraph{Kotlin}
JavaFX (\url{https://openjfx.io/}) provides something it seems.  There seem to be some basic classes in Java.AWT and Javax.ImageIO.  Seems like it is also possible to interact with OpenCV via Java bindings (best as I can tell).  Also see JavaCV \url{https://github.com/bytedeco/javacv}, BoofCV \url{https://github.com/lessthanoptimal/BoofCV} and AlgART \url{https://algart.net/java/AlgART/}

\paragraph{Manticore}
Doubtful.

\paragraph{MLTon}
Unclear.  There are many old libraries around, but none of the ones found dealt with images.  Matthew Fluet has stated (via email) that he is unaware of any libraries, and has suggested using the C FFI to work with one of the C libraries that performs the task.

\paragraph{Nim}
Yes.  At least, partial (see e.g. \url{https://github.com/nim-lang/needed-libraries/issues/77}).  Looks like NiGui \url{https://github.com/trustable-code/NiGui/blob/master/examples/example_11_image_processing_cli.nim} includes some capability for interacting with images...  Also, Flippy \url{https://github.com/treeform/flippy}; 

\paragraph{OCaml}
Yes, e.g. Bimage \url{https://github.com/zshipko/ocaml-bimage} or CamlImages \url{https://bitbucket.org/camlspotter/camlimages/src/default/}.

\paragraph{Rust}
Yes.

\subsubsection{Support for common standard image processing routines}

\paragraph{C++}

\paragraph{Clojure}
Almost certainly, though I couldn't find a current library in a quick search.

\paragraph{Go}
Most likely - haven't found it, but given Go's popularity it seems highly likely.

\paragraph{Guile}
Yes.

\paragraph{Kotlin}
At least as much as Clojure it looks like.

\paragraph{Manticore}
Couldn't find any.

\paragraph{MLTon}
Couldn't find any.

\paragraph{Nim}
Yes: \url{https://github.com/numforge/laser}

\paragraph{OCaml}
Yes (some, at least).


\paragraph{Rust}
Yes.


\subsubsection{Inter-operation with C/C++}

\paragraph{C++}

\paragraph{Clojure}
It is likely possible, but not sure that it is \emph{easy}.

\paragraph{Go}
Yes, with something called `cgo' \url{https://blog.golang.org/c-go-cgo}

\paragraph{Guile}
Yes, definitely, although perhaps not in quite the same way as other languages -- one of the goals of Guile is to permit the incorporation of Guile elements into a C program.

\paragraph{Kotlin}
Yes. \url{https://kotlinlang.org/docs/reference/native/c_interop.html}

\paragraph{Manticore}
Not entirely clear, but almost certainly can be done.

\paragraph{MLTon}
Yes.

\paragraph{Nim}
Yes.  Nim actually transpiles to C, which is then compiled as normal.

\paragraph{OCaml}
Yes.


\paragraph{Rust}
Yes.


\subsubsection{Ahead-of-time compilation to native executables}

\paragraph{C++}

\paragraph{Clojure}
Kinda with OpenJ9.  Yes with GraalVM.

\paragraph{Go}
Yes.

\paragraph{Guile}
Yes, by `embedding' the Guile program into a C program.

\paragraph{Kotlin}
Yes - though it appears that only 32-bit ARM is supported for Linux right now: \url{https://kotlinlang.org/docs/reference/native-overview.html}.

\paragraph{Manticore}
Yes - though it's not clear what backend is used by default.

\paragraph{MLTon}
Has it - through GCC and LLVM

\paragraph{Nim}
Yes (in fact, they brag they're particularly good at it)

\paragraph{OCaml}
Yes.


\paragraph{Rust}
Yes.

\subsubsection{High-quality implementations of stereo matching algorithms already available}

\paragraph{C++}

\paragraph{Clojure}
Didn't find anything.  It seems unlikely.  I did see somewhere that there is apparently a Clojure wrapper of OpenCV, though.

\paragraph{Go}
Not sure, but it's likely.

\paragraph{Guile}
Couldn't find any.  There's presumably some in C, however.

\paragraph{Kotlin}
OpenCV if nothing else (and I didn't find anything else).

\paragraph{Manticore}
Couldn't find any.

\paragraph{MLTon}
Couldn't find any.

\paragraph{Nim}
There are wrappers over OpenCV, at least.

\paragraph{OCaml}
Couldn't find any.


\paragraph{Rust}
Didn't find any, but could well be out there.

\subsubsection{Works on x86* \emph{and} ARM}

\paragraph{C++}

\paragraph{Clojure}
Not 100\% clear.  It basically comes down to whether a JVM implementation supports targeting more architectures than x84-64.  It \emph{looks like} OpenJ9 doesn't support ARM, and neither does GraalVM.

\paragraph{Go}
Yes.

\paragraph{Guile}
Looks like it does x86* and ARMv7 specifically.

\paragraph{Kotlin}
It has a native compilation feature, which supports x86-64, and one of ARM32 or ARM64 I think (I have seen references to one or the other, but not both at once).

\paragraph{Manticore}
No.  x86-64 only.

\paragraph{MLTon}
Yes.

\paragraph{Nim}
Yes (I believe it should work for whatever architectures GCC and Clang/LLVM support).

\paragraph{OCaml}
Yes.


\paragraph{Rust}
Yes.

\subsubsection{More than one individual's toy/research language, and still under active development/support}

\paragraph{C++}

\paragraph{Clojure}
Yes.

\paragraph{Go}
Yes.

\paragraph{Guile}
Doesn't have a huge team or wealthy company behind it, but it is a central part of GNU these days, and has been in development for a long time.

\paragraph{Kotlin}
Yes.

\paragraph{Manticore}
Yes, though it is definitely a research language with a small core group at present.

\paragraph{MLTon}
Yes.  The last formal release is over 2 years old, but it appears that development is still in progress to some degree.

\paragraph{Nim}
Yes.

\paragraph{OCaml}
Yes.


\paragraph{Rust}
Yes.

% \subsubsection{Support for optimisation algorithms}

\subsection{Excluded options}
The following options were investigated but excluded from consideration due to a variety of issues.  These include that they emulate a related model such as Join Calculus or Actors, but not \gls{csp}.  

\subsubsection{Actors}
\begin{itemize}
    \item Erlang / Elixir
    \item Halide with message-passing?  There's \url{https://doi.org/10.1016/j.sysarc.2017.10.005}, \url{https://doi.org/10.1007/s11265-017-1283-1}, and Aaron Epstein's Masters Thesis at MIT, ``A Distributed Backend for Halide'', but with regards to message passing they all seem to focus on MPI.%  I couldn't find anything on \gls{cml} style in Halide.
    \item \sout{Don't forget Pony...  (almost certainly not stable enough at this point)}
    \item Rust with Actix
    \item \gls{mpi}
\end{itemize}

\subsubsection{Join Calculus}
\begin{itemize}
    \item \fsharp{} with Joinads
    \item JOCaml
    \item Scala with Chymist\footnote{\url{https://github.com/Chymyst}}
    \item \sout{Polyphonic C\#/C\(\omega\)}
\end{itemize}


\section{Assessment of chosen candidates}
Sample applications, and results of profiling them.  Comparison with other pre-existing implementations.

Probably want to do one or more stress tests on each system, to see how they cope, as well as see how easy it is to program with them reasonably effectively.  E.g. implementations of the median filter ala my IVCNZ 2018 paper.

\subsection{Criteria}
How to assess?
\begin{itemize}
    \item Mean runtime (minimum is suggested to be better as it more accurately reflects ONLY the process in question, but mean is probably going to happen more often in practice - particularly relevant with garbage collections)
    \item Peak \& average memory use
    \item `Code quality' measures
\end{itemize}

\subsection{Tests}

\subsection{Test results}
What applications?...  Preferably something similar to what I actually expect to be doing.  Running these on the sample input images, perhaps set up to be a stream of images (even if it is the same one over and over again - though could that advantage some algorithms or specific implementations?), while measuring relevant metrics.  %If possible, try to deploy one of them into the field in some fashion.

% \section{Heterogeneous computing}
% I.e. effective combined use of the CPU and GPU?

\section{Final language choice}
And the winner is...


\section{Appendix}
List of languages that were investigated but fell at the first hurdle, namely not meeting one of the essential requirements above.  This is not a complete list.

\begin{itemize}
    \item \texttt{Standard ML of New Jersey} with \texttt{Concurrent ML}:  The original \gls{cml} implementation was in fact intended only for concurrent programming, and not parallel.  Thus, it does not support modern multiprocessors well.  Instead, MLTon or Manticore can perhaps be used.
    \item \texttt{Julia}:  Does not have support for parallelism in its channels operations.  ``The current version of Julia multiplexes all tasks onto a single OS thread.'' (from \url{https://docs.julialang.org/en/v1/manual/parallel-computing/#Coroutines-1}, accessed on 2 February 2020)
    \item \texttt{Concurrent C}:  Long dead.
    \item \texttt{Crystal}:  At present, Crystal does have fibers, but not support for parallelism.  ``At the moment of this writing, Crystal has concurrency support but not parallelism: several tasks can be executed, and a bit of time will be spent on each of these, but two code paths are never executed at the same exact time.'' (from \url{https://crystal-lang.org/reference/guides/concurrency.html}, accessed on 2 February 2020)
    \item \texttt{\fsharp{}} with \texttt{Hopac}:  More-or-less abandoned at this point, it seems.  Apparently (according to gossip) the original creator doesn't even work in \fsharp{} anymore, and the `current maintainer' doesn't seem to have much interest.  Also looked at .NET with \texttt{System.Threading.Channels}, but it doesn't seem to have any ability to select over channels.
    \item \texttt{Concurrent Haskell}:  There have been a number of \gls{csp}-inspired libraries implemented in Haskell.  The most recent known one, however is Communicating Haskell Processes, which has not been updated since 2014 (see \url{https://hackage.haskell.org/package/chp}).
    \item \texttt{Eiffel} with \texttt{SCOOP}:  SCOOP implements synchronous rendezvous, but does not seem to support actual message-passing.  It should be considered for future efforts, but is not close enough to \gls{cml} to be appropriate here.
    \item \texttt{Occam/Occam-\(\pi\)}:  The latest version of an Occam compiler, KRoC,\footnote{https://github.com/concurrency/kroc/} has not been updated since April 2017.
    \item \texttt{Haxe}:  It claims to be fast, and run cross-platform.  Unfortunately, no message-passing system could be found for it, in the standard library or user-created libraries.
    \item \texttt{Single Assignment C}:  No longer being maintained it seems, plus there doesn't seem to be any capacity for explicit concurrency mechanisms.
    \item \texttt{Python} with \texttt{PyCSP}:\footnote{\url{https://github.com/runefriborg/pycsp}}  Actually implements a lot of what I would need, but is unlikely to be all that performant.  Plus, it hasn't been updated in nearly four years.
    \item \texttt{Stackless Python}:\footnote{\url{http://www.stackless.com/}} This appears to have some of the basics, but no provision for selective choice.  Same story with \texttt{PyPy}\footnote{\url{https://www.pypy.org/}}.
    \item \texttt{Ada}:  In most ways it would be an excellent fit for this, but this work focuses specifically on channel-based implementations, and Ada doesn't come with channels by default (it has a similar-but-different mechanism).  One was implemented \cite{Atiya2005}, but it doesn't appear to have been incorporated into the language or made widely available.
    \item \texttt{D}:  The D language has nearly everything desired, but does not appear to have support for selection over channels.  No trace of it could be found in the standard library.  Mention of it is made in \url{https://wiki.dlang.org/Go_to_D}, but the library referred to there\footnote{\url{https://github.com/nin-jin/go.d}} seems to have been an unfinished prototype, and has not been updated in years.
    \item \texttt{Scala} with \texttt{Scala Communicating Objects}:\footnote{\url{https://www.cs.ox.ac.uk/people/bernard.sufrin/personal/CSO/}}  While it seems like it's probably a good library, it appears to have been one person's research experiment, which has no support and has not been updated for a while.
    \item \texttt{Pharo} or \texttt{Squeak}:  In most ways they would be suitable, but they don't \emph{quite} follow the \gls{cml} explicit channels approach.  They're more akin to Ada.
\end{itemize}
\chapter{Belief Propagation}

Fundamentally most of the pixel-based stereo matching algorithms (at least the ones I'm familiar with) boil down to solving an equation like

\[ E(\vec{f}) = \sum_{p \in \Omega} \Bigg[ E_{data}(p, f_p) + \sum_{q \in A(p)} E_{smooth}(f_p, f_q)\Bigg] \] with the goal of finding
\[ \arg\min_{\vec{f}} E(\vec{f})\] where \(f_p \in L\) is a label from the set of possible disparities \(L = 0, 1, \ldots, D_{max}\), \(p\) is a given pixel in the disparity map, \(\Omega\) is the set of all pixels in the disparity map, \(A(p)\) is the set of all pixels that are considered `adjacent' to \(p\) for some meaning of adjacency, and \(E\) is the total error over the whole image and \(\vec{f}\) is the vector of all disparity labels assigned to pixels in the disparity map, i.e. \( \forall_{f_p}~f_p\) is in \(\vec{f}\), and \(|\vec{f}| = |\Omega|\).

(alternative notation can be seen in Felzenswalb and Zabih 2011)

They're all essentially some form of global optimisation (though in cases where there is no smoothness function used, they degenerate to the local optimisation case because the overall lowest cost assignment becomes the lowest cost assignment at each particular location \(p\)).  The biggest differences between each of them are the functions used for the data and smoothness costs, the meaning of adjacency and the scheduling of the computations (plus whether there are iterations, etc).

Come to think of it, why can't these parts all be separated out?

All the known adjacency forms are \emph{symmetric}, meaning that \(x A y \implies y A x\), where A is the binary adjacency relation.  Thus, in every single rendezvous of processing elements, \emph{both} of them will want to both send to and receive from the other, and thus two-way exchanges are to be preferred, rather than the usual one-way of base Concurrent~ML.

A \emph{possible} optimisation might be to apply some sort of pre-processing step on each input stereo image, where all contiguous pixels of the exact same intensity are grouped together into one super-pixel (which I believe is, in fact, different from regular super-pixel approaches -- those could be trialled also), and then those super-pixels used as the message-exchanging entities.  Potential downsides to this approach, beyond the obvious that the pre-computation will take some time, are that the regular grid layout of the images, which is kinda relied upon in the adjacency/neighbourhood relations as well as in determining which pixels are along a given epipolar line.  This approach would likely reduce the number of message-passing entities, but at the cost of a greater number of messages sent by each, as it seems plausible that each one of them could end up being considered to be adjacent to many more super-pixels than the typical 4.  There have apparently been some attempts to do similar, but simply using rectangular blocks, without a great deal of success (these are discussed a bit in sect. 2.2 of Rui Gong's 2011 Masters thesis).

Introduce \gls{bp} here

Note to self.  In all instances, the final produced programs should really include some way to update each pixel/task/communicating process with its new intensity value from the new image.  So, the ideal operation is that at startup the program initialises its processing elements, then somehow supplies to them their intensity value, and then once computations are performed the PEs then send their final disparity decisions back to somewhere to be compiled into the final product.  So, timing testing will need to be done over more than just one iteration of an image.  Preferably many more.

\section{Operation of \glsentrylong{bp}}
Description of \gls{bp} goes here.  (A programmer's intro to \gls{bp}?)

\section{cP~systems}

% Will probably need to give relevant background on cP systems here also.

Look at incorporating use of PROMELA too?

\section{\glsentrylong{cml} implementation}

Perhaps this section should be moved into a `methods' chapter.  Wherever I write it, I should really explain which particular style of \gls{bp} I chose to emulate, and why.

\section{Experiments}

Perhaps this section should be moved into a `results' chapter.
\chapter{Concurrent Propagation}

\section{Operation of Concurrent Propagation}

\section{cP~systems}

\section{Concurrent~ML implementation}

Perhaps this section should be moved into a `methods' chapter.

\section{Experiments}

Perhaps this section should be moved into a `results' chapter.
\chapter{Discussion}

Where I discuss the good \& bad about this work

What should actually go in here, if anything?  Perhaps anything to be said here can just be said in the conclusion chapter?

% % \section{Comparison of \glsentrylong{mpbsm} algorithms}

% % What am I comparing them to?  Algos coded by myself?  The best I can find out there of others'?

% % \subsection{\glsentrylong{bp}}

% % \subsection{\glsentrylong{cp}}

% \section{Limitations}

% \subsection{Threats to Validity}

% \section{Alternatives for Investigation}

% \subsection{Other languages \& libraries}
% Implement my own version of \gls{cml} in Rust or Julia, etc?  Looks like Felix has most if not all of the necessary components for it.

% \subsection{Other Computational Models}
% Actors, Join Calculus, Pi Calculus

% \subsection{Potentially Useful Hardware}

% \subsubsection{CPU Additions}
% Intel's Transactional Synchronisation Extensions (TSX) -- \url{https://en.wikipedia.org/wiki/Transactional_Synchronization_Extensions}

% AMD's Advanced Synchronization Facility (ASF) -- \url{https://en.wikipedia.org/wiki/Advanced_Synchronization_Facility}

% mEDA (modified Extended Dataflow Actor) \& full/empty memory tagging (F/E bits?) e.g. \url{https://link.springer.com/chapter/10.1007/BFb0057916} \& \url{https://people.kth.se/~vladv/abstracts/TRITA-IT-0004.pdf}

% \subsubsection{Is Hyper-Threading Helpful?}

% \subsubsection{GPUs?}

\chapter{\label{chap:conc}Conclusion}

% \begin{anfxerror}{Backwards conclusion}
% José suggested an approach where you go backwards in the conclusion, starting with the most recent section and eventually reaching the start again.  This, presumably, is for instances where each earlier bit contributes to one or more later bits, and so you say \enquote{we did x, using y that we defined earlier} sort of thing.
% \end{anfxerror}

As suggested by the title, this dissertation has focused on highly concurrent solutions to computing problems.  These are, broadly, considered challenging problems (in the exact case at least) for modern electronic computers, but the high levels of concurrency -- combined with taking advantage of the unbounded memory space of \gls{mc} -- can significantly reduce their time complexity.  Each solution requires either constant time or linear time dependent on a core aspect of the problem.

The specific problems addressed were:
\begin{inparaenum}[(1)]
\item The \gls{hpp}, \gls{hcp} and \gls{tsp};
\item The \gls{gcp};
\item Image \gls{medianfilter}ing; and,
\item \gls{nmp}.
\end{inparaenum}
The first three are well-known problems with histories stretching back decades.  The last, \gls{nmp}, is (to the best of the author's knowledge) newly abstracted out from \gls{lbp}, initially motivated by attempts to model a pre-existing algorithm to perform \gls{lbp} for \gls{sm} in \gls{cps}.

None of the \glspl{ruleset} require customisation or adjustment for a given problem.  Instead, the \gls{cps} rules adapt automatically, so long as the problem to be solved is adequately specified.  This is unlike solutions sometimes seen in other forms of \gls{ps}, where a uniform or semi-uniform family of rules is defined, requiring some (potentially significant) level of pre-processing before the \glspl{ps} is applied to find the solution.  For the first three problems, the same ruleset can be used across every \gls{tlc}.  Even in the case of \gls{fne} \gls{nmp}, all that is required is assigning a reduced \gls{ruleset} derived from the base one to the \gls{pe} with fewer than four neighbours, \ie{} those on the edge, or in the corner, of the grid.

The various systems presented were investigated experimentally, using a variety of approaches.  Direct translations of the \gls{tsp} algorithm were implemented in three different programming languages, namely \fsharp{}, Erlang and Prolog.  All three worked well for small graphs, but quickly grew to consume more memory than the test system had available.  The Prolog implementation was notable, however, for being an extremely close match to the specified \gls{cps} \gls{ruleset}, requiring only 17 lines (including whitespace) to specify the complete system.

The algorithm to solve the \gls{gcp} was translated into a communicative system using an implementation of \gls{cml}, \hopac{} in \fsharp{}.  While the implementation worked well, it also highlighted that there was, in fact, relatively little communication involved, and the system had more in common with the \gls{tsp} implementation than first presumed.  Comparing the results from the simulation on various graphs against using an earlier \gls{skps} solution using \gls{mecosim} suggested that -- as largely expected due to the nature of the \adhoc{} vs general simulations (see \vref{sec:back:simulators}) -- the \gls{cps} simulation performed better.  Doing so also revealed a curious behaviour of \gls{mecosim}, where two almost-identical graphs produced markedly different runtime characteristics.

\Glspl{ruleset} to perform a variety of statistical operations were detailed in \cref{chap:median}.  All of these rulesets have a constant time complexity, \ie{} they are \bigoh{1}.  They also can be used as primitives and composed together to create further operations, as demonstrated in \eg{} \cref{sec:median:mean,sec:median:mode,sec:median:selection}.  The practical application and combination of these rules was then demonstrated by defining a ruleset to perform \gls{medianfilter}ing on a grid representing an image.  Different approaches to implementing the \gls{medianfilter} were experimented with, including one based on communicative pixels -- as with the \gls{cps} solution -- and created using \gls{cml}.  In this instance, the \gls{cml} implementation performed much worse overall than the other two presented, both in terms of running time on a conventional CPU and clarity of code.

While the development of \gls{nmp} in \cref{chap:nmp} was initially based on attempting to model \gls{lbp} in \gls{cps}, the \gls{nmp} variants presented generalise beyond \gls{lbp}.  In particular, \cref{chap:nmp} presented a generic framework for the messaging aspect, but did not prescribe the internal computations each \gls{pe} should perform.  Instead, an oracle stood in for that part of the process, with the intention that the oracle is replaced as appropriate for a given problem.  The distinguishing characteristic of \gls{nmp} compared to other messaging schemes is that each message a \gls{pe} sends to a given neighbour depends upon the messages received from the other neighbours.  The initial focus of the work was on deriving the asynchronous variant from the original \gls{gs} approach, and the \gls{ls} arose naturally out of this.

The asynchronous variant in particular was analysed and its behaviour explored in \cref{sec:nmp:analysis}.  It was demonstrated that the asynchronous variant sends the same number of messages as the \gls{gs} variant, using information that is at least as recent, giving rise to \emph{generational confluence}.  The spread of `influence' from a centre \gls{pe} to others on a grid was also visualised under differing communications channel conditions and using different formulae to update the messages.

Empirical testing of a simulation of the asynchronous variant in a \gls{fne} grid configuration confirmed that it had the expected messaging behaviour.  Moreover, detailed simulations of each variant showed that the \gls{ls} variant (very nearly) always outperforms the \gls{gs} variant -- typically by a factor of approximately 5-13\%, with the difference typically becoming larger as the size of the lattice and number of processors available both increase -- with no change in the final computed results, while the asynchronous variant is consistently faster again, often by a factor of approximately 10\% compared to the \gls{ls} variant.  The final results computed by it also tend to be near-identical to the other two variants, often within roughly 0.1\%.

\section{Contributions}
This work makes the following main contributions:

\begin{itemize}
    \item \Cref{chap:cpsystems} provides an extensive overview of \gls{cps} as it stands at the time of writing.  This is believed to be the most comprehensive summary to date.
    \item \Cref{chap:tsp} presents solutions to the \gls{hpp} and \gls{hcp}, which require a total of \(n + 1\) or \(n + 3\) steps respectively, where \(n\) is the number of nodes in the graph, with the former needing three \gls{cps} rules and the latter four.  Furthermore, it also builds on the \gls{hcp} solution to solve the \gls{tsp}, requiring the same number of steps and one more rule than the \gls{hcp}'s solution.  No earlier linear time \gls{mc} solution to the \gls{tsp} is known.
    \item \Cref{chap:gcol} presents a solution to the \gls{gcp}, which finds a valid colouring (or declares no valid colouring is possible) using only six rules, with a worst-case time complexity linear to the size of the graph.  Importantly, it works with \emph{any} number of colours, instead of the usual fixed number studied by most of the literature.
    \item \Glspl{ruleset} for many standard statistical operations on numerical multisets in \gls{cps} were provided in \cref{chap:median}.  These included:
    \begin{inparaenum}[(i)]
        \item finding the minimum and maximum;
        \item counting the number of elements, and counting the frequency of each distinct element;
        \item finding the sum, mean, and mode;
        \item sorting; and,
        \item selecting the \(n^{\text{th}}\) element from a set.
    \end{inparaenum}
    All of these, besides the minimum and maximum, are not known to have been presented earlier.  Applying these rules to the \gls{medianfilter} is also thought to be the first time this particular image operation has been modelled in any form of \gls{ps}.
    \item Building on \cref{chap:median}, and taking heavy inspiration from \gls{lbp}, \cref{chap:nmp} investigated message passing between individual \glspl{pe} on a lattice. Importantly, in the scenario under study, each message sent out depends on the last messages received from every neighbour of the preparing \gls{pe} \emph{apart} from the intended recipient of the current message.  This scenario was termed here ``\glsxtrlong{nmp}''.  Rulesets for \gls{gs}, \gls{ls} and asynchronous messaging approaches were provided.  To the best of the author's knowledge, this is the first time this type of message passing has been modelled directly according to the conceptual \gls{nmp} principles of message passing between \glspl{pe}, as opposed to other more simplistic approaches.  Importantly, this approach is also more general, as it can adapt to any form of connectivity on the lattice, beyond a grid, simply by adjusting the connections between \glspl{pe} appropriately.
    \item Empirical results assessing the systems described above, implemented to run on relatively common modern computer hardware, are provided.  These results have explored various approaches to translating \gls{cps} to run on said hardware, with varying results.
\end{itemize}

\section{Future Directions}

While this dissertation has explored new areas, it also points to worthwhile future work for \gls{cps}, and creating successful implementations of high concurrency solutions to problems.  A number of the most promising lines of enquiry are described below.

% \subsection{`Faked' Message Passing in Shared Memory}
% Roughly, most of the message passing involved here is largely, in effect, just handing around pointers to memory locations.  It would seem that the message passing itself places some overhead in the way of that.  Could there be some way to fake the message passing, so that to the program's writer it looks like an actual message passing implementation, but in reality it is just doing normal updates on mutable memory?  This \emph{might} enable the best of both worlds -- programming the algorithms according to their theory, but running in a highly efficient fashion `under-the-hood'.

% None of the below were investigated further, due to a lack of time, but they are obvious next steps to look at.

% It is not too clear how to achieve this (if it is indeed possible), but Rust would appear to be a good language to target for it.  Rust is fairly high-performance by default and enables quite a lot of low-level memory manipulation.  Moreover, its move semantics pretty much fit exactly to the concepts used here, and, on the face of it, its macro system would appear to be a convenient way to abstract over many of the details and provide a message passing façade, while actually doing efficient operations behind that.

% Extended Dataflow Actors?  Or is it Extended Dataflow Architecture?

% See also the Bulk Synchronous Parallel approach.

% It looks like the C++ \texttt{mess} library is intended to be exactly this sort of thing:  \url{https://github.com/LouisCharlesC/mess}.  The developer seems to say that the user gets to write their program in a message-passing fashion, but mess does some clever meta-programming so that there ends up being zero overhead in the end.  Also, absolutely \emph{must} address Halide, and explain why it wasn't pursued here.  Otherwise, that'll be the elephant in the room.

% https://scholar.google.co.nz/scholar?q=related:Z8GZl-HQcSkJ:scholar.google.com/&scioq=A+Static+Mapping+System+for+Logically+Shared+Memory+Parallel+Programs&hl=en&as_sdt=0,5&inst=15360723290749679499

% http://citeseerx.ist.psu.edu/viewdoc/download?doi=10.1.1.50.8739&rep=rep1&type=pdf

% https://link.springer.com/chapter/10.1007/BFb0057916

% https://link.springer.com/chapter/10.1007/3-540-63697-8_82

%%%%%%%%%%%%%%%%%%%%%%%%%%%%%%%%%%%%%%%%%%%%%%%%%%%%%%%%%%%%%%%%%%%%%%%%%%%%%%%%%

\subsection{Unification}

One-way multiset unification frequently occurs in \gls{cps}, with unification used in almost every rule presented \crefrange{chap:tsp}{chap:nmp}.  An efficient algorithm to perform this task would be highly beneficial for creating useful simulations of systems.  For example, the simulations of the \gls{tsp} algorithm written in functional programming languages (see \vref{sec:tsp:simulation}) regularly simply iterate over all relevant objects in the system, even though frequently most will be of little use in a given function call, and so the simulations could benefit from improved unification in practice.  At the time \cref{chap:tsp} was completed, no efficient algorithm for this task was known, but more recently \citeauthor{Liu2021} published \citetitle{Liu2021} \cite{Liu2021}.  Given that the \gls{tsp} simulations are near-direct translations from the \gls{cps} \glspl{ruleset} to sequential programs, they especially should be revisited in light of this development.  %  Given that they hew most closely to a direct translation from \gls{cps}, the \gls{tsp} simulations especially should be revisited in light of this development.

%%%%%%%%%%%%%%%%%%%%%%%%%%%%%%%%%%%%%%%%%%%%%%%%%%%%%%%%%%%%%%%%%%%%%%%%%%%%%%%%%

\subsection{Universal Numbers}

The solutions and systems described in this work, when performing numerical tasks, always assumed the exclusive use of natural numbers.  For many tasks, this is adequate, but many other tasks require a workable approximation of real numbers, both positive and negative.  Currently, there is no well-defined method for representing and using arbitrary real numbers in most forms of \gls{ps}, including \gls{cps}.  If \gls{cps} is to be used to solve a broader range of problems, some form of representation for real numbers will eventually be required.

There seem to be two leading candidates for such representations:  simulating IEEE-754 floating-point numbers \cite{ieee754}, which are widely supported by most modern electronic computer hardware;  or, using the concept of \emph{universal numbers (unums)} advanced more recently by \citeauthor{Gustafson2017} \cite{Gustafson2017}.  Systematising working representations of one or the other seems likely to be highly beneficial to solving future problems.  On the surface, at least, unums appear to be a better operative fit with \gls{cps}, and so perhaps should be explored first.  IEEE-754 floating-point numbers are much better supported in current commodity hardware, however.

%%%%%%%%%%%%%%%%%%%%%%%%%%%%%%%%%%%%%%%%%%%%%%%%%%%%%%%%%%%%%%%%%%%%%%%%%%%%%%%%%

\subsection{Neighbourhood Message Passing}
An obvious next step for the \gls{nmp} work is to adapt the system to the purpose of \gls{lbp} \gls{sm}, using \gls{nmp} precepts explicitly.  Time and space constraints prevented further work in this direction for this dissertation, however.

In the systems presented above, the size and shape of the grid and the communication topology between neighbours are permanently fixed at system initialisation.  Greater flexibility is usually not needed, but it might be useful in some circumstances, such as when a dynamic communication topology is necessary for a particular algorithm.  Modifying the systems from \cref{chap:nmp} accordingly and analysing the impacts on running computer programs is a clear potential extension to this paper's work.

At present, every \gls{pe} stays active throughout the system's evolution until it has sent and received all its scheduled messages.  Permitting \glspl{pe} to deactivate at appropriate points before they have reached their maximum generation count could save processing capabilities when the potential for parallelism is bounded.  As discussed with Conjecture \ref{conj:nmp:3}, however, any \gls{pe} which ceases messaging before reaching the maximum generation count can eventually affect the entire system.  Implementing effective early stopping is an unsolved problem.

These systems also have not been examined concerning communication complexity measures such as those found in \cite{Juayong2020}.  The precise results presented there are not directly applicable to this work, given the use of different \gls{ps} models, but the underlying concepts appear directly relevant.  Finding ways to define and quantify the communication complexity and `cost' (\cite{Juayong2020}'s \emph{ComR} and \emph{ComW}) seems particularly pertinent for determining the overall characteristics of a system.

%%%%%%%%%%%%%%%%%%%%%%%%%%%%%%%%%%%%%%%%%%%%%%%%%%%%%%%%%%%%%%%%%%%%%%%%%%%%%%%%%

\subsection{Alternative Hardware}
This work focused on CPU-based systems, excluding other hardware.  Fundamentally, there is a mismatch between current (\circa{} 2021) CPUs and the underlying concept of \gls{cps}.  Specifically, current CPUs provide a handful of highly capable and flexible processors (at the time of writing, typically somewhere between 4 and 12 cores), while \gls{cps} essentially assumes the presence of an unbounded number of ``smaller'' processors.  This incongruence between the model and the hardware inevitably means attempting to use the latter to simulate the former will lead to unsatisfactory results.  Other hardware should be explored if the full power of \gls{cps} (and indeed \gls{ps} in general) is to be realised, because of the potential for large-scale parallelism that devices such as \glspl{gpu} and \glspl{fpga} offer.

Given the use of numerous smaller processors sounds remarkably close to the Single-Instruction Multiple-Thread (see \cite[Ch. 4.4.1]{Hennessy2012}) nature of modern \gls{gpgpu}, it would seem that \glspl{gpu} represent a clear next step for developing simulations further.  A \gls{gpu} implementation of the \gls{nmp} variants, in particular, seems like a promising target for further investigation.  Each \gls{pe} in a \gls{nmp} system likely has small processing requirements individually, but an entire system might have many \glspl{pe}.  The cores on a \gls{gpu} are limited in their capabilities compared to a CPU's cores, but there are many more of them, and modern \glspl{gpu} can handle processes with millions of live threads.  Thus, \gls{nmp} and \glspl{gpu} are potentially a superb match.

It is not yet apparent, however, how best to join execution of \gls{cps} rules with the operation of \glspl{gpu}.  The high levels of messaging and control flow might not fit well with \gls{gpu} architecture and prove to be confounding factors, as \gls{gpu} performance is generally averse to branching.  Further work is required in this space to find an optimal strategy.  In fact, there is also almost certainly more efficiency to be extracted from CPUs, too.  Given that it can be used fairly directly for CPUs, \glspl{gpu} and \glspl{fpga}, targeting translation of \gls{cps} \glspl{ruleset} to OpenCL\footnote{\url{https://www.khronos.org/opencl/}} may prove fruitful.

\Gls{mc} overall fundamentally assumes that the processors involved operate differently from current electronic computers.  Modern electronic computers follow a model in which the processor and the memory are distinct, and that data and instructions are shuttled back and forth between the two.\footnote{Indeed, making efficient use of the memory hierarchy tends to be critical to achieving high performance in modern application development.}  All the major forms of \gls{ps}, however, assume essentially that the memory storage \emph{is} the processor, whatever form the \glspl{compartment} take.  This differing theoretical operation suggests that non-standard hardware, such as logic modules and associative memory, may more directly support \gls{mc}.  Investigating this alternative hardware thus appears to be an important line of future research.  Novel approaches \emph{may} prove superior for systems such as \glspl{fpga} and semi-distributed systems like high-performance computing clusters, though that is largely speculation at this point.

% Researching uses of alternative hardware for \gls{nmp} would be worthwhile because of the potential for large-scale parallelism offered by devices such as \glspl{gpu} and \glspl{fpga}.  A \gls{gpu} implementation of the \gls{nmp} variants, in particular, seems like a promising target for further investigation.  Each \gls{pe} in a \gls{nmp} system likely has small processing requirements individually, but an entire system might have many \glspl{pe}.  The cores on a \gls{gpu} are limited in their capabilities compared to a CPU's cores, but there are many more of them, and modern \glspl{gpu} often comfortably run processes with millions of live threads.  Thus, \gls{nmp} and \glspl{gpu} are potentially a superb match.  The high levels of messaging and control flow, however, might not fit well with \gls{gpu} architecture, as \gls{gpu} performance is generally averse to branching.  Novel approaches \emph{may} prove superior for systems such as \glspl{fpga} and semi-distributed systems like high-performance computing clusters, though that is largely speculation at this point.

% It would appear worthwhile to investigate how to implement a high-performance message passing system atop these hardware alternatives, due to the potential for many more computations per second.  Even better would be to create a heterogeneous system which can take full advantage of the strengths of each hardware type, while overcoming its weaknesses.  Precisely how to achieve efficient implementations on them is unknown.  The fact that OpenCL \fxerror[inline]{[ref]} (and, to some extent at least, OpenACC \fxerror[inline]{[ref]}) can be compiled from the same base code to different devices makes it an obvious starting point.%   \fxerror{Note to self}{There has already been at least one publication on implementing \glspl{actor} in OpenCL \fxerror[inline]{[ref]}, suggesting it is possible -- though how well synchronous message passing will work as compared to asynchronous remains to be seen.}

% The use of Full/Empty bits looks highly promising for making message passing more efficient.  It doesn't seem to have any real support in mainstream/commodity hardware, however.  There were Intel's TGX instructions, but they have been taken out of processors again.

% See also \url{https://hastlayer.com/} -- they seem to say that they do relevant stuff on \glspl{fpga}.  Interestingly, they also do unums/posits, apparently.

%%%%%%%%%%%%%%%%%%%%%%%%%%%%%%%%%%%%%%%%%%%%%%%%%%%%%%%%%%%%%%%%%%%%%%%%%%%%%%%%%

\subsection{\glsfmtname{plingua}}
Up to the time of writing, at least, \gls{cps} has often stood apart from many other \gls{ps} types in many ways.  For example, there is no representation for \gls{cps} in \gls{plingua}, nor support for simulating \gls{cps} in \gls{mecosim}.  There is no apparent reason that \gls{cps} cannot or should not be implemented in both, or successors to the two.  It seems worthwhile to develop a \gls{cps} representation in \gls{plingua} so that it can be used in the various tools and simulators based on \gls{plingua} (see \vref{sec:back:simulators}).  Support for \gls{cps} in \gls{mecosim} would make it much easier for researchers unversed in \gls{cps} to begin experimenting with it.

%\input{tex/chapter1} % I hope that you have better titles than this
%\input{chapter2} 
%\input{chapter3} 

% ====================================================
%
% ENDMATTER
%
% Appendices and bibliography 
% Pagination arabic, re-starts at 1
%
% ====================================================

\printbibliography[title={Works Cited}, heading=bibintoc]
\chapter*{Glossary}

\printglossary[type=\acronymtype]
 
\printglossary

\cleardoublepage % start afresh on a new page
\setcounter{page}{1} % re-sets the page counter
\appendixpage* % makes a page to mark beginning of appendices
% \input{appendix1} 


\end{document}