% \section{Introduction}

% \begin{anfxerror}
% Need to revise the introduction a bit to reflect better the change in focus for this chapter.
% \end{anfxerror}

% Many Computer Vision and Image Processing operations have some potential for parallelism.  Indeed, a number of them can be regarded as `embarrassingly parallel', that is to say that the process involves a considerable number of sub-process steps that do not depend on each other, and so those steps may be performed concurrently without complication.  Some of those algorithms either are explicitly characterised in terms of message passing, such as \gls{sm} with \gls{bp} \cite{Liang2011} or \gls{sgm} \cite{Drory2014}, or could be viewed as such, \eg{} \glspl{mwt} applied to images.

% \Gls{medianfilter}ing \cite[Chap. 3.4.1]{Gimelfarb2018}, \cite{Fisher2016} is an operation in image processing used to remove random `salt \& pepper' noise from images.  Such noise is characterised by pixel colouration values at the extreme high and low ends of the range of possible values.  At its simplest, \gls{medianfilter}ing recovers an approximation of the non-noisy image by taking the median of all pixel values in a window around each pixel and creating a new image using said median values for the pixels.

% This \namecref{chap:median} seeks firstly to model \gls{medianfilter}ing using \gls{cps}, as another test of the power and versatility of \gls{cps}, and secondly to explore whether using \gls{cml} -- as a method of structuring computations around message passing -- could be beneficial when applied to a \gls{mwt}, using the \gls{medianfilter} as its particular example.  The focus is on examining the potential benefit of using a different principle to structure the processing, as much as it is on the achieved results.  It is hypothesised that the same results in terms of processing the image can be achieved, but at slightly slower rates of processing due to overheads from the message passing which, strictly speaking, are unnecessary in the case of a \gls{mwt}.

% \begin{anfxerror}
% Need to revise the introduction a bit to reflect better the change in focus for this chapter.
% \end{anfxerror}

Many computer vision and image processing operations have some potential for parallelism.  Indeed, a number of them can be regarded as `embarrassingly parallel', that is to say that the process involves a considerable number of sub-process steps that do not depend on each other, and so those steps may be performed concurrently without complication.  Some of those algorithms either are explicitly characterised in terms of message passing, such as \gls{sm} with \gls{bp} \cite{Liang2011} or \gls{sgm} \cite{Drory2014}, or could potentially be viewed as such, \eg{} \glspl{mwt} applied to images.

\Gls{medianfilter}ing \cite[Chap. 3.4.1]{Gimelfarb2018}, \cite{Fisher2016} is a \gls{mwt} operation in image processing used to remove random `salt \& pepper' noise from images.  Such noise is characterised by pixel colouration values at the extreme high and low ends of the range of possible values.  At its simplest, \gls{medianfilter}ing recovers an approximation of the non-noisy image by taking the median of all pixel values in a window around each pixel and creating a new image using said median values for the pixels.

Finding the median of a multiset is a well-known fundamental statistical operation.  Likely the way that many people were first taught to perform it was to sort the relevant multiset, count the relevant multiset, halve the count, and then take the number at that position in the multiset.  All of these operations individually, and other similar ones, are relatively simple in \gls{cps} when working with numeric terms (\ie{} terms holding multiplicities of the unary digit \(\cpundig\)).  Moreover, it is also simple to combine these operations into one overarching \gls{ruleset} to carry out the requisite process.

This \namecref{chap:median} seeks firstly to model \gls{medianfilter}ing using \gls{cps}, as another test of the power and versatility of \gls{cps}, and secondly to explore whether using \gls{cml} -- as a method of structuring computations around message passing -- could be beneficial when applied to a \gls{mwt}, using the \gls{medianfilter} as its particular example.  It starts by describing how to perform a variety of fundamental statistical operations, going from less complex to more complex, and re-using earlier ones as part of the implementation of later ones.  This builds up to performing selection -- taking the \(n^{\text{th}}\) element of a sorted multiset -- which can be used to implement the median operation.  Next, this \namecref{chap:median} demonstrates the combination of selection and counting to find the median of a multiset, with antiport message passing between pixels used to build up the multisets in each pixel.  Finally, three different practical implementations of \gls{medianfilter}ing are introduced and assessed.  The first implementation is a simple naïve approach as many programmers might write, the second is a \gls{cml} very closely connected to the \gls{cps} solution, while the third sits conceptually somewhere between the two, and is based on work by \citeauthor{Braunl2001} \cite{Braunl2001}.

The focus in the final \namecref{sec:medianfilter} is on examining the potential benefit of using a different principle to structure the processing, as much as it is on the achieved results.  It is hypothesised, the \gls{cps} model notwithstanding, that the same results in terms of processing the image can be achieved, but at slightly slower rates of processing due to overheads from the message passing which, strictly speaking, are unnecessary in the case of a \gls{mwt}.