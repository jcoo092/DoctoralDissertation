\section{\label{sec:gcol:cpsys}\glsfmtname{cps} Solution to the \glsfmtname{gcp-glossary}}
In the \gls{cml} simulation of the problem, most of the work, in fact, is performed by the instantiation of new objects, rather than communication.  Thus, the problem appears to be an excellent fit to the pre-existing formulation of \gls{cps} \cite{Nicolescu2018}.

This solution assumes a graph with at most one edge between nodes (\ie{} there are not two or more edges between the same two nodes), and that there are no edges leading from a node back to itself.  Further, it takes as supplied inputs to the process conceptual finite sets \(V \subset \mathbb{N}\) representing the nodes of the graph\footnote{In fact, any finite set of arbitrary symbols could be used, but this discussion is restricted to the natural numbers for ease of reading.}; \(E \subseteq \{(i,j)~|~i, j \in V, i \neq j \}\) representing the edges of the graph; and \(K\) representing the chosen colours, where \(K\) contains whatever representation of the desired colours is considered appropriate.

Based on these definitions, for the purposes of the remainder of this \namecref{chap:gcol}, \(|K|\) is defined to be the number of colours used in the current problem, \(|V|\) the number of nodes in the graph under consideration for the current problem and \(|E|\) the number of edges in the graph.

\subsection{Pseudocode for the \glsfmtname{cps} Algorithm}
The logical operation of the system, as carried out using the rules described in detail at \cref{sec:gcol:rules}, can be demonstrated with a pseudocode representation, as shown in \cref{code:gcol:graphcol}.

\begin{algorithm}
\DontPrintSemicolon
\SetKwFor{pFor}{parallel for}{do}{endfor}
\SetKwFunction{Choose}{choose}
\SetKwFunction{Dom}{Dom}
\SetKwFunction{Graphcolouring}{Graph Colouring}
\KwIn{Set of vertices \(V\), set of edges \(E\), set of colours \(K\)}
\KwOut{A valid colouring, or an indication no valid colouring is possible}
\Begin{
\Choose{\(s \in V\), \(c \in K\)}\;
\KwRet{\Graphcolouring{\(V\), \(E\), \(K\), \(\{\{(s, c)\}\}\)}} \label{li:gcol:endofsmaller}
}

\hrulefill

\KwIn{Set of vertices \(V\), set of edges \(E\), set of colours \(K\), set of sets of partial colourings \(B\)}
\KwOut{A valid colouring \(m\), or an indication no valid colouring is possible}
\Begin{
\If{\label{li:gcol:iflgreaterthanv} \(V = \emptyset\)} {
    \Choose{\(m \in B\)}\;
    \KwRet{\(m\)}    \label{li:gcol:returnsuccess}
}
\(B' \gets \emptyset\)\;
\pFor{ \(m \in B,~x \in \Dom{m},~y \in V \setminus \Dom{m},~ d \in K \)}{    \label{li:gcol:makebstart}
    \If{\((x,y) \in E ~ \land \not\exists(x',y) \in E,~(x',d) \in m\)}{
        \(B' \gets B' \cup \{m \cup \{ (y,d) \} \}\)\;   \label{li:gcol:makebfinish}
    }
}
\eIf{ \label{li:gcol:fail1} \(B' = \emptyset\) }
   {
        \KwRet{\(\emptyset\)}  \label{li:gcol:fail2}}
    {
        \KwRet{\Graphcolouring{\(V\), \(E\), \(K\), \(B'\)}\label{li:gcol:recurse}}
}}
\caption[Pseudocode representation of the algorithm performed by the \glspl{cps} rules in \cref{ruleset:gcol:rules}]{\label{code:gcol:graphcol}Pseudocode representation of the algorithm performed by the \glspl{cps} rules in \cref{ruleset:gcol:rules}.  This employs a doubly defined tail-recursive function approach, where the algorithm begins with the upper function supplied only with the details of the graph and colours, but uses the lower function with further arguments to perform most of the processing.}
\end{algorithm}

In this notation, \(m\) is a partial functional relation to \texttt{Dom}(m); \texttt{Dom}(m) is a partial spanning tree over the graph; \((x, c) \in m\) if and only if \(c\) is the colour of node \(x\);  and \(B\) is the set of all partial functional relations built up so far.

The system works by starting with the colours and graph under consideration, then building up, in a maximally-parallel fashion, a collection of objects which represent all validly coloured partial spanning trees of the graph.  Eventually, once full spanning trees have been constructed, one is chosen at random and its colouring sent out to the environment.  Or, if no valid colouring is possible, the system will detect that by the complete absence of any partial spanning tree, and signal to the environment that no appropriate colouring is possible.  The conceptual spanning trees are built following all possible tree edges, but checking that all new frond edges\footnote{Recall that a frond edge for a spanning tree is any edge in the underlying graph which is \emph{not} included in the spanning tree \cite{Erciyes2013}} do not invalidate the colouring.

Note that while this algorithm is guaranteed to return a valid colouring if one is possible, there is \emph{no} guarantee as to precisely which valid colouring will be selected.  This \namecref{sec:gcol:cpsys} assumes that all potential colourings are equally desirable, and makes no provision for specifying that certain nodes must be coloured from a subset of the available colours.

\subsection{\label{sec:gcol:sysinit}cP System Initialisation}
System construction begins with a \gls{tlc} containing the \gls{functor} \(V' = \cpset{\cpfunc{v}{\cpfunc{n}{i}} ~|~i \in V}\), which contains the labels of the nodes in \(V\); a set \(E' \subseteq \cpset{\cpfunc{e}{\cpfunc{n}{i} \, \cpfunc{n}{j}} ~|~(i,j) \in E}\) representing the edges of the graph; a starting node \(\cpfunc{s}{h}\) where \(h \in V\) is an arbitrary node label; and a set \(K' = \cpset{\cpfunc{k}{\kappa} ~|~ \kappa \in K}\) of  the potential colours.  In the case of the three-colour problem, it might look like \(K' = \cpset{\cpfunc{k}{r}, \cpfunc{k}{g}, \cpfunc{k}{b}}\). Using separate colour symbols in this fashion means that the algorithm can operate as a \(|K|\)-colour problem, without any other modification.

While colours are represented here as separate symbols for legibility, they can, in fact, be represented equivalently by natural numbers because each colour symbol is contained within a \(k\) \gls{functor} --- one simply need fill each \gls{functor} with the appropriate number of the counting symbol to represent the desired different colours.  \Eg{} \(r\) might be represented by one copy of the unary digit, \(\cpundig\), \(g\) by two and \(b\) by three, \ie{} \(r = \cpfunc{k}{\cpundig}\), \(g = \cpfunc{k}{\cpundig^2}\) and \(b = \cpfunc{k}{\cpundig^3}\).  The same is true of the edges.  This means that, in fact, the system requires no extra symbols in its alphabet to represent the colours and edges of a given problem --- it merely re-uses atoms (see further in \cref{sec:gcol:notation}).

\subsection{\label{sec:gcol:notation}cP System Definition}

In accordance with the \gls{cps} formal notation of \cref{sec:cps:formaldescriptions}, the \gls{gcp} \glspl{cps} can be formalised as
\cptuple{GCP}{\cpset{\sigma_1}}
{\cpset{b, c, e, \allowbreak m,\allowbreak n, s, v, k, \cpundig}}
{\cpset{O_1}}
{\cref{ruleset:gcol:rules}}
{\cpset{s_1, s_2, s_3, s_4}}
{s_1} where: \(\sigma_1\) is the single \gls{tlc} of the system and \(O_1 = \cpset{\cpfunc{s}{h}} \cup V' \allowbreak \cup E' \allowbreak \cup K'\) is the multiset of initial objects contained within \(\sigma_1\), where \(h\) is an arbitrary member of \(V\).


\subsection{\label{sec:gcol:rules}Rules}

\begin{cprulesetfloat}
\begin{cpruleset}

    \cprule[rule:gcol:rules:init]{s_1}{\cpfunc{v}{\cpfunc{n}{S}Z}}\cponce{s_2}{\cpfunc{b}{\cpfunc{m}{\cpfunc{n}{S} \, \cpfunc{c}{C}} \; \cpfunc{v}{Z}} \; }
    \cppromoter{\cpfunc{k}{C}}
    \cppromoter{\cpfunc{s}{S}}
    
    \cprule[rule:gcol:rules:endsucc]{s_2}{\cpfunc{b}{M \; \cpfunc{v}{\cpempty}}}\cponce{s_3}{\cpsend{M}{env}}
    
    \cprule[rule:gcol:rules:loop]{s_2}{}\cpmaxpar{s_2}{\cpfunc{b}{\cpfunc{m}{\cpfunc{n}{X} \, \cpfunc{c}{C}} \; \cpfunc{m}{\cpfunc{n}{Y} \, \cpfunc{c}{D}} \; M \; \cpfunc{v}{Z}}}
    \cppromoter{\cpfunc{b}{\cpfunc{m}{\cpfunc{n}{X} \, \cpfunc{c}{C}} \; M \; \cpfunc{v}{\cpfunc{n}{Y}Z}}}
    \cppromoter{\cpfunc{e}{\cpfunc{n}{X} \, \cpfunc{n}{Y}}}
    \cppromoter{\cpfunc{k}{D}}
    \cpinhibitor{\Big(\cpfunc{e}{\cpfunc{n}{X'} \, \cpfunc{n}{Y}} \land \cpfunc{m}{\cpfunc{n}{X'} \, \cpfunc{c}{D}}\Big)}
    
    \cprule[rule:gcol:rules:loopclean]{s_2}{\cpfunc{b}{\cpdiscard}}\cpmaxpar{s_2}{}
    
    \cprule[rule:gcol:rules:endfail]{s_2}{}\cponce{s_4}{}
    
\end{cpruleset}
\caption[\gls{cps} \gls{ruleset} for the \glsentrylong{gcp-glossary}]{\label{ruleset:gcol:rules}A completely generic \gls{cps} \gls{ruleset} for solving the \gls{gcp}}
\end{cprulesetfloat}

The rules are presented in \cref{ruleset:gcol:rules}.  The solution consists of six rules in \emph{weak priority order}, which are invariant to the graph under consideration.  They are explained below:

\subsubsection{Description of Rules}

\begin{enumerate}
\item This rule begins system evolution.  It converts the setup \gls{functor} \(v\) into a \gls{functor} \bo{}, which is used further to instantiate new objects with the possible colourings of the system.  It merely selects the node specified by the supplied \(s\) term, and assigns a colour at random.

The \bo{} object holds a multiset of \(m\) \glspl{functor} which track nodes and their assigned colours in a potential solution and a \(v\) \gls{functor} which continues to track the reducing unexplored nodes of the graph.

This rule begins in state \(s_1\) and ends in state \(s_2\).  This rule corresponds to the upper \texttt{Graph Colouring} function in \cref{code:gcol:graphcol}.

\item This rule is one of the potential end points of the evolution of the system.  If there are no further remaining nodes to explore, \ie{} the \bo{} objects have empty \(v\) \glspl{functor}, then one of the \bo{}s is selected at random and the set \(M\) of \(m\) \glspl{functor} within said \bo{} which contain a solution to the colouring problem is output to the environment.  This rule begins in state \(s_2\) and ends in state \(s_3\).  State \(s_3\) is used to indicate to the environment that a solution has been found and the evolution of the system has terminated.  The change in state means that no other rule can be applied if this rule is applied.  This rule corresponds to \crefrange{li:gcol:iflgreaterthanv}{li:gcol:returnsuccess} of the lower \texttt{Graph Colouring} function in \cref{code:gcol:graphcol}.

\item This rule is arguably the heart of the process, and runs in maximally-parallel mode --- meaning that every possible exploration from the currently extant \bo{} objects is performed simultaneously.  In each application, for every pre-existing \bo{} object, new ones are generated with a random colouration assigned to an unexplored node, so long as there exists an edge between those nodes in the graph \emph{and} the selected colours are not the same \emph{and} there does not already exist another \(m\) object within the current \bo{} object that represents another node connected to the newly chosen node with the same colour.  This rule begins in state \(s_2\) and ends in state \(s_2\).  This rule corresponds to \crefrange{li:gcol:makebstart}{li:gcol:makebfinish} of the lower \texttt{Graph Colouring} function in \cref{code:gcol:graphcol}.

The interaction of the \(v\) objects in the output and the first \gls{promoter} work in the same fashion as \(y \in V \setminus \texttt{Dom}(M)\) in the pseudocode of \cref{code:gcol:graphcol}.  That is, this rule selects an \(n\) inside the given \bo{}'s \(v\), but naturally avoids selecting an \(n\) that is already used in one of the \bo{}'s \(m\)s because they have already been removed from \(v\).

\item This rule is a `cleaning' rule, which removes all extant \bo{} objects, thus keeping the working space relatively clean.  Recall that in \gls{cps}, existing objects are removed, and new objects instantiated at the end of the step, with the net effect that extant objects may be used as \glspl{promoter}, but the new objects may not be used by any rules.  This means that, while this rule will eliminate the \bo{} objects used by \cpruleref{rule:gcol:rules:loop} to instantiate the next generation, said objects will not be eliminated until immediately after the application of \cpruleref{rule:gcol:rules:loop} and thus the two rules do not cause conflicts.  This rule begins in state \(s_2\) and ends in state \(s_2\).

The operation of this rule occurs implicitly in the pseudocode, and does not have corresponding lines.  There is no reference to the current set of \bo{}s, \(B\), carried forward into the next call of the pseudocode function.  Instead, \(B\) is replaced with \(B'\) for the next execution of the function, and \(B\) will be cleaned up by the system in whichever way it clears away objects once all references to them have been dropped.

\item This rule is the other possible termination rule.  It merely transitions to state \(s_4\), which is used to signal to the environment that no solution is possible.  The key to this rule is that, because it transitions to state \(s_4\), it is only applicable if none of the prior rules are applicable.  This effectively means that it can only apply if all \bo{} objects have been removed from the system (because the presence of at least one \bo{} would mean that one of the prior rules will apply), indicating that every possible combination explored so far has already found a colouring conflict and thus there is no possible valid solution.  This rule begins in state \(s_2\) and ends in state \(s_4\).  This rule corresponds to \crefrange{li:gcol:fail1}{li:gcol:fail2} of the lower \texttt{Graph Colouring} function in \cref{code:gcol:graphcol}.

\end{enumerate}

\subsubsection{Definition of Terms}\label{sec:gcol:termdefs}
\paragraph{Atoms}
\begin{description}
\cptermdef{\cpempty}{The standard \gls{cps} ``empty'' term symbol.}
\end{description}

\paragraph{Functors}
\begin{description}
\cptermdef{b}{Node exploration term.  This is a collection term, which holds information pertaining to the nodes explored so far on this path and the colours assigned to those nodes, and the remaining unexplored nodes in the graph.}
\cptermdef{c}{Colouring assignment term.  Records the colour assigned to a node for a particular exploration}
\cptermdef{e}{Edge node.  Lists two locations in the graph, if they have an edge connecting them.}
\cptermdef{k}{Colour term.  Stores one of the colours potentially to be used in colouring the graph under study.}
\cptermdef{m}{Explored node term.  Records a node which has been visited in this particular exploration of the graph, and which colour the node was assigned.}
\cptermdef{n}{Node term.  Merely records the label of a node.  Used as part of other terms in the rules.}
\cptermdef{s}{Starting node.  Stores the label of the intended root node of the exploration of the graph.}
\cptermdef{v}{Unexplored nodes term.  Holds \(n\) terms with the labels of all nodes in the graph as yet unexplored in that particular exploration of the graph.}
\end{description}

\paragraph{States}
\begin{description}
\cptermdef{s_1}{System starting state.  The node in the \(s\) term is given a random colouring, and exploration of the graph begins.}
\cptermdef{s_2}{Main loop state.  All exploration of the graph and assignation of potential colours to nodes occurs in this state.}
\cptermdef{s_3}{Result state.  The system ends in this state if a valid colouring has been found.}
\cptermdef{s_4}{Negative result state.  The system ends in this state if no valid colouring is possible.}
\end{description}

\subsection{\label{sec:gcol:complexity}Complexity}
This \namecref{sec:gcol:complexity} discusses the time and space complexity of the system, with an emphasis on the maximum for each because the minima are highly graph-dependent.

\subsubsection{\label{sec:gcol:timecomplexity}Time Complexity}
The \glspl{cps} requires at most \(|V| + 1\) steps, where \(|V|\) is the total number of nodes in the graph.  In the successful case it will require that many, but for graphs where there is no possible \(|K|\)-colouring it may terminate early.

\cpRuleref{rule:gcol:rules:init} requires one application.  Rules \cpruleref*{rule:gcol:rules:endsucc} and \cpruleref*{rule:gcol:rules:endfail} between them require one application.  Rules \cpruleref*{rule:gcol:rules:loop} and \cpruleref*{rule:gcol:rules:loopclean} both require at most \(|V| - 1\) applications, but the applications occur concurrently in all cases because there is no chance for a conflict of states.  In total, that simplifies to at most \(|V| + 1\) steps, giving a time complexity of \bigoh{|V|}.  This holds regardless of the relative sizes of \(K\) and \(V\), as all possible valid colourings are simultaneously explored for a given node.  It is trivial to create a valid colouring when \(|V| \leq |K|\).   When \(|V| > |K|\), \ie{}, the number of nodes in the graph is at least one greater than the number of colours available, the maximally-parallel nature of \cpruleref{rule:gcol:rules:loop} ensures that the running time does not depend on \(|K|\).

Furthermore, in cases where \(|V| > |K|\) the system requires \emph{a minimum} of \(|K| + 2\) steps before terminating.  At each step up to step \(i = |K|\), it will always be possible to choose a colour that does not conflict with any chosen so far, simply by selecting a colour that has not been used yet.  

At step \(i = |K| + 1\), however, in the case of a complete graph, no further \bo{} objects can be created because no matter what colour is selected for the next node, it is guaranteed to conflict with one of the colours selected previously.  This means that \cpruleref{rule:gcol:rules:loopclean} will remove all current \bo{} objects, but \cpruleref{rule:gcol:rules:loop} does not replace them with new ones.  Thus, at the next step, \cpruleref{rule:gcol:rules:endfail} will transition the system to state \(s_4\), signalling to the environment that no colouring is possible, and the evolution of the system will terminate.  As such, a lower bound on the number of steps can be determined as \(|K| + 2\).  Other graphs may require more steps to reach an answer, but as described above, never more than \(|V| + 1\).

This is consistent with the upper bound, when \(|V|\) is strictly larger than \(|K|\).  \(|V|\) and \(|K|\) are both natural numbers, so the smallest possible difference between them is 1.  Thus, \(|K| + 1\) can be at most equal to \(|V|\), and therefore \(|K| + 2\) at most can be equal to \(|V| + 1\).  \Ie{},
\begin{align*}
    |K| &< |V|\\
    \implies |K| + 1 &\leq |V|\\
    \implies |K| + 2 &\leq |V| + 1
\end{align*}

Thus, the time complexity has an upper bound of \bigoh{|V|}, and a lower bound of \(\Omega(min(|K|, |V|))\).

\subsubsection{Space Complexity}

The \emph{theoretical} maximum number of objects that can be present in the system at the end of each step from step \(i = 1\) onwards is given by \cref{eq:gcol:spacemax} below, which provides an exclusive upper bound.

\begin{equation}\label{eq:gcol:spacemax}
    \textsc{Space}_{\text{max}}(i) = \underbrace{\bigg(\frac{(|V| - 1)!}{(|V| - i)!} \times |K|^{i-1}\bigg)}_{b} + \underbrace{|K|}_{k} + \underbrace{|E|}_{e}
\end{equation}

The underbraces indicate what type of \gls{functor} each part of the above counts inside the \gls{tlc}.  In practice, the system will \emph{never} contain this many objects at once, due to the restriction on creating new \bo{}s where there would be conflicting colours.  For the purposes of the above, \(0!\) (which will occur when \(i = |V|\)) is defined to be equal to 1.

After the first step, there will always be precisely one \bo{} object in the \gls{tlc}, simply by the operation of \cpruleref{rule:gcol:rules:init}, but in each proceeding step the count will depend on the exact nature of the graph and colour set under consideration.  At the point that \(i = |V| - 1\), for each \bo{} there will be only one possible choice for the next node to explore (as there will only be one \(n\) remaining in each \bo{}'s \(v\)), but each one of those choices can potentially be made with each one of the system's colours, and thus while the ``base'' number of \bo{}s will remain the same between those two steps, the total maximum possible number of \bo{}s will increase by a factor of \(|K|\).

A complete graph gives the upper limit to the number of objects that can exist, because (ignoring the operation of the \gls{inhibitor} in \cpruleref{rule:gcol:rules:loop}) every node can be reached from any other given node, and thus it provides the widest possible search --- any non-complete graph will contain an incomplete set of edges between nodes, disallowing some potential explorations from taking place.  This gives an upper bound on space complexity proportional to the number of nodes and colours in the system under consideration, \ie{} \(\mathcal{O}(|V||K|)\).  Of course, due to the operation of the \gls{inhibitor} in \cpruleref{rule:gcol:rules:loop}, no more \bo{} objects will be created for a complete graph once \(i > |K|\).

\Cref{eq:gcol:wb,eq:gcol:w} provide formulae to calculate the total space requirements of an individual \bo{} and the overall system. These equations assume that the size of a \gls{functor} that contains only atoms, such as the colour \glspl{functor} \(k\) which contain only a given number of the counting atom, is constant regardless of how many atoms it contains.  This is perhaps unrealistic in a real biological system, but is appropriate for computer simulations, where fixed-bit-length representations are likely used to represent the various elements of the system.

The size of each \bo{} can be described by \cref{eq:gcol:wb}
\begin{equation}\label{eq:gcol:wb}
    s_b(b) = c_m |M| + (c_v + c_n) |v|
\end{equation} where the various \(c_x\) terms are coefficients specifying the size of each type of object.  \(c_n\) represents the size of \(n\) \glspl{functor}, \(c_k\) represents the size of \(k\) \glspl{functor}, \(c_m = c_n + c_k\), and \(|v|\) represents the number of nodes remaining in the \(v\) \gls{functor} inside each \bo{}, while \(|M|\) similarly represents the number of \(m\) \glspl{functor} in each \bo{}.

This in turn leads to \cref{eq:gcol:w}, describing the size of the entire system.

\begin{equation}\label{eq:gcol:w}
    S = c_k |K| + c_e |E| + c_t + \sum_{b \in B}s_b(b)
\end{equation} where \(c_e\) is the size of each edge object and \(c_t\) is the size of the \gls{tlc} itself.

If all the coefficients are assumed to be equal to 1, then \cref{eq:gcol:wb} simplifies to 

\begin{equation}
    s_b(b) = |M| + 2 |v|
\end{equation}

and \cref{eq:gcol:w} to

\begin{equation}
    S = |K| + |E| + \sum_{b \in B}s_b(b) + 1
\end{equation}

\subsection{\label{sec:gcol:termination}Termination and Correctness}
The behaviour of the system in regard to termination and correctness is examined below.  The discussion in this \namecref{sec:gcol:termination} assumes that the number of steps taken is tracked in a variable \(i\).  No such subcell appears in \cref{ruleset:gcol:rules} because the rules make no use of the step number.

\begin{theorem}\label{theorem:gcol:grow}
All \bo{} objects present in the \gls{tlc} hold a monotonically growing set \(M\) of \(m\) \glspl{functor}, which represent validly coloured partial spanning trees of the whole graph, and at step \(1 \leq i \leq |V|\) in the evolution of the system, each partial spanning tree will have a length equal to \(i\).  Likewise, the magnitude of the set of unexplored node labels in the \(v\) functor decreases monotonically, is equal to \(|V| - i\).
\end{theorem}

\begin{proof}
By induction.

\textbf{Base case:}
At step \(i = 1\), a single \bo{} object will have been created by operation of \cpruleref{rule:gcol:rules:init}.  Contained within this \bo{} object is a single \(m\) \gls{functor}, which pairs a node label with a colour label.  Thus, the size of \(M\), \(|M| = 1\), \(m \in M\).  By the same rule, the \(v\) functor will be preserved but have one fewer node labels, so the magnitude of the set of labels is \(|V| - 1\).

\textbf{Induction step:}
Assume the induction hypothesis holds for step \(n \geq 1\).  Now consider step \(n + 1 \leq |V|\).

% At step \(i > 1\), any pre-existing \bo{} objects are removed by action of \cpruleref{rule:gcol:rules:loopclean}, but are replaced by \cpruleref{rule:gcol:rules:loop} with new \bo{} objects which between them cover all possible valid coloured partial spanning trees rooted at the initial node so far, after adding one further \(m\) object to \(M\), so \(|M| \geq 2\).  Thus, at step two, the extant \bo{} objects will each contain two \(m\) \glspl{functor}, one describing the initial node, and the other describing another node of the graph that has been chosen at random from among those that have an edge in the graph connecting said node to the first node.  By operation of the same rule, the number of node labels contained inside the \(v\) \gls{functor}, itself inside each \bo{} object, will decrease by one, and thus the number of remaining unexplored nodes inside it will be equal to \(|V| - i\).

At step \(n + 1\), any pre-existing \bo{} objects are removed by action of \cpruleref{rule:gcol:rules:loopclean}, but are replaced by \cpruleref{rule:gcol:rules:loop} with new \bo{} objects which between them cover all possible valid coloured partial spanning trees rooted at the initial node so far, after adding one further \(m\) object to \(M\), so \(|M|_{n + 1} = |M|_{n} + 1\).  By operation of the same rule, the number of node labels contained inside the \(v\) \gls{functor}, itself inside each \bo{} object, will decrease by one, and thus the number of remaining unexplored nodes inside it will be equal to \(|V| - (n + 1)\).

The addition of one to \(M\) is a function of the \gls{rhs} and first \gls{promoter} of \cpruleref{rule:gcol:rules:loop}.  Through pattern matching, the \gls{promoter} selects exactly one \(m\) functor individually, and the entirety of the (possibly empty) remainder as the variable \(M\).  Those precise two terms then appear again inside the new \bo{} in the \gls{rhs}, along with one further \emph{new} \(m\) functor.  This guarantees that exactly one further \(m\) is added to \(M\).  The reduction of the size of the label set inside the \(v\) functor arises from the same two elements of \cpruleref{rule:gcol:rules:loop}.  A single node label is selected in the \gls{promoter} with the term \(\cpfunc{n}{Y}\), and the (possibly empty) remainder with the variable \(Z\).  The \(Z\) is then placed back into the \(v\) functor in the new \bo{} object, but \emph{not} the label represented by \(Y\), which is instead used in the creation of the new \(m\) functor.

% At step 3, there will be three \(m\) objects inside each \bo{} object, where there are edges in the graph connecting the represented nodes.  Due to the maximally-parallel nature of \cpruleref{rule:gcol:rules:loop}, every possible partial spanning tree of size \(i\) rooted at the initial node will be represented in a \bo{} object at step \(i\).

Assuming that a colouring is possible, the system proceeds in this same fashion for each following step, until step \(i = |V| + 1\) when the \(M\) sets inside the \bo{} objects contain full spanning trees, and the \(v\) \glspl{functor} will be empty.
\end{proof}

\begin{lemma}
When a valid colouring is possible, evolution of the system will \emph{always} terminate at step \(i = |V| + 1\)
\end{lemma}

\begin{proof}
At step \(i = |V| + 1\), \cpruleref{rule:gcol:rules:loop} can no longer be applied because there will no longer be any \(n\) \glspl{functor} inside the \(v\) \glspl{functor}, as per \cref{theorem:gcol:grow}.  More importantly, \cpruleref{rule:gcol:rules:endsucc}, which enjoys priority under the weak priority ordering, becomes applicable and is thus selected as the rule to apply, ending evolution of the system with one of the \(M\) colouring sets output to the environment.

If a colouring is not possible, then eventually every possible application of \cpruleref{rule:gcol:rules:loop} will be confounded by the \gls{inhibitor}, leading to \cpruleref{rule:gcol:rules:loopclean} removing all \bo{} objects without any more replacing them.  At this point (which can itself potentially occur at step \(i = |V| + 1\), such as in the case of a 4-node complete graph with 3 colours), \cpruleref{rule:gcol:rules:endfail} will terminate evolution of the system.
\end{proof}

\begin{lemma}\label{lemma:gcol:determin}
In all cases, evolution of the system proceeds wholly deterministically between the first and final steps, exclusively.
\end{lemma}

\begin{proof}
Due to the nature of Rules \cpruleref*{rule:gcol:rules:loop} and \cpruleref*{rule:gcol:rules:loopclean}, which are the only ones applied during the middle steps of the system's evolution, there is essentially no choice made at any point.  Rules \cpruleref*{rule:gcol:rules:loop} and \cpruleref*{rule:gcol:rules:loopclean} make a non-deterministic choice in an individual application of them, but the rules are applied in a maximally parallel fashion, meaning that all possible choices are selected simultaneously.  Thus, all valid choices are explored, while the \gls{inhibitor} ensures that no invalid choices are possible.

\cpRuleref{rule:gcol:rules:endfail} also applies if and only if none of the earlier rules are applicable.  Further, there is only one possible result from applying \cpruleref{rule:gcol:rules:endfail} -- transitioning the system to state \(s_4\) --, so there is no choice involved in its application.  Therefore, the final step of the system is also deterministic in all cases where a valid colouring is impossible.

\end{proof}

\begin{lemma}\label{lemma:gcol:nondet}

The system is only non-deterministic regarding the choice of the colour of the starting node and the final selection of which valid colouring to send out to the environment, assuming that the evolution of the system finishes in state \(s_3\).
\end{lemma}

\begin{proof}
The system is deterministic for all rule applications except the first, and potentially the last, by \cref{lemma:gcol:determin}.  Application of \cpruleref{rule:gcol:rules:init} to start the system is partially non-deterministic, specifically regarding the choice of the initial colour.  Application of \cpruleref{rule:gcol:rules:endsucc} to end the evolution of the system is likewise non-deterministic only in the choice of the set of colourings to send out.

\cpRuleref{rule:gcol:rules:init} uses the pre-selected starting node, encoded in the \(s\) \gls{functor}, to create the first \bo{} object with which to evolve the system.  While that \gls{functor} will be unique by the creation conditions described above,\footnote{Should it be desired, it would be possible to have the system make a random selection between possible starting nodes, simply by starting the system with more than one \(s\) \gls{functor}.  The correctness and termination of the system will be unaffected.} it has a free choice between the available colours.  As the rule is applied exactly once due to its application mode, only one colour is selected, but the specification of the colour \gls{promoter} for \cpruleref{rule:gcol:rules:init} simply states that one of the supplied \(k\) colour objects is used.

While this selection is non-deterministic, it has no impact upon the termination of the evolution of the system, because for the standard \gls{gcp} there is no functional difference between colours.  That is, they do not imbue special properties or behaviours to their marked nodes, they merely label them.  Thus, the colour selected does limit the potential colouring sets that are available at the end by fixing the colour of the starting node, but has no impact upon which rule is applied next.

Finally, as with the selection of a colour in \cpruleref{rule:gcol:rules:init}, the selection of which extant \bo{} object to take the colouring set \(M\) out of and send out to the environment by \cpruleref{rule:gcol:rules:endsucc} is non-deterministic, as \cpruleref{rule:gcol:rules:endsucc} has a free choice in which one it selects.  From its perspective, any \bo{} object with an empty \(v\) \gls{functor} is acceptable.
\end{proof}

\begin{lemma}\label{lemma:gcol:ending}
The termination of the evolution of the system by application of \cpruleref{rule:gcol:rules:endsucc} or \cpruleref{rule:gcol:rules:endfail} is also deterministic.
\end{lemma}

\begin{proof}
Both \cpruleref{rule:gcol:rules:endsucc} and \cpruleref{rule:gcol:rules:endfail} can only be applied in specific, mutually exclusive circumstances.  In particular, it is only possible to apply \cpruleref{rule:gcol:rules:endsucc} when there is at least one \bo{} object, which additionally must have an empty \(v\) \gls{functor}.  Conversely, \cpruleref{rule:gcol:rules:endfail} may only be applied where there are absolutely no extant \bo{} objects in the system.  The only way \bo{} objects can be removed is by operation of \cpruleref{rule:gcol:rules:loopclean}, which cannot be applied in the same step as either \cpruleref{rule:gcol:rules:endsucc} or \cpruleref{rule:gcol:rules:endfail}, as they have differing final states.  This means that \emph{only} one of \cpruleref{rule:gcol:rules:endsucc} or \cpruleref{rule:gcol:rules:endfail} may be applied during a step when either of them is applied, and both are considered to terminate the evolution of the system.  If there are any \bo{}s with an empty \(v\), then \cpruleref{rule:gcol:rules:endsucc} is applied.  If there are no \bo{}s whatsoever, then \cpruleref{rule:gcol:rules:endfail} is applied.  In all other cases after the first step, Rules \cpruleref*{rule:gcol:rules:loop} and \cpruleref*{rule:gcol:rules:loopclean} are applied.
\end{proof}

\begin{theorem}
The evolution of every valid instance of this system (where the system is created in accordance with \cref{sec:gcol:notation}, and \(|V|, |E|, |K| > 0\)) will terminate, either sending a valid colouring out to the environment where a valid colouring is possible, or signalling by its state that no valid colouring is possible.
\end{theorem}

\begin{proof}
By \cref{lemma:gcol:determin,lemma:gcol:nondet} the applications of the rules are deterministic in all ways, except the choice of the colour of the starting node, and the choice of the final selected colouring.  Thus, the progression of the system is always deterministic.  By \cref{lemma:gcol:ending}, the evolution of the system is \emph{always} terminated by \cpruleref{rule:gcol:rules:endsucc} when there exist any \bo{} objects with a full set of colourings, or by \cpruleref{rule:gcol:rules:endfail} \emph{only} when no colouring is possible.  A full set of colourings inside a \bo{} is ensured under \cref{theorem:gcol:grow}.
\end{proof}

% \subsection{\label{sec:gcol:complexity}Complexity}
% This \namecref{sec:gcol:complexity} discusses the time and space complexity of the system, with an emphasis on the maximum for each because the minima are highly graph-dependent.

% \subsubsection{Time Complexity}

% \begin{lemma}\label{lemma:gcol:timecomplexmax}
% The \gls{gcp} \glspl{cps} requires \emph{at most} \(|V| + 1\) steps, where \(|V|\) is the total number of nodes in the graph.
% \end{lemma}

% \begin{proof}
% \cpRuleref{rule:gcol:rules:init} requires one application.  Rules \cpruleref*{rule:gcol:rules:endsucc} and \cpruleref*{rule:gcol:rules:endfail} between them require one application.  Rules \cpruleref*{rule:gcol:rules:loop} and \cpruleref*{rule:gcol:rules:loopclean} both require at most \(|V| - 1\) applications, but the applications occur concurrently in all cases.  In total, that simplifies to at most \(|V| + 1\) steps.  This holds regardless of the relative sizes of \(K\) and \(V\), as all possible valid colourings are simultaneously explored for a given node.  It is trivial to create a valid colouring when \(|V| \leq |K|\).   When \(|V| > |K|\), \ie{}, the number of nodes in the graph is at least one greater than the number of colours available, then the maximally-parallel nature of \cpruleref{rule:gcol:rules:loop} ensures that the running time does not depend on \(|K|\).
% \end{proof}

% \begin{lemma}\label{lemma:gcol:timecomplexmin}
% The \gls{gcp} \glspl{cps} requires \emph{at least} the minimum of \(|V| + 1\) or \(|K| + 2\) steps, where \(|V|\) is the total number of nodes in the graph and \(|K|\) the total number of colours in use.
% \end{lemma}

% \begin{proof}
% Per \cref{lemma:gcol:timecomplexmax}, when \(|V| \leq |K|\) the system will halt after \(|V| + 1\) steps.  In cases where \(|V| > |K|\) the system requires \emph{a minimum} of \(|K| + 2\) steps before terminating.  At each step up to step \(i = |K|\), it will always be possible to choose a colour that does not conflict with any chosen so far, simply by selecting a colour that has not been used yet.  

% % At step \(i = |K| + 1\), however, in the case of a complete graph, no further \bo{} objects can be created because, no matter what colour is selected for the next node, it is guaranteed to conflict with one of the colours selected previously.  This means that \cpruleref{rule:gcol:rules:loopclean} will remove all current \bo{} objects, but \cpruleref{rule:gcol:rules:loop} does not replace them with new ones.  Thus, at the next step, \cpruleref{rule:gcol:rules:endfail} will transition the system to state \(s_4\), signalling to the environment that no colouring is possible, and the evolution of the system will terminate.  As such, a lower bound on the number of steps can be determined as \(|K| + 2\).%  Other graphs may require more steps to reach an answer, but as described above, never more than \(|V| + 1\).

% In the case of a complete graph, at step \(i = |K| + 1\) no further \bo{} objects can be created; regardless of what colour is selected for the next node, it is guaranteed to conflict with one of the colours selected previously.  Consequently, \cpruleref{rule:gcol:rules:loopclean} will remove all current \bo{} objects, but \cpruleref{rule:gcol:rules:loop} does not replace them with new ones.  Thus, at the next step, \cpruleref{rule:gcol:rules:endfail} will transition the system to state \(s_4\), signalling to the environment that no colouring is possible, and the evolution of the system will terminate.  As such, a lower bound on the number of steps can be determined as \(|K| + 2\).%  Other graphs may require more steps to reach an answer, but as described above, never more than \(|V| + 1\).
% \end{proof}

% \begin{theorem}\label{theorem:gcol:earlytermination}
% In the successful case the \gls{gcp} \glspl{cps} will always require \(|V| + 1\) steps, but for graphs where there is no possible \(|K|\)-colouring it may terminate early.
% \end{theorem}

% \begin{proof}
% Consequence of \cref{lemma:gcol:timecomplexmax,lemma:gcol:timecomplexmin}. This is consistent with the upper bound, when \(|V|\) is strictly larger than \(|K|\).  \(|V|\) and \(|K|\) are both natural numbers, so the smallest possible difference between them is 1.  Thus, \(|K| + 1\) can be at most equal to \(|V|\), and therefore \(|K| + 2\) at most can be equal to \(|V| + 1\).  \Ie{},
% \begin{align*}
%     |K| &< |V|\\
%     \implies |K| + 1 &\leq |V|\\
%     \implies |K| + 2 &\leq |V| + 1
% \end{align*}
% \end{proof}

% \begin{theorem}
% The \gls{gcp} \glspl{cps} has an upper-bound time complexity of \bigoh{|V|} and a lower-bound time complexity of \(\Omega(\min{(|K|, |V|)})\)
% \end{theorem}

% \begin{proof}
% Direct consequence of \cref{theorem:gcol:earlytermination}.
% \end{proof}

% \subsubsection{Space Complexity}

% % The \emph{theoretical} maximum number of objects that can be present in the system at the end of each step from step 1 onwards is given by \cref{eq:gcol:spacemax} below, which provides an exclusive upper bound.

% % \begin{equation}\label{eq:gcol:spacemax}
% %     \textsc{Space}_{\text{max}}(i) = \underbrace{\bigg(\frac{(|V| - 1)!}{(|V| - i)!} \times |K|^{i-1}\bigg)}_{b} + \underbrace{|K|}_{k} + \underbrace{|E|}_{e}
% % \end{equation}

% % The underbraces indicate what type of \gls{functor} each part of the above counts inside the \gls{tlc}.  In practice, the system will \emph{never} contain this many objects at once, due to the restriction on creating new \bo{}s where there would be conflicting colours.  For the purposes of the above, \(0!\) (which will occur when \(i = |V|\)) is defined to be equal to 1.

% \begin{theorem}
% The maximum number of objects that can be present in the system at the end of each step \(i = 1\) onwards is given by \cref{eq:gcol:spacemax}, which provides an upper bound.

% \begin{equation}\label{eq:gcol:spacemax}
%     \textsc{Space}_{\text{max}}(i) = \underbrace{\bigg(\frac{(|V| - 1)!}{(|V| - i)!} \times |K|^{i-1}\bigg)}_{b} + \underbrace{|K|}_{k} + \underbrace{|E|}_{e} + \underbrace{1}_{t}
% \end{equation}
% \end{theorem}

% \begin{proof}
% The underbraces indicate what type of \gls{functor} each part of \cref{eq:gcol:spacemax} counts inside the \gls{tlc} (where \(t\) represents the \gls{tlc} itself).  For the purposes of \cref{eq:gcol:spacemax}, \(0!\) (which will occur when \(i = |V|\)) is defined to be equal to 1.

% Always, after the first step, there will be precisely one \bo{} object in the \gls{tlc}, simply by the operation of \cpruleref{rule:gcol:rules:init}, but in each succeeding step the count will depend on the exact nature of the graph and colour set under consideration.  The true maximum will only be observed for complete graphs, and then only for steps \(i \leq |K| + 1\).

% At the point that \(i = |V| - 1\) and there exists still at least one \bo{}, for each \bo{} there will be only one possible choice for the next node to explore (as there will only be one \(n\) remaining in each \bo{}'s \(v\)).  Each one of those choices can potentially be made with each one of the system's colours, and thus while the “base'' number of \bo{}s will remain the same between those two steps, the total maximum possible number of \bo{}s will increase by a factor of \(|K|\).

% % A complete graph gives the upper limit to the number of objects that can exist because (ignoring the operation of the \gls{inhibitor} in \cpruleref{rule:gcol:rules:loop}) every node can be reached from any other given node, and thus it provides the widest possible search -- any non-complete graph will contain an incomplete set of edges between nodes, disallowing some potential explorations from taking place.  This gives an upper bound on space complexity proportional to the number of nodes and colours in the system under consideration, \ie{}. \(\mathcal{O}(|V||K|)\).  Of course, due to the operation of the \gls{inhibitor} in \cpruleref{rule:gcol:rules:loop}, no more \bo{} objects will be created for a complete graph once \(i > |K|\).

% A complete graph gives the upper limit to the number of objects that can exist because (ignoring the operation of the \gls{inhibitor} in \cpruleref{rule:gcol:rules:loop}) every node can be reached from any other given node, and thus it provides the widest possible search --- any non-complete graph will contain an incomplete set of edges between nodes, disallowing some potential explorations from taking place.  This gives an upper bound on space complexity proportional to the number of nodes and colours in the system under consideration, \ie{}. \bigoh{|V| |K|}.
% \end{proof}

% \begin{corollary}
% Due to the operation of the \gls{inhibitor} in \cpruleref{rule:gcol:rules:loop}, no more \bo{} objects will be created for a complete graph once \(i > |K|\).  Thus, for problem instances where \(|K| < |V|\), \cref{eq:gcol:spacemax} provides a \emph{theoretical} maximum space complexity for larger values of \(i\).
% \end{corollary}

% \begin{remark}
% \Cref{eq:gcol:wb,eq:gcol:w} provide formulae to calculate the total space requirements of an individual \bo{} and the overall system. These equations assume that the size of a \gls{functor} that contains only atoms, such as the colour \glspl{functor} \(k\) which contain only a given number of the counting atom, is constant regardless of how many atoms it contains.  This is perhaps unrealistic in a real biological system, but is appropriate for electronic computer simulations, where fixed-bit-length representations are likely used to represent the various elements of the system.

% \begin{equation}\label{eq:gcol:wb}
%     s_b(b) = c_m |M| + (c_v + c_n) |v|
% \end{equation}

% The size of each \bo{} can be described by \cref{eq:gcol:wb}, where the various \(c_x\) terms are coefficients specifying the size of each type of object.  \(c_n\) represents the size of \(n\) \glspl{functor}, \(c_k\) represents the size of \(k\) \glspl{functor}, \(c_m = c_n + c_k\), and \(|v|\) represents the number of nodes remaining in the \(v\) \gls{functor} inside each \bo{}, while \(|M|\) similarly represents the number of \(m\) \glspl{functor} in each \bo{}.

% This in turn leads to \cref{eq:gcol:w}, describing the size of the entire system.

% \begin{equation}\label{eq:gcol:w}
%     S = c_k |K| + c_e |E| + c_t + \sum_{b \in B}s_b(b)
% \end{equation} where \(c_e\) is the size of each edge object and \(c_t\) is the size of the \gls{tlc} itself.

% If all the coefficients are assumed to be equal to 1, then \cref{eq:gcol:wb} simplifies to 

% \begin{equation}
%     s_b(b) = |M| + 2 |v|
% \end{equation}

% and \cref{eq:gcol:w} to

% \begin{equation}
%     S = |K| + |E| + 1 + \sum_{b \in B}s_b(b)
% \end{equation}
% \end{remark}

% \subsection{\label{sec:gcol:termination}Progression, Termination, and Correctness}
% The behaviour of the system concerning its progress, termination, and correctness is examined below.  The discussion in this \namecref{sec:gcol:termination} assumes that the number of steps taken is tracked in a variable \(i\).  No such subcell appears in \cref{ruleset:gcol:rules} because the rules do not need it.

% % \begin{lemma}\label{lemma:gcol:monotonicm}
% % Until the system reaches state \(s_3\) or \(s_4\), the set \(M\) of \(m\) functors, which represent validly coloured partial spanning trees of the whole graph, grows monotonically in each \bo{}.
% % \end{lemma}

% % \begin{proof}
% % At step 1, a single \bo{} object will have been created by operation of \cpruleref{rule:gcol:rules:init}.  Contained within this \bo{} object is a single \(m\) \gls{functor}, which pairs a node label with a colour label.  Thus, the size of \(M\), \(|M| = 1\), \(m \in M\).
% % \end{proof}

% \begin{lemma}\label{lemma:gcol:monotonicm}
% Until the system reaches state \(s_3\) or \(s_4\), the set \(M\) of \(m\) functors grows monotonically in each \bo{}.
% \end{lemma}

% \begin{proof}
% The operation of \cpruleref{rule:gcol:rules:init} creates a single \bo{} object, which holds a single \(m\) \gls{functor}.  Until reaching state \(s_3\), there is \emph{no} method to remove an \(m\) \glspl{functor} from within a \bo{}, but \cpruleref{rule:gcol:rules:loop} adds one further \(m\) to every remaining \bo{} at each step.
% \end{proof}

% % \begin{remark}
% % One could perhaps argue that there is one exception to \cref{lemma:gcol:monotonicm}:  The \bo{} objects which are removed from a system entirely.  The sets \(M\) in those \glspl{functor} could conceivably be argued as shrinking to magnitude 0 by operation of rule (4), thus violating the condition of monotonic growth.  This work takes the alternative viewpoint that, with the elimination of the containing \bo{}s, the sets entirely cease to exist, and therefore attempting to ascribe a magnitude to them is meaningless.
% % \end{remark}

% % \begin{theorem}\label{theorem:gcol:grow}
% % All \bo{} objects present in the \gls{tlc} hold a monotonically growing set \(M\) of \(m\) \glspl{functor}, which represent validly coloured partial spanning trees of the whole graph; and, at step \(1 \leq i \leq \min{(|K|, |V|)}\) in the evolution of the system, each partial spanning tree will have a length equal to \(i\).
% % \end{theorem}

% \begin{theorem}
% The set \(M\) of \(m\) functors in a \bo{} object correspond to partial spanning trees of the graph to be coloured.
% \end{theorem}

% \begin{proof}
% A spanning tree is, by definition, a tree over an underlying connected graph that includes every node, but only sufficient edges with which to connect every node \emph{without cycles}.  A partial spanning tree is any connected subset of the nodes and edges of a spanning tree.  Therefore, when incrementally increasing the size of the tree, the next node and edge to add \emph{must} connect to a pre-existing part of the partial spanning tree, but without introducing a cycle.

% The \gls{rhs} and first two \glspl{promoter} of \cpruleref{rule:gcol:rules:loop} ensure that the next node (label) added to \(M\) is connected to the existing graph.  As described in \cref{sec:gcol:termdefs}, the \(e\) functors represent the extant edges between nodes in the problem graph.  By sharing common variables in the definition of \cpruleref{rule:gcol:rules:loop} with the \(e\) functor, both the \gls{rhs} and first promoter are required to select only those nodes where an edge exists between them, with one of the two nodes involved already a part of the graph.  This ensures that the set remains connected.

% The depletion of the \(v\) functor as each node is introduced into the tree likewise prevents cycles.  When adding edges incrementally, a cycle can only arise in the circumstance that an edge is instantiated between two nodes which are already members of the tree.  The shrinking set in the \(v\) \gls{functor} guarantees that the next edge introduced connects to a node which is not yet in the tree.  %Proof by contradiction?
% \end{proof}

% \begin{theorem}\label{theorem:gcol:grow}
% % At step \(1 \leq i \leq \min{(|K|, |V|)}\) in the evolution of the system, \(|M| = i\).  That is, each partial spanning tree will have a length equal to \(i\).
% At step \(1 \leq i \leq \min{(|K|, |V|)}\) in the evolution of the system, \(|M| = i\).  That is, each partial spanning tree represented by a set \(M\) will have a length equal to \(i\).
% \end{theorem}

% \begin{proof}
% % By induction on the step number, \(i\).  There are two aspects to prove here.  Firstly, that the size of the set \(M\) does indeed grow monotonically, and secondly that the sets \(M\) are partial spanning trees of the whole graph with length equal to \(i\).

% % \textbf{Base case:}    Let \(i = 1\).  At step 1, a single \bo{} object will have been created by operation of \cpruleref{rule:gcol:rules:init}.  Contained within this \bo{} object is a single \(m\) \gls{functor}, which pairs a node label with a colour label.  Thus, the size of \(M\), \(|M| = 1\), \(m \in M\).  A subgraph of one node, such as \(M\), is trivially a partial spanning tree.

% % \textbf{Induction step:}  Assume the induction hypothesis holds at step \(n\), \(1 \leq n < \min{(|K|, |V|) - 1}\).  Now consider step \(n + 1\).  At step \(n + 1\), any pre-existing \bo{} objects are removed, but are replaced with new \bo{} objects which between them cover all possible valid coloured partial spanning trees rooted at the initial node so far, after adding one further \(m\) object to \(M\), so \(|M| \geq 2\) by operation of \cpruleref{rule:gcol:rules:loop} in \cref{ruleset:gcol:rules}.  In fact, \(|M| = n + 1\) because \cpruleref{rule:gcol:rules:loop} is the only way to add a new \(m\) object to the set \(M\), and it only ever adds one new \(m\) in a given application.  The first promoter ensures this happens, because \bo{}s which don't meet the requirements are not instantiated.  Furthermore, there is \emph{no} rule to remove any \(m\) from a \bo{} besides \cpruleref{rule:gcol:rules:endsucc}, which only applies at step \(|V| + 1\) and removes the entire set \(M\), thus the size of \(M\) cannot decrease during any earlier step.

% % By operation of the same rule, the number of node labels contained inside the \(v\) \gls{functor}, itself inside each \bo{} object, will decrease by one, and thus the number of remaining unexplored nodes inside it will be equal to \(|V| - (n + 1)\).

% By induction on the step number, \(i\).%  There are two aspects to prove here.  Firstly, that the size of the set \(M\) does indeed grow monotonically, and secondly that the sets \(M\) are partial spanning trees of the whole graph with length equal to \(i\).

% \textbf{Base case:}    Let \(i = 1\).  At step 1, a single \bo{} object will have been created by operation of \cpruleref{rule:gcol:rules:init}.  Contained within this \bo{} object is a single \(m\) \gls{functor}, which pairs a node label with a colour label.  Thus, \(|M| = i = 1\).%  A subgraph of one node, such as \(M\), is trivially a partial spanning tree.

% \textbf{Induction step:}  Assume the induction hypothesis holds at step \(n\), \(1 \leq n < \min{(|K|, |V|)}\).  By \cref{lemma:gcol:monotonicm}, \(|M|\) is guaranteed not to decrease.  Thus, it will suffice to show that it increases by 1 for each step.  By \cref{lemma:gcol:timecomplexmin,lemma:gcol:timecomplexmax}, step \(n\) will be an application of Rules \cpruleref*{rule:gcol:rules:loop} and \cpruleref*{rule:gcol:rules:loopclean}.  While \cpruleref{rule:gcol:rules:loopclean} removes all the current \bo{}s, \cpruleref{rule:gcol:rules:loop} replaces them with updated copies where a valid expansion is possible.  The promoter of the latter rule uses pattern matching to select a single \(m\) functor from the set \(M\) along with the remainder of the set (expressed through the variable \(M\)).  It then adds those same objects back into the new \bo{}, plus exactly one new \(m\) functor. Thus, at each step \(n\), \(|M|\) increases by one.  %Now consider step \(n + 1\).  At step \(n + 1\), any pre-existing \bo{} objects are removed, but are replaced with new \bo{} objects which between them cover all possible validly coloured partial spanning trees rooted at the initial node so far, after adding one further \(m\) object to \(M\), so \(|M| \geq 2\) by operation of \cpruleref{rule:gcol:rules:loop} in \cref{ruleset:gcol:rules}.  In fact, \(|M| = n + 1\) because \cpruleref{rule:gcol:rules:loop} is the only way to add a new \(m\) object to the set \(M\), and it only ever adds one new \(m\) in a given application.  The first promoter ensures this happens, because \bo{}s which don't meet the requirements are not instantiated.  Furthermore, there is \emph{no} rule to remove any \(m\) from a \bo{} besides \cpruleref{rule:gcol:rules:endsucc}, which only applies at step \(|V| + 1\) and removes the entire set \(M\), thus the size of \(M\) cannot decrease during any earlier step.

% % By operation of the same rule, the number of node labels contained inside the \(v\) \gls{functor}, itself inside each \bo{} object, will decrease by one, and thus the number of remaining unexplored nodes inside it will be equal to \(|V| - (n + 1)\).

% % At step one, a single \bo{} object will have been created by operation of \cpruleref{rule:gcol:rules:init}.  Contained within this \bo{} object is a single \(m\) \gls{functor}, which pairs a node label with a colour label.  Thus, the size of \(M\), \(|M| = 1\), \(m \in M\).

% % At step \(i > 1\), any pre-existing \bo{} objects are removed, but are replaced with new \bo{} objects which between them cover all possible valid coloured partial spanning trees rooted at the initial node so far, after adding one further \(m\) object to \(M\), so \(|M| \geq 2\).  Thus, at step two, the extant \bo{} objects will each contain two \(m\) \glspl{functor}, one describing the initial node, and the other describing another node of the graph that has been chosen at random from among those that have an edge in the graph connecting said node to the first node.  By operation of the same rule, the number of node labels contained inside the \(v\) \gls{functor}, itself inside each \bo{} object, will decrease by one, and thus the number of remaining unexplored nodes inside it will be equal to \(|V| - i\).

% % At step 3, there will be three \(m\) objects inside each \bo{} object, where there are edges in the graph connecting the represented nodes.  Due to the maximally-parallel nature of \cpruleref{rule:gcol:rules:loop}, every possible partial spanning tree of size \(i\) rooted at the initial node will be represented in a \bo{} object at step \(i\).  Assuming that a colouring is possible, the system proceeds in the same fashion for each following step, until step \(i = |V| + 1\) when the \(M\) sets inside the \bo{} objects contain full spanning trees, and the \(v\) \glspl{functor} will be empty.

% % At that point, \cpruleref{rule:gcol:rules:loop} can no longer be applied (because there will no longer be any \(n\) \glspl{functor} inside the \(v\) \glspl{functor}).  More importantly, \cpruleref{rule:gcol:rules:endsucc}, which enjoys priority under the weak priority ordering, becomes applicable and is thus selected as the rule to apply, ending evolution of the system with one of the \(M\) colouring sets output to the environment.

% % If a colouring is not possible, then eventually every possible application of \cpruleref{rule:gcol:rules:loop} will be confounded by the \gls{inhibitor}, leading to \cpruleref{rule:gcol:rules:loopclean} removing all \bo{} objects without any more replacing them.  At this point (which can itself potentially occur at step \(i = |K| + 1\), such as in the case of a 4-node complete graph with 3 colours), \cpruleref{rule:gcol:rules:endfail} will terminate evolution of the system.
% \end{proof}

% % \begin{remark}
% % In fact, in all cases where a valid colouring is possible, the length of the spanning tree represented by \(M\) will be equal to the step number \(i \in [1, |V|]\).  In all cases where no colouring is possible, the length will always be equal to the step number \(i \in [1, |K|)\).
% % \end{remark}

% \begin{lemma}\label{lemma:gcol:determin}
% In all cases, evolution of the system proceeds wholly deterministically between the first and final steps, exclusively.
% \end{lemma}

% \begin{proof}
% Due to the nature of Rules \cpruleref*{rule:gcol:rules:loop} and \cpruleref*{rule:gcol:rules:loopclean}, which are the only ones applied during the middle steps of the system's evolution, there is essentially no choice made at any point.  Rules \cpruleref*{rule:gcol:rules:loop} and \cpruleref*{rule:gcol:rules:loopclean} make a non-deterministic choice in an individual application of them, but the rules are applied in a maximally parallel fashion, meaning that all possible choices are selected concurrently.  Thus, all valid choices are explored, while \cpruleref{rule:gcol:rules:loop}'s \gls{inhibitor} ensures that no invalid choices are possible.
% \end{proof}

% % \begin{remark}\label{remark:gcol:endfaildetermin}
% % \cpRuleref{rule:gcol:rules:endfail} is also deterministic.  It applies if and only if none of the earlier rules are applicable.  Further, there is only one possible result from applying \cpruleref{rule:gcol:rules:endfail} -- transitioning the system to state \(s_4\) -- so there is no choice involved in its application.  Therefore, the final step of the system is also deterministic when no valid colouring is possible.
% % \end{remark}

% \begin{lemma}\label{lemma:gcol:endfaildetermin}
% The final step of the system is also deterministic when no valid colouring is possible.
% \end{lemma}

% \begin{proof}
% \cpRuleref{rule:gcol:rules:endfail} is also deterministic.  It applies if and only if none of the earlier rules are applicable and no valid colouring is possible.  Further, there is only one possible result from applying \cpruleref{rule:gcol:rules:endfail} -- transitioning the system to state \(s_4\) -- so there is no choice involved in its application.
% \end{proof}

% \begin{theorem}\label{theorem:gcol:nondet}
% The system is only non-deterministic regarding the choice of the colour of the starting node and the final selection of which valid colouring to send out to the environment, assuming that the evolution of the system finishes in state \(s_3\).
% \end{theorem}

% \begin{proof}
% The system is deterministic for all rule applications except the first, and potentially the last, by \cref{lemma:gcol:determin,lemma:gcol:endfaildetermin}.  Application of \cpruleref{rule:gcol:rules:init} to start the system is partially non-deterministic, specifically regarding the choice of the initial colour.  Application of \cpruleref{rule:gcol:rules:endsucc} to end the evolution of the system is likewise non-deterministic only in the choice of the set of colourings to send out.

% \cpRuleref{rule:gcol:rules:init} uses the pre-selected starting node, encoded in the \(s\) \gls{functor}, to create the first \bo{} object with which to evolve the system.  While that \gls{functor} will be unique by the creation conditions described above,\footnote{Should it be desired, it would be possible to have the system make a random selection between possible starting nodes, simply by starting the system with more than one \(s\) \gls{functor}.  The correctness and termination of the system will be unaffected.} it has a free choice between the available colours.  As the rule is applied exactly once due to its application mode, only one colour is selected, but the specification of the colour \gls{promoter} for \cpruleref{rule:gcol:rules:init} simply states that one of the supplied \(k\) colour objects is used.

% While this selection is non-deterministic, it has no impact upon the termination of the evolution of the system, because for the standard \gls{gcp} there is no functional difference between colours.  That is, they do not imbue special properties or behaviours to their marked nodes, they merely label them.  Thus, the colour selected does limit the potential colouring sets that are available at the end by fixing the colour of the starting node, but has no impact upon which rule is applied next.

% Finally, as with the selection of a colour in \cpruleref{rule:gcol:rules:init}, the selection of which extant \bo{} object whence to take the colouring set \(M\) and send it out to the environment by \cpruleref{rule:gcol:rules:endsucc} is non-deterministic, as \cpruleref{rule:gcol:rules:endsucc} has a free choice in which one it selects.  From its perspective, any \bo{} object with an empty \(v\) \gls{functor} is acceptable.
% \end{proof}

% \begin{lemma}\label{lemma:gcol:ending}
% The termination of the evolution of the system by application of \cpruleref{rule:gcol:rules:endsucc} or \cpruleref{rule:gcol:rules:endfail} is also deterministic.
% \end{lemma}

% \begin{proof}
% Both \cpruleref{rule:gcol:rules:endsucc} and \cpruleref{rule:gcol:rules:endfail} can only be applied in specific, mutually exclusive circumstances.  In particular, it is only possible to apply \cpruleref{rule:gcol:rules:endsucc} when there is at least one \bo{} object, which additionally must have an empty \(v\) \gls{functor}.  Conversely, \cpruleref{rule:gcol:rules:endfail} may only be applied where there are absolutely no extant \bo{} objects in the system.  The only way \bo{} objects can be removed is by operation of \cpruleref{rule:gcol:rules:loopclean}, which cannot be applied in the same step as either \cpruleref{rule:gcol:rules:endsucc} or \cpruleref{rule:gcol:rules:endfail}, as they have differing final states.  This means that \emph{only} one of \cpruleref{rule:gcol:rules:endsucc} or \cpruleref{rule:gcol:rules:endfail} may be applied during a step when either of them is applied, and both are considered to terminate the evolution of the system.  If there are any \bo{}s with an empty \(v\), then \cpruleref{rule:gcol:rules:endsucc} is applied.  If there are no \bo{}s whatsoever, then \cpruleref{rule:gcol:rules:endfail} is applied.  In all other cases after the first step, Rules \cpruleref*{rule:gcol:rules:loop} and \cpruleref*{rule:gcol:rules:loopclean} are applied.
% \end{proof}

% \begin{theorem}
% The evolution of every valid instance of this system (where the system is created in accordance with \cref{sec:gcol:notation}, and \(|V|, |E|, |K| > 0\)) will terminate, either sending a valid colouring out to the environment where a valid colouring is possible, or signalling by its state that no valid colouring is possible.
% \end{theorem}

% \begin{proof}
% By \cref{lemma:gcol:determin,lemma:gcol:endfaildetermin,theorem:gcol:nondet} the applications of the rules are deterministic in all ways, except the choice of the colour of the starting node, and the choice of the final selected colouring.  Thus, the progression of the system is always deterministic.  By \cref{lemma:gcol:ending}, the evolution of the system is \emph{always} terminated by \cpruleref{rule:gcol:rules:endsucc} when there exist any \bo{} objects with a full set of colourings, or by \cpruleref{rule:gcol:rules:endfail} \emph{only} when no colouring is possible.  A full set of colourings inside a \bo{} is ensured under \cref{theorem:gcol:grow}.
% \end{proof}