\section{Comparison with Other \glsfmtname{mc} Solutions}

There have been a small number of papers address the \gls{gcp}.  Many of those, however, have addressed only the 3-colouring sub-problem, rather than the general \gls{gcp}.  \Cref{sec:gcol:gcpsol} addresses the papers which solve the general \gls{gcp}, summarised in \cref{tab:gcol:gcolalgocomp}.  \Cref{sec:gcol:3colsol} addresses the papers which solve the 3-colouring problem specifically.  Given the importance of \cite{Gheorghe2013} to this \namecref{chap:gcol}, \cref{sec:gcol:skpcomp} is dedicated to comparing it with the current work.% it is given a dedicated \namecref{sec:gcol:skpcomp} where it and this \namecref{chap:gcol}'s solution are compared in greater detail.

\subsection{\label{sec:gcol:skpcomp}Comparison with the Simple Kernel P Systems Solution}
The \gls{cps} solution requires only five rules, which are \emph{invariant} to the problem graph under study, as opposed to defining a family of rules that are customised to the graph at hand.  Likewise for the alphabet.  The alphabet consists of only 9 \glspl{functor} and the \gls{cps} `empty' atom (\(\cpempty\)), but with the nesting of atoms and \glspl{functor} these combine into further forms.  As the semantic meaning of each never changes, this does not add a great deal of complexity.  The \gls{cps} solution requires roughly equivalent starting objects to the \gls{skps} solution.

\begin{table}
\centering
\begin{tabular}{@{}lcc@{}}
\toprule
Type/specification                & \gls{skps}        & \gls{cps} \\ \midrule
Alphabet                          & \(|V|(|V|-1)/2 + 7|V| + 10\) & \(10\)         \\
Rules                             & \(2|V|~\&~2|V| + 7\)       & 6          \\
Max. \# of subcompartments & \(3^|V| + 1\)             & N/A          \\
Number of steps                   & \(2|V| + 3\)             & \(\leq |V| + 1\)         \\ \bottomrule
\end{tabular}%
\caption[\Glsxtrlong{skps} vs. \gls{cps} comparison]{\glsfmtname{skps} vs. \gls{cps}.  \gls{cps} does not have a concept of subcompartments in the same way as \gls{skps}, and thus has no entry for the maximum number of subcompartments.  This distinction is discussed in \cref{sec:gcol:diffsubcomp}}
\label{tab:gcol:skpcomp}
\end{table}

\Cref{tab:gcol:skpcomp} provides a comparison of the \gls{skps} solution presented in \cite{Gheorghe2013} compared with the solution presented above, and is based upon Table 1 in \cite{Gheorghe2013}.  

\subsubsection{\label{sec:gcol:diffsubcomp}Differing Concepts of Subcompartments}
Atoms are simple objects, but one could perhaps argue that \glspl{functor} should be regarded as somewhere between full \gls{skps} subcompartments and simple objects because they are non-atomic and can potentially hold objects (including other \glspl{functor}) within themselves.  The key difference between subcompartments and \glspl{functor} is that the former have rules of their own and typically evolve separately to their containing \gls{compartment}, whereas the latter have \emph{no} computing/evolutionary power of their own, and are simply nested objects operated upon by their \gls{tlc}'s rules.  A subcompartment could potentially have all its parent \glspl{compartment} removed and still function as though it now has the skin membrane, but not so \glspl{functor}.  This means that \gls{cps} does not have a concept of subcompartments completely comparable to \gls{skps}', with instead the \glspl{functor} in use being closer to normal inanimate objects.

\subsubsection{Rules Length}
The ``price'' for reducing the number of rules and symbols involved in the system is arguably making the rules themselves more complex.\footnote{These rules are fairly easy to understand nevertheless --- they are simply longer and use more symbols than those of other systems.  Each rule's specification is individually more complex, but their instantiations and interactions are much less so than seen in many other \gls{ps}.}  Much the same as the \gls{skps}~system in \cite{Gheorghe2013} allows for a \textcquote[][p.~829]{Gheorghe2013}{more succinct (in terms of number of rules, objects, number of cells and execution steps) \textelp{specification than an earlier \glspl{tlps} from \cite{Diaz-Pernil2008}} at the expense of a greater length of the rules}, so too does the \glspl{cps} again allow for a more compact solution to the problem in exchange for longer rules.

If one chooses to exclude the guards specified, then the longest rule in \cite{Gheorghe2013} has a length of \(5\).  Including the guards, however, (which seems more realistic), then the longest \gls{skps} rule has a length of \(3 + 3|V|(|V| - 1)/2\) while the longest rule in \cref{ruleset:gcol:rules}, \cpruleref{rule:gcol:rules:loop}, has a \emph{fixed} length of 14, determined by counting the number of distinct atoms, \glspl{functor} and variables that appear in it.  This is regarded as an acceptable price to pay, considering what this enables the system to achieve otherwise.  Which style is most appropriate fundamentally depends on the specifics of a given situation, and the relative costs of the alphabet size, rule length etc. in implementing a system in a chosen form.

%%%%%%%%%%%%%%%%%%%%%%%%%%%%%%%%%%%%%%%%%%%%%%%%%%%%%%%%%%%%%%%%%%%%%%%%%%%%%%%%%%%%%%%%%%%%%%%%%%%%%%

\subsection{\label{sec:gcol:gcpsol}Other Solutions to the \glsfmtname{gcp-glossary}}

While there have been a number of publications discussing the 3-colouring problem (\cref{sec:gcol:3colsol}), there have been surprisingly few which work on the general \gls{gcp}.  It is unclear why this should be the case, but it limits the number of direct comparisons which can be made.  These comparisons (where the proposed solution uses only \gls{ps} concepts) are summarised in \cref{tab:gcol:gcolalgocomp}.

\begin{table}
\centering
\begin{tabular}{llccc} 
\toprule
\textbf{Algorithm} & \begin{tabular}[c]{@{}l@{}}\textbf{P systems}\\\textbf{type}\end{tabular} & \begin{tabular}[c]{@{}c@{}}\textbf{Num. of}\\\textbf{rules}\end{tabular} & \begin{tabular}[c]{@{}c@{}}\textbf{Size of}\\\textbf{alphabet}\end{tabular} & \begin{tabular}[c]{@{}c@{}}\textbf{Time}\\\textbf{complexity}\end{tabular}  \\ 
\midrule
\cite{Gheorghe2013} & \gls{skps} & 10 & 14 & \bigoh{n}\\
\cite{Tanaka2012} & asynchronous \gls{ps} & 23 & 13 & \bigoh{n^2}\\
\cite{Umetsu2019} & asynchronous \gls{ps} & 28 & 15 & \bigoh{n^2}\\
This work & \gls{cps} & 6 & 10 & \bigoh{n}\\
\bottomrule
\end{tabular}
\caption[Comparison of known \gls{ps} solutions to the \glsentrylong{gcp-glossary}.]{Comparison of known \gls{ps} solutions to the \glsentrylong{gcp-glossary}.  \(n\) is the number of nodes in the graph.}
\label{tab:gcol:gcolalgocomp}
\end{table}

%%%%%%%%%%%%%%%%%%%%%%%%%%%%%%%%%%%%%%%%%%%%%%%

\citeauthor{Tanaka2012} \cite{Tanaka2012} present asynchronous \gls{ps} to solve the \gls{gcp} for a \(k\)-colouring scenario, and to find the \emph{minimum} number of colours required for a given graph. Curiously, the \(k\)-colouring solution is another instance where the system presented is a recogniser system, even though all the information required to describe a solution is generated during the system's evolution.  The overall minimising approach is relatively similar to \cref{sec:gcol:cpsys}, with an extra step at the end to select a colour using the minimum number of colours, akin to the end of the \gls{tsp} solution in \cref{sec:tsp:algotsp}.
The \(k\)-colouring variant is used for \cref{tab:gcol:gcolalgocomp}, as it is closer in spirit to the system in \cref{sec:gcol:cpsys}.

%%%%%%%%%%%%%%%%%%%%%%%%%%%%%%%%%%%%%%%%%%%%%%%

\citeauthor{Umetsu2019} \cite{Umetsu2019} build on \cite{Tanaka2012}'s asynchronous \glspl{ps} to find the minimum colouring for a graph.  They state that they use the ``branch and bound'' technique to assist in the exploration of the graph, saying \textcquote[][p.~242]{Umetsu2019}{If \textelp{} adjacent vertices have the same color, the partial color assignment of vertices is discarded as a bounding operation.}  This is the same principle as the \gls{compartment} pruning used in \cref{sec:gcol:cml}.  The main advance of this work over \cite{Tanaka2012} seems to be that it should have a (potentially much) smaller memory requirement for practical simulations.

%%%%%%%%%%%%%%%%%%%%%%%%%%%%%%%%%%%%%%%%%%%%%%%

\citeauthor{Andreu-Guzman2020} \cite{Andreu-Guzman2020} employ a hybrid membrane algorithm combined with genetic algorithms (of the style first put forward by \citeauthor{Nishida2006} \cite{Nishida2006}).  They propose a new sub-type of membrane algorithm, called \emph{Dynamic operators in a One-level Genetic Algorithm \glsfmtname{ps}}.  The authors find that their system overall significantly outperforms an earlier work that used only genetic algorithms, strongly suggesting a definite benefit to using the membrane algorithm approach.

%%%%%%%%%%%%%%%%%%%%%%%%%%%%%%%%%%%%%%%%%%%%%%%%%%%%%%%%%%%%%%%%%%%%%%%%%%%%%%%%%%%%%%%%%%%%%%%%%%%%%%

\subsection{\label{sec:gcol:3colsol}Other Solutions to the 3-Colouring Problem}

%%%%%%%%%%%%%%%%%%%%%%%%%%%%%%%%%%%%%%%%%%%%%%%

\citeauthor{Diaz-Pernil2008} \cite{Diaz-Pernil2008} presented a \gls{tlps} solution to solve the 3-colouring problem in linear time (an earlier version, \cite{Diaz-Pernil2007}, is the earliest known presented \glspl{ps} to solve a form of the \gls{gcp} within polynomial time).  The system works by using cell division to generate the entire space of possible colourings of the graph, and then using communication between cells and the environment to build up potential solutions and eliminate invalid ones.  This paper also presents a recogniser system, despite seemingly generating the information needed to describe a solution.

%%%%%%%%%%%%%%%%%%%%%%%%%%%%%%%%%%%%%%%%%%%%%%%

\citeauthor{Turcanu2012} \cite{Turcanu2012} presented a solution to the 3-colouring problem using a \glspl{tlps}. This paper appears to be an adjunct to \cite{Gheorghe2013} and discusses also the \gls{skps} from that paper.  Both solutions were simulated in \gls{mecosim} in order to compare them.  The models were further translated into Event-B \cite{Abrial2010}, and formally verified using the ProB model checker \cite{Leuschel2008} for Rodin \cite{Abrial2010a}.  The authors concluded that both systems can solve the problem, but that the pruning behaviour of the \gls{skps} leads to a more efficient computer implementation.  They also mentioned that their model checking only worked for relatively small problems, for similar reasons to why the \gls{tsp} solution in \cref{chap:tsp} could only be simulated for relatively small complete graphs.

%%%%%%%%%%%%%%%%%%%%%%%%%%%%%%%%%%%%%%%%%%%%%%%

\citeauthor{Christinal2018} \cite{Christinal2018} (building on the earlier \cite{Mathu2015}) use a variant of \gls{tlps} \enquote{with protein on cells} to create a recogniser system for the 3-colouring problem.  The authors contrast their solution with those of \cite{Diaz-Pernil2008} and \cite{Gheorghe2013}, finding that theirs requires fewer rules and steps than \cite{Diaz-Pernil2008} (though on the same order), but substantially more still than \cite{Gheorghe2013}.  It is unclear if \citeauthor{Christinal2018} consider their solution to be superior, or simply implemented with a different type of \gls{ps}.

%%%%%%%%%%%%%%%%%%%%%%%%%%%%%%%%%%%%%%%%%%%%%%%

\citeauthor{Niu2016} \cite{Niu2016} use timed \gls{tlps} with cell division to construct a time-free recogniser solution to the 3-colouring problem. Their \gls{ruleset} appears to be quite similar to that of \cite{Turcanu2012}.  They demonstrate that their time-free uniform family of rules can solve the 3-colouring decision problem for all timings, and in linear time with the use of maximal parallelism.

%%%%%%%%%%%%%%%%%%%%%%%%%%%%%%%%%%%%%%%%%%%%%%%

\citeauthor{Wang2009} \cite{Wang2009} present another recogniser solution to the 3-colouring decision problem using a \gls{tlps} with \enquote{cell separation}, which they use for the same technique that would ordinarily be called cell division in \gls{mc}.\footnote{It seems unlikely that the authors are native English speakers, and perhaps not fluent ones either, so this is probably simply a translation issue.}  Their solution seems to require a substantially larger alphabet and \gls{ruleset} (again, as with the other solutions besides the \gls{cps} one presented \cref{sec:gcol:cpsys}, both the rules and symbols are uniform families) than other solutions, but as this paper is much earlier than most others, it seems plausible that the later works built on this one.  The authors also indicate their solution has a time complexity of \bigoh{n + m}, whereas other solutions generally require \bigoh{n}.  That is, the time complexity depends on both the number of nodes and the number of edges.

%%%%%%%%%%%%%%%%%%%%%%%%%%%%%%%%%%%%%%%%%%%%%%%

