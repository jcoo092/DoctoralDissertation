\section{\label{sec:tsp:conc}Summary}
This chapter defined a succinct \gls{cps} algorithm for solving the \gls{tsp} in \bigoh{n} time, by using the capacity of \gls{cps} to create and manipulate complex subcells in only a few high-level steps.  This algorithm builds on a simpler version for finding Hamiltonian paths and cycles, requires only a fixed set of five rules, and takes \(n + 3\) steps to find a solution for any connected digraph of size \(n\), an improvement on the previous best known \gls{ps}-based solution to the \gls{tsp}.

Simple examples were provided to demonstrate the operation of the algorithm.  The \gls{tsp} algorithm can operate, without modification to the \gls{ruleset}, on any arbitrary weighted graph with a Hamiltonian cycle.   The algorithm requires only a specification of the graph encoded as subcells, and could be extended to detect the absence of a Hamiltonian cycle.