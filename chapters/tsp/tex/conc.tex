\section{\label{sec:tsp:conc}Summary}
This chapter defined a succinct \gls{cps} algorithm for solving the \gls{tsp} in \bigoh{n} time, by using the capacity of \gls{cps} to create and manipulate complex subcells in only a few high-level steps.  This algorithm builds on a simpler version for finding Hamiltonian Paths and Cycles, requires only a fixed set of five rules, and takes \(n + 3\) steps to find a solution for any connected digraph of size \(n\), an improvement on the previous best known \gls{ps}-based solution to the \gls{tsp}.

Simple examples were provided to demonstrate the operation of the algorithm.  The \gls{tsp} algorithm can operate, without modification to the \gls{ruleset}, on any arbitrary weighted graph with a Hamiltonian Cycle.   The algorithm requires only a specification of the graph encoded as subcells, and could be extended to detect the absence of a Hamiltonian Cycle.

% \begin{table}
% \centering
% \caption{Comparison of known exact \gls{ps} solutions to the \glsxtrlong{tsp}}
% \label{tab:tsp:algocomp}
% \setlength{\tabcolsep}{5pt}
% \begin{tabular}{@{}lcc@{}}
% \toprule
% \textbf{Algorithm}  & \multicolumn{1}{l}{\textbf{\begin{tabular}[c]{@{}c@{}}Num. of\\  rules\end{tabular}}} & \multicolumn{1}{l}{\textbf{\begin{tabular}[c]{@{}c@{}}Run time\\  order\end{tabular}}} \\ \midrule
% Guo \& Dai \cite{Guo2017}          & $\sim$50                                   & \bigoh{n^2}                                         \\
% Cooper \& Nicolescu & 5                                          & \bigoh{n}                                          \\ \bottomrule
% \end{tabular}
% \end{table}