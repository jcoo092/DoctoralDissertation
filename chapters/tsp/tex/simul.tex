\section{\label{sec:tsp:simulation}Simulations}
To demonstrate that this approach can be applied in practice to small problem graphs, sample simulations were written in SWI-Prolog, \fsharp{} and Erlang for the \gls{tsp} algorithm.  In each case, the programs were written with an emphasis on matching the \gls{cps} algorithm, rather than with a focus on memory or time efficiency.  Better implementations from a real-world-use viewpoint could likely be created, but they may not reflect the \gls{cps} rules as faithfully.  The Prolog program in particular matches very closely to the \gls{cps} algorithm, and so is fully presented here.  A complete program listing for the Prolog program's rules corresponding to the algorithm is in \cref{app:tsp:codeprolog}, while \cref{app:tsp:probprolog} defines the problem graph.  A copy of the source code for each simulation is available at \url{https://github.com/jcoo092/cP-Systems-TSP}.  While care was taken to keep these simulations similar to each other, differences between the languages inevitably mean they are not identical.

All three languages are reasonably well-suited to implementing \gls{cps}.  As mentioned, the Prolog program in particular maps well to the algorithm, requiring only 7 Prolog rules in total, plus a variable number of facts specifying the problem graph, 17 in this case.  The functional elements of \fsharp{} such as higher-order functions also allow a reasonable approximation, if perhaps not with quite the fidelity of Prolog.  Erlang, possibly owing to the fact it was originally implemented atop Prolog, appears to fall in between the two approaches, leaning more towards \fsharp{}.  It is clear from these programs that there is at least one potential close mapping from \gls{cps} to both logic and functional programming languages.  The emphasis here, however, was on demonstrating the similarities of the languages to \gls{cps}, and not on performance.  As such, no attempt has been made to optimise the simulations.

To gauge their ``real-world'' effectiveness, the simulations were informally tested with increasing digraph sizes.  All three ably coped with digraph sizes of up to 10 vertices, returning an answer at most in a matter of a few seconds.  \fsharp{} and Erlang struggled somewhat at 11 vertices, with the latter taking more than one minute to complete, while Prolog quickly failed with an `out of stack memory' error.  The \fsharp{} simulation was tested on a digraph of size 12, but the test was terminated after 90 minutes running time due to time constraints, failing to complete and return a minimum cost path.  For this reason, a 12 vertex run was not attempted with Erlang.

It is unsurprising that memory limits may become an issue, considering that for a totally connected 11-vertex digraph, at the 11th step almost 40 million (\(11!\)) subcells would be required.  These results suggest that, while the fundamental process does indeed lead to determining lowest-cost paths, much more effective use of memory will be required to make software simulations practical for highly connected digraphs of any significant size (\ie{}, greater than approximately 11 or 12 vertices).

A comparison of \cref{fig:tsp:utree,fig:tsp:dtree} suggests, however, that for graphs that are not totally connected, the implementations may cope much better, as would be expected.  With fewer paths to explore, the growth in the number of objects, and thus the space requirements, will be smaller.  This is not unique by any means to the \gls{cps} version of the \gls{tsp} --- any method to solve the \gls{tsp} which involves a breadth-first search of the graph, considering whether nodes have already been visited, will experience the same. 

\subsection{Prolog Simulation}

It is perhaps interesting to contrast the Prolog method for exploring the problem space and finding an answer with that of \gls{cps}.  On the surface they appear quite similar --- a handful of rules and some initial statements specific to the problem, combined with unification.   In some ways, however, they also appear to be opposing duals of each other.  Prolog in general, and thus with this program, works on a top-down backward-chaining approach, whilst \gls{cps} works on a bottom-up forward-chaining approach.  Whereas the rules of \cref{ruleset:tsp:tsp} describe the process for evolving the system \foreign{en masse} and taking advantage of \gls{mc}'s unbounded parallelism capabilities, the Prolog program describes how to search for a solution among the possibilities.  In essence, the latter describes a chain of dependencies and constraints, and leaves it up to the runtime to work back along the chain recursively until it can determine acceptable unifications for every variable involved.

That is to say that (sequential) Prolog tries to do the least work and use the least space possible, so it starts with the definition of the requested inference (an invocation of \texttt{go} in the program), and only evaluates elements of that definition as it discovers they are needed in order to provide a result.  In the case of the digraph exploration, this essentially means that it tries to perform a linear depth-first search of the digraph, stopping as soon as it has an answer.  With the \gls{tsp} it inevitably must evaluate every cycle to know which has minimum cost, however.

Conversely, the \gls{cps} algorithm starts with the definition of the problem graph, and does all possible work to find the result in what is essentially a breadth-first search.  All possible cycles from the root node are instantiated and eventually checked for minimum cost. The ability of \gls{cps} to perform these instantiations and the comparison completely in parallel means that a relatively small number of steps are required to find the minimum --- though at the cost of a potentially (exponentially) large space and processing complexity.

\begin{listing}
\caption[Complete SWI-Prolog code for the \glsxtrlong{tsp} algorithm]{\label{app:tsp:codeprolog}Complete SWI-Prolog code for the rules of the \gls{tsp} algorithm}
\inputminted[linenos,breaklines,frame=lines,autogobble,firstline=5]{prolog}{chapters/tsp/code/tsp.pl.txt}
\end{listing}

\begin{listing}
\caption{\label{app:tsp:probprolog}SWI-Prolog code defining the example problem undirected graph G shown in \cref{fig:tsp:ugraph}}
\inputminted[linenos,breaklines,frame=lines,autogobble,lastline=3]{prolog}{chapters/tsp/code/tsp.pl.txt}
\end{listing}

% The code presented in \cref{app:tsp:codeprolog} is divided into five parts, largely corresponding to the rules in \cref{ruleset:tsp:tsp}.  \cpRuleref{rule:tsp:tsp:clean} has no matching implementation in the Prolog simulation, as the clean-up of objects is delegated to the Prolog system.  Conversely, the Prolog system requires an extra declaration unneeded in the \glspl{cps}; specifically, the \texttt{go} predicate on line 17.  This instructs the Prolog runtime to seek a valid unification for the variable \texttt{M} on the \gls{lhs} which satisfies the constraints on the \gls{rhs}.  Lines 5 \& 6 implement \cpRuleref{rule}

% \Cref{app:tsp:probprolog} matches exactly to \cref{objs:tsp:obj1}.  The same set of \(e\) functors and the one \(v\) functor are reproduced.  As with the system in \cref{ruleset:tsp:tsp}, the program in \cref{app:tsp:codeprolog} uses these ground terms as the input data with which to compute the final result.

The code presented in \cref{app:tsp:codeprolog} is divided into five parts, corresponding moderately to the rules in \cref{ruleset:tsp:tsp}.  \cpRuleref{rule:tsp:tsp:clean} has no matching implementation in the Prolog simulation, as the clean-up of objects is delegated to the Prolog system.  Conversely, the Prolog system requires an extra declaration unneeded in the \glspl{cps}; specifically, the \texttt{go} predicate on line 17.  This instructs the Prolog runtime to seek a valid unification for the variable \texttt{M} on the \gls{lhs} which satisfies the constraints on the \gls{rhs}.  It also bundles within itself the selection of a starting node in \cpruleref{rule:tsp:tsp:start}, and the output of the final selection that ends the \glspl{cps}'s evolution in \cpruleref{rule:tsp:tsp:min}.  The minimum-finding performed in the latter rule is separated out into the three complementary definitions of \texttt{minh} on lines 13-15.  Lines 8 \& 9 of \cref{app:tsp:codeprolog} closely match \cpruleref{rule:tsp:tsp:explore}.  The \gls{lhs} of \cpruleref{rule:tsp:tsp:makezs} is embodied by lines 5 \& 6, while the \gls{rhs} of the same rule roughly matches to line 11.



\Cref{app:tsp:probprolog} matches exactly to \cref{objs:tsp:obj1}.  The same \(e\) functors and one \(v\) functor are reproduced.  As with the system in \cref{ruleset:tsp:tsp}, the program in \cref{app:tsp:codeprolog} uses these ground terms as the input data with which to compute the final result.