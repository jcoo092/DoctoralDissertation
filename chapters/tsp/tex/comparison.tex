\section{Comparison with Other \glsfmtname{mc} Solutions}

%%%%%%%%%%%%%%%%%%%%%%%%%%%%%%%%%%%%%%%%%%%%%%%%%%%%%%%%%%%%%%%%%%%%%%%%%%%%%%%%%%%%%%%%%%%%%%%%%%%%%%%

\subsection{The \glsfmtname{hcp-glossary}}

% The \gls{hpp}/\gls{hcp} has been examined before through the lens of \gls{ps}, with different variants of \gls{ps} used such as \gls{tlps} \cite{Martin-Vide2003}, \gls{ps} with active membranes \cite{Pan2006,Song2013}, Recogniser \gls{ps} \cite{Chen2009}, Asynchronous \gls{ps} \cite{Tagawa2012} and \gls{snps} \cite{Xue2013}.  \citeauthor{Jimenez2003} used the \gls{hpp} as an example computation when discussing complexity classes in \gls{ps} \cite{Jimenez2003}.  In fact, it was demonstrated relatively early in the history of \gls{ps} that some variants at least could be used to solve the \gls{hpp} in linear time \cite{Mutyam2001}.  This last approach, using \gls{ps} with membrane creation, is arguably the closest to the approach used in this \lcnamecref{chap:tsp}, which involves the repeated instantiation of subcells.

The \gls{hpp} \& \gls{hcp} have been examined before through the lens of \gls{ps}.  In fact, it was demonstrated relatively early in the history of \gls{ps} that some variants could be used to solve the \gls{hpp} in linear time \cite{Mutyam2001}.  \citeauthor{Jimenez2003} used the \gls{hpp} as an example computation when discussing complexity classes in \gls{ps} \cite{Jimenez2003}.  %  This approach, using \gls{clps} with membrane creation, is arguably the closest to the approach used in this \lcnamecref{chap:tsp}, which involves the repeated instantiation of subcells.
Other work has used various types of \gls{ps} used such as \gls{tlps} \cite{Martin-Vide2003}, \gls{clps} with active membranes \cite{Pan2006,Chen2009}, timed \gls{ps} \cite{Song2013}, \gls{clps} with symport/antiport rules \cite{Orellana-Martin2019a}, ``asynchronous'' \gls{ps} \cite{Tagawa2012} (using a different concept of asynchrony than \cref{sec:back:syncasync}) and \gls{snps} \cite{Xue2013}.

\begin{table}
\centering
\begin{tabular}{llccc} 
\toprule
\textbf{Algorithm} & \begin{tabular}[c]{@{}l@{}}\textbf{P systems}\\\textbf{type}\end{tabular} & \begin{tabular}[c]{@{}c@{}}\textbf{Num. of}\\\textbf{ rules}\end{tabular} & \begin{tabular}[c]{@{}c@{}}\textbf{Size of}\\\textbf{alphabet}\end{tabular} & \begin{tabular}[c]{@{}c@{}}\textbf{Time}\\\textbf{complexity}\end{tabular}  \\ 
\midrule
% \cite{Chen2009} & \gls{clps} & 15 & 17 & \bigoh{n^2}\\
% \cite{Guo2017} & \gls{clps} & 56 & 13 & \bigoh{n^2}\\
\cite{Orellana-Martin2019a} & \gls{clps} & 51 & 29 & \bigoh{n^3}\\
\cite{Tagawa2012} & asynchronous \gls{ps} & 8 & 10 & \bigoh{n^2}\\
% \cite{Zhang2019} & \gls{snps} & 4 & 1 & \bigoh{n}\\
This work & \gls{cps} & 5 & 11 & \bigoh{n}\\
\bottomrule
\end{tabular}
\caption{Comparison of known exact \gls{ps} solutions to the \gls{hcp}.  \(n\) is the number of nodes in the graph.}
\label{tab:tsp:hcpalgocomp}
\end{table}

\Cref{tab:tsp:hcpalgocomp} summarises key measures about each of the other papers on solving the \gls{hcp} with \gls{mc} that are discussed below --- the \gls{hpp} is excluded as the (comparatively) simpler problem.  Other systems all require some degree of customisation based on the problem at hand, generally parameterized over the number of nodes or edges in the graph, making it difficult to provide a precise estimate of the number of rules and alphabet size (or even the time complexity).  That does imply, however, that the true numbers of distinct rules and objects required are generally polynomial, not fixed.  This means that \emph{for all other systems besides the current \glspl{cps}, these are underestimates}.  The number of rules and the alphabet size in the \cref{tab:tsp:hcpalgocomp} are taken from the distinct rule and symbol definitions listed in each paper.  Both measures have been counted on a good-faith best-effort basis from the included system definitions.

Furthermore, many of the described systems solve the decision problem version of the \gls{hpp} or \gls{hcp}, rather than the variant where one finds and describes such a path or cycle, which the systems presented in \cref{sec:tsp:algohpp} do.

%%%%%%%%%%%%%%%%%%%%%%%%%%%%%%%%%%%%%%%%%%%%%%%%%%%%%%%%%%%%%

In \cite{Orellana-Martin2019a}, \citeauthor{Orellana-Martin2019a} describe a solution to the \gls{hcp} that works similarly to the \glspl{cps} in \cref{sec:tsp:algohpp}.  Their system also generates all possible paths using \gls{clps} with symport/antiport rules and membrane division, and then finds the result from among them.  Surprisingly, despite generating the information needed to output a solution, the presented system is instead a recogniser \glspl{ps} \cite[Ch.~12.2]{Paun2010b}, meaning that it only decides whether there is a valid Hamiltonian cycle present.

%%%%%%%%%%%%%%%%%%%%%%%%%%%%%%%%%%%%%%%%%%%%%%%%%%%%%%%%%%%%%

\citeauthor{Chen2009} \cite{Chen2009} also give a recogniser \gls{ps}, using \gls{clps} with active membranes, but only for the \gls{hpp}.  Like \cref{sec:tsp:hpp} it essentially uses a brute-force exploration of the entire graph.  An important difference, however, is that the system described in \cite{Chen2009} seemingly ignores whether there is an edge between two nodes until the final stage of the process, meaning that it may create invalid nested membranes without benefit.

%%%%%%%%%%%%%%%%%%%%%%%%%%%%%%%%%%%%%%%%%%%%%%%%%%%%%%%%%%%%%

Notably, \cite{Martin-Vide2003} both first introduced \gls{tlps}, and demonstrated that \gls{tlps} could be used to solve the \gls{hpp} in linear time.  The solution output by the described system encodes a Hamiltonian path (or indicates that none exist).  The system is also notable for being relatively compact in terms of rules and symbols, seemingly by using representing the nodes in the graph with one cell each, and using communications between them to describe possible paths.

%%%%%%%%%%%%%%%%%%%%%%%%%%%%%%%%%%%%%%%%%%%%%%%%%%%%%%%%%%%%%

\citeauthor{Pan2006} provided another recognising solution to the \gls{hpp} in \cite{Pan2006}.  The focus of this work was on proving that \gls{clps} with active membranes, but forgoing the use of some types of rules, can still solve the \gls{hpp} in polynomial time by a uniform family of rules.  Importantly, one of the types of rules used, described in \cite{Pan2006} as ``d-separation rules'' do increase the number of membranes in the system, keeping to the tradition of trading off space complexity for time complexity.

%%%%%%%%%%%%%%%%%%%%%%%%%%%%%%%%%%%%%%%%%%%%%%%%%%%%%%%%%%%%%

\citeauthor{Song2013} focused on solving the \gls{hpp} while addressing the assumption of most types of \gls{ps} operate on a global clock, where all compartments evolve at the same time.  They conclude that their system can solve the \gls{hpp} in polynomial time, regardless of the timings involved.  Given that most other systems use rules which are only enabled in the presence of particular objects, it seems doubtful that other systems would fail if every compartment in the system did not evolve simultaneously, however.

%%%%%%%%%%%%%%%%%%%%%%%%%%%%%%%%%%%%%%%%%%%%%%%%%%%%%%%%%%%%%

\citeauthor{Tagawa2012} use an asynchronous \gls{ps} to solve the \gls{hcp} in \cite{Tagawa2012}, though the definition of asynchrony used there is significantly different to that used in this dissertation (see \vref{sec:back:syncasync}).  In the case of \cite{Tagawa2012} asynchrony is used to mean that anywhere from one to all possible rules are applied in parallel at a given step.\footnote{This is an interesting scenario, and possibly more realistic than the usual assumptions underpinning many \gls{ps} variants, but the naming is curious.}  The underlying system excluding asynchronicity appears to be \gls{clps} with active membranes.  The authors conclude that their system can solve the \gls{hcp} (the decision form) within either \bigoh{n!} to \bigoh{n^2} steps, depending on the level of parallelism involved.

%%%%%%%%%%%%%%%%%%%%%%%%%%%%%%%%%%%%%%%%%%%%%%%%%%%%%%%%%%%%%

\citeauthor{Xue2013} used \gls{snps} to solve the \emph{directed} \gls{hpp}, \ie{} the \gls{hpp} when the problem graph is specifically a digraph.  Also unusually compared to many other papers, the authors permit any neuron in the system to be used as an output neuron, whereas other work throughout \gls{mc} typically assumes a specific output compartment.  It is unclear if this has any practical impact, however.

%%%%%%%%%%%%%%%%%%%%%%%%%%%%%%%%%%%%%%%%%%%%%%%%%%%%%%%%%%%%%

Overall, most other work on the \gls{hpp} and \gls{hcp} has focused only on the \gls{hpp}, and usually only the decision problem, rather than attempting to output a description of a valid path or cycle (recall that a description of a valid path or cycle suffices to demonstrate that it exists).  None of the other know \gls{mc} solutions have used either a completely fixed \gls{ruleset} or alphabet, instead using (semi-)uniform families of such.

%%%%%%%%%%%%%%%%%%%%%%%%%%%%%%%%%%%%%%%%%%%%%%%%%%%%%%%%%%%%%%%%%%%%%%%%%%%%%%%%%%%%%%%%%%%%%%%%%%%%%%%

\subsection{The \glsfmtname{tsp-glossary}}

% \begin{table}[htb]
% \centering
% \caption{Comparison of known exact \gls{ps} solutions to the \gls{tsp}}
% \label{tab:tsp:algocomp}
% \setlength{\tabcolsep}{5pt}
% \begin{tabular}{@{}lcc@{}}
% \toprule
% \textbf{Algorithm}  & \multicolumn{1}{l}{\textbf{\begin{tabular}[c]{@{}c@{}}Num. of\\  rules\end{tabular}}} & \multicolumn{1}{l}{\textbf{\begin{tabular}[c]{@{}c@{}}Run time\\  order\end{tabular}}} \\ \midrule
% Guo \& Dai \cite{Guo2017}          & $\sim$50                                   & \bigoh{n^2}                                         \\
% Cooper \& Nicolescu & 5                                          & \bigoh{n}                                          \\ \bottomrule
% \end{tabular}
% \end{table}

% \usepackage{booktabs}

A small amount of work has been done on the \gls{tsp} from the perspective of \gls{mc}, beginning with the work of \citeauthor{Nishida2006} \cite{Nishida2006}, who used a combination of a membrane structure and pre-existing methods to search for an approximate solution in a space of solutions to the \gls{tsp} for a given digraph.  Others built on this approach by integrating techniques such as Genetic Algorithms \cite{Manalastas2013,He2014}, Ant Colony Optimisation \cite{Zhang2011} and Active Evolution \cite{Song2015}, along with more complex approaches for multiple salesman problems \cite{He2015}.%  A paper by \citeauthor{Chen2011} \cite{Chen2011} was apparently also written on the topic, but no copy could be located.

Surprisingly, however, given the multitude of solutions demonstrated for the \gls{hcp}, all these \gls{tsp} papers have been written from the perspective of approximate solutions to the \gls{tsp}.  They typically take an approach of using membranes to divide up the search space of potential solutions, whilst applying other pre-existing techniques to the process.  These papers have used membranes to structure a search space, but arguably have not fully embraced the \gls{ps} model, \eg{} none of them have specified typical \gls{ps} rewriting rules, instead applying other techniques within the subcells, and using \gls{ps} rules only to move potential solutions between cells.  Thus, it is challenging to compare them directly with \gls{mc}-only solutions.  The work by \citeauthor{Guo2017} \cite{Guo2017} is the first known solution to use only \gls{ps}.

\begin{table}
\centering
\begin{tabular}{llccc} 
\toprule
\textbf{Algorithm} & \begin{tabular}[c]{@{}l@{}}\textbf{P systems}\\\textbf{type}\end{tabular} & \begin{tabular}[c]{@{}c@{}}\textbf{Num. of}\\\textbf{ rules}\end{tabular} & \begin{tabular}[c]{@{}c@{}}\textbf{Size of}\\\textbf{alphabet}\end{tabular} & \begin{tabular}[c]{@{}c@{}}\textbf{Time}\\\textbf{complexity}\end{tabular}  \\ 
\midrule
\cite{Aman2021} & \gls{tlps} & 1 & 8 & \bigoh{n}\\
\cite{Guo2017} & \gls{clps} & 56 & 13 & \bigoh{n^2}\\
% \cite{Orellana-Martin2019a} & \gls{clps} & 51 & 29 & \bigoh{n^3}\\
\cite{Zhang2019} & \gls{snps} & 4 & 1 & \bigoh{n}\\
This work & \gls{cps} & 5 & 13 & \bigoh{n}\\
\bottomrule
\end{tabular}
\caption{Comparison of known exact \gls{ps} solutions to the \gls{tsp}.  \(n\) is the number of nodes in the graph.}
\label{tab:tsp:algocomp}
\end{table}

% \Cref{tab:tsp:algocomp} summarises key measures about each of the other papers on solving the \gls{tsp} with \gls{mc} that are discussed below.  It includes only those which work wholly in \gls{mc} because it is unclear how to make a meaningful comparison with those which use a hybrid approach, \eg{} \cite{Nishida2006}.  Other systems all require some degree of customisation based on the problem at hand, generally parameterized over the number of nodes or edges in the graph, making it difficult to provide a precise count of the number of rules and alphabet size (or even the time complexity) if the authors themselves have not done so.  That does imply, however, that the true numbers of distinct rules and objects required are generally polynomial, not fixed.  The number of rules and the alphabet size in the table are taken from the distinct rule and symbol definitions listed, \ie{} they are usually \emph{classes} of rule or symbol requiring instantiations based on the problem.  This means that \emph{for all other systems besides the current \glspl{cps}, these are underestimates}.  Both measures have been counted on a good-faith best-effort basis from the included system definitions.

\Cref{tab:tsp:algocomp} summarises key measures about each of the other papers on solving the \gls{tsp} with \gls{mc} that are discussed below.  It includes only those which work wholly in \gls{mc} because it is unclear how to make a meaningful comparison with those which use a hybrid approach, \eg{} \cite{Nishida2006}.  Other systems all require some degree of customisation based on the problem at hand, generally parameterized over the number of nodes or edges in the graph, making it difficult to provide a precise estimate of the number of rules and alphabet size (or even the time complexity).  That does imply, however, that the true numbers of distinct rules and objects required are generally polynomial, not fixed.  This means that \emph{for all other systems besides the current \glspl{cps}, these are underestimates}.  The number of rules and the alphabet size in the \cref{tab:tsp:algocomp} are taken from the distinct rule and symbol definitions listed in each paper.  Both measures have been counted on a good-faith best-effort basis from the included system definitions.

%  \Eg{} in \cite{Orellana-Martin2019a}, \citeauthor{Orellana-Martin2019a} do examine the rule and symbol complexity, finding that both the number of rules and alphabet size on the order of \(\Theta(n^6)\).  Likewise, \cite{Zhang2019} is stated as having an alphabet size of one (the spike), but information is encoded in the labels of the neurons yet excluded from the alphabet, while labels are explicitly included with the \gls{cps} solution.

%%%%%%%%%%%%%%%%%%%%%%%%%%%%%%%%%%%%%%%%%%%%%%%%%%%%%%%%%%%%%

In \cite{Guo2017} -- the original inspiration for this \lcnamecref{chap:tsp} -- \citeauthor{Guo2017} specify a \gls{clps} that builds an exploration tree starting from the root node, conceptually very similar to \cref{fig:tsp:utree,fig:tsp:dtree}.  In fact, the most significant difference between this work and \cite{Guo2017} is the number of rules and number of distinct symbols used.

%%%%%%%%%%%%%%%%%%%%%%%%%%%%%%%%%%%%%%%%%%%%%%%%%%%%%%%%%%%%%

% \citeauthor{Aman2021} \cite{Aman2021} define \gls{tlps} with costs, where there is a cost associated to the channels between cells, and use it to solve the \gls{tsp}.
% \citeauthor{Aman2021} also provide a working implementation of their system using the Priced-Time Maude rewriting engine \cite{Bendiksen2009}.  They comment: \blockquote{By \textdel{providing an implementation of}\textelp{using} tissue P systems with costs in Priced-Timed Maude, it is unnecessary to create the exponential space by means of the rules of P systems \textins{as is done in \cref{sec:tsp:algotsp}}, but just describe how the objects can move non-deterministically through \textdel{synapses}\textelp{channels}.  The exponential space of configurations is created by running the Priced-Timed Maude tool, that is also used to filter the final configurations and provide the desired solutions (if they exist).}

\citeauthor{Aman2021} \cite{Aman2021} define \gls{tlps} with costs, where there is a cost associated to the channels between cells, and use it to solve the \gls{tsp}.
\citeauthor{Aman2021} also provide a working implementation of their system using the Priced-Time Maude rewriting engine \cite{Bendiksen2009}.  They comment: \blockquote{By \textelp{using} tissue P systems with costs in Priced-Timed Maude, it is unnecessary to create the exponential space by means of the rules of P systems \textins{as is done in \cref{sec:tsp:algotsp}}, but just describe how the objects can move non-deterministically through \textelp{channels}.  The exponential space of configurations is created by \textelp{} the Priced-Timed Maude tool, that is also used to filter the final configurations and provide the desired solutions (if they exist).}
They state that the simulation takes \qty{28}{\milli\second} to produce \enquote{the desired configurations} using the same graph G as \cref{fig:tsp:ugraph}.  The listed specification of the system somewhat closely matches \cref{app:tsp:probprolog}.
Only one rule type is listed in this work, making it appear compact, but this system also assumes that a root node has been pre-selected, and does not take any action to output a result.  It is also unclear where exactly the path with minimum cost is selected.  It appears that this is delegated to Maude.

%%%%%%%%%%%%%%%%%%%%%%%%%%%%%%%%%%%%%%%%%%%%%%%%%%%%%%%%%%%%%

\citeauthor{Qi2018} \cite{Qi2018} presented a probabilistic solution using \gls{snps} combined with genetic algorithms (something they call ``optimisation \gls{snps}'').  Unfortunately, the authors do not give a formal or systematic description of their system.  They state that they implemented it in MATLAB, but again provide scant details on the results.  This prevents an adequate comparison, though given that their system is probabilistic a direct comparison would not be possible anyway.

%%%%%%%%%%%%%%%%%%%%%%%%%%%%%%%%%%%%%%%%%%%%%%%%%%%%%%%%%%%%%

\citeauthor{Wei2019} \cite{Wei2019} combine a \enquote{dynamic active membrane system} (seemingly \gls{clps} with active membranes) with particle swarm optimisation \cite{Sun2016} to produce another probabilistic solution.  Their solution appears to share commonalities with \citeauthor{Nishida2006}'s \cite{Nishida2006}, in that it too is primarily based on the alternative computing method, and membranes are simply used for structuring the computation.  They do not detail their \glspl{ps}, nor provide clear results.

%%%%%%%%%%%%%%%%%%%%%%%%%%%%%%%%%%%%%%%%%%%%%%%%%%%%%%%%%%%%%

\citeauthor{Zhang2019} \cite{Zhang2019} presented a classical \gls{snps} solution for the \gls{tsp}.  As with many classic \gls{mc} solutions to challenging problems \cite{Paun1999a} (including, that of \cref{sec:tsp:algotsp}), their solution creates new neurons as it proceeds, creating an exponential-sized system to solve the problem quickly.  Indeed, the overall functioning of the system shares much in common with \cref{ruleset:tsp:tsp}.  Notably, the \gls{ruleset} only requires the specification and adaptation of only rule types.  While these still must be customised to the specific weighted graph at hand, it is nevertheless relatively compact compared to solutions for many other problems.

%%%%%%%%%%%%%%%%%%%%%%%%%%%%%%%%%%%%%%%%%%%%%%%%%%%%%%%%%%%%%