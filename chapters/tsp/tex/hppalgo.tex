\section{\label{sec:tsp:algohpp}cP~Systems Hamiltonian Path and Cycle Algorithms}
% We discuss in this section only digraphs, recalling that all undirected graphs can be represented quite simply as a directed graph with arcs in both directions simulating edges -- this is in fact how we represent undirected graphs in our algorithm.

This section discusses only digraphs, since all undirected graphs can be represented quite simply as a directed graph with arcs in both directions simulating edges --- this is in fact how undirected graphs in are represented in the below algorithm.

\subsection{Formal System Definition}
A cP~system can be described as a 6-tuple as shown below.

\cptupletemplate{}

\(T\) is the set of top-level cells at the start of the evolution of the system; \(A\) is the alphabet of the system; \(O\) is the set of multisets of initial objects in the top-level cells; \(R\) is the set of rulesets for each top-level cell, \(S\) is the set of possible states of the system, and \(\Bar{s} \in S\) is the starting state of the system. Further, \(|T| = |O| = |R|\).

\(T = \{1\}\), i.e. the entire system consists of only one top-level cell.  \(A = \) (the set of all symbols used).  \(O = \{E \cup v\}\).  For the \gls{hpp}, \(R\) is the ruleset in \cref{ruleset:tsp:hpp}, and for the \gls{hcp} it is \cref{ruleset:tsp:hcp}.  For both, \(S = \{s_1, s_2, s_3, s_4\}\), and \(\Bar{s} = s_1\).

\cptuple{HPP}{\cpset{1}}{\cpset{e, f, h, p, p', s, t, u, v}}{\cpset{E \cup \cpset{v}}}{\text{\cref{ruleset:tsp:hpp}}}{\cpset{s_1, s_2, s_3, s_4}}{s_1}

\cptuple{HCP}{\cpset{1}}{\cpset{e, f, h, p, p', s, t, u, v}}{\cpset{E \cup \cpset{v}}}{\text{\cref{ruleset:tsp:hcp}}}{\cpset{s_1, s_2, s_3, s_4}}{s_1}

\subsection{Hamiltonian Path}
The algorithm follows a simple approach, essentially a simple maximally parallel breadth-first search of the digraph.  We start with a top-level cell enclosed by the skin membrane, and populated with subcells describing the problem digraph (recall that subcells in \gls{cps} are the namesake complex objects).  From there, a starting vertex of the problem digraph is randomly selected, and the top-level cell is then populated with the other initial required subcells.  The evolution of the system then proceeds synchronously level-by-level through the different potential paths of the cycle by creating new subcells encoding the digraph traversals up to that point, expanding all possible paths from a given vertex which exclude any of the previously visited vertices.

The algorithm requires a fixed set of \textbf{only four rules}, presented in \cref{ruleset:tsp:hpp}.  Note that we do not define a class of rules and suggest that there should be instantiations of it to fit the specific problem, but instead we define precisely four rules that apply the same to all problems conforming to the basic assumptions described below.

At the beginning of the computation, assume an elementary cell with the skin membrane, and that the set \(E\) of subcells of the form \(E = \{e(f(i)\,t(j))\}_{i,j \in \mathbb{N}; ~ i \neq j}\) encoding the arcs of the problem digraph is already present inside the skin membrane.  Subcell \(e\) represents an arc; \(f\) the origin vertex; and \(t\) the destination vertex.  We further assume that the subcell \( v(v(X),v(Y)\dots)\), listing the vertices of the problem digraph is present, though this could be derived from the subcells in \(E\), if required.  The system begins in state 1.

\begin{cprulesetfloat}
    \begin{cpruleset}
        \cprule{s_1}{\cpfunc{v}{\cpfunc{v}{R} Y}}{1}{s_2}{\cpfunc{s}{\cpfunc{u}{Y} \; \cpfunc{p}{\cpfunc{h}{R}\cpfunc{p}{}}}}
        
        \cprule{s_2}{}{1}{s_3}{\cpfunc{p'}{P}}
        \cppromoter{\cpfunc{s}{\cpfunc{u}{} \; \cpfunc{p}{P}}}
        
        \cprule{s_2}{}{+}{s_2}{\cpfunc{s}{\cpfunc{u}{Z} \; \cpfunc{p}{\cpfunc{h}{T} \cpfunc{p}{\cpfunc{h}{F} \cpfunc{p}{P}}}}}
        \cppromoter{\cpfunc{s}{\cpfunc{u}{\cpfunc{v}{T}Z} \; \cpfunc{p}{\cpfunc{h}{F} \cpfunc{p}{P}}}}
        \cppromoter{\cpfunc{e}{\cpfunc{f}{F} \; \cpfunc{t}{T}}}
        
        \cprule{s_2}{\cpfunc{s}{\_}}{+}{s_2}{}
    \end{cpruleset}
    \caption[Ruleset for the \glsentrylong{hpp-glossary}]{\label{ruleset:tsp:hpp}Ruleset for our \gls{hpp} \gls{cps} algorithm.}
\end{cprulesetfloat}

Rule 1 begins the computation by selecting an arbitrary vertex \(R\) from the subcell \(v\) to become the starting point of the cycle, and creating the first \(s\) subcell.  The \(s\) subcells represent steps down the tree of the exploration of the digraph (see \cref{fig:tsp:utree} for an example), with the first step representing the selection of the starting vertex \(R\).  Each \(s\) subcell comprises two components: \(u\), a set representing the remaining unexplored vertices in the digraph; and \(p\) which acts as a list as described above, and keeps track of the cycle's path so far.  The \(R\) variable used in this rule represents the root of the computation, which is not used further in this algorithm, but becomes important for the \gls{hcp} and \gls{tsp}.  The rule is applied in exactly-once mode, and the system transitions to state 2.  Application of this rule takes one step.

Rule 2 is listed earlier despite being applied after rules 3 and 4, to enjoy an advantage in the weak priority ordering.  After rules 3 and 4 have been applied enough times to leave the system only with \(s\) subcells that have no further nodes to explore in their \(u\) subcell, rule 2 will select and output a path \(p'\) chosen at random from among those deduced so far.  This rule is applied in exactly-once mode, and application of the rule takes one step, with the system ending in state 3.  The application of this rule represents the end of the evolution of the system.

Rule 3 is arguably the heart of the algorithm.  So long as there are one or more vertex labels remaining in the unexplored vertex subcells \(u\) and a relevant \(e\) subcell available, this rule will be applied to each extant \(s\) subcell, and create new derivative \(s\) subcells that represent another step in the exploration of the digraph/another level in the exploration tree.  The next selected vertex \(T\) for each instantiation will be removed from \(u\) and prepended to \(p\).  This rule is applied in maximally-parallel parallel mode, and the system remains in state 2.  Application of this rule takes one step per remaining vertex in the digraph, or \(n - 1\) steps in total.

Rule 4 runs in parallel with rule 3, and simply removes the extant \(s\) subcells from the system.  Due to the parallel nature of \gls{clps}, where any number of rules can be applied concurrently so long as they do not conflict with each other, this rule can work in conjunction with rule 3 without issue, because changes to subcells are not performed until the end of the step.  Note that neither rules 3 nor 4 can be applied alongside rule 2, because rule 2 changes the system's state, and therefore application of it conflicts with the latter two rules.  Both later rules are applied at the same time, so that at the end of their application, the new \(s\) subcells have been created, and the pre-existing ones deleted.  This rule is applied in maximally-parallel mode, and the system remains in state 2.  Application of this rule takes one step per vertex in the digraph, or \(n - 1\) steps in total -- of note is that these steps are the same ones used for the application of rule 3, and occur simultaneously.

\subsection{Hamiltonian Cycle}
Finding a Hamiltonian cycle requires an expansion of the algorithm, specifically to record the initial starting vertex, and ensure that the paths produced also end there.  The modified rules taking account of the changes are presented in \cref{ruleset:tsp:hcp}.

\cpresetrulenumber

\begin{cprulesetfloat}
    \begin{cpruleset}
        
        \cprule{s_1}{\cpfunc{v}{\cpfunc{v}{R} Y}}{1}{s_2}{\cpfunc{s}{\cpfunc{r}{R} \; \cpfunc{u}{Y} \; \cpfunc{p}{\cpfunc{h}{R} \cpfunc{p}{}}}}
        
        \cprule{s_2}{\cpfunc{s}{\cpfunc{r}{R} \; \cpfunc{u}{} \; \cpfunc{p}{\cpfunc{h}{F} \cpfunc{p}{P}}}}{+}{s_3}{\cpfunc{z}{\cpfunc{p}{\cpfunc{h}{R} \cpfunc{p}{\cpfunc{h}{F} \cpfunc{p}{P}}}}}
        \cppromoter{\cpfunc{e}{\cpfunc{f}{F} \; \cpfunc{t}{R}}}
        
        \cprule{s_2}{}{+}{s_2}{\cpfunc{s}{\cpfunc{r}{R} \; \cpfunc{u}{Z} \; \cpfunc{p}{\cpfunc{h}{T} \cpfunc{p}{\cpfunc{h}{F} \cpfunc{p}{P}}}}}
        \cppromoter{\cpfunc{s}{\cpfunc{r}{R} \; \cpfunc{u}{\cpfunc{v}{T} Z} \; \cpfunc{p}{\cpfunc{h}{F} \cpfunc{p}{P}}}}
        \cppromoter{\cpfunc{e}{\cpfunc{f}{F} \; \cpfunc{t}{T}}}
        
        \cprule{s_2}{\cpfunc{s}{\_}}{+}{s_2}{}
        
        \cprule{s_3}{}{1}{s_4}{\cpfunc{p'}{P}}
        \cppromoter{\cpfunc{z}{\cpfunc{p}{P}}}
        
    \end{cpruleset}
    \caption[Ruleset for the \glsentrylong{hcp-glossary}]{\label{ruleset:tsp:hcp}Ruleset for our \gls{hcp} \gls{cps} algorithm.}
\end{cprulesetfloat}

The major changes to this from the \gls{hpp} algorithm are the addition of an \(r\) subcell to every \(s\) subcell, the change of rule 2 and the addition of a rule 5. The \(r\) subcell stores a vertex -- the \(R\) selected in rule 1 -- to keep track of the starting point of the cycle, where the cycle will also have to end.  This is used by the new rule 2, which instead of outputting a randomly selected path, constructs new \(z\) subcells out of the \(s\) subcells.  These new subcells contain the full path of the cycle, but are only instantiated if there is an \(e\) subcell representing an arc on the graph from the final vertex on the path back to the origin vertex \(R\), by the action of the promoter.  Finally, rule 5 selects at random from one of these new \(z\) subcells and outputs a final path subcell \(p'\).  Rule 5 runs in exactly-once mode.  Application of this rule takes one step, and the system ends in state 4, which represents the new termination point of the evolution of the system.

The time complexity of this algorithm can be summarised as Proposition \ref{prop:tsp:runtime}:

\begin{proposition}
In total, the Hamiltonian cycle (path) algorithm takes \(n + 3\) (\(n + 1\)) operations, giving the algorithm a time complexity of \(\mathcal{O}(n)\).
\label{prop:tsp:runtime}
\end{proposition}