\section{\label{sec:tsp:variations}Variations on the Algorithms}
\Cref{sec:tsp:algohpp,sec:tsp:algotsp} presented algorithms to find a single Hamiltonian path or cycle, with minimum cost in the case of a weighted digraph (\ie{} the \gls{tsp}).  With minimal modifications, however, other results may be obtained from the algorithm, such as using a specific starting vertex or generating all possible paths/cycles.

To use a specific vertex as the starting point of the algorithm, \cpruleref{rule:tsp:tsp:start} may be skipped by starting the system in state \(s_2\), with an \(s\) subcell containing the chosen vertex as the head of the \(p\) subcell supplied to the \gls{tlc}.

% For the \gls{hcp} and \gls{tsp}, \cpruleref{rule:tsp:tsp:min} could be applied in maximally-parallel mode without any other changes to produce all possible Hamiltonian cycles (with minimum cost for the \gls{tsp}).  This results in the output into the \gls{tlc} of multiple \(p'\) subcells, to be handled further as appropriate.  Likewise for each problem, if maximally-parallel mode is used on \cpruleref{rule:tsp:tsp:start}, then the paths/cycles starting at every vertex will be generated.  All these potentially have the effect of increasing the space complexity of the system, which is not an issue for \gls{ps} with their infinite available space, but may impact software simulations.

For the \gls{hcp} and \gls{tsp}, \cpruleref{rule:tsp:tsp:min} could be applied in maximally-parallel mode without any other changes to produce all possible Hamiltonian cycles (with minimum cost for the \gls{tsp}).  This results in the output into the \gls{tlc} of multiple \(p'\) subcells, to be handled further as appropriate.  Likewise for each problem, if maximally-parallel mode is used on \cpruleref{rule:tsp:tsp:start}, then the paths/cycles starting at every vertex will be generated, at the cost of increasing the space complexity.

\Cref{sec:tsp:algotsp} assumed that all arc weights are strictly positive integers.  Weights of zero can be accommodated relatively easily.  Should every possible cycle have a minimum weight of at least 1, then no changes are needed.  If it is possible for there to be a cycle with a total weight of zero, then a sixth rule must be introduced, before the current \cpruleref{rule:tsp:tsp:min}, which in turn becomes \cpruleref{rule:tsp:minzerorules:min}.  This new \cpruleref{rule:tsp:minzerorules:zero} simply detects a minimum of zero.  The new rules are shown in \cref{rules:tsp:minzerorules}.

\cpchangerulenumber{4}

\begin{cprulesetfloat}
\begin{cpruleset}
    \cprule[rule:tsp:minzerorules:zero]{s_3}{}{1}{s_4}{\cpfunc{p'}{P} \; \cpfunc{c'}{\cpempty}}
    \cppromoter{\cpfunc{z}{\cpfunc{p}{P} \, \cpfunc{c}{\cpempty}}}
    
    \cprule[rule:tsp:minzerorules:min]{s_3}{}{1}{s_4}{\cpfunc{p'}{P} \; \cpfunc{c'}{C \cpundig W}}
    \cppromoter{\cpfunc{z}{\cpfunc{p}{P} \, \cpfunc{c}{C \cpundig W}}}
    \cpinhibitor{\cpfunc{z}{\cpfunc{p}{\_} \, \cpfunc{c}{C}}}
    
\end{cpruleset}
\caption[Rules to find the minimum cost path when it may be zero]{\label{rules:tsp:minzerorules}Rules to find the minimum cost path in the \gls{tsp} algorithm, when that path cost may be zero}
\end{cprulesetfloat}