\section{\label{sec:lr:formalmodels}Formal Models of Concurrent Computation}

The models briefly reviewed in this \lcnamecref{sec:lr:formalmodels} are just a small sample of the formal models for computation that have been devised.  An important aspect of each of the ones mentioned below is that they explicitly deal with concurrency in some fashion, thus meriting their (brief) discussion.  Ultimately, \gls{mc} is used as the model of choice for the remainder of the dissertation after this \lcnamecref{chap:lr}, but it undoubtedly has been influenced by, or developed contemporaneously with, each of the below.  Indeed, close study of each model inevitably reveals commonalities between them.

\subsection{\label{subsec:lr:csp}\Glsfmtlong{csp}}

\emph{\Gls{csp}} is a \emph{process algebra} and abstract model of concurrent computation put forward by Hoare \cite{Hoare1985,Roscoe2011}.  A typical sequential computation is represented by a \emph{process}.  Processes' \textcquote[][p.~478]{Roscoe2011}{behaviour is described in terms of the occurrence and availability of abstract entities called \textit{events}}.  Should more than one event be available simultaneously for a given process, then one will be chosen non-deterministically.  This choice is internal to the process, and not influenced by, or visible to, other processes.

Concurrency is introduced by the existence of multiple processes.  In general, the processes evolve independently, responding to events as they come.  Should a particular event appear in the alphabet of multiple processes, however, then all processes \emph{must} choose to participate in that event at the same time.  Should all processes involved make such a choice, they engage in a synchronous multi-way atomic synchronisation (hence `communicating').  \gls{csp} has provided significant inspiration for concurrency design in a number of programming languages, notably including Ada \cite{Defense1983,Taft2013}, Occam \cite{Elizabeth1987}, Google's Go \cite{Meyerson2014} (not to be confused with the earlier language Go! \cite{Clark2004}, which itself was explicitly designed for concurrency) and \gls{cml} \cite{Reppy2011} (see further \vref{sec:lr:cml}).  

\subsubsection{Pi Calculus}
Pi Calculus was created by Robin Milner and associates in the early 1990s, building upon \gls{csp}.  Milner appreciated \gls{csp}, which advanced concurrency models by explicitly incorporating \emph{synchronised} interaction, something his earlier Calculus for Communicating Systems \cite{Milner1980} had lacked  \cite{Milner1993}.  Milner still regarded \gls{csp} as incomplete, however, in that it had no support for the concept of `mobility' --- \ie{} the ability of the system to reconfigure itself during operation.  Pi Calculus was created as an attempt to build upon those earlier systems but present a complete calculus of concurrent computation in much the same way that Lambda Calculus \cite{Barendregt1984} is a complete calculus for sequential computation.\footnote{Milner also pointed out that sequential computation is a special case of concurrent computation.}

Both \gls{csp} and Pi Calculus are effective for modelling concurrent systems (see \eg{} \cite{Roscoe2011}).  They have a drawback, however, in that they tend to be very verbose.  Specifications of behaviour for processes are intertwined with the descriptions of processes' states.  They are effective for specifying a system formally, and verifying the behaviour of said system by defined reductions (see \eg{} \cite[Ch.~3.2]{Varela2013}), but for larger systems the notational burden can make it difficult to understand what each process can and will do.

\subsection{\label{subsec:lr:actors}\texorpdfstring{\Glspl{actor}}{Actors}}
The \emph{\gls{actor}} \cite{Agha1986,Agha1997} model was introduced by Carl Hewitt \cite{Hewitt1973}.  Much like \gls{csp} \& its cousins, the \gls{actor} model is based around the concept of sequential, separate but communicating processes which exchange messages.  Again, the processes make decisions independently and proceed based on their communications.  A key difference, however, is that in the \gls{actor} model the message exchanges are \emph{asynchronous}.  Each \gls{actor} has its own \emph{mailbox}, and may send messages to other \glspl{actor} so long as it knows their identity (which is equivalent for this purpose to a concept of a postal address for the \gls{actor}), \emph{but does not wait at all for a response before proceeding}.  The other major differentiator for \glspl{actor} is that the \emph{only} way to communicate with one another is to know the intended recipient's identity/address.  Logical shared memory is explicitly disallowed,\footnote{It is entirely possible in practice, of course, to implement an \gls{actor} system on a shared memory system.  The model expressly forbids any notion of a resource shared in any fashion besides \glspl{actor} requesting a result from another \gls{actor} who has exclusive access to the resource, however.} which prevents changing who an \gls{actor} communicates with simply by \eg{} changing the holder of a channel endpoint, but also means that actors can be distributed by a runtime without (directly) affecting the practical programming.

The \gls{actor} model is popular for concurrent programming, possibly owing to its intuitive concept.  The fact that communication is asynchronous makes \glspl{actor} much more suitable for modelling distributed systems without shared memory than \gls{csp} or similar --- \glspl{actor} can send messages and proceed without (necessarily) needing to wait for a response, instead continuing forward based on the messages they have received.  By contrast, a typical distributed system with synchronous communication would have prohibitive time costs, given the relative slow speed of typical links between distributed computers as compared to their capacity for local processing.  Many \gls{actor} systems have been implemented for different programming languages (\eg{} \cite{Varela2001,Srinivasan2008,Charousset2016,Bernstein2016}), and in fact it is a core component, and perhaps largely responsible for the success, of Erlang/OTP \cite{Armstrong2010,Armstrong2013,Vinoski2012}.  A relatively new language, Pony \cite{Clebsch2015,Clebsch2017}, takes this even further.

Not only do \glspl{actor} map well onto many systems which can be modelled as independent units communicating with each other, in practice they also can prove useful for controlling access to shared resources.  For example, \citeauthor{Terrell2018} describes using an \gls{actor} to control access to a dictionary data structure shared and frequently updated by multiple threads \cite{Terrell2018}.  After experimenting with using other dictionary structures designed for concurrency, \citeauthor{Terrell2018} eventually wrapped a basic dictionary in an \gls{actor}, interposing between the dictionary and the accessing threads.  The queue behaviour of the \gls{actor}'s mailbox inherently imposes an ordering on requests to access the underlying dictionary's contents.   Moreover, the \gls{actor} is the only one to access the dictionary.  These two factors combined obviate the need for any form of locking or other synchronisation.  Thus, the \gls{actor} approach proved the most performant.

Rather differently, by using a concept of ``virtual actors'' -- roughly equivalent to the concept of virtual memory vs physical memory -- \citeauthor{Bernstein2016} were able to create a system for cloud-based actors that could scale almost linearly and provide an extremely reliable service \cite{Bernstein2016}.

\Glspl{actor}, when used for non-trivial real-world software, have been criticised at times \eg{} \cite{Welsh2013,Stucchio2013}.  While some of the criticisms described are implementation-specific (relating to Akka, a Scala \gls{actor} library), a common thread is that \glspl{actor} do not compose well.  This has the negative consequence that it is difficult to combine an \gls{actor} with anything else to create a new abstraction, and can require extensive modifications in source code to make relatively simple logical changes.

\subsection{\label{sec:lr:othermodels}Other Models}

\paragraph{Petri Nets}
Perhaps the earliest formalised model of concurrent computation still in wide use is the \emph{Petri Net} \cite{Dennis2011}, first conceived of by Carl Petri to describe chemical processes \cite{Petri2008}.  A basic Petri Net is comprised of places, tokens and transitions, where tokens move between places via arcs and all places are separated by transitions (and \viceversa).

The main weakness of Petri Nets is that, in general, they require the definition of a fixed system from the outset, and do not easily permit the evolution of the system during ``runtime'' as appropriate.  Of Petri Nets, \citeauthor{Varela2013} says \textcquote[][p.~36]{Varela2013}{Petri nets \textelp{} in the \textelp{} form described \textins{in \cite{Varela2013}}, \textelp{} are not able to model the dynamicity of concurrent software\textins{.}  \textelp{} Petri nets \textelp{} do not support the compositional reasoning afforded by more modern models of concurrency.}

\paragraph{Join Calculus}
In some ways, Join Calculus -- created in the 1990s by \citeauthor{Fournet1996} \cite{Fournet1996,Fournet2002} -- is the most similar of the models discussed in this \lcnamecref{sec:lr:formalmodels} to \gls{mc}.  Both work with multisets, are inspired by the interactions and movement of atoms, and define computations through rulesets.  Join Caculus takes its inspiration direct from Chemistry and chemical reactions, however, with no reference to Biology.  Where evolution in \gls{mc} occurs within cells, reactions in Join Calculus take place in the ``chemical abstract machine.''

Join Calculus has proved fruitful for assisting in the design of new concepts in practical programming, including JoCaml \cite{Fournet2003}, Joinads \cite{Petricek2011}\footnote{See also \url{http://tryjoinads.org/docs/use/joins.html}.} and Reagents \cite{Turon2012}.  Despite this utility, however, Join Calculus seems largely to have been neglected by those outside the theoretical Computer Science and programming languages communities.

\paragraph{\label{sec:lr:pram}Parallel Random Access Machines}

\emph{\Glspl{pram}} are a model for concurrent computation arguably closer to the operation of real electronic computers.  

\begin{anfxnote}
It seems like I should probably include a summary of PRAMs here too.
\end{anfxnote}