\section{\glsentrytext{ps}/\glsentrytext{mc}}
\gls{ps}, also known as \gls{mc} (the two terms are generally used interchangeably), is a bio-inspired model of computing created by Păun in the late 1990s \cite{tPaun98a,Paun2000}, originally conceived of by considering the process of chemical reactions and exchanges that occur inside living biological cells \& the membranes within, and regarding this process as a form of computation.

Păun describes \gls{mc} \cite[p.~VII]{Paun2002} as:
\begin{quote}
Membrane computing is a branch of natural computing which abstracts from
the structure and the functioning of living cells. In the basic model, the membrane
systems - also called P systems - are distributed parallel computing
devices, processing multisets of objects, synchronously, in the compartments
delimited by a membrane structure. The objects, which correspond to chemicals
evolving in the compartments of a cell, can also pass through membranes.
The membranes form a hierarchical structure - they can be dissolved, divided,
created, and their permeability can be modified. A sequence of transitions between
configurations of a system forms a computation. The result of a halting
computation is the number of objects present at the end of the computation
in a specified membrane, called the output membrane. The objects can also
have a structure of their own that can be described by strings over a given
alphabet of basic molecules - then the result of a computation is a set of
strings.
\end{quote}

\Gls{mc} works analogously to a typical modern electronic computer, in that the system stores data, and processes \& updates those data based on a predefined program, with a view to arriving at a computable answer based on the starting state and any inputs to the system \cite{Paun2002,Paun2010b}.  In the case of \gls{ps}, the data are multisets of symbols, representing various chemicals and their quantities.  These are found inside one or more cells, based on real biological cells, which (to a certain extent at least) form a hybrid between main memory and the processing units of a computer.  The instructions of the program itself are provided by rules, which specify transformations of objects and interactions with the surrounding environment and other membranes or cells.

There are now, broadly, three main families of \gls{ps} variants:  \gls{clps} \cite{Paun2001,Paun2002}, \gls{tlps} \cite{tMaPaPaRo01a,Martin-Vide2003} and \gls{snps} \cite{Ionescu2006}.\footnote{Several other variants have been created, but most are used infrequently, if ever.  Apart from Numerical \gls{ps} and \gls{cps}, described in \autoref{subsec:numpsys} and \autoref{subsec:cpsys} respectively, this work will not address them.}  \Gls{clps} is the original, and sees objects compartmented into `membranes', which are arranged in a graphical tree structure with the outermost membrane (separating the cell from its environment) as the root of the tree.  In most variants, objects can evolve inside a membrane, but also be communicated between membranes (and the environment).  Furthermore, membranes can divide or dissolve themselves, and may have one or more special properties, such as `polarization' \cite{Paun1999a}.

Conversely, \gls{tlps} and \gls{snps} both arrange their computing compartments, named `cells' or `neurons' respectively, as nodes in arbitrary digraphs, with the edges between them representing connecting channels.  Whereas \gls{clps} emphasise the evolution of multisets of objects inside compartments, \gls{tlps} and \gls{snps} emphasise communication between separate cells/neurons, and many \gls{tlps} variants do not include any capacity for internal evolution inside cells -- if new objects are required, they are imported via communication with the environment, which is considered to possess an unlimited number of all objects, but has no rules of its own.

While \gls{tlps} have arbitrary alphabets, only one object is used in \gls{snps}, the `spike'.  This means that \gls{tlps} are frequently much like \gls{clps} in that they have custom objects for each purpose, with the key difference (usually) being in how the \glspl{prox} are arranged relative to each other and the choice between the two motivated primarily by which one seems like a better fit to the computation to be modelled.

Conversely, \gls{snps} represent everything through the use of differing quantities of the spike, kept in different neurons.  This means that it can be more complex to model certain problems, but also arguably means that \gls{snps} are, \textit{prima facie}, closer to Lambda Calculus \cite{Barendregt1984} and Church Numerals \fxerror[inline]{[ref]}, as well as Register Machines \fxerror[inline]{[ref]} (and indeed Register Machines have been simulated with \gls{snps} \fxerror[inline]{[ref]}).  All three approaches have been proven Turing-universal though \fxerror[inline]{[ref]}, so all three should be capable of expressing the same computations in different forms.  Furthermore, because \gls{snps} can easily be represented numerically, they lend themselves well to vector/matrix representations \cite{Zeng2010}.  This means that, potentially, \gls{snps} implementations can take advantage of high-performance techniques such as using \gls{blas} and/or \glspl{gpu} \cite{Aboy2019}.

Arguably, the most notable and important aspects of \gls{ps} models are that they:  i) Generally have no space limit.  That is, they contain an arbitrary number of cells, objects and membranes;  ii) Across all cells and membranes, all rules that can be applied are applied, as many times as possible given the current number of objects available.  These two features mean that \gls{ps} have unbounded space and processing capacity, which can be used to solve traditionally computationally-difficult problems relatively quickly \cite{Sosik2003,Jimenez2003,Paun1999a,Henderson2020}.  Most of these solutions, however, rely on trading time complexity for space complexity.  While this works in the theoretical framework, electronic simulations of the systems do not have access to unlimited instantaneous memory space, meaning many of the fast solutions are impractical with current real-world computers, e.g. \cite{Cooper2019} \fxnote[inline]{[refs]}.

\Gls{mc} is not just a theoretical model with limited practical use, however.  Besides Image Processing \& Computer Vision (see \autoref{subsec:imgprocpsys}), \gls{ps} variants have been applied to a range of fields, from power grid management to robotic control systems \cite{Zhang2017}.  [P-Lingua and simulation systems, e.g. MeCoSim]

\subsection{\label{subsec:numpsys}Numerical \glsentrytext{ps}}
Numerical P systemsss

\subsection{\label{subsec:cpsys}\glsentrytext{cps}}
\fxerror*{Expand/explain}{\cite{Nicolescu2014b,Nicolescu2017}}

\gls{cps} is another variant of \gls{ps}, developed by Nicolescu and collaborators in the early 2010s \fxerror[inline]{[ref]}.  It is largely based on \gls{clps}, and can be seen, to some extent at least, as a higher-level abstraction over it \cite{Nicolescu2018}.  It can also incorporate elements of \gls{tlps}, however, in that it includes concepts of channels and message passing between cells \cite{Henderson2019}.  Nicolescu, Ipate \& Wu demonstrated that not only is \gls{cps} capable of performing the same tasks as other \gls{ps} variants, but also can be used fairly cleanly to model typical computer programs \cite{Nicolescu2014a}.

The major advantage of \gls{cps} over traditional \gls{clps} is a simplification in the specification of complete systems to solve a given problem.  \gls{clps} (as well as \gls{tlps} and \gls{snps}) typically require the definition of a family of rulesets customised to the specific instance of the problem at hand, whereas \gls{cps} usually requires only the definition of a fixed (usually much shorter) set of rules that cover all possible instances.  Only inputs to the system need vary to solve different instances of the problem, e.g. in \cite{Cooper2019} only five fixed rules were needed to solve any instance of the Travelling Salesman Problem, with only customisation of the input objects (in this case, elements describing the nodes and edges of the graph) required.

\subsection{\label{sec:tsp:cpsystems}\texorpdfstring{\gls{cps}}{cP systems} : \texorpdfstring{\gls{ps}}{P systems} with Complex Symbols}
% --------------------------------------------------

In the interests of self-containment, we present here some material describing the background of \gls{cps}, for the benefit of readers as yet unfamiliar with the topic.  More extensive presentation of \gls{cps} has appeared most recently in \cite{Nicolescu2018}, and it is recommended that the interested reader peruse that paper as well.  There are two notable additions shown here that are not in \cite{Nicolescu2018}, however: the stronger semantics for inhibitors, to fully implement logical negation; and the minimum-finding algorithm explained in \autoref{sec-min}, used in solving the \gls{tsp}.  We wish to point out that, while \gls{cps} is transitively bio-inspired through its basis in \gls{ps}, it has not been developed with the aim of simulating or modelling real-world biology, and instead is intended as a useful theoretical model for computation.

\subsubsection{Complex symbols as subcells}

\emph{Complex symbols} or \emph{subcells}, 
play the roles of cellular micro-compartments or substructures,
such as organelles, vesicles or cytoophidium assemblies (``snakes''),
which are embedded in cells or travel between cells, 
but without having the full processing power of a complete cell.
In our proposal, \emph{subcells} represent nested labelled data compartments
with no processing power of their own;
instead, they are acted upon by the rules of their enclosing cells.

Our basic vocabulary consists of \emph{atoms} and \emph{variables}, 
collectively known as \emph{simple symbols}.
\emph{Complex symbols} are similar to Prolog-like \emph{first-order terms}, 
recursively built from \emph{multisets} of atoms and variables.
Together, complex symbols and simple symbols (atoms, variables) are called \emph{symbols},
%We omit a formal grammar for them here.
and can be defined by the following formal grammar:

\begin{framed}
\vspace{-0.5cm}
\begin{small}
\begin{alltt}
    <symbol> ::= <atom> | <variable> | <term> 
    <term> ::= <functor> '(' <argument> ')'
    <functor> ::= <atom>
    <argument> ::= \(\lambda\) | ( <symbol> )+
\end{alltt}
\end{small}
\vspace{-0.5cm}
\end{framed}

\emph{Atoms} are typically denoted by lower case letters (or, occasionally, digits), 
such as $a$, $b$, $c$, \(\cpundig\). 
\emph{Variables} are typically denoted by uppercase letters, 
such as $X$, $Y$, $Z$.
\emph{Functors} are term (subcell) labels; here functors can only be atoms, not variables.

For improved readability, we also consider \emph{anonymous variables}, which are denoted by underscores (``$\_$'').
Each underscore occurrence represents a \emph{new} unnamed variable
and indicates that something, in which we are not interested, must fill that slot.

Symbols that do \emph{not} contain variables are called \emph{ground}, e.g.:
\begin{itemize}
\item Ground symbols:
$a$, $a(\lambda)$, $a(b)$, $a(b c)$, $a(b^2 c)$, $a(b(c))$, $a(bc(\lambda))$, $a(b(c)d(e))$, $a(b(c)d(e))$, $a(b(c)d(e(\lambda)))$, $a(bc^2 d)$.

\smallskip
\item Symbols which are not ground:
$X$, $a(X)$, $a(bX)$, $a(b(X))$, $a(XY)$, $a(X^2)$, $a(XdY)$,  $a(Xc())$, $a(b(X)d(e))$, $a(b(c)d(Y))$, $a(b(X^2)d(e(Xf^2)))$;
also, using anonymous variables: $\_$, $a(b\_)$, $a(X\_)$, $a(b(X)d(e(\_)))$.

\smallskip
\item This term-like construct which starts with a variable is not a symbol (this grammar defines first-order terms only):
$X(a Y)$.
\end{itemize}

Note that we may abbreviate the expression of complex symbols 
by removing inner $\lambda$'s as explicit references to the empty multiset, 
e.g.~$a(\lambda) = a()$.

In \emph{concrete} models, \emph{cells} may contain \emph{ground} symbols only (no variables).
Rules may however contain \emph{any} kind of symbols, atoms, variables and terms (whether ground or not).

\medskip
\noindent
\textbf{Unification.} 
All symbols which appear in rules (ground or not) can be (asymmetrically) \emph{matched} against \emph{ground} terms,
using an ad-hoc version of \emph{pattern matching}, 
more precisely, a \emph{one-way first-order syntactic unification} (one-way, because cells may not contain variables).
An atom can only match another copy of itself, but
a variable can match any multiset of ground terms (including $\lambda$).
This may create a combinatorial \emph{non-determinism}, 
when a combination of two or more variables are matched against the same multiset,
in which case an arbitrary matching is chosen. 
For example:
\begin{itemize}
\item Matching $a(b(X)fY) = a(b(cd(e))f^2g)$ deterministically creates a single set of unifiers:
$X, Y = cd(e), fg$.

\smallskip
\item Matching $a(XY^2) = a(de^2f)$ deterministically creates a single set of unifiers: 
$X, Y = df, e$.

\smallskip
\item Matching $a(b(X)c(\cpundig X)) = a(b(\cpundig^2)c(\cpundig^3))$ deterministically creates one single unifier: 
$X = \cpundig^2$.

\smallskip
\item Matching $a(b(X)c(\cpundig X)) = a(b(\cpundig^2)c(\cpundig^2))$ fails.

\smallskip
\item Matching $a(XY) = a(df)$ non-deterministically creates one of the following four sets of unifiers: 
$X, Y = \lambda, df$; $X, Y = df, \lambda$; $X, Y = d, f$; $X, Y = f, d$. 
\end{itemize}

\iffalse
\noindent
\textbf{Performance note.}
If the rules avoid any matching non-determinism, then
this proposal should not affect the performance of P~simulators running on existing machines.
Assuming that bags are already taken care of, e.g.~via hash-tables,
our proposed unification probably adds an almost linear factor.
Let us recall that, in similar contexts (no occurs check needed), 
Prolog unification algorithms can run in $O(n g(n))$ steps,
where $g$ is the inverse Ackermann function.
Our conjecture must be proven though, 
as the novel presence of multisets may affect the performance.
\fi

% -------------------------------------------------

\subsubsection{High-level or generic rules}

Typically, our rules use \emph{states} and are applied top-down, in the so-called \emph{weak priority} order.

\smallskip
\noindent
\textbf{Pattern matching.}
Rules are matched against cell contents using the aforementioned \emph{pattern matching},
which involves the rule's left-hand side, promoters and inhibitors -- 
promoters and inhibitors are further discussed below, in a following paragraph.

Generally, variables have \emph{global rule scope};
these are assumed to be introduced by \emph{existential} quantifiers preceding the rule
-- with the exception of inhibitors, which may introduce \emph{local variables}, 
as further discussed below. 

The matching is \emph{valid} only if, after substituting variables by their values, 
the rule's right-hand side contains ground terms only
(so \emph{no} free variables are injected in the cell or sent to its neighbours),
as illustrated by the following sample scenario:
\begin{itemize}
\item The cell's \emph{current content} includes the \emph{ground term}:\\
%\smallskip
$n(a \, \phi(b \, \phi(c) \, \psi(d)) \, \psi(e))$

\smallskip
\item The following (state-less) \emph{rewriting rule} is considered: \\ 
%\smallskip
$n(X \, \phi(Y \, \phi(Y_1) \, \psi(Y_2)) \, \psi(Z)) ~ \rightarrow ~ v(X) \: n(Y \, \phi(Y_2) \, \psi(Y_1)) \: v(Z)$

\smallskip
\item Our pattern matching determines the following \emph{unifiers}: \\
%\smallskip
$X = a$, $Y = b$, $Y_1 = c$, $ Y_2 = d$, $Z = e$.

\smallskip
\item This is a \emph{valid} matching and, after \emph{substitutions}, 
the rule's \emph{right-hand} side gives the \emph{new content}: \\
%\smallskip
$v(a) ~ n(b \, \phi(d) \, \psi(c)) ~ v(e)$
\end{itemize}

\noindent
\textbf{Generic rules format.}
We consider rules of the following \emph{generic} format 
(we call this format generic, because it actually defines templates involving variables):
\begin{framed}
\vspace{-0.6cm}
\begin{align*}
\emph{current-state} ~~ \emph{symbols} \dots ~ \rightarrow_\alpha ~ & \emph{target-state} ~~ (\emph{in-symbols}) \dots ~~ \\
 & (\emph{out-symbols})_\delta \dots \\
 & | ~  \emph{promoters} \dots ~~ \neg ~  \emph{inhibitors} \dots
\end{align*}
\vspace{-0.8cm}
\end{framed}
Where:
\begin{itemize}
\item \emph{current-state} and \emph{target-state} are atoms or terms;

\smallskip
\item \emph{symbols}, \emph{in-symbols}, \emph{promoters} and \emph{inhibitors} are symbols;

\smallskip
\item \emph{in-symbols} become available after the end of the current step only, as in traditional \gls{ps}  (we can imagine that these are sent via an ad-hoc fast \emph{loopback} channel); 

\smallskip
\item subscript $\alpha$ $\in$ $\{1$, $+\}$, 
indicates the application mode,
as further discussed in the example below;

\smallskip
\item \emph{out-symbols} are sent, at the end of the step, to the cell's structural neighbours.
These symbols are enclosed in round parentheses which further indicate 
their destinations, above abbreviated as $\delta$. 
The most usual scenarios include: 

\begin{itemize}
\item $(a)\downarrow_i$ indicates that $a$ is sent over outgoing arc $i$ (unicast); 

\item $(a)\downarrow_{i,j}$ indicates that $a$ is sent over outgoing arcs $i$ and $j$(multicast); 

\item $(a)\downarrow_\forall$ indicates that $a$ is sent over all outgoing arcs (broadcast). 
\end{itemize}

All symbols sent via one \emph{generic rule} to the same destination form one single \emph{message} and they travel together as one single block (even if the generic rule is applied in mode $\scriptstyle + \displaystyle$).
\end{itemize}

\smallskip
\noindent
\textbf{Promoters and inhibitors.}
To define additional useful matchings expressively, 
our promoters and inhibitors may also use virtual ``equality'' terms, 
written in infix format, with the $=$ operator.
For example, including the term $(ab = XY)$ indicates the following additional matching constraints on variables $X$ and $Y$: either $X, Y = ab, \lambda$; or $X, Y = a, b$; or $X, Y = b, a$; or $X, Y = \lambda, ab$.

To usefully define inhibitors as logical negations,
variables which only appear in the scope of an inhibitor are assumed to have \emph{local scope}. 
These variables are assumed to be defined by \emph{existential} quantifiers, immediately after the negation. 
Semantically, this is equivalent as introducing these variables at the global rule level, 
but by \emph{universal} quantifiers, after all other global variables,
which are introduced by \emph{existential} quantifiers.

As an illustration, consider a cell containing $a(c) ~ a(ccc)$ and contrast two rules, 
containing the following two sample promoter/inhibitor pairs 
(for brevity, other rule details are omitted here).

\lstset{xleftmargin=.5in, xrightmargin=.5in} 
\begin{lstlisting}
... $\mid$  $a(cXY)$     $\neg$  $a(X)$    #\hfill (1)\enspace#
... $\mid$  $a(cZ)$     $\neg$  $(Z=XY)$  $a(X)$    #\hfill (2)\enspace#
\end{lstlisting}

These two rules appear quite similar and their inhibitor tests share the same expression: 
NO $a(X)$ may be present in the cell.

Rule (1) uses two global variables, $X, Y$. 
According to its promoter, $a(cXY)$, these variables can be matched in four different ways:
(1a) $X, Y = \lambda, \lambda$; (1b) $X, Y = cc, \lambda$; (1c) $X, Y = \lambda, cc$; (1d) $X, Y = c, c$.
Three different unifications, (1a), (1b), (1c), pass the inhibitor test, 
as there are no cell terms $a()$, $a(cc)$, $a()$, respectively. 
Unification (1d) fails the inhibitor test, because there IS one cell term $a(c)$.

Rule (2) uses one global variable, $Z$, and two local inhibitor variables, $X, Y$.
According to its promoter, $a(cZ)$, variable $Z$ can be matched in two different ways: 
(2a) $Z = \lambda$; (2b) $Z = cc$.
Unification (2a) passes the inhibitor test, because it only generates one local unification,
$X, Y = \lambda, \lambda$, and there is NO cell term $a()$.
Unification (2b) fails the inhibitor test, because it generates all the following three local unifications:
(2b1) $X, Y = cc, \lambda$; (2b2) $X, Y = \lambda, cc$; (2b3) $X, Y = c, c$; 
and there IS a cell term corresponding to (2b3), $a(c)$.

The pattern of rule (2) will be used later, in \autoref{sec-min}, 
to define a single step minimum-finding ruleset.

\smallskip
\noindent
\textbf{Application modes -- $1$ and $+$.}
To explain our two rule application modes, $1$ and $+$,
let us consider a cell, $\sigma$, containing three counter-like complex symbols,
$c(\cpundig^2)$, $c(\cpundig^2)$, $c(\cpundig^3)$,
and the two possible application modes of the following high-level ``decrementing'' rule:
\vspace{-0.2cm}
\begin{framed}
\vspace{-0.5cm}
$$(\rho_\alpha) ~S_1 ~c(\cpundig \, X) \rightarrow_{\alpha} S_2 ~c(X),\\
\mathrm{where} \; \alpha \in \{\scriptstyle 1 \displaystyle, \scriptstyle + \displaystyle\}.$$
\vspace{-0.8cm}
\end{framed}
%\vspace{-0.3cm}

The left-hand side of rule $\rho_\alpha$, $c(\cpundig \, X)$, can be unified in three different ways,
to each one of the three $c$ symbols extant in cell $\sigma$.
Conceptually, we instantiate this rule in three different ways,
each one tied and applicable to a distinct symbol:
\begin{eqnarray*}
& (\rho_1)  & ~S_1 ~c(\cpundig^2) \rightarrow S_2 ~c(\cpundig),\\
& (\rho_2)  & ~S_1 ~c(\cpundig^2) \rightarrow S_2 ~c(\cpundig),\\
& (\rho_3) & ~S_1 ~c(\cpundig^3) \rightarrow S_2 ~c(\cpundig^2).
\end{eqnarray*}

\begin{enumerate}
\item If $\alpha = \: \scriptstyle 1 \displaystyle$, rule~$\rho_1$ 
non-deterministically selects and applies one of these virtual rules $\rho_1$, $\rho_2$, $\rho_3$.
Using $\rho_1$ or $\rho_2$, 
cell $\sigma$ ends with counters $c(\cpundig)$, $c(\cpundig^2)$, $c(\cpundig^3)$.
Using $\rho_3$,
cell $\sigma$ ends with counters $c(\cpundig^2)$, $c(\cpundig^2)$, $c(\cpundig^2)$.

\smallskip
\item If $\alpha = \: \scriptstyle + \displaystyle$, rule~$\rho_+$ 
applies in parallel all these virtual rules $\rho_1$, $\rho_2$, $\rho_3$.
Cell $\sigma$ ends with counters $c(\cpundig)$, $c(\cpundig)$, $c(\cpundig^2)$.
\end{enumerate}

Semantically, the $+$ mode is equivalent to a virtual sequential while loop around the same rule in $1$ mode, which is repeated until it is no more applicable.  Note, however, that all such applications of the rule are carried out concurrently in a single step.

\smallskip
\noindent
\textbf{Special cases.}
Simple scenarios involving generic rules are sometimes 
semantically equivalent to sets of non-generic rules defined via bounded loops.
For example, consider the rule
$$
S_1 ~ a(x(I) \; y(J)) ~ \rightarrow_+ ~ S_2 ~ b(I) ~ c(J),
$$
where the cell's contents guarantee that $I$ and $J$ 
only match integers in ranges $[1,n]$ and $[1,m]$, respectively.
Under these assumptions, 
this rule is equivalent to the following set of non-generic rules:
$$
S_1 ~ a_{i,j} ~ \rightarrow S_2 ~ b_i ~ c_j, ~ \forall i \in [1,n], j \in [1,m].
$$

However, unification is a much more powerful concept, 
which cannot be generally reduced to simple bounded loops.

\smallskip
\noindent
\textbf{Benefits.}
This type of generic rules allows (i) a reasonably fast parsing and processing of subcomponents, and
(ii) algorithm descriptions with \emph{fixed-size alphabets} and \emph{fixed-sized rulesets}, 
independent of the size of the problem and number of cells in the system (often \emph{impossible} with only atomic symbols).

% \smallskip
% \noindent
% \textbf{Synchronous vs asynchronous.}
% In our models, we do not make any \emph{syntactic} difference between the synchronous and asynchronous scenarios;
% this is strictly a \emph{runtime} assumption~\cite{N-CMC-LNCS-2012}.
% Any model is able to run on both the synchronous and asynchronous runtime ``engines'',
% albeit the results may differ.
% Our asynchronous model matches closely the standard definition for asynchonicity used in distributed algorithms;
% however, this is not needed in this paper so we don't follow this topic here.

% -------------------------------------------------

% --------------------------------------------------
\subsubsection{Data structures in \gls{cps}}\label{sec-data-structures}
% --------------------------------------------------

In this subsection we sketch the design of two high-level data structures, 
similar to the data structures used in high-level pseudocode or %high-level 
programming languages:
natural numbers and lists, together with alternative more legible notations
% numbers, relations, functions, associative arrays, lists, trees, strings, 
% together with alternative more readable notations.

\medskip
\noindent
\textbf{Natural numbers.} Natural numbers can be represented via \emph{multisets} containing repeated occurrences of the \emph{same} atom.
For example, considering that $\cpundig$ represents an ad-hoc unary digit, 
the following complex symbols can be used to describe 
the contents of a virtual integer \emph{variable} $a$: 
$a () = a(\lambda)$ --- the value of $a$ is 0;
$a(\cpundig^3)$ --- the value of $a$ is 3.
For concise expressions, we may alias these number representations by their corresponding numbers, e.g.~$a() \equiv a(0), b(\cpundig^3) \equiv b(3)$.
Nicolescu et al.~\cite{Nicolescu2014,RN-HW-ROMJIST14} show how the basic arithmetic operations can be efficiently modelled by \gls{ps} with complex symbols.

Here follows a list of simple arithmetic expressions, assignments and comparisons:

\lstset{xleftmargin=.5in, xrightmargin=.5in} 
\begin{lstlisting}
  $x = 0$ $\equiv$ $x(\lambda)$
  $x = 1$ $\equiv$ $x(\cpundig)$
  $x = 2$ $\equiv$ $x(\cpundig \cpundig)$
  $x = n$ $\equiv$ $x(\cpundig^n)$
  
  $x \leftarrow y + z$ $\equiv$ $y(Y) ~ z(Z) ~ \rightarrow ~ x(YZ)$ #\hfill\textsl{destructive add}\enspace#
  $x \leftarrow y + z$ $\equiv$ $ \rightarrow ~ x(YZ) ~ \mid ~ y(Y) ~ z(Z)$ #\hfill\textsl{preserving add}\enspace#
  
  $x = y$ $\equiv$  $x(X) ~y(X)$ #\hfill\textsl{equality}\enspace#
  $x \leq y$ $\equiv$  $x(X) ~y(XY)$ #\hfill\textsl{less than or equal to}\enspace#
  $x <  y$ $\equiv$  $x(X) ~y(X1Y)$ #\hfill\textsl{strictly less than}\enspace#
\end{lstlisting}

Note that strictly less than (\(<\)) requires the extra \(1\), because \(Y\) can match on \(\lambda\).

% \medskip
% \noindent
% \textbf{Relations and functions.} Consider the \emph{binary relation} $r$, 
% defined by: 
% $r = \{ (a, b)$, $(b, c)$, $(a, d)$, $(d, c) \}$ (which has a diamond-shaped graph). 
% Using complex symbols, relation $r$ can be represented as a \emph{multiset} with four $r$ items,
% $\{ r(\kappa(a) ~ \upsilon(b))$, $r(\kappa(b) ~ \upsilon(c))$, $r(\kappa(a) ~ \upsilon(d))$, $r(\kappa(d) ~ \upsilon(c)) \}$, 
% where ad-hoc atoms $\kappa$ and $\upsilon$ introduce \emph{domain} and \emph{codomain} values (respectively).
% We may also alias the items of this multiset by a more expressive notation such as: $\{ (a \stackrel{r}\rightleftarrows b)$, $(b \stackrel{r}\rightleftarrows c)$, $(a \stackrel{r}\rightleftarrows d)$, $(d \stackrel{r}\rightleftarrows c) \}$.

% If the relation is a \emph{functional relation}, then we can emphasise this by using another operator, such as ``mapsto''. For example, the functional relation 
% $f = \{ (a, b)$, $(b, c)$, $(d, c) \}$ can be represented by multiset
% $\{ f(\kappa(a) ~ \upsilon(b))$, $f(\kappa(b) ~ \upsilon(c))$, $f(\kappa(d) ~ \upsilon(c)) \}$ or by the more suggestive notation: 
% $\{ (a \stackrel{f}\mapsto b)$, $(b \stackrel{f}\mapsto c)$, $(d \stackrel{f}\mapsto c) \}$.
% To highlight the actual mapping value, instead of $a \stackrel{f}\mapsto b$,
% we may also use the succinct abbreviation $f[a] = b$.

% In this context, the $\rightleftarrows$ and $\mapsto$ operators are considered to have a high associative priority, so the enclosing parentheses are mostly used for increasing the readability.

% \medskip
% \noindent
% \textbf{Associative arrays.} Consider the \emph{associative array} $x$, 
% with the following key-value mappings (i.e. functional relation): 
% $\{ \cpundig \mapsto a; \cpundig^3 \mapsto c; \cpundig^7 \mapsto g \}$. 
% Using complex symbols, array $x$ can be represented as a multiset with three items,
% $\{ x(\kappa(\cpundig)\,\upsilon(a))$, $x(\kappa(\cpundig^3)\,\upsilon(c))$, $x(\kappa(\cpundig^7)\,\upsilon(g)) \}$, 
% where ad-hoc atoms $\kappa$ and $\upsilon$ introduce keys and values (respectively).
% We may also alias the items of this multiset by the more expressive notation
% $\{ \cpundig \stackrel{x}\mapsto a$, $\cpundig^3 \stackrel{x}\mapsto c$, $\cpundig^7 \stackrel{x}\mapsto g \}$.

\medskip
\noindent
\textbf{Lists.} Consider the \emph{list} $y$, containing the following sequence of values: 
$[u; v; w]$. 
List $y$ can be represented as the complex symbol
$y(\, \gamma(u~\gamma(v~\gamma(w~\gamma()))))$, 
where the ad-hoc atom $\gamma$ represents the list constructor \emph{cons} and $\gamma()$ the empty list.
We may also alias this list by the more expressive equivalent notation
$y(u\,|\,v\,|\,w)$
-- or by $y(u\,|\,y')$, $y'(v\,|\,w)$ --
where operator $\mid$ separates the head and the tail of the list.
The notation $z(|)$ is shorthand for $z(\gamma())$ and indicates an empty list, $z$.

% \medskip
% \noindent
% \textbf{Trees.} Consider the \emph{binary tree} $z$, described by the structured expression \\
% $(a, (b), (c, (d), (e)))$, 
% i.e.~$z$ points to a root node which has: 
% (i) the value $a$; 
% (ii) a left node with value $b$; and 
% (iii) a right node with value $c$, left leaf $d$, and right leaf $e$. 
% Tree $z$ can be represented as the complex symbol
% $z(a ~ \phi(b) ~ \psi(c ~ \phi(d) ~ \psi(e)))$, 
% where ad-hoc atoms $\phi, \psi$ introduce left subtrees, right subtrees (respectively).

% \medskip
% \noindent
% \textbf{Strings.} Consider the \emph{string} $s = ``abc"$, 
% where $a$, $b$, and $c$ are atoms. 
% Obviously, string $s$ can be interpreted as the list $s = [a; b; c]$, i.e.
% string $s$ can be represented as the complex symbol
% $s(\, \gamma(a~\gamma(b~\gamma(c~\gamma()))))$, etc.

% --------------------------------------------------
\subsubsection{Efficient minimum-finding with cP~rules}\label{sec-min}
% --------------------------------------------------

Consider an unstructured multiset $A \subseteq \mathbb{N}$ of size $n$. 
It is well known that (1) any sequential algorithm that finds its minimum needs at least $n$ steps, and 
(2) any parallel algorithm that finds its minimum needs at least $\log n$ parallel steps.

Without loss of generality, consider a cP~system cell, in state $S_1$, where multiset $A$ is given via functor $a$; 
e.g., multiset $A = \{ 1, 2, 2, 5 \}$ is represented as $a(1) a(2) a(2) a(5)$.
The following rulesets implement various versions of a cP~system minimum-finding algorithm.
All these rulesets transit to state $S_2$ and construct a term with functor $b$, containing $\min A$.
Some of these are destructive processes; if otherwise desired, one could first make a copy of the initial multiset $A$.

The following destructive ruleset is an emulation of the classical sequential minimum finding algorithm, which takes $n$ steps:

\lstset{xleftmargin=.5in, xrightmargin=.5in} 
\begin{lstlisting}
$S_1$  $a(X)$  $\rightarrow_{1}$  $S_2$  $b(X)$ 
$S_2$  $a(XY)$  $b(X)$  $\rightarrow_{1}$  $S_2$  $b(X)$     #\hfill  $a \geq b  $ \enspace #
$S_2$  $a(X)$  $b(X1Y)$  $\rightarrow_{1}$  $S_2$  $b(X)$   #\hfill  $a < b  $ \enspace #
\end{lstlisting}

The following destructive ruleset is an emulation of the classical parallel minimum finding algorithm, which takes $\log n$ steps.
As long as there are more than one term $a$, the ruleset loops in state $S_1$, keeping minima between pairs.
When only one $a$ remains (containing the minimum value), the ruleset transits to state $S_2$ and tags the minimum. 

\lstset{xleftmargin=.5in, xrightmargin=.5in} 
\begin{lstlisting}
$S_1$  $a(XY)$  $a(X)$  $\rightarrow_{+}$  $S_1$  $a(X)$     
$S_1$  $a(X)$  $a(X1Y)$  $\rightarrow_{+}$  $S_1$  $a(X)$    
$S_1$  $a(X)$  $\rightarrow_{1}$  $S_2$  $b(X)$  
\end{lstlisting}

However, using the full associative power of \gls{cps}, we can find a non-destructive version with two rules, 
which works in \emph{just two steps} (regardless of the set cardinality). 
This is a substantial improvement over existing classical algorithms (both sequential and parallel). 
It starts by making a full copy of $a$ as $b$, in one $+$-parallel step, 
and then deletes all non-minimal $b$ values in another $+$-parallel step. 

\lstset{xleftmargin=.5in, xrightmargin=.5in} 
\begin{lstlisting}
$S_1$  $\rightarrow_{+}$  $S_1'$  $b(X)$    $\mid$  $a(X)$  
$S_1'$  $b(X1Y)$  $\rightarrow_{+}$  $S_2$    $\mid$  $a(X)$  
\end{lstlisting}

Note that, if the minimum value appears several times in multiset $A$, 
then we will end with the same multiplicity of $b$'s, each one containing the same value, $\min A$.
If this is required, there are several ways to select only one copy and delete the rest --
but we do not further deal with this issue here.

Moreover, using the full power of cP~inhibitors (as logical negations, with local variables), 
we can even non-destructively solve the problem in just \emph{one single step},
with one or two rules.
This version is implemented by the following ruleset:

\lstset{xleftmargin=.5in, xrightmargin=.5in} 
\begin{lstlisting}
$S_1$  $\rightarrow_{1}$  $S_2$  $b()$    $\mid$  $a()$
$S_1$  $\rightarrow_{1}$  $S_2$  $b(1Z)$     $\mid$  $a(1Z)$     $\neg$  $(Z=XY)$  $a(X)$
\end{lstlisting}

If $A$ contains zero, then there is a term $a()$, and: (1) the first rule applies, constructing $b()$; (2) the second rule is not applicable.
Otherwise (if there is no zero in $A$): (1) the first rule is not applicable; (2) the second rule constructs $b(1Z)$, 
a value which exists among $a$'s, as $a(1Z)$, but there is NO other $a$ containing a strictly lesser value, such as $a(X)$,
where $X$ is a sub-multiset of $Z$, $X \subseteq Z$.
In the end, the newly constructed $b$ will contain one copy of the minimum value of multiset $A$.

If multiset $A$ does not contain zero values, i.e. $A \subseteq \mathbb{N}^+$, then the first rule can be safely omitted (as it will never be applicable). 
A similar ruleset can be devised for finding the maximum of a given set of natural numbers.

\subsection{\label{subsec:imgprocpsys}Image Processing and Computer Vision in P~systems}
\fxerror*{Expand/explain}{\cite{Zhang2012}}

Perhaps owing to the unbounded potential space and parallelism of \gls{ps}, combined with the embarrassingly parallel nature of many tasks in Image Processing \& Computer Vision, the latter has proved to be fertile ground for the former, although not every publication puts its model to the test with a computerised simulation, or if it does, the authors may only provide scant details \cite{Diaz-Pernil2019}.  

Christinal, Díaz-Pernil \& Real \cite{Christinal2011} described a family of \gls{tlps} to perform region-based segmentation of both 2D and 3D images.  Despite their family of systems requiring only two cells, it also needed custom rule sets based on the size of the images as well as the number of colours present, with a number of rules per set proportional to the same measurements.  The paper showed the results of simulating the system, but provides no details on performance.

Díaz-Pernil \textit{et al.} \cite{Diaz-Pernil2013} commented that ``... commonly [a] parallel algorithm needs to be re-designed with only slight references to the [sequential original].  ... the design of a new parallel implementation not inspired by the sequential one allows ... the proposal of new creative solutions.''  They then demonstrated this fact by designing a new edge detection and segmentation algorithm named `A Graphical P (AGP) segmentator', inspired by the Sobel operator (see e.g. \cite{Nixon2012}) and using the segmentation method from \cite{Christinal2011}, which they modelled in \gls{tlps}.  The authors implemented their new algorithm on a \gls{gpu} and compared it with an implementation of the 3x3 and 5x5 Sobel operators, finding that theirs had near-identical runtimes but superior edge detection capabilities.

Díaz-Pernil \textit{et al.} \cite{Diaz-Pernil2013a} further explored modelling classic image processing techniques by implementing Guo \& Hall's binary image skeletonisation technique \cite{Guo1989} with \gls{snps}.  The overall system's rules templates are reasonably simple, but include references to a set \(DEL\) (used as a lookup to determine whether a cell should turn white or stay black) which does not appear to be modelled inside the system, meaning that it is not self-contained.  The authors simulated this system on a \gls{gpu}, but found that their implementation was upwards of twice as slow as another pre-existing implementation.  Confusingly, however, they state that one of the reasons for this is ``that the use of an alphabet with only one object, the spike \(a\), does not fit in the GPU architecture''.  This statement is difficult to understand, given that spikes can easily be represented as simple integers.  The authors also commented that the synchronous nature of the model is unrealistic, and imposing a global clock upon the system can be problematic.

Nicolescu \cite{Nicolescu2014} alternatively applied \gls{cps} to image skeletonisation based on Guo \& Hall's technique \cite{Guo1989}, presenting three forms of a solution: Synchronous versions that use multiple or a single cell (essentially the latter replicates the former via the use of sub-membranes), and an asynchronous multi-cell version.  The asynchronous version no longer assumes that all messages are passed between cells simultaneously and instantaneously, compensating for this by increasing the number of messages used.  This form, while arguably more realistic to modern computers, requires a greater message complexity. A prospective \gls{actor}-model-based (see \autoref{subsec:actors}) simplified implementation using \fsharp{}'s \texttt{MailboxProcessor} \cite[ch.~11]{Syme2015a} was presented also, but no results from running it were reported.

Nicolescu \cite{Nicolescu2015a} further applied \gls{cps} to seeded region growing of grayscale images.  The described system used a two-level approach, based on the `Structured Grid Dwarf' of the 13 Berkeley Dwarves \cite{Asanovic2006}, where the image was divided into rectangular blocks of multiple pixels.  Each block was modelled with a single cell, inter-block processing was carried out via message passing, and intra-block processing was performed by typical object evolution.  It was again suggested that this would fit well to the \gls{actor} model.

Díaz-Pernil \textit{et al.} \cite{Diaz-Pernil2016} built upon their AGP segmentator algorithm to create a version that works with RGB images rather than grayscale and applied it to a common medical Computer Vision task, isolating the `optic disc' in images of the inner eye.  With this they used the skeletonisation algorithm from \cite{Diaz-Pernil2013a} and a number of other steps not based on \gls{ps} to produce a complete imaging pipeline.  The authors implemented this on a \gls{gpu}, and found that their system was both more accurate and faster than previous systems.

% Most directly relevant to the current work are \cite{GimelFarb2013a,Gimelfarb2011,Nicolescu2014b}, which model Dynamic Programming \gls{sm} in \gls{ps}, and indeed saw the genesis of \gls{cp} (\autoref{subsec:concprop}).  

Most directly relevant to the current work are \cite{GimelFarb2013a,Gimelfarb2011,Nicolescu2014b}, which model Dynamic Programming \gls{sm} in \gls{ps}, and indeed saw the genesis of \gls{cp}.  

\subsubsection{\glsentrytext{ps} on \glsentrylongpl{gpu}}
In many instances, a \gls{ps} model for a problem involves many separate small elements processing their data separately, and perhaps updating each other's state at the end of a step.  Given that this sounds remarkably close to the Single-Instruction Multiple-Thread \cite[Ch. 4.4.1]{Hennessy2012} nature of modern \gls{gpgpu}, it is no surprise that there has been much work put into simulating \gls{ps} on \glspl{gpu}.

\fxerror*{Expand/explain all these}{\cite{Cecilia2010,Cecilia2010a,Cecilia2013,Macias-Ramos2015,Martinez-Del-Amor2015,Martinez-Del-Amor2013a,Maroosi2014,Maroosi2014a}}

