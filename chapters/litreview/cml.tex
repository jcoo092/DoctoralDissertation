\section{\glsentrylong{cml-glossary} \& related}

\begin{anfxerror}{\gls{cml} no longer relevant?}
    This entire section is now not nearly so relevant.  Not sure what to replace it with/modify it to, however.  Maybe something about reported results of implementations of the models of concurrent computation?  (In which case, I could probably just fold it into that section)
\end{anfxerror}

\Gls{cml} \cite{Reppy1991,Panangaden1997} is an approach to concurrent programming originally developed by John Reppy (based on his earlier `Pegasus Meta-Language' \cite{Reppy1988}).  It was created originally as a library in Standard ML of New Jersey, but its concepts have subsequently appeared elsewhere.   \Gls{cml} is designed to avoid many of the problems with concurrency that arise in traditional sequential programming, where the use of locks, mutexes and semaphores etc. are frequently required, and often lead to the potential introduction of problems such live/deadlocks, data races and extreme resource contention.  This is achieved by changing the approach to concurrent programming to one of logically separate, internally sequential processing elements that share data as required by `passing messages'\footnote{This is the logical concept, but there is not strictly any specific required software implementation.} between themselves.  In CML, these logical processing elements are referred to as threads, and they exchange messages over channels synchronously, i.e. there is a temporal overlap between one thread offering to send, and another to receive, over the same channel, and the first to offer blocks until the second makes its offer.

Reppy describes a concurrent program as one that supports multiple sequential sub-programs conceptually executing in parallel separately, but interacting through shared resources to achieve a common goal.  CML is concerned with the scenario where said interactions are explicit, and in order to facilitate that \enquote{CML takes the unique approach of supporting \emph{higher-order concurrent programming}} (emphasis Reppy's), whereby communication and synchronisation are made into first-class members of the language, similar to how functional programming languages made functions into first-class members of themselves \cite[Preface]{Reppy2007}.

