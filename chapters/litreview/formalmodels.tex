\section{Formal Models of Concurrent Computation}
Perhaps the earliest (or at least, the earliest that is still widely known) model of concurrent computation is the Petri Net.

\cite{Varela2013}

\subsection{\glsentrylong{csp} \& Pi~Calculus}
\gls{csp} is a `process algebra' and abstract model of concurrent computation put forward by Hoare \cite{Hoare1985,Roscoe2011}.  A typical sequential computation is represented by a `process'.  Processes' ``behaviour is described in terms of the occurrence and availability of abstract entities called \textit{events}'' \cite[p.~478]{Roscoe2011}.  Should more than one event be available simultaneously for a given process, then one will be chosen non-deterministically.  This choice is internal to the process, and not influenced by or visible to any other process.  %E.g. for a situation where there is one possible event \(a\) at a given point in time, the process \(P\) will choose that event and then proceed according to the result, written as: \[ a \rightarrow P(a) \]

Concurrency is introduced by the existence of multiple processes.  In general, the processes evolve independently, responding to events as they come.  Should a particular event appear in the alphabet of multiple processes, however, then all processes \emph{must} choose to participate in that event at the same time.  Should all processes involved make such a choice, they engage in a synchronous multi-way atomic synchronisation (hence `communicating').  \gls{csp} has provided significant inspiration for concurrency design in a number of programming languages, notably including Ada \cite{Defense1983,Taft2013}, Occam \cite{Elizabeth1987}, Google's Go \cite{Meyerson2014} (not to be confused with the earlier language Go! \cite{Clark2004}, which itself was explicitly designed for concurrency) and \gls{cml} \cite{Reppy2011}. 

Milner appreciated \gls{csp}, which advanced concurrency models by explicitly incorporating \emph{synchronised} interaction, something Milner's earlier Calculus for Communicating Systems \cite{Milner1980} had lacked  \cite{Milner1993}.  Milner still regarded \gls{csp} as incomplete, however, in that it had no support for the concept of `mobility' -- i.e. the ability of the system to reconfigure itself during operation.  Pi Calculus was created as an attempt to build upon those earlier systems but present a complete calculus of concurrent computation in much the same way that Lambda Calculus \cite{Barendregt1984} is a complete calculus for sequential computation.\footnote{Milner also pointed out that sequential computation is, in fact, a special case of concurrent computation.}

\subsection{\label{subsec:actors}Actors}
The \gls{actor} \cite{Agha1986} model was introduced by Hewitt \fxnote[inline]{[ref]}.  Much like \gls{csp} \& its cousins, the \gls{actor} model is based around the concept of separated, sequential but communicating processes which exchange messages.  Again, the processes make decisions and proceed based on their communications.  A key difference, however, is that in the \gls{actor} model the message exchanges are \emph{a}synchronous.  Each \gls{actor} has its own `mailbox', and may send messages to other actors so long as it knows their name (which is equivalent for this purpose to a concept of an address for the \gls{actor}), \emph{but does not wait at all for a response before proceeding}.

The \gls{actor} model is a popular one for concurrent programming, possibly owing to its intuitive concept.  The fact that communication is asynchronous makes Actors much more suitable for modelling distributed systems without shared memory than \gls{csp} or similar -- Actors can send messages and proceed without (necessarily) needing to wait for a response, instead continuing to process based on the messages they themselves have received.  By contrast, a system with synchronous communication would have prohibitive time costs, given the relative slow speed of typical links between distributed computers as compared to their capacity for local processing.  Many \gls{actor} systems have been implemented for different programming languages (e.g. \cite{Varela2001,Srinivasan2008,Charousset2016} \fxnote[inline]{[need more refs?]}), and in fact it is a core component, and perhaps largely responsible for the success, of Erlang/OTP \cite{Armstrong2010,Armstrong2013}.  A relatively new language, Pony \cite{Clebsch2015,Clebsch2017}, takes this even further.
\begin{anfxnote}{Positive example for actors?}
In Concurrency in .NET, Terrell described using an actor to control access to a shared resource and stated that it was an extremely effective solution.
\end{anfxnote}

\Glspl{actor}, when used for non-trivial real-world software, have been criticised at times \cite{Welsh2013,Stucchio2013}.  While some of the criticisms described are implementation-specific (Akka, a Scala \gls{actor} library), a common thread is that \glspl{actor} do not compose well.  This has the negative consequence that it is difficult to combine an \gls{actor} with anything else to create a new abstraction, and can require extensive modifications in source code to make relatively simple changes.

\subsection{Join Calculus}

\begin{anfxwarning}{Join Calc refs}
Chemical abstract machine, joinads, Scala communicating objects (IIRC, that was join calculus based), reagents
\end{anfxwarning}

\subsection{Others}
Mobility calculus.  Others?

\subsection{Criticisms}

Gorlatch \cite{Gorlatch2004} argued against basic message passing, decrying it as an unnecessary and unhelpful complication and favouring \fxerror*{What does `collective operations' mean?}{`collective' operations} instead.  This criticism focused upon \gls{mpi} as it was at the time, however, and made no reference to either Actors or \gls{csp}. \fxerror{Does this truly belong here?}
