%-----------------------------------------------------------------------------------
\section{\label{sec:cps:compoundterms}Indexed Notation for Compound Terms}
%-----------------------------------------------------------------------------------

\emph{Indexed compound terms} are reasonably common in \gls{cps}.  That is, terms tagged with one or more sub-terms used to distinguish different instances of the same \gls{functor} type.  They are still classic \gls{cps} objects with nested terms, but some of the subterms are used only to differentiate instances of the encompassing compound term.  This is roughly analogous to tagging a term with its index in a logical vector/array or its key in a typical dictionary/associative array data structure.

For example, in \cref{chap:nmp}, compound \(v\) terms appear in multiple rules.  Ordinarily, these would be represented as nested terms along the lines of
\[ \cpfunc{v}{\cpfunc{v'}{N} \; \cpfunc{v''}{G}\; D} \]
These are used inside \glspl{tlc} to represent sets of tagged data.  They are indexed by neighbour \(N\) and tagged with a `generation count' \(G\) (further explained in \cref{sec:nmp:pespecific}).  Both of these values track metadata about a datum.  Lastly, the datum stored by the encompassing term is given.  In many common programming languages, accessing each datum might be written like \texttt{v[N]}, where \texttt{v} is a dictionary indexed by neighbour;  \(G\) could be stored as a property of the data structure used to represent the data.  The \(v\) \glspl{functor} can instead be written as \[ \cpvv{N}{G}{D} \] for a shorthand that can be expanded back out to a complete form automatically.  The first pair of parentheses selects \glspl{functor} by neighbour, the second records the generation, while the final pair shows the actual contents.  This is \emph{purely} a notational convenience, without impact upon the application of the rules and evolution of the system.  When a concrete instance of a \(v\) compound term is indexed by a ground term (i.e. not a variable) \(k\), e.g. \(\cpvv{k}{\_}{\_}\), it may be referred to as \emph{k-tagged}.

\lstset{xleftmargin=.5in, xrightmargin=.5in} 
\begin{lstlisting}
  $\cpfunc{v}{\cpfunc{v'}{N} \; \cpfunc{v''}{G}\; D}$ #\hfill nested compound term \enspace#
  $\cpvv{N}{G}{D}$ #\hfill indexed compound term \enspace#
\end{lstlisting}