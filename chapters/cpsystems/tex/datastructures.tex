%-----------------------------------------------------------------------------------
\section{Data Structures in \glsfmtname{cps}}\label{sec:cps:datastructures}
%-----------------------------------------------------------------------------------

This \lcnamecref{sec:cps:datastructures} sketches the design of some high-level data structures, 
similar to the data structures used in high-level pseudocode or programming languages:
numbers, relations, functions, associative arrays, lists, trees, strings, 
together with alternative, more readable notations.

%-------------------------------------------------------
\subsection{\label{sec:cps:natnums}Natural Numbers}
%-------------------------------------------------------
Natural numbers can be represented via \emph{multisets} containing repeated occurrences of the \emph{same} atom.
For example, considering that \(\cpundig\) represents an ad hoc unary digit, 
the following complex symbols can be used to describe 
the contents of a virtual integer \emph{variable} \(a\): 
\(\cpfunc{a}{\,} = \cpfunc{a}{\cpempty} = \cpfunc{a}{\cpundig^0}\) --- the value of \(a\) is 0;
\(\cpfunc{a}{\cpundig^3}\) --- the value of \(a\) is 3.

\lstset{xleftmargin=.5in, xrightmargin=.5in} 
\begin{lstlisting}
  $\cpfunc{e}{\cpempty} \equiv \cpfunc{e}{0} \equiv \cpfunc{e}{}$ #\hfill\textsl{empty functor}\enspace#
  $\cpfunc{a}{\cpundig\cpundig\cpundig} \equiv \cpfunc{a}{\cpundig^3} \equiv \cpfunc{a}{3}$ #\hfill\textsl{the number three}\enspace#
\end{lstlisting}

For concise expressions, these number representations may be aliased by their corresponding numbers, \eg{}~\(\cpfunc{a}{\,} \equiv \cpfunc{a}{0}, \cpfunc{b}{\cpundig^3} \equiv \cpfunc{b}{3}\).  Numerical operations are simulated with unary arithmetic \cite{Aman2019,Bonchis2006}.
Nicolescu \etal{} \cite{Nicolescu2014,RN-HW-ROMJIST14} show how the basic arithmetic operations can be modelled efficiently by \gls{ps} with complex symbols.

Here follows a list of simple arithmetic expressions, assignments, and comparisons:

\lstset{xleftmargin=.5in, xrightmargin=.5in} 
\begin{lstlisting}
  $x = 0$ $\equiv$ $\cpfunc{x}{\lambda}$
  $x = 1$ $\equiv$ $\cpfunc{x}{\cpundig}$
  $x = 2$ $\equiv$ $\cpfunc{x}{\cpundig \cpundig}$
  $x = n$ $\equiv$ $\cpfunc{x}{\cpundig^n}$
  
  $x \leftarrow y \cpmaxpar z$ $\equiv$ $\cpfunc{y}{Y} ~ \cpfunc{z}{Z} ~ \rightarrow ~ \cpfunc{x}{YZ}$ #\hfill\textsl{destructive add}\enspace#
  $x \leftarrow y + z$ $\equiv$ $ \rightarrow ~ \cpfunc{x}{YZ} ~ \mid ~ \cpfunc{y}{Y} ~ \cpfunc{z}{Z}$ #\hfill\textsl{preserving add}\enspace#
  
  $x = y$ $\equiv$  $\cpfunc{x}{X} ~\cpfunc{y}{X}$ #\hfill\textsl{equality}\enspace#
  $x \leq y$ $\equiv$  $\cpfunc{x}{X} ~\cpfunc{y}{XY}$ #\hfill\textsl{less than or equal to}\enspace#
  $x <  y$ $\equiv$  $\cpfunc{x}{X} ~\cpfunc{y}{X1Y}$ #\hfill\textsl{strictly less than}\enspace#
\end{lstlisting}
\noindent
`Strictly less than' (\(<\)) requires the extra \(1\), because \(Y\) can match on \(\cpempty\).

\subsubsection{Less-than and Greater-than in \glsfmtname{promoter} and \glsfmtname{inhibitor}}

When dealing with natural numbers\footnote{In fact, this applies to any rule dealing with only a single type of atom, but rules dealing with natural numbers are by far the most common of these.} in rules it is possible to enforce `greater than' and `less than' relationships between variables through the use of sub-multiset specifications in \glspl{promoter} and \glspl{inhibitor}.  For instance, to state that variable \(A\) must be less than or equal to variable \(B\), \ie{} \(A \leq B\), one may write: \(|~ A \subseteq B\).  Or, to state that \(B\) must be strictly less than \(C\), \ie{} \(B < C\), one may write: \(|~ B \subsetneq C\).

These relations hold because all natural numbers are represented by repeated uses of the same object -- the unary digit \(\cpundig\) -- and fewer copies of the same object are always a sub-multiset of the greater multiset.  The use of this is similar to the equality condition introduced in \cref{sec:cps:prominhi}, but works only for variables representing numbers, whereas the equality condition applies to all variables. 

%-------------------------------------------------------
\subsection{Relations and Functions}
%-------------------------------------------------------
Consider the \emph{binary relation} \(r\), as defined by: 
\(r = \{ (a, b),\, (b, c),\, (a, d),\, (d, c) \}\) (which has a diamond-shaped graph). 
Using complex symbols, relation \(r\) can be represented as a \emph{multiset} with four \(r\) items,
\(\cpset{ \cpfunc{r}{\cpfunc{\kappa}{a} ~ \cpfunc{\upsilon}{b}},\, \cpfunc{r}{\cpfunc{\kappa}{b} ~ \cpfunc{\upsilon}{c}},\, \cpfunc{r}{\cpfunc{\kappa}{a} ~ \cpfunc{\upsilon}{d}},\, \cpfunc{r}{\cpfunc{\kappa}{d} ~ \cpfunc{\upsilon}{c}} }\), 
where \adhoc{} atoms \(\kappa\) and \(\upsilon\) introduce \emph{domain} and \emph{codomain} values (respectively).
The items of this multiset may also be aliased by a more expressive notation such as: \(\{ (a \stackrel{r}\rightleftarrows b)\), \((b \stackrel{r}\rightleftarrows c)\), \((a \stackrel{r}\rightleftarrows d)\), \((d \stackrel{r}\rightleftarrows c) \}\).

If the relation is a \emph{functional relation}, this can be emphasised using another operator, such as ``\textsf{mapsto}''. For example, the functional relation 
\(f = \cpset{ (a, b),\, (b, c),\, (d, c) }\) can be represented by the multiset
\(\cpset{ \cpfunc{f}{\cpfunc{\kappa}{a} ~ \cpfunc{\upsilon}{b}},\, \cpfunc{f}{\cpfunc{\kappa}{b} ~ \cpfunc{\upsilon}{c}},\, \cpfunc{f}{\cpfunc{\kappa}{d} ~ \cpfunc{\upsilon}{c}} }\) or by the more suggestive notation: 
\(\{ (a \stackrel{f}\mapsto b)\), \((b \stackrel{f}\mapsto c)\), \((d \stackrel{f}\mapsto c) \}\).
To highlight the actual mapping value, instead of \(a \stackrel{f}\mapsto b\),
one may also use the succinct abbreviation \(f[a] = b\).

In this context, the \(\rightleftarrows\) and \(\mapsto\) operators are considered to have a high associative priority, so the enclosing brackets are primarily used to increase readability.

%-------------------------------------------------------
\subsection{Associative Arrays}
%-------------------------------------------------------
Consider the \emph{associative array} \(x\), 
with the following key-value mappings (\ie{} functional relation): 
\(\{ \cpundig \mapsto a; \cpundig^3 \mapsto c; \cpundig^7 \mapsto g \}\). 
Using complex symbols, array \(x\) can be represented as a multiset with three items,
\(\cpset{ \cpfunc{x}{\cpfunc{\kappa}{\cpundig}\,\cpfunc{\upsilon}{a}},\, \cpfunc{x}{\cpfunc{\kappa}{\cpundig^3}\,\cpfunc{\upsilon}{c}},\, \cpfunc{x}{\cpfunc{\kappa}{\cpundig^7}\,\cpfunc{\upsilon}{g}} }\), 
where \adhoc{} atoms \(\kappa\) and \(\upsilon\) introduce keys and values (respectively).
This too may be aliased by the more expressive notation
\(\{ \cpundig \stackrel{x}\mapsto a\), \(\cpundig^3 \stackrel{x}\mapsto c\), \(\cpundig^7 \stackrel{x}\mapsto g \}\).

%-------------------------------------------------------
\subsection{\label{sec:cps:lists}Lists}
%-------------------------------------------------------
Consider the \emph{list} \(y\), containing the following sequence of values: 
\([u; v; w]\). 
List \(y\) can be represented as the complex symbol
\(\cpfunc{y}{\, \cpfunc{\gamma}{u~\cpfunc{\gamma}{v~\cpfunc*{\gamma}{w~\cpfunc*{\gamma}{}}}}}\), 
where the \adhoc{} atom \(\gamma\) represents the list constructor \emph{cons} and \(\cpfunc{\gamma}{\,}\) the empty list.
This may be aliased by the more expressive equivalent notation
\(\cpfunc{y}{u\,|\,v\,|\,w}\)
-- or by \(\cpfunc{y}{u\,|\,y'}\), \(y'(v\,|\,w)\) --
where operator \(\mid\) separates the head and the tail of the list.
The notation \(\cpfunc{z}{|}\) is shorthand for \(\cpfunc{z}{\cpfunc{\gamma}{\,}}\) and indicates an empty list, \(z\).

%-------------------------------------------------------
\subsection{Trees}
%-------------------------------------------------------
Consider the \emph{binary tree} \(z\), described by the structured expression \\
\((a, (b), (c, (d), (e)))\), 
\ie{}~\(z\) points to a root node which has: 
\begin{inparaenum}[(i)]
\item the value \(a\); 
\item a left node with value \(b\); and 
\item a right node with value \(c\), left leaf \(d\), and right leaf \(e\). 
\end{inparaenum}
Tree \(z\) can be represented by the complex symbol
\(\cpfunc{z}{a ~ \cpfunc{\phi}{b} ~ \cpfunc{\psi}{c ~ \cpfunc{\phi}{d} ~ \cpfunc{\psi}{e}}}\), 
where \adhoc{} atoms \(\phi, \psi\) introduce left and right subtrees, respectively.

%-------------------------------------------------------
\subsection{Strings}
%-------------------------------------------------------
Consider the \emph{string} \(s = ``abc"\), 
where \(a\), \(b\), and \(c\) are atoms. 
Obviously, string \(s\) can be interpreted as the list \(s = [a; b; c]\), \ie{}
string \(s\) can be represented as the complex symbol
\(\cpfunc{s}{\, \cpfunc{\gamma}{a~\cpfunc{\gamma}{b~\cpfunc*{\gamma}{c~\cpfunc*{\gamma}{}}}}}\), etc.