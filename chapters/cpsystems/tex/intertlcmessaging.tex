%-----------------------------------------------------------------------------------
\section{\label{sec:cps:intertlcmess}Inter-\glsfmtname{tlc} Messaging}
%-----------------------------------------------------------------------------------

\Glspl{tlc} may exchange messages over channels (differently to the behaviour of out-symbols described in \cref{sec:cps:genericrules}).  Each \gls{tlc} may hold one or more appropriately labelled endpoints for any relevant channels, and may attempt both to send and to receive messages via those endpoints in its rules.  A message is written encapsulated inside angle brackets and marked with either an exclamation mark on the \gls{rhs} or a question mark on the \gls{lhs} to represent sending or receiving, respectively.  \Eg{} \(\cpsend{\cpfunc{a}{b}}{c}\) would represent a message \(\cpfunc{a}{b}\) to be sent via channel \(c\), and \(\cprecv{\cpfunc{d}{e}}{f}\) would represent a message \(\cpfunc{d}{e}\) to be received via channel \(f\).

Both sending and receiving use pattern matching.  For the sending case, any \gls{cps} term which matches the pattern in the rule may be removed from the \gls{tlc} and placed into a buffer multiset at the other end of the channel.  Receiving works similarly in that any object stored in the channel's buffer multiset, which matches the pattern of a receipt rule, may be withdrawn.  If more than one object in the buffer matches the pattern, one of them is selected non-deterministically.  Importantly, this means that ordinary \gls{cps} channels do \emph{not} operate as \gls{fifo} queues by default.

\lstset{xleftmargin=.5in, xrightmargin=.5in} 
\begin{lstlisting}
  $\cpsend{\cpfunc{a}{b}}{c}$ #\hfill\textsl{send}\enspace#
  $\cprecv{\cpfunc{d}{e}}{f}$ #\hfill\textsl{receive}\enspace#
\end{lstlisting}

%-------------------------------------------------------
\subsection{\label{sec:cps:antiport}Antiport Communication Rules in \glsfmtname{cps}}
%-------------------------------------------------------

Antiport rules \cite{Orellana-Martin2019,Paun2002} allow for the bidirectional exchange of objects between membranes/cells/neurons during a single rule execution, with the restriction that objects \emph{must} travel in both directions.  Thus, if one side is only ready to send or receive, rather than both, the rule cannot run at the next step.  An important ramification of this is that it prevents deadlock from both sides of the exchange waiting on the other to send a message.

In the context of \gls{cps}, this means that a given rule must involve receipt over a channel on the left-hand-side, and sending on the \emph{same} channel on the right-hand-side.  To emphasise that the rule requires antiport behaviour, the sending and receiving terms have an extra \(!\) or \(?\), respectively.

\lstset{xleftmargin=.5in, xrightmargin=.5in} 
\begin{lstlisting}
  $\cpantisend{\cpfunc{a}{b}}{c}$ #\hfill\textsl{antiport send}\enspace#
  $\cpantirecv{\cpfunc{d}{e}}{f}$ #\hfill\textsl{antiport receive}\enspace#
\end{lstlisting}

For example, \cpruleinline{ \cprule*{s_1}{\cpantirecv{\cpfunc{a}{B}}{c} \; \cpfunc{d}{E}}{\cponce}{s_2}{\cpantisend{\cpfunc{d}{E}}{c} \; \cpfunc{a}{B}}} would be valid because the same channel is used on both sides of the rule.

%-------------------------------------------------------
\subsection{\label{sec:cps:blocking}Blocking vs. Non-blocking Message Receipt in \glsfmtname{cps}}
%-------------------------------------------------------

In \gls{cps} all outgoing communications from a \gls{tlc} to others is non-blocking by default.  The channels connecting the cells buffer message objects if needed, and thus an outgoing message can always be accepted by the channel even if the holder of the other end of the channel is not yet ready to receive the message.  Receiving messages via channels is also ordinarily a non-blocking operation in \gls{cps}, albeit for a different reason.  If there are no eligible messages on the channel, either buffered by the channel itself or offered by the cell holding the other end of the channel, then the rules to receive over that channel will not apply at the next step regardless of whichever other rules may or may not be applied.

It may be helpful in some circumstances, however, to simulate the nature of a blocking receipt.  This can be achieved with the use of additional dedicated states.  The beginning state for the intended blocking receipt should be unused as the beginning state for any other rule (except for another aspect of the same blocking receipt).  This state is also used as the ending state for another rule, which either is the end of another process in the computation or used as a test to determine whether to enter into a blocking receipt.  The ending state for the blocking receipt rule should return the cell to its standard process.

The overall effect of using the unique state is that it ensures no other rule may be used inside a particular \gls{tlc} at a given step.  Effectively, the \gls{tlc} becomes quiescent until another \gls{tlc} makes an appropriate offer to send a message to the first cell.  At that point, one or more messages are exchanged as appropriate, and the receiving \gls{tlc} returns to its standard processing otherwise.  This is used in \cref{sec:nmp:pespecific} (Rules \cpruleref*{rule:nmp:proxspec:movetoend} \& \cpruleref*{rule:nmp:proxspec:recvfromneighs} in \vref{ruleset:nmp:proxspec}).

% -------------------------------------------------

%-------------------------------------------------------
\subsection{\label{sec:cps:syncasync}Synchronous vs Asynchronous Messaging in \glsfmtname{cps}}
%-------------------------------------------------------

\gls{cps} models do not make any \emph{syntactic} difference between synchronous and asynchronous messaging (\cref{sec:back:syncasync}) scenarios;
this is strictly a \emph{runtime} assumption~\cite{Nicolescu2012}.
Any model can run on both the synchronous and asynchronous runtime ``engines'', albeit the results may differ.  All \glspl{tlc} still follow the same global clock for each step, but the start and end of their rounds may differ under asynchronicity, due to varying arrival times of messages on channels.

\paragraph{Non-determinism}
In general, and as with \gls{mc} generally, both scenarios are non-deterministic.  If complete determinism is required, it must be carefully built into the system's \gls{ruleset}.  This fact is most obvious in the asynchronous scenario, when messages can arrive at any arbitrary time, but even the synchronous scenario may necessitate non-deterministic choices at times. \Eg{} when there is a choice between possible
unifications, as in \cref{sec:cps:unification}; or, when there are multiple messages in a channel’s bag
which match the pattern of the receiving \gls{tlc}’s receipt rule but because the relevant rule runs in \(\cponce\) mode only one is to be selected.

Determinism in the final result only is known as \emph{confluence}.  This means that, while different evolutions of the system may lead to different intermediate states, the final result computed will always be the same.  Confluence too must be built into the \gls{ruleset} if it is required.