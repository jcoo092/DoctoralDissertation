%-----------------------------------------------------------------------------------
\section{High-level or Generic Rules}
%-----------------------------------------------------------------------------------

%-------------------------------------------------------
\subsection{\label{sec:cps:genericrules}Generic Rules Format}
%-------------------------------------------------------
Consider rules of the following \emph{generic} format 
(called generic because it defines templates involving variables):
\begin{framed}
\vspace{-0.6cm}
\begin{align*}
\textit{current-state} ~~ \textit{symbols} \dots ~ \rightarrow_\alpha ~ & \textit{target-state} ~~ (\textit{in-symbols}) \dots ~~ \\
 & (\textit{out-symbols})_\delta \dots \\
 & | ~  \textit{promoters} \dots ~~ \neg ~  \textit{inhibitors} \dots
\end{align*}
\vspace{-0.8cm}
\end{framed}
Where:
\begin{itemize}
\item \textit{current-state} and \textit{target-state} are atoms or terms;  these are traditionally written in the form \(s_x\) where \(x\) is a sequential numeral (\eg{} \(s_1\), \(s_2\) \etc{}), but any valid ground atom or term as appropriate is permitted.

\smallskip
\item \textit{symbols}, \textit{in-symbols}, \textit{promoters} and \textit{inhibitors} are symbols;

\smallskip
\item \textit{in-symbols} become available after the end of the current step only, as in traditional \gls{ps}  (one can imagine that these are sent via an \adhoc{} fast \textit{loopback} channel); 

\smallskip
\item subscript \(\alpha\) \(\in\) \(\{\cponce\), \(\cpmaxpar\}\), 
indicates the application mode, discussed further in \cref{sec:cps:applicationmodes};

\smallskip
\item \textit{out-symbols} are sent to the cell's structural neighbours at the end of the step.
These symbols are enclosed in angle brackets that indicate 
their \fxerror{Should this be moved to an ``intra-\gls{tlc} messaging'' section?}{destinations}, abbreviated above as \(\delta\). 
The most usual scenarios include: 

\begin{itemize}
\item \(\cpoutsymbol{a}{i}\) indicates that \(a\) is sent over outgoing arc \(i\) (unicast); 

\item \(\cpoutsymbol{a}{i,j}\) indicates that \(a\) is sent over outgoing arcs \(i\) and \(j\)(multicast); 

\item \(\cpoutsymbolall{a}\) indicates that \(a\) is sent over all outgoing arcs (broadcast). 
\end{itemize}

All symbols sent via one \emph{generic rule} to the same destination form one single \emph{message}, and they travel together as one single block (even if the generic rule is applied in mode \(\cpmaxpar\)).
\end{itemize}

%-------------------------------------------------------
\subsection{Pattern Matching}
%-------------------------------------------------------
\fxerror{Should this be combined with the unification section above?}{Rules are matched against cell contents using the aforementioned \emph{pattern matching},
which involves the rule's \emph{\gls{lhs}}, promoters and inhibitors.}

Generally, variables have \emph{global rule scope};
these are assumed to be introduced by \emph{existential} quantifiers preceding the rule
--- except for inhibitors, which may introduce \emph{local variables}, 
as further discussed in \vref{sec:cps:prominhi}. 

The matching is \emph{valid} only if, after unification/substituting variables by their values, 
the rule's \emph{\gls{rhs}} contains ground terms only
(so \emph{no} free variables are injected in the cell or sent to its neighbours),
as illustrated by the following sample scenario:
\begin{itemize}
\item The cell's \emph{current content} includes the \emph{ground term}:\\
\(\cpfunc{n}{a \, \cpfunc{\phi}{b \, \cpfunc{\phi}{c} \, \cpfunc{\psi}{d}} \, \cpfunc{\psi}{e}}\)

\smallskip
\item The following (state-less) \emph{rewriting rule} is considered: \\ 
\(\cpfunc{n}{X \, \cpfunc{\phi}{Y \, \cpfunc{\phi}{Y_1} \, \cpfunc{\psi}{Y_2}} \, \cpfunc{\psi}{Z}} ~ \rightarrow ~ \cpfunc{v}{X} \: \cpfunc{n}{Y \, \cpfunc{\phi}{Y_2} \, \cpfunc{\psi}{Y_1}} \: \cpfunc{v}{Z}\)

\smallskip
\item The pattern matching determines the following \emph{unifiers}: \\
\(X = a\), \(Y = b\), \(Y_1 = c\), \( Y_2 = d\), \(Z = e\).

\smallskip
\item This is a \emph{valid} matching and, after \emph{substitutions}, 
the rule's \gls{rhs} gives the \emph{new content}: \\
\(\cpfunc{v}{a} ~ \cpfunc{n}{b \, \cpfunc{\phi}{d} \, \cpfunc{\psi}{c}} ~ \cpfunc{v}{e}\)
\end{itemize}

%-------------------------------------------------------
\subsection{\label{sec:cps:prominhi}Promoters and Inhibitors}
%-------------------------------------------------------

Promoters are objects that \emph{must} be present within the cell for the rule to be applicable but are \emph{not} removed by the rule.  Conversely, inhibitors are objects that \emph{must not} be present for the rule to be applicable, although the rule may create them.  If promoters are present, they are denoted following a \(|\) per promoter, and inhibitors by \(\neg\), \eg{} \(|\,\cpfunc{a}{A}\) or \(\neg\,\cpfunc{b}{B}\).  Promoters and inhibitors are traditionally written below the main rule body, but this is not strictly required.  Promoters are typically written first, followed by inhibitors, but this too is not compulsory.  All promoters and inhibitors have equal priority among themselves for a given attempted unification of a rule.

\lstset{xleftmargin=.5in, xrightmargin=.5in} 
\begin{lstlisting}
  $| ~ \cpfunc{a}{A}$ #\hfill\textsl{promoter}\enspace#
  $\neg ~ \cpfunc{b}{B}$ #\hfill\textsl{inhibitor}\enspace#
\end{lstlisting}

To define additional useful matchings expressively, 
promoters and inhibitors may also use virtual ``equality'' terms, 
written in infix format, with the \(=\) operator.
For example, including the term \((ab = XY)\) indicates the following additional matching constraints on variables \(X\) and \(Y\): either \(X, Y = ab, \cpempty\); or \(X, Y = a, b\); or \(X, Y = b, a\); or \(X, Y = \cpempty, ab\).

To define inhibitors as logical negations usefully,
variables that only appear in the scope of an inhibitor are assumed to have \emph{local scope}. 
Furthermore, these variables are assumed to be defined by \emph{existential} quantifiers, immediately after the negation. 
Semantically, this is equivalent to introducing these variables at the global rule level, 
but by \emph{universal} quantifiers, after all other global variables;
introduced by \emph{existential} quantifiers.

As an illustration, consider a cell containing \(\cpfunc{a}{c} ~ \cpfunc{a}{ccc}\) and contrast two rules, 
containing the following two sample promoter/inhibitor pairs 
(for brevity, other rule details are omitted here).

\lstset{xleftmargin=.5in, xrightmargin=.5in} 
\begin{lstlisting}
... $\mid$  $\cpfunc{a}{cXY}$     $\neg$  $\cpfunc{a}{X}$    #\hfill (1)\enspace#
... $\mid$  $\cpfunc{a}{cZ}$     $\neg$  $(Z=XY)$  $\cpfunc{a}{X}$    #\hfill (2)\enspace#
\end{lstlisting}

These two rules appear very similar, and their inhibitor tests share the same expression: 
\emph{No} \(\cpfunc{a}{X}\) may be present in the cell.

Rule (1) uses two global variables, \(X, Y\). 
According to its promoter, \(\cpfunc{a}{cXY}\), these variables can be matched in four different ways:
(1a) \(X, Y = \cpempty, \cpempty\); (1b) \(X, Y = cc, \cpempty\); (1c) \(X, Y = \cpempty, cc\); (1d) \(X, Y = c, c\).\footnote{Strictly speaking, this matching is possible in multiple ways, with different \(c\) atoms assigned to each variable.  For current purposes, though, they are equivalent and so are reduced to the one result.}
Three different unifications, (1a), (1b), (1c), pass the inhibitor test, 
as there are no cell terms \(\cpfunc{a}{\,}\), \(\cpfunc{a}{cc}\), \(\cpfunc{a}{\,}\), respectively. 
Unification (1d) fails the inhibitor test because there \emph{is} one cell term \(\cpfunc{a}{c}\).

Rule (2) uses one global variable, \(Z\), and two local inhibitor variables, \(X, Y\).
According to its promoter, \(\cpfunc{a}{cZ}\), variable \(Z\) can be matched in two different ways: 
(2a) \(Z = \cpempty\); (2b) \(Z = cc\).
Unification (2a) passes the inhibitor test because it only generates one local unification,
\(X, Y = \cpempty, \cpempty\), and there is NO cell term \(\cpfunc{a}{\,}\).
Unification (2b) fails the inhibitor test because it generates all three following local unifications:
(2b1) \(X, Y = cc, \cpempty\); (2b2) \(X, Y = \cpempty, cc\); (2b3) \(X, Y = c, c\); 
and there \emph{is} a cell term corresponding to (2b3), \(\cpfunc{a}{c}\).

%-------------------------------------------------------
\subsection{\label{sec:cps:applicationmodes}Application Modes: Exactly-once and Maximally-parallel}
%-------------------------------------------------------
To explain the two rule application modes, \(\cponce\) and \(\cpmaxpar\) (referred to as \emph{exactly-once} and \emph{maximally-parallel} respectively) consider a cell, \(\sigma\), containing three counter-like complex symbols,
\(\cpfunc{c}{\cpundig^2}\), \(\cpfunc{c}{\cpundig^2}\), \(\cpfunc{c}{\cpundig^3}\),
and the two possible application modes of the following high-level ``decrementing'' rule:
\vspace{-0.2cm}
\begin{framed}
\vspace{-0.5cm}
\[(\rho_\beta) ~s_1 ~\cpfunc{c}{\cpundig \, X} \rightarrow_{\alpha} s_2 ~\cpfunc{c}{X},\\
\mathrm{where} \; \alpha \in \{\cponce, \cpmaxpar\}.\]
\vspace{-0.8cm}
\end{framed}

The \gls{lhs} of rule \(\rho_\beta\), \(\cpfunc{c}{\cpundig \, X}\), can be unified in three different ways,
to each one of the three \(c\) symbols extant in cell \(\sigma\).
Conceptually, this rule may be instantiated in three different ways,
each one tied and applicable to a distinct symbol:
\begin{eqnarray*}
& (\rho_1)  & ~s_1 ~\cpfunc{c}{\cpundig^2} \rightarrow_{\alpha} s_2 ~\cpfunc{c}{\cpundig},\\
& (\rho_2)  & ~s_1 ~\cpfunc{c}{\cpundig^2} \rightarrow_{\alpha} s_2 ~\cpfunc{c}{\cpundig},\\
& (\rho_3) & ~s_1 ~\cpfunc{c}{\cpundig^3} \rightarrow_{\alpha} s_2 ~\cpfunc{c}{\cpundig^2}.
\end{eqnarray*}

\begin{enumerate}
\item If \(\alpha = \: \cponce\), rule~\(\rho_1\) 
non-deterministically selects and applies one of these virtual rules \(\rho_1\), \(\rho_2\), \(\rho_3\).
Using \(\rho_1\) or \(\rho_2\), 
cell \(\sigma\) ends with counters \(\cpfunc{c}{\cpundig}\), \(\cpfunc{c}{\cpundig^2}\), \(\cpfunc{c}{\cpundig^3}\).
Using \(\rho_3\),
cell \(\sigma\) ends with counters \(\cpfunc{c}{\cpundig^2}\), \(\cpfunc{c}{\cpundig^2}\), \(\cpfunc{c}{\cpundig^2}\).

\smallskip
\item If \(\alpha = \: \cpmaxpar\), rule~\(\rho_\cpmaxpar\) 
applies in parallel all these virtual rules \(\rho_1\), \(\rho_2\), \(\rho_3\).
Cell \(\sigma\) ends with counters \(\cpfunc{c}{\cpundig}\), \(\cpfunc{c}{\cpundig}\), \(\cpfunc{c}{\cpundig^2}\).
\end{enumerate}

Semantically, the \(\cpmaxpar\) mode is equivalent to a virtual sequential \texttt{while} loop around the same rule in \(\cponce\) mode, which is repeated until it is no longer applicable.  All such applications of the rule are carried out concurrently in a single step, however.

Like most other \gls{ps} variants, \gls{cps} ordinarily evolve synchronously in a stepwise fashion following an implicit global clock.  Rules are applied based on whether the available multiset(s) within the system match the rules' \gls{lhs} and promoters, and not the inhibitors.  The consumption of removed objects plus the creation of new objects happens instantaneously at the end of a step.

%-------------------------------------------------------
\subsection{Benefits}
%-------------------------------------------------------
This type of generic rules allows
\begin{inparaenum}[(i)]
\item a reasonably fast parsing and processing of subcomponents; and
\item algorithm descriptions with \emph{fixed-size alphabets} and \emph{fixed-sized \glspl{ruleset}}, 
independent of the size of the problem and the number of cells in the system (often \emph{impossible} with only atomic symbols).
\end{inparaenum}

%-------------------------------------------------------
\subsection{Special Cases}
%-------------------------------------------------------
Simple scenarios involving generic rules are sometimes 
semantically equivalent to sets of non-generic (ground) rules defined via bounded loops.
For example, consider the rule
\[
s_1 ~ \cpfunc{a}{IJ} ~ \rightarrow_\cpmaxpar ~ s_2 ~ \cpfunc{b}{I} ~ \cpfunc{c}{J},
\]
where the cell's contents guarantee that \(I\) and \(J\) 
only match integers in ranges \([1,n]\) and \([1,m]\), respectively.
Under these assumptions, 
this rule is equivalent to the following set of non-generic rules:
\[
s_1 ~ a_{i,j} ~ \rightarrow s_2 ~ b_i ~ c_j, ~ \forall i \in [1,n], j \in [1,m].
\]

However, unification is a much more powerful concept, 
which cannot be reduced generally to simple bounded loops.

%-------------------------------------------------------
\subsection{\label{sec:cps:syncasync}Synchronous vs Asynchronous}
%-------------------------------------------------------
% % \gls{cps}' models do not make any \emph{syntactic} difference between synchronous and asynchronous scenarios;
% % this is strictly a \emph{runtime} assumption~\cite{Nicolescu2012}.
% % Any model can run on both the synchronous and asynchronous runtime ``engines'',
% % albeit the results may differ.
% % The asynchronous model conceptually closely matches the standard definition of asynchronicity used in distributed computing \cite{Balanescu2011,Nicolescu2014}, and \emph{not} the definition of asynchronicity followed in \cite{Cavaliere2009,Frisco2012,Song2013} and related.  This asynchronous concept differs from the synchronous one primarily in three key respects:
% % \begin{inparaenum}[(1)]
% % \item There is \emph{no} concept of a global clock.  Each step for a given \gls{tlc} begins after an arbitrary delay;
% % \item Each step for a \gls{tlc} takes zero time.  That is, `internal' evolution of the \gls{tlc} is instantaneous, once the step has begun after the aforementioned delay;
% % \item Messages sent between \glspl{tlc} via channels take a random non-zero length of time to travel along the channel.
% % \end{inparaenum}

% % \citeauthor{Fokkink2013} says of asynchronous communications: \textcquote[][p.~7]{Fokkink2013}{\textelp{} that sending and receiving a message are distinct events at the sending and receiving \textdel{process}\textins{\glspl{tlc}}, respectively. The delay of a message in a channel is arbitrary but finite.}

% % \begin{anfxerror}{Still need to tidy up this section!}
% % Include a definition of asynchronous from distributed computing — probably hunt for something in Lynch's book.
% % \end{anfxerror}

% \gls{cps}' models do not make any \emph{syntactic} difference between synchronous and asynchronous scenarios;
% this is strictly a \emph{runtime} assumption~\cite{Nicolescu2012}.
% Any model can run on both the synchronous and asynchronous runtime ``engines'',
% albeit the results may differ.
% The asynchronous model conceptually closely matches the standard definition of asynchronicity used in distributed computing \cite{Balanescu2011,Nicolescu2014}, and \emph{not} the definition of asynchronicity followed in \cite{Cavaliere2009,Frisco2012,Song2013} and related.  This asynchronous concept differs from the synchronous one primarily in three key respects:
% \begin{inparaenum}[(1)]
% \item There is \emph{no} concept of a global clock.  Each step for a given \gls{tlc} begins after an arbitrary delay;
% \item Each step for a \gls{tlc} takes zero time.  That is, `internal' evolution of the \gls{tlc} is instantaneous, once the step has begun after the aforementioned delay;
% \item Messages sent between \glspl{tlc} via channels take a random non-zero length of time to travel along the channel.
% \end{inparaenum}

% \citeauthor{Fokkink2013} says of asynchronous communications: \textcquote[][p.~7]{Fokkink2013}{\textelp{} that sending and receiving a message are distinct events at the sending and receiving \textdel{process}\textins{\glspl{tlc}}, respectively. The delay of a message in a channel is arbitrary but finite.}

% \begin{anfxerror}{Still need to tidy up this section!}
% Include a definition of asynchronous from distributed computing — probably hunt for something in Lynch's book.
% \end{anfxerror}

\gls{cps} models do not make any \emph{syntactic} difference between synchronous and asynchronous scenarios;
this is strictly a \emph{runtime} assumption~\cite{Nicolescu2012}.
Any model can run on both the synchronous and asynchronous runtime ``engines'', albeit the results may differ.
The concepts of synchronous and asynchronous here are those of the standard timing models in distributed computing.
These apply to messaging between \glspl{tlc}.  For systems with only a single \gls{tlc}, there is no meaningful difference.

\paragraph{Rounds}
Under both synchronous and asynchronous scenarios, all \glspl{tlc} work in \emph{rounds} (sometimes also referred to as ``macro-steps'').  A round consists of three repeating sequential sub-steps:
\begin{enumerate}
    \item \emph{Receive}:  receive one or more incoming messages
    \item \emph{Process}:  perform any requisite local computations, and update the local state
    \item \emph{Send}:  send out new messages as appropriate based on the processing from the previous step
\end{enumerate}

\paragraph{Synchronous}
Under the synchronous model, all \glspl{tlc} evolve in lock-step.  They may carry out the sub-steps at different speeds, but begin and end rounds together.  When considering timing of messaging, there are two standard approaches:  to consider the processing sub-step as taking one time unit and the transit time (\ie{} the time between when the message is sent by the sender and when it is received by the recipient) taking zero time units; or, to consider the processing sub-step as taking zero time units and the transit time taking one time unit.  The synchronous model shares some similarities with, but is \emph{not} logically equivalent to \gls{csp} or \gls{pram}.

\paragraph{Asynchronous}
Under the asynchronous model, there is (unsurprisingly) no natural synchronisation between \glspl{tlc}.  All \glspl{tlc} proceed through the sub-steps at their own pace.  The processing sub-step is considered to take zero time units, and transit times are considered to take any number of time units --- or, alternatively, while processing remains as taking zero time units, transit times take somewhere between zero and one (inclusive) time units.  This makes the second synchronous approach a special case of the asynchronous approach.  The asynchronous model is \emph{extremely} similar to the \gls{actor} model.

\paragraph{Non-determinism}
In general, and as with \gls{mc} generally, both scenarios are non-deterministic.  This fact is most obvious in the asynchronous scenario, when messages can arrive at any arbitrary time, but even the synchronous scenario may necessitate non-deterministic choices at times.  If complete determinism is required, it must be carefully built into the system's ruleset.  Determinism in the final result only is known as \emph{confluence}.  This means that, while different evolutions of the system may lead to different intermediate states, the final result computed will always be the same.

\paragraph{Echo}
To illustrate the difference between the synchronous and asynchronous scenarios, consider the illustrations in [echo figures].  The Echo algorithm is used to establish a spanning tree over a graph of nodes/\glspl{tlc}, in this case rooted at node 1.

In the synchronous case, all messages sent out are received simultaneously, and all nodes proceed through their rounds together.  In the asynchronous case, however, messages may arrive at arbitrary times, and nodes proceed through their rounds entirely independently of one another.  They react to messages as they arrive, and might do nothing for a time while other nodes are working if no new messages arrive during that time.

% The asynchronous model conceptually closely matches the standard definition of asynchronicity used in distributed computing \cite{Balanescu2011,Nicolescu2014}, and \emph{not} the definition of asynchronicity followed in \cite{Cavaliere2009,Frisco2012,Song2013} and related.  This asynchronous concept differs from the synchronous one primarily in three key respects:
% \begin{inparaenum}[(1)]
% \item There is \emph{no} concept of a global clock.  Each step for a given \gls{tlc} begins after an arbitrary delay;
% \item Each step for a \gls{tlc} takes zero time.  That is, `internal' evolution of the \gls{tlc} is instantaneous, once the step has begun after the aforementioned delay;
% \item Messages sent between \glspl{tlc} via channels take a random non-zero length of time to travel along the channel.
% \end{inparaenum}

For more on the topics discussed in this \lcnamecref{sec:cps:syncasync}, the interested reader is referred \fxerror{Need citations for Lynch and Tel}{to} \cite{Fokkink2013}.

\begin{anfxwarning}
What of the usual \gls{ps} global synchronous clock?
\end{anfxwarning}