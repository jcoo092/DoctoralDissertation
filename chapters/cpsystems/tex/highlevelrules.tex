%-----------------------------------------------------------------------------------
\section{High-level or Generic Rules}
%-----------------------------------------------------------------------------------

%-------------------------------------------------------
\subsection{\label{sec:cps:genericrules}Generic Rules Format}
%-------------------------------------------------------
Consider rules of the following \emph{generic} format 
(called generic because it defines templates involving variables):
\begin{framed}
\vspace{-0.6cm}
\begin{align*}
\textit{current-state} ~~ \textit{symbols} \dots ~ \rightarrow_\alpha ~ & \textit{target-state} ~~ (\textit{in-symbols}) \dots ~~ \\
 & (\textit{out-symbols})_\delta \dots \\
 & | ~  \textit{\glspl{promoter}} \dots ~~ \neg ~  \textit{\glspl{inhibitor}} \dots
\end{align*}
\vspace{-0.8cm}
\end{framed}
Where:
\begin{itemize}
\item \textit{current-state} and \textit{target-state} are atoms or terms;  these are traditionally written in the form \(s_x\) where \(x\) is a sequential numeral (\eg{} \(s_1\), \(s_2\) \etc{}), but any valid ground atom or term as appropriate is permitted.

\smallskip
\item \textit{symbols}, \textit{in-symbols}, \textit{\glspl{promoter}} and \textit{\glspl{inhibitor}} are symbols;

\smallskip
\item \textit{in-symbols} become available after the end of the current step only, as in traditional \gls{ps}  (one can imagine that these are sent via an \adhoc{} fast \textit{loopback} channel); 

\smallskip
\item subscript \(\alpha\) \(\in\) \(\{\cponce\), \(\cpmaxpar\}\), 
indicates the application mode, discussed further in \cref{sec:cps:applicationmodes};

\smallskip
\item \textit{out-symbols} are sent to the cell's structural neighbours at the end of the step.
These symbols are enclosed in angle brackets that indicate 
their destinations, abbreviated above as \(\delta\). 
The most usual scenarios include: 

\begin{itemize}
\item \(\cpoutsymbol{a}{i}\) indicates that \(a\) is sent over outgoing arc \(i\) (unicast); 

\item \(\cpoutsymbol{a}{i,j}\) indicates that \(a\) is sent over outgoing arcs \(i\) and \(j\)(multicast); 

\item \(\cpoutsymbolall{a}\) indicates that \(a\) is sent over all outgoing arcs (broadcast). 
\end{itemize}

All symbols sent via one \emph{generic rule} to the same destination form one single \emph{message}, and they travel together as one single block (even if the generic rule is applied in mode \(\cpmaxpar\)).
\end{itemize}

%-------------------------------------------------------
\subsection{Pattern Matching}
%-------------------------------------------------------
Rules are matched against cell contents using the aforementioned \emph{pattern matching},
which involves the rule's \emph{\gls{lhs}}, \glspl{promoter} and \glspl{inhibitor}.

Generally, variables have \emph{global rule scope};
these are assumed to be introduced by \emph{existential} quantifiers preceding the rule
--- except for \glspl{inhibitor}, which may introduce \emph{local variables}, 
as further discussed in \vref{sec:cps:prominhi}. 

The matching is \emph{valid} only if, after unification/substituting variables by their values, 
the rule's \emph{\gls{rhs}} contains ground terms only
(so \emph{no} free variables are injected in the cell or sent to its neighbours),
as illustrated by the following sample scenario:
\begin{itemize}
\item The cell's \emph{current content} includes the \emph{ground term}:\\
\(\cpfunc{n}{a \, \cpfunc{\phi}{b \, \cpfunc{\phi}{c} \, \cpfunc{\psi}{d}} \, \cpfunc{\psi}{e}}\)

\smallskip
\item The following (state-less) \emph{rewriting rule} is considered: \\ 
\(\cpfunc{n}{X \, \cpfunc{\phi}{Y \, \cpfunc{\phi}{Y_1} \, \cpfunc{\psi}{Y_2}} \, \cpfunc{\psi}{Z}} ~ \rightarrow ~ \cpfunc{v}{X} \: \cpfunc{n}{Y \, \cpfunc{\phi}{Y_2} \, \cpfunc{\psi}{Y_1}} \: \cpfunc{v}{Z}\)

\smallskip
\item The pattern matching determines the following \emph{unifiers}: \\
\(X = a\), \(Y = b\), \(Y_1 = c\), \( Y_2 = d\), \(Z = e\).

\smallskip
\item This is a \emph{valid} matching and, after \emph{substitutions}, 
the rule's \gls{rhs} gives the \emph{new content}: \\
\(\cpfunc{v}{a} ~ \cpfunc{n}{b \, \cpfunc{\phi}{d} \, \cpfunc{\psi}{c}} ~ \cpfunc{v}{e}\)
\end{itemize}

%-------------------------------------------------------
\subsection{\label{sec:cps:prominhi}Promoters and Inhibitors}
%-------------------------------------------------------

Promoters are objects that \emph{must} be present within the cell for the rule to be applicable but are \emph{not} removed by the rule.  Conversely, \glspl{inhibitor} are objects that \emph{must not} be present for the rule to be applicable, although the rule may create them.  If \glspl{promoter} are present, they are denoted following a \(|\) per \gls{promoter}, and \glspl{inhibitor} by \(\neg\), \eg{} \(|\,\cpfunc{a}{A}\) or \(\neg\,\cpfunc{b}{B}\).  Promoters and \glspl{inhibitor} are traditionally written below the main rule body, but this is not strictly required.  Promoters are typically written first, followed by \glspl{inhibitor}, but this too is not compulsory.  All \glspl{promoter} and \glspl{inhibitor} have equal priority among themselves for a given attempted unification of a rule.

\lstset{xleftmargin=.5in, xrightmargin=.5in} 
\begin{lstlisting}
  $| ~ \cpfunc{a}{A}$ #\hfill\textsl{\gls{promoter}}\enspace#
  $\neg ~ \cpfunc{b}{B}$ #\hfill\textsl{\gls{inhibitor}}\enspace#
\end{lstlisting}

To define additional useful matchings expressively, 
\glspl{promoter} and \glspl{inhibitor} may also use virtual ``equality'' terms, 
written in infix format, with the \(=\) operator.
For example, including the term \((ab = XY)\) indicates the following additional matching constraints on variables \(X\) and \(Y\): either \(X, Y = ab, \cpempty\); or \(X, Y = a, b\); or \(X, Y = b, a\); or \(X, Y = \cpempty, ab\).

To define \glspl{inhibitor} as logical negations usefully,
variables that only appear in the scope of an \gls{inhibitor} are assumed to have \emph{local scope}. 
Furthermore, these variables are assumed to be defined by \emph{existential} quantifiers, immediately after the negation. 
Semantically, this is equivalent to introducing these variables at the global rule level, 
but by \emph{universal} quantifiers, after all other global variables;
introduced by \emph{existential} quantifiers.

As an illustration, consider a cell containing \(\cpfunc{a}{c} ~ \cpfunc{a}{ccc}\) and contrast two rules, 
containing the following two sample \gls{promoter}/\gls{inhibitor} pairs 
(for brevity, other rule details are omitted here).

\lstset{xleftmargin=.5in, xrightmargin=.5in} 
\begin{lstlisting}
... $\mid$  $\cpfunc{a}{cXY}$     $\neg$  $\cpfunc{a}{X}$    #\hfill (1)\enspace#
... $\mid$  $\cpfunc{a}{cZ}$     $\neg$  $(Z=XY)$  $\cpfunc{a}{X}$    #\hfill (2)\enspace#
\end{lstlisting}

These two rules appear very similar, and their \gls{inhibitor} tests share the same expression: 
\emph{No} \(\cpfunc{a}{X}\) may be present in the cell.

Rule (1) uses two global variables, \(X, Y\). 
According to its \gls{promoter}, \(\cpfunc{a}{cXY}\), these variables can be matched in four different ways:
(1a) \(X, Y = \cpempty, \cpempty\); (1b) \(X, Y = cc, \cpempty\); (1c) \(X, Y = \cpempty, cc\); (1d) \(X, Y = c, c\).\footnote{Strictly speaking, this matching is possible in multiple ways, with different \(c\) atoms assigned to each variable.  For current purposes, though, they are equivalent and so are reduced to the one result.}
Three different unifications, (1a), (1b), (1c), pass the \gls{inhibitor} test, 
as there are no cell terms \(\cpfunc{a}{\,}\), \(\cpfunc{a}{cc}\), \(\cpfunc{a}{\,}\), respectively. 
Unification (1d) fails the \gls{inhibitor} test because there \emph{is} one cell term \(\cpfunc{a}{c}\).

Rule (2) uses one global variable, \(Z\), and two local \gls{inhibitor} variables, \(X, Y\).
According to its \gls{promoter}, \(\cpfunc{a}{cZ}\), variable \(Z\) can be matched in two different ways: 
(2a) \(Z = \cpempty\); (2b) \(Z = cc\).
Unification (2a) passes the \gls{inhibitor} test because it only generates one local unification,
\(X, Y = \cpempty, \cpempty\), and there is NO cell term \(\cpfunc{a}{\,}\).
Unification (2b) fails the \gls{inhibitor} test because it generates all three following local unifications:
(2b1) \(X, Y = cc, \cpempty\); (2b2) \(X, Y = \cpempty, cc\); (2b3) \(X, Y = c, c\); 
and there \emph{is} a cell term corresponding to (2b3), \(\cpfunc{a}{c}\).

%-------------------------------------------------------
\subsection{\label{sec:cps:applicationmodes}Application Modes: Exactly-once and Maximally-parallel}
%-------------------------------------------------------
To explain the two rule application modes, \(\cponce\) and \(\cpmaxpar\) (referred to as \emph{exactly-once} and \emph{maximally-parallel} respectively) consider a cell, \(\sigma\), containing three counter-like complex symbols,
\(\cpfunc{c}{\cpundig^2}\), \(\cpfunc{c}{\cpundig^2}\), \(\cpfunc{c}{\cpundig^3}\),
and the two possible application modes of the following high-level ``decrementing'' rule:
\vspace{-0.2cm}
\begin{framed}
\vspace{-0.5cm}
\[(\rho_\beta) ~s_1 ~\cpfunc{c}{\cpundig \, X} \rightarrow_{\alpha} s_2 ~\cpfunc{c}{X},\\
\mathrm{where} \; \alpha \in \{\cponce, \cpmaxpar\}.\]
\vspace{-0.8cm}
\end{framed}

The \gls{lhs} of rule \(\rho_\beta\), \(\cpfunc{c}{\cpundig \, X}\), can be unified in three different ways,
to each one of the three \(c\) symbols extant in cell \(\sigma\).
Conceptually, this rule may be instantiated in three different ways,
each one tied and applicable to a distinct symbol:
\begin{eqnarray*}
& (\rho_1)  & ~s_1 ~\cpfunc{c}{\cpundig^2} \rightarrow_{\alpha} s_2 ~\cpfunc{c}{\cpundig},\\
& (\rho_2)  & ~s_1 ~\cpfunc{c}{\cpundig^2} \rightarrow_{\alpha} s_2 ~\cpfunc{c}{\cpundig},\\
& (\rho_3) & ~s_1 ~\cpfunc{c}{\cpundig^3} \rightarrow_{\alpha} s_2 ~\cpfunc{c}{\cpundig^2}.
\end{eqnarray*}

\begin{enumerate}
\item If \(\alpha = \: \cponce\), rule~\(\rho_1\) 
non-deterministically selects and applies one of these virtual rules \(\rho_1\), \(\rho_2\), \(\rho_3\).
Using \(\rho_1\) or \(\rho_2\), 
cell \(\sigma\) ends with counters \(\cpfunc{c}{\cpundig}\), \(\cpfunc{c}{\cpundig^2}\), \(\cpfunc{c}{\cpundig^3}\).
Using \(\rho_3\),
cell \(\sigma\) ends with counters \(\cpfunc{c}{\cpundig^2}\), \(\cpfunc{c}{\cpundig^2}\), \(\cpfunc{c}{\cpundig^2}\).

\smallskip
\item If \(\alpha = \: \cpmaxpar\), rule~\(\rho_\cpmaxpar\) 
applies in parallel all these virtual rules \(\rho_1\), \(\rho_2\), \(\rho_3\).
Cell \(\sigma\) ends with counters \(\cpfunc{c}{\cpundig}\), \(\cpfunc{c}{\cpundig}\), \(\cpfunc{c}{\cpundig^2}\).
\end{enumerate}

Semantically, the \(\cpmaxpar\) mode is equivalent to a virtual sequential \texttt{while} loop around the same rule in \(\cponce\) mode, which is repeated until it is no longer applicable.  All such applications of the rule are carried out concurrently in a single step, however.

Like most other \gls{ps} variants, \gls{cps} ordinarily evolve synchronously in a stepwise fashion following an implicit global clock.  Rules are applied based on whether the available multiset(s) within the system match the rules' \gls{lhs} and \glspl{promoter}, and not the \glspl{inhibitor}.  The consumption of removed objects plus the creation of new objects happens instantaneously at the end of a step.

%-------------------------------------------------------
\subsection{Benefits}
%-------------------------------------------------------
This type of generic rules allows
\begin{inparaenum}[(i)]
\item a reasonably fast parsing and processing of subcomponents; and
\item algorithm descriptions with \emph{fixed-size alphabets} and \emph{fixed-sized \glspl{ruleset}}, 
independent of the size of the problem and the number of cells in the system (often \emph{impossible} with only atomic symbols).
\end{inparaenum}

%-------------------------------------------------------
\subsection{Special Cases}
%-------------------------------------------------------
Simple scenarios involving generic rules are sometimes 
semantically equivalent to sets of non-generic (ground) rules defined via bounded loops.
For example, consider the rule
\[
s_1 ~ \cpfunc{a}{IJ} ~ \rightarrow_\cpmaxpar ~ s_2 ~ \cpfunc{b}{I} ~ \cpfunc{c}{J},
\]
where the cell's contents guarantee that \(I\) and \(J\) 
only match integers in ranges \([1,n]\) and \([1,m]\), respectively.
Under these assumptions, 
this rule is equivalent to the following set of non-generic rules:
\[
s_1 ~ a_{i,j} ~ \rightarrow s_2 ~ b_i ~ c_j, ~ \forall i \in [1,n], j \in [1,m].
\]

However, unification is a much more powerful concept, 
which cannot be reduced generally to simple bounded loops.