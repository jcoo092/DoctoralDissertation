%-----------------------------------------------------------------------------------
\section{\label{sec:cps:termsandrulesrevisited}Symbols And Rules Revisited}
%-----------------------------------------------------------------------------------
The expansion of \gls{cps} to include communication between \glspl{tlc} -- either through the use of channels or the out-symbols system -- plus \glspl{ms} introduces new syntactic elements.  Thus, the presentation of terms in \cref{sec:cps:terms} and generic rules in \cref{sec:cps:genericrules} is no longer complete.  This \namecref{sec:cps:termsandrulesrevisited} revisits both and expands them to encompass the new elements.

%-------------------------------------------------------
\subsection{\label{sec:cps:symbolsrevisited}Symbols Revisited}
%-------------------------------------------------------

The addition of channels to \glspl{tlc} requires an expansion of the \gls{cps} grammar to express them.  Specifically, the communicand, antiport, communication, out-symbol and \gls{ms} productions are all added as a consequence.  Recall that atoms and variables may be any arbitrary symbols.  By convention, however, atoms are represented with lower-case letters, and variables by upper-case letters.  All other productions are formed by combinations of those, along with a few other fixed symbols, and so atoms and variables are now referred to as \emph{primitives}.

\begin{framed}
\vspace{-0.6cm}
\begin{small}
\begin{bnf*}
    \bnfprod*{primitive}{\bnfpn{atom} \bnfor \bnfpn{variable}}\\
    \bnfprod*{symbol}{\bnfpn{primitive} \bnfor \bnfpn{term} \bnfor \bnfpn{channel} \bnfor}\\
    \bnfmore*{\bnfpn{antiport} \bnfor \bnfpn{\gls{ms}}}\\
    \bnfprod*{functor}{\bnfpn{atom}}\\
    \bnfprod*{argument}{\bnfes \bnfor \bnfsp \bnfpn{symbol} \bnfsp \bnfts{+}}\\
    \bnfprod*{term}{\bnfpn{functor} \bnfsp \bnfts{`('} \bnfsp \bnfpn{argument} \bnfsp \bnfts{`)'}}\\
    \bnfprod*{communicand}{\bnfts{`\(\langle\)'} \bnfsp \bnfpn{argument} \bnfsp \bnfts{`\(\rangle\)'}}\\
    \bnfprod*{\gls{ms}}{\bnfpn{functor} \bnfsp \bnfts{`\{'} \bnfsp \bnfpn{argument} \bnfsp \bnfts{`\}'}}\\
    \bnfprod*{in-symbol}{\bnfpn{symbol}}\\
    \bnfprod*{channel}{\bnfpn{communicand} \bnfsp \bnfts{`?'} \bnfsp \bnfpn{primitive} \bnfor}\\
    \bnfmore*{\bnfpn{communicand} \bnfsp \bnfts{`!'} \bnfsp \bnfpn{primitive}}\\
    \bnfprod*{antiport}{\bnfpn{communicand} \bnfsp \bnfts{`??'} \bnfsp \bnfpn{primitive} \bnfor}\\
    \bnfmore*{\bnfpn{communicand} \bnfsp \bnfts{`!!'} \bnfsp \bnfpn{primitive}}\\
    \bnfprod*{out-symbol}{\bnfpn{communicand} \bnfsp \bnfts{`\(\downarrow\)'} \bnfsp ( \bnfpn{atom} \bnfsp (\bnfts{`,'} \bnfpn{atom})* \bnfor \bnfts{`\(\forall\)'} )}\\
    \bnfprod*{communication}{\bnfpn{channel} \bnfor \bnfpn{antiport} \bnfor \bnfpn{out-symbol}}
\end{bnf*}
\end{small}
\vspace{-0.8cm}
\end{framed}

%-------------------------------------------------------
\subsection{\label{sec:cps:genericrulesrevisited}Generic Rules Revisited}
%-------------------------------------------------------

\begin{framed}
\vspace{-0.6cm}
\begin{align*}
 current-state  ~~  symbols  \dots ~ \rightarrow_\alpha ~ &  target-state  ~~ ( in-symbols ) \dots ~~ \\
 & ( out-symbols )_\delta \dots \\
 & | ~  \glspl{promoter} \dots ~~ \neg ~  \glspl{inhibitor} \dots
\end{align*}
\vspace{-0.8cm}
\end{framed}

It is not reflected in the grammar in \cref{sec:cps:symbolsrevisited}, but, as described in the relevant sections earlier, certain productions may appear only on the left-hand or right-hand side of the middle arrow of a \gls{cps} rule.  In particular, channels and antiports with a question mark may appear only on the left-hand side, while channels and antiports with an exclamation mark, and all out-symbols, may only appear on the right-hand side.  Furthermore, every antiport and \gls{ms} on the left-hand side must have a matching partner on the right-hand side (and vice versa).