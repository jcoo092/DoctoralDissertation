\section{Alternative `locally-synchronous' rules}
It is possible to define a ruleset where each \gls{pe} is \emph{\gls{ls}}, where a \gls{ls} \gls{pe} is one that waits until it has received \emph{all} of its expected incoming messages for a given generation before continuing.  The difference between the \gls{ls} and asynchronous \gls{nmp} processes is that a \gls{pe} working in the asynchronous style will start a new round after receiving any messages -- it only needs to receive one before returning to a new round.  Whereas a \gls{ls} \gls{pe} will remain (effectively) in a blocking wait until it has received messages for the next generation from \emph{all} of its neighbours, however long that takes.

To go from the asynchronous ruleset to a \gls{ls} one requires only changing \cref{ruleset:nmp:proxspec}'s rule 10, splitting it into two related rules, as shown in \cref{ruleset:nmp:localsync}.  Combined, the two in effect serve to change the stopping condition of the message receipt process.  Where the asynchronous rules merely block until a message arrives over \emph{any} channel, the \gls{ls} rules block until exactly one message has been received from \emph{every} neighbour.  Rule 10a terminates the blocking wait once all messages are received.  Rule 10b continues to receive messages but prevents the receipt of more than one message from a neighbour during the same round.

Similarly, the pseudocode from \cref{alg:nmp:pespecific} requires only two changes, on lines \ref{line:nmp:recvwhile} and \ref{line:nmp:recvfor}.  On line \ref{line:nmp:recvwhile}, the condition for the \texttt{while} loop changes from \(|R| = 0\) to \(R \not= A\), reflecting that the \gls{pe} now waits until it has received a message from \emph{every} neighbour, rather than one or more.  On line \ref{line:nmp:recvfor}, the set from which the \gls{pe} receives changes from \(A\) to \(A \setminus R\), reflecting that only a single message should be received from each neighbour during the current round of \gls{nmp}.

\begin{cprulesetfloat}
    \begin{cpruleset}
    
        \cprulecustnum{s_6}{}{1}{s_2}{}{10a}
        \cppromoter{\cpfunc{r}{X} \; \cpfunc{r}{Y} \; \cpfunc{r}{Z} \; \cpfunc{r}{W}}
        \cppromoter{\cpfunc{a}{\cpfunc{n}{X}\,\cpfunc{n}{Y}\,\cpfunc{n}{W}}}
        
        \cprulecustnum{s_6}{\cprecv{\cpvv{\_}{G\cpundig}{D}}{X} & & & \\ & \cpvv{X}{G}{\_}}{+}{s_6}{\cpvv{X}{G\cpundig}{D} &\\ & & & & \cpfunc{r}{X}}{10b}
        \cpinhibitor{\cpfunc{r}{X}}
        
    \end{cpruleset}
    \caption[Alternative forms of \cref{ruleset:nmp:proxspec}'s rule 10]{\label{ruleset:nmp:localsync}Alternative forms of \cref{ruleset:nmp:proxspec}'s rule 10 for a \gls{pe} operates in a \gls{ls}, rather than asynchronous, fashion.}
\end{cprulesetfloat}

In the context of \gls{cps}, there is no meaningful difference in the end result of the \gls{gs} ruleset of \cref{sec:nmp:systemwide} and the \gls{ls} rules.  In practice with implementations on standard electronic computers, however, there likely is a difference in total running times due to eliminating the need to synchronise across the entire lattice.  Each \gls{pe} now needs to wait only for its slowest neighbour during a given round, rather than the entire lattice waiting for the overall slowest \gls{pe}.  The comparative performance of all three of the `globally' synchronous, \gls{ls}, and asynchronous rules, is investigated further in \cref{sec:nmp:timingexp}.

The \gls{ls} rules necessarily work similarly to the asynchronous rules regarding receipts.  In both cases, the next messages due from every neighbour may not arrive during the same rule execution moment.  The only real difference is that the \gls{ls} version loops until it has received a message for the next generation from every neighbour, instead of moving on after receiving any message.  Sending may differ between the two, however.  In the case of the asynchronous system, it is possible for a \gls{pe} not to have received all the messages required to compute the new outgoing message for a given neighbour at the start of a round.  This is untrue for the \gls{ls} version, which by its nature guarantees that a \gls{pe} has sufficient information for every outgoing message at the start of the next round.  The rules for the \gls{ls} version could thus be simplified so that the sending process much more closely matches the \gls{gs} version if desired.

