\section{\label{sec:nmp:globalsync}\texorpdfstring{\Glsfmtplural{gs}}{Globally-synchronous} Messaging}

\cpresetrulenumber

This \namecref{sec:nmp:globalsync} presents the first \gls{nmp} variant, wherein the entire system evolves as one, and every \gls{pe} executes the same rule(s) simultaneously.  For the entirety of this \namecref{chap:nmp}, assume that all \glspl{pe}, channels and the oracle (\(o\)) behave correctly and follow their respective protocols, without faults, corruptions, lost messages, \etc{}  This significantly simplifies the presented systems by allowing the omission of error handling concepts and describing only the intended operation.

\begin{figure}
    \centering
    \includegraphics[width=0.9\textwidth]{chapters/nmp/images/iota_proxels_oracle_v7.pdf}
    \caption[The communication topology of a \glsxtrlong{nmp} system from the perspective of an arbitrary \glsxtrshort{pe} in a \gls{fne} arrangement]{The communication topology of the system from the perspective of an arbitrary \gls{pe} in a \gls{fne} arrangement.}
    \label{fig:nmp:iota_proxels_environment_oracle}
\end{figure}

The layout of the system, from the perspective of an arbitrary \gls{pe} in a \gls{fne}, is depicted in \cref{fig:nmp:iota_proxels_environment_oracle}.  The central circle is this \gls{pe} and is labelled with \(\iota\) to represent its ID in the overall system.  It is connected to each neighbour and the oracle by two-way channels.  The neighbours themselves are anonymous, but \(\iota\)'s end of each channel is labelled with a number from 1-4, representing the neighbours who are expected to sit above, to the right, below and to the left of the central \gls{pe}, respectively.  The oracle is smaller, blurred and surrounded with a dashed line, to reflect that it is a stand-in for an unspecified process that ordinarily would take place \emph{inside} the \gls{pe}.

In the current approach, each \gls{pe} has \emph{no} direct knowledge of its neighbouring \glspl{pe}.  Instead, it interacts with channels that connect to those neighbours; the channels interpose between the \glspl{pe}.  Furthermore, each label for a neighbour in a \gls{pe} is \emph{not} the system's ID for that neighbour, but the \gls{pe}'s label for a connecting channel endpoint.  Nevertheless, as a shortcut, \glspl{pe} will simply be referred to as communicating with their neighbours.  Messages exchanged by \glspl{pe} are termed \emph{\gls{nm} messages}, while messages exchanged with an oracle may be referred to as \emph{\gls{oq} messages}.

In the \gls{gs} version, the entire system evolves in lock-step, and thus is always in the same phase and \gls{cps} state.  The evolution follows a basic process, consisting of three conceptually distinct phases, the first two of which may be repeated.  Specifically, the system proceeds as follows (the rules are listed in \vref{ruleset:nmp:systemwide} and explained in \vref{sec:nmp:globalsync:rulesdesc}):

\begin{enumerate}
    \item\label{enumitem:nmp:pu} \Gls{oq} (Rules \cpruleref*{rule:nmp:systemwide:loopdecrement}, \cpruleref*{rule:nmp:systemwide:sendtooracle} \& \cpruleref*{rule:nmp:systemwide:recvfromoracle})
    \item\label{enumitem:nmp:nm} \Gls{nm} (rule \cpruleref*{rule:nmp:systemwide:antiport})
    \item\label{enumitem:nmp:final} Finalisation (Rules \cpruleref*{rule:nmp:systemwide:loopend}, \cpruleref*{rule:nmp:systemwide:oraclefinalise} \& \cpruleref*{rule:nmp:systemwide:end})
\end{enumerate}

\Cref{fig:nmp:systemwidestatemachine} depicts this progression as a state machine.  States \(s_1\) and \(s_2\) cover the \gls{oq} phase; \(s_3\) the \gls{nm} phase; and \(s_4\) and \(s_5\) the finalisation phase.  State \(s_6\) is the halting state of the system's evolution.  In an application of \gls{nmp} phases \ref{enumitem:nmp:nm} and \ref{enumitem:nmp:final} will not change.  Only phase \ref{enumitem:nmp:pu} is replaced.  The system's evolution is sketched at a high level in \cref{alg:nmp:systemwide2}.

The \gls{oq} phase is when each \gls{pe} performs its internal computation. As mentioned earlier, these computations are problem-dependent and thus cannot be modelled generically. Instead, this \namecref{chap:nmp} uses communication with an oracle assigned to each \gls{pe} to simulate this aspect of \gls{nmp}. The \gls{nm} phase is the main focus of this work. This is where \glspl{pe} exchange messages with their neighbours, based on data received earlier from other neighbours. The finalisation phase occurs after each \gls{pe} reaches a stopping point. At this point, each \gls{pe} performs a final computation to determine its output for the lattice, incorporating the latest data received from neighbours.

\begin{figure}
    \centering
    \includegraphics{chapters/nmp/images/systemwidestatemachine.pdf}
    \caption[State machine of the progression of the system-wide \glsxtrlong{nmp} \gls{cps} rules.]{State machine of the progression of the system-wide \gls{nmp} \gls{cps} rules.  The vertices are labelled with states and the arcs with the rule(s) which cause the state transitions.}
    \label{fig:nmp:systemwidestatemachine}
\end{figure}

\begin{algorithm}
\DontPrintSemicolon
\KwIn{Iteration counter \(i\) and array \(v\) of initial data}
\KwOut{Final result \(z\)}
\SetKwFor{pForEach}{parallel foreach}{do}{endfch}
\Begin{
    \tcc{\(\alpha \Leftarrow \langle \beta \rangle\) denotes receiving object \(\alpha\) on channel \(\beta\);\newline\(\gamma \Rightarrow \langle \delta \rangle\) denotes sending object \(\gamma\) on channel \(\delta\);\newline\(\epsilon \Leftrightarrow \langle \phi \rangle\) denotes antiport exchange, sending \emph{and} receiving (swapping) objects \(\epsilon\) on channel \(\phi\).}\;
    
    \While{\(i > 0\)}{
        \tcc{\Glsentrylong{oq}}
        \(i \gets i - 1\)\;
        \pForEach{\(x \in \{1, 2, 3, 4\}\)}{
            \(w[x] \gets v[x]\)\;
            \(w[x] \Rightarrow \langle o \rangle\)
        }
        \lpForEach{\((x,d) \Leftarrow \langle o \rangle\)}{
            \(v[x] \gets d\)\
        }
        
        \;\tcc{\Glsentrylong{nm}}
        \lpForEach{\(x \in \{1, 2, 3, 4\}\)}{\(v[x] \Leftrightarrow \langle x \rangle\)}
    }
    \;\tcc{Finalisation}
    \pForEach{\(x \in \{1, 2, 3, 4\}\)}{
        \(w'[x] \gets v[x]\)\;
        \(w'[x] \Rightarrow \langle o \rangle\)
    }
    \(z \Leftarrow \langle o \rangle\)\;
}
\caption[Pseudocode of the \glsxtrlong{nmp} process in the \gls{gs} system]{\label{alg:nmp:systemwide2}Pseudocode description of the process for an individual \gls{pe} in the \gls{gs} system}
\end{algorithm}

Assume before the start of the system's evolution that the correct number of \glspl{pe} are already in place, and they each contain an appropriately initialised iteration counter and their relevant channel endpoints only.  Everything else required by each \gls{pe} will be supplied by the oracle, or generated by rules during the \gls{pe}'s evolution.  The rules for the \gls{gs} system are listed in \cref{ruleset:nmp:systemwide} and explained in \cref{sec:nmp:globalsync:rulesdesc}.  This \namecref{sec:nmp:globalsync} first defines the \gls{gs} \gls{nmp} system, explains the intended operation of the \gls{ruleset}, then describes each of the ground terms used.

\subsection{System Definition}
Recall from \cref{sec:cps:formaldescriptions} that a given \gls{cps} implementation can be defined as a 6-tuple:

\cptuplechans{\text{NMP-GS}}{\cpset{\sigma_1, \dots, \sigma_{\text{max}}}}{A}{O}{C}{\text{\cref{ruleset:nmp:systemwide}}}{\cpset{s_1, s_2, s_3, s_4, s_5, s_6}}{s_1}

\(T\) is a set of \glspl{tlc}, each representing a single \gls{pe}, numbered from one to the total size of the lattice.  These numbers correspond to the \(\iota\) of \cref{fig:nmp:iota_proxels_environment_oracle}.  \(A\) is the set of all terms defined in \cref{sec:nmp:globalsync:definitions}.  Each \gls{pe}'s starting multiset in \(O\) is a generation counter \gls{functor} \(i\) and its set of initial data \(V\).  Each \gls{pe}'s set in \(C\) is \(\{1, 2, 3, 4\}\) for any \gls{pe} inside the grid, while those on a border or corner will include one or two fewer entries, respectively, in their set (see also \cref{sec:nmp:ruleslessthanfour} for further discussion of this).

\subsection{\label{sec:nmp:globalsync:rulesdesc}Description of Rules}

\begin{cprulesetfloat}
    \begin{cpruleset}
        
        % Else move to finishing
        \cprule[rule:nmp:systemwide:loopend]{s_1}{i(\lambda)}{\cponce}{s_4}{}
        
        % Decrement iterator
        \cprule[rule:nmp:systemwide:loopdecrement]{s_1}{i(I\cpundig)}{\cponce}{s_2}{i(I)}
        
        % Send to oracle
        \cprule[rule:nmp:systemwide:sendtooracle]{s_1}{\cpvq{X}{D}}{\cpmaxpar}{s_2}{\cpsend{\cpvqw{X}{D}}{o}}
        
        % Receive from oracle
        \cprule[rule:nmp:systemwide:recvfromoracle]{s_2}{\cprecv{\cpvqw{X}{D}}{o}}{\cpmaxpar}{s_3}{\cpvq{X}{D}}
        
        \\
        
        % Exchange messages
        \cprule[rule:nmp:systemwide:antiport]{s_3}{\cpvq{X}{D} & & &\\ & \cpantirecv{\cpvq{\_}{D'}}{X}}{\cpmaxpar}{s_1}{\cpvq{X}{D'} &\\ & & & & \cpantisend{\cpvq{X}{D}}{X}}
        
        \\
        
        % Send to oracle (finalisation)
        \cprule[rule:nmp:systemwide:oraclefinalise]{s_4}{\cpvq{X}{D}}{\cpmaxpar}{s_5}{\cpsend{w'\perfectunary{IncreaseHeight}{(}{)}{X}\perfectunary{IncreaseHeight}{(}{)}{D}}{o}}
        
        % Oracle returns results
        \cprule[rule:nmp:systemwide:end]{s_5}{\cprecv{\cpfunc{z}{Z}}{o}}{\cpundig}{s_6}{\cpfunc{z}{Z}}
        
    \end{cpruleset}
    \caption[Complete \gls{ruleset} for \gls{gs} \glsxtrlong{nmp}]{\label{ruleset:nmp:systemwide}Complete \gls{ruleset} for \gls{gs} \gls{nmp}, using an oracle to perform update computations}
\end{cprulesetfloat}

\begin{enumerate}
    \item If \(i\) is empty, continue to the finalisation phase.
    \item Decrement \(i\) when sending \(w\) messages to the oracle for \gls{oq}.
    \item Convert the \(v\) messages into \(w\) messages and send them to the oracle to compute the new messages to send to neighbours.
    \item Receive back new \(w\) objects from the oracle and convert them to \(v\) objects.
    \item Swap \(v\) messages with each neighbour \(X\), using antiport communication (see \cref{sec:cps:antiport}).
    \item Send the \(v\) objects as \(w'\) to the oracle for finalisation computation.
    \item Receive the final result \(\cpfunc{z}{Z}\) from the oracle and halt.
\end{enumerate}

\subsection{\label{sec:nmp:globalsync:definitions}Definitions of Terms}

\paragraph{Atoms}
\begin{description}
    \cptermdef{o}{The label of the channel used to communicate with the oracle.}
    \cptermdef{\cpempty}{The \gls{cps} `empty' atom (see \cref{sec:cps:complexsymbols,sec:cps:natnums}).}
\end{description}

\paragraph{\Glspl{functor}}
\begin{description}
    \cptermdef{i}{A generation counter.  Used to count the number of generations of \gls{nm} remaining before moving to the finalisation phase.}
    \cptermdef{v}{The \(v\) compound terms described in \cref{sec:cps:compoundterms} \emph{except} without the generation counter --- the \(i\) counter serves the same purpose for the \gls{gs} variant.  These serve as \gls{nm} messages.}
    \cptermdef{w}{The same as the \(v\) terms, but used as messages to the oracle for \gls{oq}.}
    \cptermdef{w'}{The same as the above \(w\) objects, but marked to indicate to the oracle that they are to be used for finalisation rather than \gls{oq}.}
    \cptermdef{z}{Final output \gls{functor}.  Holds the result of the \gls{pe}’s computation.}
\end{description}

\paragraph{States}
\begin{description}
    \cptermdef{s_1}{The opening \gls{oq} phase state, where data are sent to the oracle and the generation counter is decremented.}
    \cptermdef{s_2}{Data update receipt state. Updated data are received back from the oracle.}
    \cptermdef{s_3}{\Gls{nm} state.  Messages are swapped with neighbours.}
    \cptermdef{s_4}{Finalisation phase transition state}
    \cptermdef{s_5}{State for transmission to the oracle for the final result, and awaiting its response.}
    \cptermdef{s_6}{State for receipt of the result from the oracle and halting of evolution.}
\end{description}