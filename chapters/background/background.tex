\chapter{\label{chap:back}Background}
This \lcnamecref{chap:back} summarises some pertinent related topics that are relevant to the central body of this dissertation.  Each is only a short summary, sufficient to communicate key ideas, provide necessary background information, and demonstrate that there are almost always multiple related theoretical concepts and approaches to solving problems.  In none of these cases does the synopsis come close to covering the entire span of the given area comprehensively.  Each part merely tries to introduce and illuminate its topic sufficiently for the purposes of this work.  The interested reader who is unfamiliar with any of these topics is strongly advised to refer to the materials cited.

% See \url{http://forneylab.org/} for something seemingly quite similar to what I'm doing...;  fglib \url{https://github.com/danbar/fglib};  IGraph library \url{https://igraph.org/};  LGNpy \url{https://github.com/ostwalprasad/LGNpy};

This \lcnamecref{chap:back} specifically discusses formal models of concurrent computation, synchronous \& asynchronous communication, \gls{cml} and \gls{mc}.  \Gls{mc} is, in fact, another model of concurrent computation, but given its central position in this work, it is given a dedicated separate \lcnamecref{sec:back:mc}.  The same can be said about \gls{cps}, which is given its own \lcnamecref{chap:tsp}.

\section{\label{sec:back:formalmodels}Formal Models of Concurrent Computation}
The models briefly reviewed in this \namecref{sec:back:formalmodels} are just a small sample of the formal models for computation that have been devised.  An important aspect of each of the ones mentioned below is that they explicitly deal with concurrency in some fashion, thus meriting their (abridged) discussion.  Ultimately, \gls{mc} is used as the model of choice for the remainder of the dissertation after this \namecref{chap:back}, but it undoubtedly has been influenced by, or developed contemporaneously with, each of the below.  Indeed, close study of each model inevitably reveals commonalities between them.

\subsection{\label{subsec:back:csp}Communicating Sequential Processes}

\emph{\Gls{csp}} is a \emph{process algebra} and abstract model of concurrent computation put forward by Hoare \cite{Hoare1985,Roscoe2011}.  A typical sequential computation is represented by a \emph{process}.  Processes' \textcquote[][p.~478]{Roscoe2011}{behaviour is described in terms of the occurrence and availability of abstract entities called \textit{events}}.  Should more than one event be available simultaneously for a given process, then one will be chosen non-deterministically.  This choice is internal to the process, and not influenced by, or visible to, other processes.

Concurrency is introduced by the existence of multiple processes.  In general, the processes continue independently, responding to events as they come.  Should a particular event appear in the alphabet of multiple processes, however, then all processes \emph{must} choose to participate in that event at the same time.  Should all processes involved make such a choice, they engage in a synchronous multi-way atomic synchronisation (hence ``communicating'').  \gls{csp} has provided significant inspiration for concurrency design in a number of programming languages, notably including Ada \cite{Defense1983,Taft2013}, Occam \cite{Elizabeth1987}, Google's Go \cite{Meyerson2014} (not to be confused with the earlier language Go! \cite{Clark2004}, which itself was explicitly designed for concurrency) and \gls{cml} \cite{Reppy2011} (see further \vref{sec:back:cml}).  An especially interesting practical application of \gls{csp} itself is finding flaws -- and fixes to the flaws -- in information security protocols \cite{Roscoe1995,Lowe1996,Koltuksuz2010}.

\subsubsection{\(\pi\) Calculus}
\(\pi\) calculus was created by Robin Milner and associates in the early 1990s, building upon \gls{csp}.  Milner appreciated \gls{csp}, which advanced concurrency models by explicitly incorporating \emph{synchronised} interaction, something his earlier Calculus for Communicating Systems \cite{Milner1980} had lacked  \cite{Milner1993}.  Milner still regarded \gls{csp} as incomplete, however, in that it had no support for the concept of `mobility' --- \ie{} the ability of the system to reconfigure itself during operation.  In the context of \(\pi\) calculus, this is achieved by passing references to links between processes via other links, thus enabling a dynamic communication system. \(\pi\) calculus was created as an attempt to build upon earlier systems but present a complete calculus of concurrent computation, in much the same way that Lambda Calculus \cite{Barendregt1984} is a complete calculus for sequential computation.\footnote{Milner also noted that sequential computation is a special case of concurrent computation \cite{Milner1993}.}

Both \gls{csp} and \(\pi\) calculus are effective for modelling concurrent systems (see \eg{} \cite{Roscoe2011}).  They have a drawback, however, in that they tend to be very verbose.  Specifications of behaviour for processes are intertwined with the descriptions of processes' states.  They are effective for specifying a system formally, and verifying the behaviour of said system by defined reductions (see \eg{} \cite[Ch.~3.2]{Varela2013}), but for larger systems the notational burden can make it difficult to understand what each process can and will do.

\subsection{\label{subsec:back:actors}\texorpdfstring{\Glspl{actor}}{Actors}}
The \emph{\gls{actor}} model was introduced by Carl Hewitt \cite{Hewitt1973} and substantially further developed for practical programming use by Gul Agha and others \cite{Agha1986,Agha1997}.  Much like \gls{csp} \& its cousins, the \gls{actor} model is based around the concept of sequential, separate but communicating processes which exchange messages.  Again, the processes make decisions independently and proceed based on their communications.  A key difference, however, is that in the \gls{actor} model the message exchanges are \emph{asynchronous}.  Each \gls{actor} has its own \emph{mailbox}, and may send messages to other \glspl{actor} so long as it knows their identity (which is equivalent for this purpose to a concept of a postal address for the \gls{actor}), \emph{but does not wait at all for a response before proceeding}.  Moreover, for \glspl{actor}, the \emph{only} way to communicate with one another is to know the intended recipient's identity/address.  Logical shared memory is explicitly disallowed,\footnote{It is entirely possible in practice, of course, to implement an \gls{actor} system on a shared memory system.  The model expressly forbids any notion of a resource shared in any fashion besides \glspl{actor} requesting a result from another \gls{actor} who has exclusive access to the resource, however.} which prevents changing who an \gls{actor} communicates with simply by \eg{} changing the holder of a channel endpoint, but also means that actors can be distributed by a runtime without (directly) affecting the practical programming.

The \gls{actor} model is popular for concurrent programming, possibly owing to its intuitive concept.  The fact that communication is asynchronous makes \glspl{actor} much more suitable for modelling distributed systems without shared memory than \gls{csp} or similar --- \glspl{actor} can send messages and proceed without (necessarily) needing to wait for a response, instead continuing forward based on the messages they have received.  By contrast, a typical distributed system with synchronous communication would have prohibitive time costs, given the relative slow speed of typical links between distributed computers as compared to their capacity for local processing.  Many \gls{actor} systems have been implemented for different programming languages (\eg{} \cite{Varela2001,Srinivasan2008,Charousset2016,Bernstein2016}), and in fact it is a core component, and perhaps largely responsible for the success, of Erlang/OTP \cite{Armstrong2010,Armstrong2013,Vinoski2012}.  A relatively new language, Pony \cite{Clebsch2015,Clebsch2017}, takes this even further.

\Glspl{actor}, when used for non-trivial real-world software, have been criticised at times \eg{} \cite{Welsh2013,Stucchio2013}.  While some of the criticisms described are implementation-specific (relating to Akka, a Scala \gls{actor} library), a common thread is that \glspl{actor} do not compose well.  This has the negative consequence that it is difficult to combine an \gls{actor} with anything else to create a new abstraction, and can require extensive modifications in source code to make relatively simple logical changes.

\subsection{\label{sec:back:othermodels}Other Models}

\paragraph{Petri Nets}
Perhaps the earliest formalised model of concurrent computation still in wide use is the \emph{Petri net} \cite{Dennis2011}, first conceived of by Carl Petri to describe chemical processes \cite{Petri2008}.  A basic Petri net consists of places, tokens and transitions, where tokens move between places via arcs and all places are separated by transitions (and \viceversa).

The main weakness of classical Petri nets is that they require the definition of a fixed system from the outset, and do not easily permit the evolution of the system during ``runtime'' as appropriate.  Of Petri nets, \citeauthor{Varela2013} says \textcquote[][p.~36]{Varela2013}{Petri nets \textelp{} in the \textelp{} form described \textins{in \cite{Varela2013}}, \textelp{} are not able to model the dynamicity of concurrent software\textins{.}  \textelp{} Petri nets \textelp{} do not support the compositional reasoning afforded by more modern models of concurrency.}

\paragraph{\label{sec:back:pram}Parallel Random Access Machines}

\emph{\Glspl{pram}} are a model for concurrent computation arguably closer to the operation of real electronic computers than the others described above.  \citeauthor{JaJa2011} defines a \gls{pram} as \textcquote{JaJa2011}{an abstract model for parallel computation which assumes that all the processors operate synchronously under a single clock and are able to randomly access a large shared memory. In particular, a processor can execute an arithmetic, logic, or memory access operation within a single clock cycle}.  Unlike the process algebras described above, the \gls{pram} model does not, in general, impose a particular way to describe a \gls{pram} algorithm, making it sometimes quite a useful aid in developing parallel algorithms.

\Glspl{pram} developed as a natural extension to random access machines (RAMs), which, as the name suggests, provide a model that is reasonably close to typical uniprocessor electronic computers.  A \gls{pram} is a collection of processors operating synchronously (\ie{} on the same clock), each labelled with a unique identifier, and with access to a global shared memory.  Processors may operate independently of each other, but many described \gls{pram} algorithms in fact operate in a \emph{single-program multiple-data} fashion where all processors follow the same program and so are only semi-independent, at best.  These attributes mean that a \gls{pram} often looks quite similar (at a superficial level, at least) to the operation of a typical \gls{gpu}.  The systems described in \cref{chap:nmp} also bear more than a passing resemblance to \glspl{pram}.

\subsubsection{GAMMA, the Chemical Abstract Machine, and Join Calculus}

\paragraph{GAMMA}

\paragraph{The Chemical Abstract Machine}

\paragraph{Join Calculus}
In some ways, join calculus -- created in the 1990s by \citeauthor{Fournet1996} \cite{Fournet1996,Fournet2002} -- is the most similar of the models discussed in this \namecref{sec:back:formalmodels} to \gls{mc}.  Both work with multisets, are inspired by the interactions and movement of atoms, and define computations through rulesets.  Join calculus takes its inspiration directly from chemistry and chemical reactions, however, with no reference to biology.  Where evolution in \gls{mc} occurs within cells, reactions in join calculus take place in the ``chemical abstract machine.''

Join calculus has proved fruitful for assisting in the design of new concepts in practical programming, including JoCaml \cite{Fournet2003}, Joinads \cite{Petricek2011}\footnote{See also \url{http://tryjoinads.org/docs/use/joins.html}.} and Reagents \cite{Turon2012}.  Despite this utility, however, join calculus seems largely to have been neglected by those outside the theoretical computer science and programming languages communities.
%-------------------------------------------------------
\section{\label{sec:back:syncasync}Synchronous and Asynchronous Messaging in Distributed Algorithms}
%-------------------------------------------------------

% \gls{cps} models do not make any \emph{syntactic} difference between synchronous and asynchronous scenarios;
% this is strictly a \emph{runtime} assumption~\cite{Nicolescu2012}.
% Any model can run on both the synchronous and asynchronous runtime ``engines'', albeit the results may differ.
% The concepts of synchronous and asynchronous here are those of the standard timing models in distributed algorithms \cite{Lynch1996}.
% These apply only to messaging between processes.  For systems with only a single process, there is no meaningful difference.  A lone process will continue evolving without messaging until the system halts (if ever).

% \gls{cps} models do not make any \emph{syntactic} difference between synchronous and asynchronous scenarios;
% this is strictly a \emph{runtime} assumption~\cite{Nicolescu2012}.
% Any communicating computational model can run on both the synchronous and asynchronous runtime ``engines'', albeit the results may differ.
% The concepts of synchronous and asynchronous messaging used in this dissertation are those of the standard timing models in distributed algorithms \cite{Lynch1996}.
% These apply only to messaging between processes.  For systems with only a single process, there is no meaningful difference.  A lone process will continue evolving without messaging until the system halts (if ever).

This dissertation uses the standard timing models in distributed algorithms \cite{Lynch1996} for messaging where relevant.  For reference, the most important relevant concepts are explained and illustrated below.  For more on the topics discussed in this \lcnamecref{sec:back:syncasync}, the interested reader is referred to \cite{Fokkink2013,Lynch1996,Tel2000}.

\subsection{Rounds}
Under both synchronous and asynchronous messaging, all processes work in \emph{rounds} (sometimes also referred to as ``macro-steps'').  A round consists of three repeating sequential sub-steps:
\begin{enumerate}
    \item \emph{\textsc{Receive}}:  receive one or more incoming messages
    \item \emph{\textsc{Process}}:  perform any requisite local computations, and update the local state
    \item \emph{\textsc{Send}}:  send out new messages as appropriate based on the processing from the previous step
\end{enumerate}
This applies only to messaging between processes.  For systems with a solitary process, there is no meaningful difference because the process will essentially remain in the \textsc{process} sub-step.  \Ie{} a lone process will continue working without messaging until the system halts (if ever).

% As alluded to above, if there is only a single process, then it will essentially always remain in the process sub-step, until reaching a point where it stops evolving entirely.  The concept of rounds is superfluous in this situation.  Using or forgoing rounds can be a by-product of the specified \gls{ruleset}.  If there are no rules involving inter-process communication, then the process never enters the send or receive sub-steps.

\subsection{Synchronous Messaging}
Under the synchronous model, all processes proceed in lock-step.  They may carry out the sub-steps at different speeds, but begin and end rounds together.  When considering timing of messaging, there are two standard approaches:  to consider the \textsc{process} sub-step as taking one time unit and the transit time (\ie{} the time between when the message is sent by the sender and when it is received by the recipient) taking zero time units; or, to consider the \textsc{process} sub-step as taking zero time units and the transit time taking one time unit.

\subsection{Asynchronous Messaging}
Under the asynchronous model, there is (unsurprisingly) no natural synchronisation between processes.  All processes proceed through the sub-steps at their own pace.  The \textsc{process} sub-step is considered to take zero time units, and transit times are considered to take any number of time units.  Alternatively, by normalisation, while the \textsc{process} sub-step still takes zero time units, transit times take somewhere between zero and one (inclusive) time units.  This makes the second synchronous approach a special case of the asynchronous approach.  The asynchronous model is similar to the theoretical \gls{actor} model.

% \subsection{Non-determinism}
% In general, and as with \gls{mc} generally, both scenarios are non-deterministic.  This fact is most obvious in the asynchronous scenario, when messages can arrive at any arbitrary time, but even the synchronous scenario may necessitate non-deterministic choices at times.  \Eg{} when there is a choice between possible unifications, as in \cref{sec:cps:unification}; or, when there are multiple messages in a channel's bag which match the pattern of the receiving process's receipt rule but only one is to be selected,\footnote{Channels are not naturally \gls{fifo} in \gls{cps}.  See \vref{sec:cps:intertlcmess} for more.} likely because the relevant rule runs in \(\cponce\) mode.

% If complete determinism is required, it must be carefully built into the system's ruleset.  Determinism in the final result only is known as \emph{confluence}.  This means that, while different evolutions of the system may lead to different intermediate states, the final result computed will always be the same.

% \subsection{Non-determinism}
% In general, both models are non-deterministic.  This fact is most obvious in the asynchronous scenario, when messages can arrive at any arbitrary time, but even the synchronous scenario may necessitate non-deterministic choices at times.    %\Eg{} when there is a choice between possible unifications, as in \cref{sec:cps:unification}; or, when there are multiple messages in a channel's bag which match the pattern of the receiving process's receipt rule but only one is to be selected,\footnote{Channels are not naturally \gls{fifo} in \gls{cps}.  See \vref{sec:cps:intertlcmess} for more.} likely because the relevant rule runs in \(\cponce\) mode.

% If complete determinism is required, it must be carefully built into the system's ruleset.  Determinism in the final result only is known as \emph{confluence}.  This means that, while different evolutions of the system may lead to different intermediate states, the final result computed will always be the same.

\subsection{Echo}
To illustrate the difference between the synchronous and asynchronous scenarios, consider the illustrations in \cref{fig:back:echosync,fig:back:echoasync}.\footnote{These figures were created by Dr. Radu Nicolescu, and have been reproduced here with his permission.}  The Echo algorithm \cite[Ch.~4.3]{Fokkink2013} is used to establish a spanning tree over a graph of nodes/processes, in this case rooted at node 1.  The algorithm works in two phases, named \emph{broadcast} and \emph{convergecast}.  Echo starts in the broadcast phase with an initiator node sending out a message to each of its neighbours.  Each of those recipients then marks the sender as its parent, and in turn sends a broadcast message to every other one of its neighbours, then waits to receive a broadcast message from each of those other neighbours.  This pattern repeats across the graph.  Once the response broadcast messages are received, each node then sends a convergecast message back to its parent.  This carries on until finally the initiator receives the convergecast messages back from its neighbours.

In \cref{fig:back:echosync,fig:back:echoasync}, the first green node is the initiator, and each other node turns green when it receives its first message.  Edges between nodes turn into green arrows when a node chooses its parent.  The direction of the arrow points to said parent.  Red arrows indicate the direction of travel for a broadcast message, while blue arrows indicate the direction of travel for a convergecast message.  Solid red \& green arrows indicate that a message is received in that round by the recipient, while a dashed line indicates that the message is still in transit.

\begin{figure}[htbp]
    \centering
    \subcaptionbox{Round Zero}{\includegraphics[width=0.32\textwidth]{chapters/background/images/echo/sync/notext_f0_0.pdf}}
    \subcaptionbox{Round One}{\includegraphics[width=0.32\textwidth]{chapters/background/images/echo/sync/notext_f0_1.pdf}}
    \subcaptionbox{Round Two}{\includegraphics[width=0.32\textwidth]{chapters/background/images/echo/sync/notext_f0_2.pdf}}
    \subcaptionbox{Round Three}{\includegraphics[width=0.32\textwidth]{chapters/background/images/echo/sync/notext_f0_3.pdf}}
    % \subcaptionbox{Round Four}{\includegraphics[width=0.32\textwidth]{chapters/background/images/echo/sync/notext_f0_4.pdf}}
    \caption{Progression of the synchronous echo algorithm, starting from round zero before any messages are sent.  Arrows in red mean broadcast messages, while arrows in blue mean convergecast messages.}
    \label{fig:back:echosync}
\end{figure}

\begin{figure}[htbp]
    \centering
    \subcaptionbox{Round Zero}{\includegraphics[width=0.32\textwidth]{chapters/background/images/echo/async/notext_f0_0.pdf}}
    \subcaptionbox{Round One}{\includegraphics[width=0.32\textwidth]{chapters/background/images/echo/async/notext_f0_1.pdf}}
    \subcaptionbox{Round Two}{\includegraphics[width=0.32\textwidth]{chapters/background/images/echo/async/notext_f0_2.pdf}}
    \subcaptionbox{Round Three\label{fig:back:echoasync:3}}{\includegraphics[width=0.32\textwidth]{chapters/background/images/echo/async/notext_f0_3.pdf}}
    \subcaptionbox{Round Four}{\includegraphics[width=0.32\textwidth]{chapters/background/images/echo/async/notext_f0_4.pdf}}
    \subcaptionbox{Round Five}{\includegraphics[width=0.32\textwidth]{chapters/background/images/echo/async/notext_f0_5.pdf}}
    % \subcaptionbox{Round Six}{\includegraphics[width=0.32\textwidth]{chapters/background/images/echo/async/notext_f0_6.pdf}}
    % \subcaptionbox{Round Seven}{\includegraphics[width=0.32\textwidth]{chapters/background/images/echo/async/notext_f0_7.pdf}}
    % \subcaptionbox{Round Eight}{\includegraphics[width=0.32\textwidth]{chapters/background/images/echo/async/notext_f0_8.pdf}}
    % \subcaptionbox{Round Nine}{\includegraphics[width=0.32\textwidth]{chapters/background/images/echo/async/notext_f0_9.pdf}}
    % \caption{Progression of the asynchronous echo algorithm, starting from round zero before any messages are sent.  Arrows in red mean broadcast messages, while arrows in blue mean convergecast messages.  Despite node 1 being the initiator node, node 4 considers node 3 to be its parent because it first receives a broadcast message from node 3.}
    % \label{fig:back:echoasync}
% \end{figure}

% \begin{figure}\ContinuedFloat
%     \centering
    % \subcaptionbox{Round Zero}{\includegraphics[width=0.32\textwidth]{chapters/background/images/echo/async/notext_f0_0.pdf}}
    % \subcaptionbox{Round One}{\includegraphics[width=0.32\textwidth]{chapters/background/images/echo/async/notext_f0_1.pdf}}
    % \subcaptionbox{Round Two}{\includegraphics[width=0.32\textwidth]{chapters/background/images/echo/async/notext_f0_2.pdf}}
    % \subcaptionbox{Round Three}{\includegraphics[width=0.32\textwidth]{chapters/background/images/echo/async/notext_f0_3.pdf}}
    % \subcaptionbox{Round Four}{\includegraphics[width=0.32\textwidth]{chapters/background/images/echo/async/notext_f0_4.pdf}}
    % \subcaptionbox{Round Five}{\includegraphics[width=0.32\textwidth]{chapters/background/images/echo/async/notext_f0_5.pdf}}
    \subcaptionbox{Round Six}{\includegraphics[width=0.32\textwidth]{chapters/background/images/echo/async/notext_f0_6.pdf}}
    \subcaptionbox{Round Seven}{\includegraphics[width=0.32\textwidth]{chapters/background/images/echo/async/notext_f0_7.pdf}}
    \subcaptionbox{Round Eight}{\includegraphics[width=0.32\textwidth]{chapters/background/images/echo/async/notext_f0_8.pdf}}
    % \subcaptionbox{Round Nine}{\includegraphics[width=0.32\textwidth]{chapters/background/images/echo/async/notext_f0_9.pdf}}
    % \caption{Progression of the asynchronous echo algorithm (cont.), starting from round zero before any messages are sent.  Arrows in red mean broadcast messages, while arrows in blue mean convergecast messages.  Despite node 1 being the initiator node, node 4 considers node 3 to be its parent because it first receives a broadcast message from node 3.}
    % \label{fig:back:echoasync}
    \caption{Progression of the asynchronous echo algorithm, starting from round zero before any messages are sent.  Arrows in red mean broadcast messages, while arrows in blue mean convergecast messages.  Despite node 1 being the initiator node, node 4 considers node 3 to be its parent because it first receives a broadcast message from node 3.}
    \label{fig:back:echoasync}
\end{figure}

In the synchronous case, all messages sent out are received simultaneously, and all nodes proceed through their rounds together.  In the asynchronous case, however, messages may arrive at arbitrary times, and nodes proceed through their rounds entirely independently of one another.  The ``round numbers'' in the figure mark the point where at least one message has been received by a node.  The nodes react to messages as they arrive, and might do nothing for a time while other nodes are working if no new messages arrive during that time.  Under the asynchronous model, depending on communications topology and speeds, it is possible for the first broadcast message to reach a process to have followed a less direct route from the initiator than might be possible.  Exactly this is seen in \cref{fig:back:echoasync:3}, where node 4 first receives a message from node 3 (and thus marks node 3 as its parent), despite a direct connection to the initiator, node 1.

% The asynchronous model conceptually closely matches the standard definition of asynchronicity used in distributed computing \cite{Balanescu2011,Nicolescu2014}, and \emph{not} the definition of asynchronicity followed in \cite{Cavaliere2009,Frisco2012,Song2013} and related.  This asynchronous concept differs from the synchronous one primarily in three key respects:
% \begin{inparaenum}[(1)]
% \item There is \emph{no} concept of a global clock.  Each step for a given process begins after an arbitrary delay;
% \item Each step for a process takes zero time.  That is, `internal' evolution of the process is instantaneous, once the step has begun after the aforementioned delay;
% \item Messages sent between processes via channels take a random non-zero length of time to travel along the channel.
% \end{inparaenum}

% \begin{anfxwarning}
% What of the usual \gls{ps} global synchronous clock?
% \end{anfxwarning}
\section{\label{sec:back:cml}\glsfmtlong{cml-glossary}}

While parallelism is often the best (perhaps only) way to achieve improvements in execution time for different algorithms once an efficient sequential implementation has been created, it is a notoriously challenging affair \cite{Shun2017}.  When working at the level of directly manipulating threads, such as using the pthreads found in POSIX-compliant operating systems, programmers are exposed to a high level of risk of inadvertently introducing concurrency bugs, such as data races, deadlocks and livelocks.  A panoply of approaches to overcoming this challenge, both theoretical and practical, have been proposed and developed over the years, with varying degrees of success, \eg{} \cite{Boyapati2002,Bocq2012,Seinstra2004}.  Many, perhaps most, large-scale programming languages that use a runtime include some form of parallelism simplification within their standard libraries, \eg{} the Executor system in Java \cite[Ch. 4]{Fernandez2012}, and Swift \& Objective-C's Grand Central Dispatch \cite{Maskrey2018}.

Most simplifications fairly directly target either data-parallelism by simultaneously applying the same operation over multiple elements in arrays, \eg{} SIMD instructions in CPUs \cite{Hughes2015}; or task-parallelism by making provisions for the fork-join model \cite{McCool2012}.  These simplifications can be very useful, but not all instances of parallelism fit neatly into their models.  Algorithms that are well-modelled by the \Gls{csp} and \Gls{actor} models, such as those explicitly centred around concepts of message passing, are not necessarily easy to express using either SIMD or fork-join instructions.

\Gls{cml} \cite{Reppy1991,Panangaden1997} is an approach to concurrent programming originally developed by John Reppy (based on his earlier `Pegasus Meta-Language' \cite{Reppy1988}).  \Gls{cml} was created to provide a framework for creating concurrent programs with synchronous communications on single-core machines,\footnote{In fact, the original implementation \emph{relied} on the fact that the processor was single-core under-the-covers.} and was later extended to permit parallelism \cite{Reppy2009a}.  It was created originally as a library in Standard ML of New Jersey (where ML refers to the earlier programming language \textit{Meta Language}), whence the ML part of the name, but its concepts have subsequently appeared elsewhere.  The basic concept of communicating via channels has experienced a renaissance in recent years, likely due at least in part to its inclusion as a core feature of Go \cite{Meyerson2014}, but \gls{cml} has a more advanced system that Go (at the time of writing) does not fully support.

\Gls{cml} is designed to avoid many of the problems with concurrency that arise in traditional sequential programming, where the use of locks, mutexes and semaphores etc. are frequently required, and often lead to the potential introduction of problems such live-/deadlocks, data races and extreme resource contention.  This is achieved by changing the approach to concurrent programming to one of logically separate, internally sequential processing elements that share data as required by `passing messages'\footnote{This is the logical concept, but there is not strictly any specific required software implementation.} between themselves.  In \gls{cml}, these logical processing elements are referred to as threads, and they exchange messages over channels \emph{synchronously} (called \emph{rendezvous}), \ie{} there is a temporal overlap between one thread offering to send, and another to receive, over the same channel, and the first to offer blocks until the second makes its offer.  When two processes are offering appropriately on either side of an exchange, rendezvous takes place.

Reppy describes a concurrent program as one that supports multiple sequential sub-programs conceptually executing in parallel separately, but interacting through shared resources to achieve a common goal.  \Gls{cml} is concerned with the scenario where said interactions are explicit, and in order to facilitate that \enquote{\gls{cml} takes the unique approach of supporting \emph{higher-order concurrent programming}} (emphasis Reppy's), whereby communication and synchronisation are made into first-class members of the language, similar to how functional programming languages made functions into first-class members of themselves \cite[Preface]{Reppy2007}.

\begin{anfxerror}{Expand this?}
Describe more of how \gls{cml} works?  Is there not something in the gcol chapter that can be shifted into here, at least?
\end{anfxerror}
\section{\label{sec:back:mc}\Glsfmtname{mc}/\glsfmtname{ps}}

\emph{\Gls{mc}}, also known as \emph{\gls{ps}} (the two terms are generally used interchangeably), is a bio-inspired model of computing created by Gheorghe Păun in the late 1990s \cite{tPaun98a,Paun2000}.  It was originally conceived of by considering the process of chemical reactions and exchanges that occur inside living biological cells and the membranes within, and regarding this process as a form of computation.  \Gls{mc} was identified in 2016 by the National Research Council of Canada as \textcquote[][p. 17]{Wiseman2016}{a rigorous and comprehensive framework that provides a parallel distributed framework with flexible evolution rules.}

\citeauthor{Paun2002} describes \gls{mc} as:
\blockcquote[][p.~VII]{Paun2002}{Membrane computing is a branch of natural computing which abstracts from the structure and the functioning of living cells. In the basic model, the membrane systems -- also called P systems -- are distributed parallel computing devices, processing multisets of objects, synchronously, in the compartments delimited by a membrane structure. The objects, which correspond to chemicals evolving in the compartments of a cell, can also pass through membranes. The membranes form a hierarchical structure --- they can be dissolved, divided, created, and their permeability can be modified. A sequence of transitions between configurations of a system forms a computation. The result of a halting computation is the number of objects present at the end of the computation in a specified membrane, called the output membrane. The objects can also have a structure of their own that can be described by strings over a given alphabet of basic molecules - then the result of a computation is a set of strings.}

\begin{anfxerror}{P systems Diagram?}
Include a copy of the membrane layout diagram from Paun?
\end{anfxerror}

\Gls{ps} works analogously to a typical modern electronic computer, in that the system stores data and processes \& updates those data based on a predefined program, with a view to arriving at a computable answer based on the starting state and any inputs to the system \cite{Paun2002,Paun2010b}.  In classical \gls{ps}, the data are multisets of symbols, representing various chemicals and their quantities.  These are found inside one or more \emph{cells},\footnote{Loosely based on real biological cells.} which (to a certain extent at least) form a hybrid between main memory and the processing units of a computer.  The instructions of the program itself are provided by \emph{rules}, which specify transformations of objects and interactions with the surrounding environment and other membranes or cells.

There are now, broadly, three main families of \gls{ps} variants:  \gls{clps} \cite{Paun2001,Paun2002}, \gls{tlps} \cite{tMaPaPaRo01a,Martin-Vide2003} and \gls{snps} \cite{Ionescu2006}.\footnote{Several other variants have been created, but most are used infrequently, if ever.  Most recent work in \gls{mc} has focused on sub-variants of \gls{clps}, \gls{tlps}, \gls{snps} and \gls{cps}.}  \Gls{clps} is the direct descendant of the original classical \gls{ps}, and sees objects compartmented into \emph{membranes}, which are arranged in a graphical tree structure with the outermost \emph{skin} membrane -- which separates the cell from its environment -- as the root of the tree.  In most variants, objects can evolve inside a membrane, but also be communicated between membranes (and the environment).  Furthermore, membranes can \emph{divide} or \emph{dissolve} themselves, and may have one or more special properties, such as \emph{polarization} \cite{Paun1999a}.

On the other hand, \gls{tlps} and \gls{snps} both arrange their computing compartments -- named \emph{cells} or \emph{neurons}, respectively -- as nodes in arbitrary digraphs, with the edges between them representing connecting channels or synapses.  Whereas \gls{clps} emphasises the evolution of multisets of objects inside membranes of a given cell, \gls{tlps} and \gls{snps} emphasise communication between separate cells/neurons, and might not include any capacity for internal evolution inside cells.  If new objects are required, they are imported via communication with the environment, which possesses an unlimited number of all objects but has no rules of its own.

While \gls{tlps} have arbitrary alphabets, only one object is used in \gls{snps}, the \emph{spike}.  This means that \gls{tlps} are frequently much like \gls{clps} in that they have custom objects for each purpose, with the key difference (usually) being in the arrangement of the compartments/membranes/cells relative to each other and the choice between the two motivated primarily by which one better fits the computation to be modelled.

Conversely, \gls{snps} represent everything through the use of differing quantities of the spike, kept in different neurons.  This means that it can be more complex to model certain problems, but also arguably means that \gls{snps} are, \textit{prima facie}, closer to Lambda Calculus \cite{Barendregt1984} and Church Numerals (see \eg{} \cite{Koopman2014,Hinze2005}), as well as Register Machines (see \eg{} \cite{Korec1996}) (and indeed Register Machines have been simulated with \gls{snps} \cite{Pan2010}).  All three main types of \gls{ps}, in some form, have been proven Turing-universal though \cite{Bernardini2005,Chen2008,Freund2005}, so all three should be capable of expressing the same computations in different forms.  Furthermore, because \gls{snps} can be easily represented numerically, they lend themselves well to vector/matrix representations \cite{Zeng2010,Martinez-del-Amor2021,Gheorghe2021,Hu2016}.  This means that, potentially, \gls{snps} implementations can take advantage of high-performance techniques such as directly using \gls{blas} \& \gls{lapack} and/or \glspl{gpu} \cite{Aboy2019}.

Arguably, the most noteworthy and important aspects of \gls{ps} models are that:
\begin{inparaenum}[(i)]
\item They have no space limit.  That is, they contain an unbounded number of cells, objects and membranes;
\item Usually, across all cells and membranes, all rules that can be applied are applied, as many times as possible given the current number of objects available.
\end{inparaenum}
These two features mean that \gls{ps} have unbounded space and processing capacity, which can be used to solve traditionally computationally difficult problems relatively quickly \cite{Sosik2003,Jimenez2003,Paun1999a,Henderson2020}.  Most of these solutions, however, rely on trading time complexity for space complexity.  While this works in the theoretical framework, electronic simulations of the systems do not have access to unlimited instantaneous memory space, meaning many of the fast solutions are impractical with current real-world computers, \eg{} \cite{Cooper2019,Cooper2019a} \fxnote[inline]{[refs]} (see further \vref{sec:psystemsuses}).

\citeauthor{Valencia-Cabrera2019} said of this:
\blockcquote[][p.~213]{Valencia-Cabrera2019}{We do not know if we will have those machines able to solve NP-complete problems in polynomial time, in many cases even linear time, but \textins{that does} not necessarily mean we will have to wait until that moment in biochemical technology to find some relevant use of P systems. As Babbage kept working on his ideas, not simply waiting until the precise moment when Turing, Von Neumann, and their contemporaries witnessed the first electronic computers based on similar principles, membrane computing must keep moving, finding new ways to provide a step further.}

Nevertheless, modelling a problem in \gls{mc} can lead to new insights or improved formulations of solutions, as occurs in \cref{chap:nmp}.  For example, in \cite{GimelFarb2013a} (building on \cite{Gimelfarb2011}) \citeauthor{GimelFarb2013a} describe how formulating Symmetric Dynamic Programming \gls{sm} in terms of \gls{ps} led to finding a bug in the implementation, \textcquote[][p.~24]{GimelFarb2013a}{but also (and what is much more important) refactor this algorithm, based on our cell structure.  The result is a more robust and flexible version, which allows us to fine tune its parameters and enhance its capabilities, without rewriting it from scratch.}  Furthermore, as reported in \cite{Nicolescu2014b}, this exercise led directly to the creation of a new \gls{sm} algorithm, Concurrent Propagation \cite{Gimelfarb2012}.  \citeauthor{Pang2018} \cite{Pang2018} also claim significant benefits from modelling certain problems in a novel variant of Enzymatic Numerical \gls{ps} \cite{Pavel2010}, but it is unclear how much of the stated benefit compared to their baseline implementation arises instead from the use of a \gls{gpu}.

\begin{anfxerror}{Finish 'em!}
Need to finish the below subsection...
\end{anfxerror}

\subsection{Objects, Rules and Steps}
All known \gls{ps} types fundamentally operate on a similar basis:  One or more sets of rules -- \emph{\gls{ruleset}} -- are defined, describing how the \emph{objects} present in the system's compartments change at each \emph{step}.  As mentioned above, the objects are usually multisets of arbitrary symbolic \emph{atoms}, with the exceptions of \gls{snps} which uses the spike as its only symbol, and Numerical \gls{ps}, which uses ordinary numbers in place of atoms.

\Gls{ps} types normally operate synchronously and assume the presence of a global clock.  At each clock ``tick'', every compartment compares its extant objects and its \emph{evolution rules}, determines which rules are applicable given the current objects, and then deletes the objects used in the rules, replacing them with new ones as the rules dictate.  This execution of the system is referred to as the system's evolution.  All \gls{ps} evolve, and therefore all types have evolution rules (though they may not be referred to as such).  Other types of rules are possible, including: \emph{dissolution} and \emph{division} rules in \gls{clps}, where membranes either dissolve and release their objects into the their parent membrane, or replicate themselves (essentially performing mitosis) and distribute their contents among the new membranes; \emph{forgetting} rules in \gls{snps} whereby one or more copies of the spike are removed from a given neuron;

All rules use the same basic model.  

\subsubsection{Weak Priority Order}

Many, perhaps most, types of \gls{ps} \glspl{ruleset} use a \emph{weak-priority} ordering.  This means that some rules will be tested for applicability ahead of others, on some priority basis, but earlier applicable rules only prevent later applicable rules from being applied if there is a conflict between the two.  The most common way that this conflict can arise is by two rules trying to use the same pre-existing object in the compartment.

Generally speaking, an individual rule will select for use one or more copies of one or more objects the multiset.  At the end of the rule's application, these objects are deleted and replaced with any new ones the rule specifies.  Since the rule will delete the chosen objects, it would not make sense for another rule to be able also to use and then delete the same objects.  Therefore, the first rule to select (or take hold of or seize \etc{}) a given object prevents any other rule from using it too, and thus the first rule has priority over later rules.

The typical method of defining rules' priority is to use \emph{top-down} ordering.  This simply means that the rules presented first in a \gls{ruleset} have 

\subsection{Computer Representations and Simulations of \glsfmtname{ps}}

\subsubsection{\Adhoc{} and General Simulations}
There are arguably two main approaches to simulating \gls{ps}:
\begin{inparaenum}[a)]
\item ``\Adhoc{}'' simulations, where a separate program is written specifically for a given type of problem and its \gls{ps} solution; and
\item ``General'' simulations, where a separate simulation engine capable of simulating one or more types of \glspl{ps} is created independent of a given problem, and is supplied problem-specific configurations.  The engine uses the configuration to initialise the simulated system, and works through the problem from there.
\end{inparaenum}

The main advantage of the \adhoc{} style is the ability to adjust and optimise the simulation's implementation to suit the \gls{ps} variant used, and the problem at hand.  In general, \adhoc{} simulations would be expected to require less resources to find the answer, \eg{} running faster and/or using less memory.  The major disadvantage of the \adhoc{} approach is that a new simulation must be developed for each problem studied, requiring more time and greater levels of technical skill while reducing flexibility.  The main advantages of the general approach are greater flexibility from the produced program -- \ie{} it can simulate more problems -- and a broadening of the people who can experiment with different \gls{ps} variants and problems to those with lower levels of programming expertise.

General simulations permit specialisation and a division of labour, meaning one person can look into new \gls{ps} variants and problems to apply them to, while another person focuses on developing and improving the simulation engine itself.  This is a clear upside, but there is equally a downside: lacking problem-specific knowledge, the general simulations usually do not perform all potential optimisations, meaning that there could be unavoidable upper bounds on the efficiency of a simulation, no matter the specific problem at hand.  Furthermore, general simulations must run inside another program, whereas \adhoc{} simulations can be created as independent, native executables.

Traditionally, this has been an `either/or' problem, where one can take either a wholly \adhoc{} approach or a wholly generalist approach.  \citeauthor{Perez-Hurtado2019} more recently introduced a \gls{ps} ``compiler'' \texttt{pcc} \cite{Perez-Hurtado2019}, which can produce a standalone native executable from a non-programmatic specification of a particular \glspl{ps} --- thus providing a third, middle-ground option.  They say of this compiler: \textquote{the goal of \textins{\texttt{pcc}} is twofold: On the one hand, it purports to be a good assistant for researchers while verifying their designs, even working with experimental models. On the other hand, it provides optimized simulators for real applications, such as robotics or simulation of biological phenomena.}  It was not used in this dissertation, as \texttt{pcc} did not support \gls{cps} at the relevant time, but the idea holds great promise for the future.

\subsubsection{\label{sec:back:simulators}\Glsfmtname{ps} Simulators}

\citeauthor{Valencia-Cabrera2019} provide a summary of the development of simulators for \Gls{ps} since the field's inception in the late 1990s \cite{Valencia-Cabrera2019}.\footnote{The authors also provide a timeline of practical works in \gls{ps} at \url{https://github.com/RGNC/plingua}.  Some of the software described in this \lcnamecref{sec:back:simulators} is available at \url{http://ppage.psystems.eu/index.php/Software/}.}  Unsurprisingly, most early simulators were \adhoc{} and created for a specific purpose, focusing on one problem domain and simulating one \gls{ps} variant.  Many were intended for formal verification of models as much as they were for practical use \cite{Gutierrez-Naranjo2007}.  These early simulations were written in a wide variety of programming languages, including (comparatively) lesser-used languages such as Haskell, Prolog and LISP.  Notably, \citeauthor{Ciobanu2004} created a simulator specifically for distributed computing, using C++ and \gls{mpi} \cite{Ciobanu2004}.

As the number of \gls{ps} variants defined, and simulations to experiment with them, expanded greatly, it began to make more sense to create general simulators which did not require detailed customisation for every experiment.  A handful of these multi-purpose simulators began to appear, including (among others): PSim \cite{Bianco2007,Bianco2007a}; a transpiler from Systems Biology Markup Language (SBML) to C Language Integrated Production System (CLIPS) \cite{NepomucenoChamorro2005};  and a web-based simulator which also made use of CLIPS \cite{Bonchis2005}.  Of particular interest from this period is \cite{Acampora2007}, which specifically targeted the creation of a paralel and distributed multi-agent system, to take advantage of the concurrency inherent in most \gls{ps} variants and models.

While these simulators were a clear step to re-usability, they still largely targeted only a specific \gls{ps} variant or sub-variant.  There was another issue in that there was no standard for representing an individual \glspl{ps}.  Simulators generally either used their own custom specification system, such as a special-purpose XML schema, or attempted to make use of a representation created for another purpose, such as SBML.  To address these two shortcomings of the existing systems, researchers at the Universidad de Sevilla (University of Seville) in Spain created \gls{plingua}\footnote{\url{http://www.p-lingua.org/wiki/index.php/Main_Page}, \url{https://github.com/RGNC/plingua}} \cite{Diaz-Pernil2008a,Garcia-Quismondo2010} and \gls{mecosim}\footnote{\url{http://www.p-lingua.org/mecosim/}} \cite{Perez-Hurtado2010}.

\Gls{plingua} is a declarative markup language, used to specify specific systems and their initial configurations.  Arguably, it has become the dominant specification language of the computerised \gls{ps} world.  Crucially, \gls{plingua} also allows for the specification of new \gls{ps} variants and extensions to existing ones, giving it a much greater potential flexibility.  It is primarily built around the Java library PLinguaCore, which provides functionality to translate between various representations of \gls{ps} specifications.  One of the simulators to make heavy use of \gls{plingua} is \gls{mecosim}.

\Gls{mecosim} is a Java-based general-purpose \gls{mc} simulator.  It uses \gls{plingua} and spreadsheets to define the evolution of a given \gls{ps} type, as well as the problem to be solved --- both the rules and starting state of the system.  The particular strengths of \gls{mecosim} are that, once a particular type of \glspl{ps} has been defined it is completely re-usable, and the simulator permits rapid experimentation with different designs without any programming.  Sadly, however, it appears that both \gls{plingua} and \gls{mecosim} have effectively been abanonded, as neither seems to have been substantively updated in some years.  It is not currently clear if there is any unifying simulator of specification language to supersede them.

\begin{anfxwarning}{Some citations to include}
\cite{Raghavan2020,Raghavan2020a,Raghavan2016}
\end{anfxwarning}

\subsection{\label{sec:psystemsuses}Practical Applications of \glsfmtname{ps}}
\Gls{mc} is not just a theoretical model with limited practical use.  Besides Image Processing \& Computer Vision (see \vref{subsec:imgprocpsys}), \gls{ps} variants have been applied to a range of fields, from power grid management to robotic control systems \cite{Zhang2017}.

\begin{anfxwarning}{Some citations to include}
\cite{Zhang2020,Colomer2010,Gheorghe2010,Liu2016,Huang2016,Perez-Hurtado2010,Verlan2012,Syropoulos2004,Liu2020,Lefticaru2011,Oltean2008}
\end{anfxwarning}

\begin{anfxerror}{Finish this}
\citeauthor{Florea2017} proposed using Enzymatic Numerical \gls{ps} for robotics \cite{Florea2017,Florea2016,Florea2017a,Florea2019,Florea2016a}.
\end{anfxerror}

\subsubsection{\label{subsec:imgprocpsys}\fxwarning{How relevant still is this section?}{Image Processing and Computer Vision in \glsfmtname{ps}}}
\begin{anfxwarning}{Some citations to include}
\cite{Zhang2012,Yuan2019}
\end{anfxwarning}

Perhaps owing to the potentially unbounded space and parallelism capacity of \gls{ps}, combined with the embarrassingly parallel nature of many tasks in Image Processing \& Computer Vision, the latter have proved to be fertile ground for the former, although not every publication puts its model to the test with a computerised simulation, or if it does, the authors may only provide scant details \cite{Diaz-Pernil2019}.

\citeauthor{Christinal2011} \cite{Christinal2011} described a family of \gls{tlps} to perform region-based segmentation of both 2D and 3D images.  Despite their family of systems requiring only two cells, it also needed custom rule sets based on the size of the images as well as the number of colours present, with a number of rules per set proportional to the same measurements.  The paper showed the results of simulating the system, but provides no details on performance.

\citeauthor{Diaz-Pernil2013} \cite{Diaz-Pernil2013} commented that \enquote{\textelp{} commonly \textins{a} parallel algorithm needs to be re-designed with only slight references to the \textins{sequential original}.  \textelp{} the design of a new parallel implementation not inspired by the sequential one allows \textelp{} the proposal of new creative solutions.}  They then demonstrated this fact by designing a new edge detection and segmentation algorithm named `A Graphical P (AGP) segmentator', inspired by the Sobel operator (see \eg{} \cite{Nixon2012}) and using the segmentation method from \cite{Christinal2011}, which they modelled in \gls{tlps}.  The authors implemented their new algorithm on a \gls{gpu} and compared it with an implementation of the \numproduct{3x3} and \numproduct{5x5} Sobel operators, finding that theirs had near-identical runtimes but superior edge detection capabilities.

\citeauthor{Diaz-Pernil2013a} \cite{Diaz-Pernil2013a} further explored modelling classic image processing techniques by implementing \citeauthor{Guo1989}'s binary image skeletonisation technique \cite{Guo1989} with \gls{snps}.  The overall system's rules templates are reasonably simple, but include references to a set \(\mathit{DEL}\) (used as a lookup to determine whether a cell should turn white or stay black) which does not appear to be modelled inside the system, meaning that it is not self-contained.  The authors simulated this system on a \gls{gpu}, but found that their implementation was upwards of twice as slow as another pre-existing implementation.  Confusingly, however, they state that one of the reasons for this is \enquote{that the use of an alphabet with only one object, the spike \(a\), does not fit in the GPU architecture}.  This statement is perplexing, given that spikes can easily be represented as simple integers.  The authors also commented that the synchronous nature of the model is unrealistic, and imposing a global clock upon the system can be problematic.

\citeauthor{Nicolescu2014} \cite{Nicolescu2014} alternatively applied \gls{cps} to image skeletonisation based on \citeauthor{Guo1989}'s technique \cite{Guo1989}, presenting three forms of a solution: Synchronous versions that use multiple or a single cell (essentially the latter replicates the former via the use of sub-membranes), and an asynchronous multi-cell version.  The asynchronous version no longer assumes that all messages are passed between cells simultaneously and instantaneously, compensating for this by increasing the number of messages used.  This form, while arguably more realistic to modern computers, requires a greater message complexity. A simplified \Gls{actor}-model-based (see \cref{subsec:back:actors}) implementation using \fsharp{}'s \texttt{MailboxProcessor} \cite[ch.~11]{Syme2015a} was presented, but no results from running it were reported.

\citeauthor{Nicolescu2015} \cite{Nicolescu2015} further applied \gls{cps} to seeded region growing of greyscale images.  The described system used a two-level approach, based on the `Structured Grid Dwarf' of the 13 Berkeley Dwarves \cite{Asanovic2006}, where the image was divided into rectangular blocks of multiple pixels.  Each block was modelled with a single cell, inter-block processing was carried out via message passing, and intra-block processing was performed by typical object evolution.  It was again suggested that this would fit well to the \Gls{actor} model.

\citeauthor{Diaz-Pernil2016} \cite{Diaz-Pernil2016} built upon the AGP segmentator algorithm to create a version that works with RGB images rather than greyscale and applied it to a common medical Computer Vision task, isolating the `optic disc' in images of the inner eye.  Along with this, they used the skeletonisation algorithm from \cite{Diaz-Pernil2013a} and a number of other steps not based on \gls{ps} to produce a complete imaging pipeline.  The authors implemented this on a \gls{gpu}, and found that their system was both more accurate and faster than previous systems.

% \subsubsection{\glsfmtname{ps} on \glsxtrlongpl{gpu}}
% In many instances, a \gls{ps} model for a problem involves many independent small elements processing their data separately, and perhaps updating each other's state at the end of a step.  Given that this sounds remarkably close to the Single-Instruction Multiple-Thread \cite[Ch. 4.4.1]{Hennessy2012} nature of modern \gls{gpgpu}, it is no surprise that there has been much work put into simulating \gls{ps} on \glspl{gpu}.

% \begin{anfxwarning}{More citations}
% \cite{Cecilia2010,Cecilia2010a,Cecilia2013,Macias-Ramos2015,Martinez-Del-Amor2015,Martinez-Del-Amor2013a,Maroosi2014,Maroosi2014a}
% \end{anfxwarning}