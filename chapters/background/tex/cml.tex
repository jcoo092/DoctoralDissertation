\section{\label{sec:back:cml}\glsfmtlong{cml-glossary}}

While parallelism is often the best (perhaps only) way to achieve improvements in execution time for different algorithms once an efficient sequential implementation has been created, it is a notoriously challenging affair \cite{Shun2017}.  When working at the level of directly manipulating threads, such as using the pthreads found in POSIX-compliant operating systems, programmers are exposed to a high level of risk of inadvertently introducing concurrency bugs, such as data races, deadlocks and livelocks.  A panoply of approaches to overcoming this challenge, both theoretical and practical, have been proposed and developed over the years, with varying degrees of success, \eg{} \cite{Boyapati2002,Bocq2012,Seinstra2004}.  Many, perhaps most, large-scale programming languages that use a runtime include some form of parallelism simplification within their standard libraries, \eg{} the Executor system in Java \cite[Ch. 4]{Fernandez2012}, and Swift \& Objective-C's Grand Central Dispatch \cite{Maskrey2018}.

Most simplifications fairly directly target either data-parallelism by simultaneously applying the same operation over multiple elements in arrays, \eg{} SIMD instructions in CPUs \cite{Hughes2015}; or task-parallelism by making provisions for the fork-join model \cite{McCool2012}.  These simplifications can be very useful, but not all instances of parallelism fit neatly into their models.  Algorithms that are well-modelled by the \Gls{csp} and \Gls{actor} models, such as those explicitly centred around concepts of message passing, are not necessarily easy to express using either SIMD or fork-join instructions.

\Gls{cml} \cite{Reppy1991,Panangaden1997} is an approach to concurrent programming originally developed by John Reppy (based on his earlier `Pegasus Meta-Language' \cite{Reppy1988}).  \Gls{cml} was created to provide a framework for creating concurrent programs with synchronous communications on single-core machines,\footnote{In fact, the original implementation \emph{relied} on the fact that the processor was single-core under-the-covers.} and was later extended to permit parallelism \cite{Reppy2009a}.  It was created originally as a library in Standard ML of New Jersey (where ML refers to the earlier programming language \textit{Meta Language}), whence the ML part of the name, but its concepts have subsequently appeared elsewhere.  The basic concept of communicating via channels has experienced a renaissance in recent years, likely due at least in part to its inclusion as a core feature of Go \cite{Meyerson2014}, but \gls{cml} has a more advanced system that Go (at the time of writing) does not fully support.

\Gls{cml} is designed to avoid many of the problems with concurrency that arise in traditional sequential programming, where the use of locks, mutexes and semaphores etc. are frequently required, and often lead to the potential introduction of problems such live-/deadlocks, data races and extreme resource contention.  This is achieved by changing the approach to concurrent programming to one of logically separate, internally sequential processing elements that share data as required by `passing messages'\footnote{This is the logical concept, but there is not strictly any specific required software implementation.} between themselves.  In \gls{cml}, these logical processing elements are referred to as threads, and they exchange messages over channels \emph{synchronously} (called \emph{rendezvous}), \ie{} there is a temporal overlap between one thread offering to send, and another to receive, over the same channel, and the first to offer blocks until the second makes its offer.  When two processes are offering appropriately on either side of an exchange, rendezvous takes place.

Reppy describes a concurrent program as one that supports multiple sequential sub-programs conceptually executing in parallel separately, but interacting through shared resources to achieve a common goal.  \Gls{cml} is concerned with the scenario where said interactions are explicit, and in order to facilitate that \enquote{\gls{cml} takes the unique approach of supporting \emph{higher-order concurrent programming}} (emphasis Reppy's), whereby communication and synchronisation are made into first-class members of the language, similar to how functional programming languages made functions into first-class members of themselves \cite[Preface]{Reppy2007}.

\begin{anfxerror}
Expand this?  Describe more of how \gls{cml} works?  Is there not something in the gcol chapter that can be shifted into here, at least?
\end{anfxerror}