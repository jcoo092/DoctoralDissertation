\section{\label{sec:back:mc}\Glsfmtname{mc}/\glsfmtname{ps}}

\emph{\Gls{mc}}, also known as \emph{\gls{ps}} (the two terms are generally used interchangeably), is a bio-inspired model of computing created by Gheorghe Păun in the late 1990s \cite{tPaun98a,Paun2000}.  It was originally conceived of by considering the process of chemical reactions and exchanges that occur inside living biological cells and the membranes within, and regarding this process as a form of computation.  \Gls{mc} was identified in 2016 by the National Research Council of Canada as \textcquote[][p. 17]{Wiseman2016}{a rigorous and comprehensive framework that provides a parallel distributed framework with flexible evolution rules.}

\citeauthor{Paun2002} describes \gls{mc} as:
\blockcquote[][p.~VII]{Paun2002}{Membrane computing is a branch of natural computing which abstracts from the structure and the functioning of living cells. In the basic model, the membrane systems -- also called P systems -- are distributed parallel computing devices, processing multisets of objects, synchronously, in the compartments delimited by a membrane structure. The objects, which correspond to chemicals evolving in the compartments of a cell, can also pass through membranes. The membranes form a hierarchical structure --- they can be dissolved, divided, created, and their permeability can be modified. A sequence of transitions between configurations of a system forms a computation. The result of a halting computation is the number of objects present at the end of the computation in a specified membrane, called the output membrane. The objects can also have a structure of their own that can be described by strings over a given alphabet of basic molecules - then the result of a computation is a set of strings.}

\begin{anfxerror}
Include a copy of the membrane layout diagram from Paun?
\end{anfxerror}

\Gls{ps} works analogously to a typical modern electronic computer, in that the system stores data and processes \& updates those data based on a predefined program, with a view to arriving at a computable answer based on the starting state and any inputs to the system \cite{Paun2002,Paun2010b}.  In classical \gls{ps}, the data are multisets of symbols, representing various chemicals and their quantities.  These are found inside one or more \emph{cells},\footnote{Loosely based on real biological cells.} which (to a certain extent at least) form a hybrid between main memory and the processing units of a computer.  The instructions of the program itself are provided by \emph{rules}, which specify transformations of objects and interactions with the surrounding environment and other membranes or cells.

There are now, broadly, three main families of \gls{ps} variants:  \gls{clps} \cite{Paun2001,Paun2002}, \gls{tlps} \cite{tMaPaPaRo01a,Martin-Vide2003} and \gls{snps} \cite{Ionescu2006}.\footnote{Several other variants have been created, but most are used infrequently, if ever.  Most recent work in \gls{mc} has focused on sub-variants of \gls{clps}, \gls{tlps}, \gls{snps} and \gls{cps}.}  \Gls{clps} is the direct descendant of the original classical \gls{ps}, and sees objects compartmented into \emph{membranes}, which are arranged in a graphical tree structure with the outermost \emph{skin} membrane -- which separates the cell from its environment -- as the root of the tree.  In most variants, objects can evolve inside a membrane, but also be communicated between membranes (and the environment).  Furthermore, membranes can \emph{divide} or \emph{dissolve} themselves, and may have one or more special properties, such as \emph{polarization} \cite{Paun1999a}.

On the other hand, \gls{tlps} and \gls{snps} both arrange their computing compartments -- named \emph{cells} or \emph{neurons}, respectively -- as nodes in arbitrary digraphs, with the edges between them representing connecting channels or synapses.  Whereas \gls{clps} emphasises the evolution of multisets of objects inside membranes of a given cell, \gls{tlps} and \gls{snps} emphasise communication between separate cells/neurons, and might not include any capacity for internal evolution inside cells.  If new objects are required, they are imported via communication with the environment, which possesses an unlimited number of all objects but has no rules of its own.

While \gls{tlps} have arbitrary alphabets, only one object is used in \gls{snps}, the \emph{spike}.  This means that \gls{tlps} are frequently much like \gls{clps} in that they have custom objects for each purpose, with the key difference (usually) being in the arrangement of the compartments/membranes/cells relative to each other and the choice between the two motivated primarily by which one better fits the computation to be modelled.

Conversely, \gls{snps} represent everything through the use of differing quantities of the spike, kept in different neurons.  This means that it can be more complex to model certain problems, but also arguably means that \gls{snps} are, \textit{prima facie}, closer to Lambda Calculus \cite{Barendregt1984} and Church Numerals (see \eg{} \cite{Koopman2014,Hinze2005}), as well as Register Machines (see \eg{} \cite{Korec1996}) (and indeed Register Machines have been simulated with \gls{snps} \cite{Pan2010}).  All three main types of \gls{ps}, in some form, have been proven Turing-universal though \cite{Bernardini2005,Chen2008,Freund2005}, so all three should be capable of expressing the same computations in different forms.  Furthermore, because \gls{snps} can be easily represented numerically, they lend themselves well to vector/matrix representations \cite{Zeng2010,Martinez-del-Amor2021,Gheorghe2021,Hu2016}.  This means that, potentially, \gls{snps} implementations can take advantage of high-performance techniques such as directly using \gls{blas} \& \gls{lapack} and/or \glspl{gpu} \cite{Aboy2019}.

Arguably, the most noteworthy and important aspects of \gls{ps} models are that:
\begin{inparaenum}[(i)]
\item They have no space limit.  That is, they contain an unbounded number of cells, objects and membranes;
\item Usually, across all cells and membranes, all rules that can be applied are applied, as many times as possible given the current number of objects available.
\end{inparaenum}
These two features mean that \gls{ps} have unbounded space and processing capacity, which can be used to solve traditionally computationally difficult problems relatively quickly \cite{Sosik2003,Jimenez2003,Paun1999a,Henderson2020}.  Most of these solutions, however, rely on trading time complexity for space complexity.  While this works in the theoretical framework, electronic simulations of the systems do not have access to unlimited instantaneous memory space, meaning many of the fast solutions are impractical with current real-world computers, \eg{} \cite{Cooper2019,Cooper2019a} \fxnote[inline]{[refs]} (see further \vref{sec:psystemsuses}).

\citeauthor{Valencia-Cabrera2019} said of this:
\blockcquote[][p.~213]{Valencia-Cabrera2019}{We do not know if we will have those machines able to solve NP-complete problems in polynomial time, in many cases even linear time, but \textins{that does} not necessarily mean we will have to wait until that moment in biochemical technology to find some relevant use of P systems. As Babbage kept working on his ideas, not simply waiting until the precise moment when Turing, Von Neumann, and their contemporaries witnessed the first electronic computers based on similar principles, membrane computing must keep moving, finding new ways to provide a step further.}

Nevertheless, modelling a problem in \gls{mc} can lead to new insights or improved formulations of solutions, as occurs in \cref{chap:nmp}.  For example, in \cite{GimelFarb2013a} (building on \cite{Gimelfarb2011}) \citeauthor{GimelFarb2013a} describe how formulating Symmetric Dynamic Programming \gls{sm} in terms of \gls{ps} led to finding a bug in the implementation, \textcquote[][p.~24]{GimelFarb2013a}{but also (and what is much more important) refactor this algorithm, based on our cell structure.  The result is a more robust and flexible version, which allows us to fine tune its parameters and enhance its capabilities, without rewriting it from scratch.}  Furthermore, as reported in \cite{Nicolescu2014b}, this exercise led directly to the creation of a new \gls{sm} algorithm, Concurrent Propagation \cite{Gimelfarb2012}.  \citeauthor{Pang2018} \cite{Pang2018} also claim significant benefits from modelling certain problems in a novel variant of Enzymatic Numerical \gls{ps} \cite{Pavel2010}, but it is unclear how much of the stated benefit compared to their baseline implementation arises instead from the use of a \gls{gpu}.

\begin{anfxerror}
Need to include sections describing the general procession of a \gls{mc} system, e.g. that there are rules, and things evolve synchronously in steps, etc.  Also, need to include a description of ``top-down weak-priority order''.
\end{anfxerror}

\subsection{Objects, Rules and Steps}
All known \gls{ps} types fundamentally operate on a similar basis:  One or more sets of rules -- \emph{\gls{ruleset}} -- are defined, describing how the \emph{objects} present in the system's compartments change at each \emph{step}.  As mentioned above, the objects are usually multisets of arbitrary symbolic \emph{atoms}, with the exceptions of \gls{snps} which uses the spike as its only symbol, and Numerical \gls{ps}, which uses ordinary numbers in place of atoms.

\Gls{ps} types normally operate synchronously and assume the presence of a global clock.  At each clock ``tick'', every compartment compares its extant objects and its \emph{evolution rules}, determines which rules are applicable given the current objects, and then deletes the objects used in the rules, replacing them with new ones as the rules dictate.  This execution of the system is referred to as the system's evolution.  All \gls{ps} evolve, and therefore all types have evolution rules (though they may not be referred to as such).  Other types of rules are possible, including: \emph{dissolution} and \emph{division} rules in \gls{clps}, where membranes either dissolve and release their objects into the their parent membrane, or replicate themselves (essentially performing mitosis) and distribute their contents among the new membranes; \emph{forgetting} rules in \gls{snps} whereby one or more copies of the spike are removed from a given neuron;

All rules use the same basic model.  

\subsubsection{Weak Priority Order}

Many, perhaps most, types of \gls{ps} \glspl{ruleset} use a \emph{weak-priority} ordering.  This means that some rules will be tested for applicability ahead of others, on some priority basis, but earlier applicable rules only prevent later applicable rules from being applied if there is a conflict between the two.  The most common way that this conflict can arise is by two rules trying to use the same pre-existing object in the compartment.

Generally speaking, an individual rule will select for use one or more copies of one or more objects the multiset.  At the end of the rule's application, these objects are deleted and replaced with any new ones the rule specifies.  Since the rule will delete the chosen objects, it would not make sense for another rule to be able also to use and then delete the same objects.  Therefore, the first rule to select (or take hold of or seize \etc{}) a given object prevents any other rule from using it too, and thus the first rule has priority over later rules.

The typical method of defining rules' priority is to use \emph{top-down} ordering.  This simply means that the rules presented first in a \gls{ruleset} have 

\subsection{Computer Representations and Simulations of \glsfmtname{ps}}

\subsubsection{\Adhoc{} and General Simulations}
There are arguably two main approaches to simulating \gls{ps}:
\begin{inparaenum}[a)]
\item ``\Adhoc{}'' simulations, where a separate program is written specifically for a given type of problem and its \gls{ps} solution; and
\item ``General'' simulations, where a separate simulation engine capable of simulating one or more types of \glspl{ps} is created independent of a given problem, and is supplied problem-specific configurations.  The engine uses the configuration to initialise the simulated system, and works through the problem from there.
\end{inparaenum}

The main advantage of the \adhoc{} style is the ability to adjust and optimise the simulation's implementation to suit the \gls{ps} variant used, and the problem at hand.  In general, \adhoc{} simulations would be expected to require less resources to find the answer, \eg{} running faster and/or using less memory.  The major disadvantage of the \adhoc{} approach is that a new simulation must be developed for each problem studied, requiring more time and greater levels of technical skill while reducing flexibility.  The main advantages of the general approach are greater flexibility from the produced program -- \ie{} it can simulate more problems -- and a broadening of the people who can experiment with different \gls{ps} variants and problems to those with lower levels of programming expertise.

General simulations permit specialisation and a division of labour, meaning one person can look into new \gls{ps} variants and problems to apply them to, while another person focuses on developing and improving the simulation engine itself.  This is a clear upside, but there is equally a downside: lacking problem-specific knowledge, the general simulations usually do not perform all potential optimisations, meaning that there could be unavoidable upper bounds on the efficiency of a simulation, no matter the specific problem at hand.  Furthermore, general simulations must run inside another program, whereas \adhoc{} simulations can be created as independent, native executables.

Traditionally, this has been an `either/or' problem, where one can take either a wholly \adhoc{} approach or a wholly generalist approach.  \citeauthor{Perez-Hurtado2019} more recently introduced a \gls{ps} ``compiler'' \texttt{pcc} \cite{Perez-Hurtado2019}, which can produce a standalone native executable from a non-programmatic specification of a particular \glspl{ps} --- thus providing a third, middle-ground option.  They say of this compiler: \textquote{the goal of \textins{\texttt{pcc}} is twofold: On the one hand, it purports to be a good assistant for researchers while verifying their designs, even working with experimental models. On the other hand, it provides optimized simulators for real applications, such as robotics or simulation of biological phenomena.}  It was not used in this dissertation, as \texttt{pcc} did not support \gls{cps} at the relevant time, but the idea holds great promise for the future.

\subsubsection{\label{sec:back:simulators}\Glsfmtname{ps} Simulators}

\citeauthor{Valencia-Cabrera2019} provide a summary of the development of simulators for \Gls{ps} since the field's inception in the late 1990s \cite{Valencia-Cabrera2019}.\footnote{The authors also provide a timeline of practical works in \gls{ps} at \url{https://github.com/RGNC/plingua}.  Some of the software described in this \lcnamecref{sec:back:simulators} is available at \url{http://ppage.psystems.eu/index.php/Software/}.}  Unsurprisingly, most early simulators were \adhoc{} and created for a specific purpose, focusing on one problem domain and simulating one \gls{ps} variant.  Many were intended for formal verification of models as much as they were for practical use \cite{Gutierrez-Naranjo2007}.  These early simulations were written in a wide variety of programming languages, including (comparatively) lesser-used languages such as Haskell, Prolog and LISP.  Notably, \citeauthor{Ciobanu2004} created a simulator specifically for distributed computing, using C++ and \gls{mpi} \cite{Ciobanu2004}.

As the number of \gls{ps} variants defined, and simulations to experiment with them, expanded greatly, it began to make more sense to create general simulators which did not require detailed customisation for every experiment.  A handful of these multi-purpose simulators began to appear, including (among others): PSim \cite{Bianco2007,Bianco2007a}; a transpiler from Systems Biology Markup Language (SBML) to C Language Integrated Production System (CLIPS) \cite{NepomucenoChamorro2005};  and a web-based simulator which also made use of CLIPS \cite{Bonchis2005}.  Of particular interest from this period is \cite{Acampora2007}, which specifically targeted the creation of a paralel and distributed multi-agent system, to take advantage of the concurrency inherent in most \gls{ps} variants and models.

While these simulators were a clear step to re-usability, they still largely targeted only a specific \gls{ps} variant or sub-variant.  There was another issue in that there was no standard for representing an individual \glspl{ps}.  Simulators generally either used their own custom specification system, such as a special-purpose XML schema, or attempted to make use of a representation created for another purpose, such as SBML.  To address these two shortcomings of the existing systems, researchers at the Universidad de Sevilla (University of Seville) in Spain created \gls{plingua}\footnote{\url{http://www.p-lingua.org/wiki/index.php/Main_Page}, \url{https://github.com/RGNC/plingua}} \cite{Diaz-Pernil2008a,Garcia-Quismondo2010} and \gls{mecosim}\footnote{\url{http://www.p-lingua.org/mecosim/}} \cite{Perez-Hurtado2010}.

\Gls{plingua} is a declarative markup language, used to specify specific systems and their initial configurations.  Arguably, it has become the dominant specification language of the computerised \gls{ps} world.  Crucially, \gls{plingua} also allows for the specification of new \gls{ps} variants and extensions to existing ones, giving it a much greater potential flexibility.  It is primarily built around the Java library PLinguaCore, which provides functionality to translate between various representations of \gls{ps} specifications.  One of the simulators to make heavy use of \gls{plingua} is \gls{mecosim}.

\Gls{mecosim} is a Java-based general-purpose \gls{mc} simulator.  It uses \gls{plingua} and spreadsheets to define the evolution of a given \gls{ps} type, as well as the problem to be solved --- both the rules and starting state of the system.  The particular strengths of \gls{mecosim} are that, once a particular type of \glspl{ps} has been defined it is completely re-usable, and the simulator permits rapid experimentation with different designs without any programming.  Sadly, however, it appears that both \gls{plingua} and \gls{mecosim} have effectively been abanonded, as neither seems to have been substantively updated in some years.  It is not currently clear if there is any unifying simulator of specification language to supersede them.

\begin{anfxwarning}{Some citations to include}
\cite{Raghavan2020,Raghavan2020a,Raghavan2016}
\end{anfxwarning}

\subsection{\label{sec:psystemsuses}Practical Applications of \glsfmtname{ps}}
\Gls{mc} is not just a theoretical model with limited practical use.  Besides Image Processing \& Computer Vision (see \vref{subsec:imgprocpsys}), \gls{ps} variants have been applied to a range of fields, from power grid management to robotic control systems \cite{Zhang2017}.

\begin{anfxwarning}{Some citations to include}
\cite{Zhang2020,Colomer2010,Gheorghe2010,Liu2016,Huang2016,Perez-Hurtado2010,Verlan2012,Syropoulos2004,Liu2020,Lefticaru2011,Oltean2008}
\end{anfxwarning}

\begin{anfxerror}{Finish this}
\citeauthor{Florea2017} proposed using Enzymatic Numerical \gls{ps} for robotics \cite{Florea2017,Florea2016,Florea2017a,Florea2019,Florea2016a}.
\end{anfxerror}

\subsubsection{\label{subsec:imgprocpsys}\fxwarning{How relevant still is this section?}{Image Processing and Computer Vision in \glsfmtname{ps}}}
\begin{anfxwarning}{Some citations to include}
\cite{Zhang2012,Yuan2019}
\end{anfxwarning}

Perhaps owing to the potentially unbounded space and parallelism capacity of \gls{ps}, combined with the embarrassingly parallel nature of many tasks in Image Processing \& Computer Vision, the latter have proved to be fertile ground for the former, although not every publication puts its model to the test with a computerised simulation, or if it does, the authors may only provide scant details \cite{Diaz-Pernil2019}.

\citeauthor{Christinal2011} \cite{Christinal2011} described a family of \gls{tlps} to perform region-based segmentation of both 2D and 3D images.  Despite their family of systems requiring only two cells, it also needed custom rule sets based on the size of the images as well as the number of colours present, with a number of rules per set proportional to the same measurements.  The paper showed the results of simulating the system, but provides no details on performance.

\citeauthor{Diaz-Pernil2013} \cite{Diaz-Pernil2013} commented that \enquote{\textelp{} commonly \textins{a} parallel algorithm needs to be re-designed with only slight references to the \textins{sequential original}.  \textelp{} the design of a new parallel implementation not inspired by the sequential one allows \textelp{} the proposal of new creative solutions.}  They then demonstrated this fact by designing a new edge detection and segmentation algorithm named `A Graphical P (AGP) segmentator', inspired by the Sobel operator (see \eg{} \cite{Nixon2012}) and using the segmentation method from \cite{Christinal2011}, which they modelled in \gls{tlps}.  The authors implemented their new algorithm on a \gls{gpu} and compared it with an implementation of the \numproduct{3x3} and \numproduct{5x5} Sobel operators, finding that theirs had near-identical runtimes but superior edge detection capabilities.

\citeauthor{Diaz-Pernil2013a} \cite{Diaz-Pernil2013a} further explored modelling classic image processing techniques by implementing \citeauthor{Guo1989}'s binary image skeletonisation technique \cite{Guo1989} with \gls{snps}.  The overall system's rules templates are reasonably simple, but include references to a set \(\mathit{DEL}\) (used as a lookup to determine whether a cell should turn white or stay black) which does not appear to be modelled inside the system, meaning that it is not self-contained.  The authors simulated this system on a \gls{gpu}, but found that their implementation was upwards of twice as slow as another pre-existing implementation.  Confusingly, however, they state that one of the reasons for this is \enquote{that the use of an alphabet with only one object, the spike \(a\), does not fit in the GPU architecture}.  This statement is perplexing, given that spikes can easily be represented as simple integers.  The authors also commented that the synchronous nature of the model is unrealistic, and imposing a global clock upon the system can be problematic.

\citeauthor{Nicolescu2014} \cite{Nicolescu2014} alternatively applied \gls{cps} to image skeletonisation based on \citeauthor{Guo1989}'s technique \cite{Guo1989}, presenting three forms of a solution: Synchronous versions that use multiple or a single cell (essentially the latter replicates the former via the use of sub-membranes), and an asynchronous multi-cell version.  The asynchronous version no longer assumes that all messages are passed between cells simultaneously and instantaneously, compensating for this by increasing the number of messages used.  This form, while arguably more realistic to modern computers, requires a greater message complexity. A simplified \Gls{actor}-model-based (see \cref{subsec:back:actors}) implementation using \fsharp{}'s \texttt{MailboxProcessor} \cite[ch.~11]{Syme2015a} was presented, but no results from running it were reported.

\citeauthor{Nicolescu2015} \cite{Nicolescu2015} further applied \gls{cps} to seeded region growing of greyscale images.  The described system used a two-level approach, based on the `Structured Grid Dwarf' of the 13 Berkeley Dwarves \cite{Asanovic2006}, where the image was divided into rectangular blocks of multiple pixels.  Each block was modelled with a single cell, inter-block processing was carried out via message passing, and intra-block processing was performed by typical object evolution.  It was again suggested that this would fit well to the \Gls{actor} model.

\citeauthor{Diaz-Pernil2016} \cite{Diaz-Pernil2016} built upon the AGP segmentator algorithm to create a version that works with RGB images rather than greyscale and applied it to a common medical Computer Vision task, isolating the `optic disc' in images of the inner eye.  Along with this, they used the skeletonisation algorithm from \cite{Diaz-Pernil2013a} and a number of other steps not based on \gls{ps} to produce a complete imaging pipeline.  The authors implemented this on a \gls{gpu}, and found that their system was both more accurate and faster than previous systems.

% \subsubsection{\glsfmtname{ps} on \glsxtrlongpl{gpu}}
% In many instances, a \gls{ps} model for a problem involves many independent small elements processing their data separately, and perhaps updating each other's state at the end of a step.  Given that this sounds remarkably close to the Single-Instruction Multiple-Thread \cite[Ch. 4.4.1]{Hennessy2012} nature of modern \gls{gpgpu}, it is no surprise that there has been much work put into simulating \gls{ps} on \glspl{gpu}.

% \begin{anfxwarning}{More citations}
% \cite{Cecilia2010,Cecilia2010a,Cecilia2013,Macias-Ramos2015,Martinez-Del-Amor2015,Martinez-Del-Amor2013a,Maroosi2014,Maroosi2014a}
% \end{anfxwarning}