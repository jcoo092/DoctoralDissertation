\chapter{Introduction}
Some wordiness will go here.

\section{My first section}
Post-introductory wordiness

% \Gls{bp} can be described in terms of message passing, but also can be given fairly efficient \gls{gpu} implementations.  How do the two connect?

\begin{anfxwarning}{What goes in the intro?}
From \url{https://abacus.bates.edu/~ganderso/biology/resources/writing/HTWsections.html}:  ``the Introduction must answer the questions, "What was I studying? Why was it an important question? What did we know about it before I did this study? How will this study advance our knowledge?"''
\end{anfxwarning}

\section{Research Questions}
\begin{enumerate}
    \item Does modelling \gls{nmp} in a theoretical setting like \gls{cps} help in some way?
    \item Does an asynchronous model of \gls{nmp} reveal any benefit over the traditional synchronous versions?
    \item Does the practical implementation of \gls{nmp}, applied to \gls{bp} for \gls{sm} show any improvement over other techniques?
\end{enumerate}

\subsection{Research Question One}

\subsection{Research Question Two}

\subsection{Research Question Three}

\section{Hypotheses}

\subsection{Hypothesis One}

\subsection{Hypothesis Two}

\subsection{Hypothesis Three}

\section{Outline}
Where I summarise the upcoming dissertation (this section probably needs a different/better name).  Where in the intro do I explain the novelty?