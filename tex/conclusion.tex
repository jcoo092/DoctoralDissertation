\chapter{Conclusion}

\fxerror{Need to write a proper conclusion}{I did some stuff!}

\section{Future Directions}

\begin{anfxwarning}{Where should this go?}
    Similar material is also covered in the discussion chapter.  Should much of the below be shifted into there?  Alternatively, should much of the stuff from the discussion chapter move to here?  Should there even be a discussion chapter?
\end{anfxwarning}

A major theme of this dissertation has been exploiting large-scale parallelism, and concurrency more broadly, to reduce the time complexity of solutions to problems.  Practical implementations of those solutions sometimes provide no measurable benefit over traditional approaches, however.  Developing the capacity to simulate \gls{cps} further, in particular developing more advanced techniques for translating \gls{cps} rules to efficient parallel code, should be a priority if the benefits of \gls{cps} are to be realised.

\begin{anfxerror}
A big challenge to the above is that many of the approaches described in this dissertation assume unbounded near-instantaneous memory, \emph{and} an unbounded number of available processors.
\end{anfxerror}

% \subsection{\glsentrytext{cml}}
% \Gls{cml} seems like the obvious approach to take from here.\footnote{In fact, considerable time and effort was spent on attempting to identify a suitable \gls{cml} implementation to use for just this purpose.  Unfortunately, in the end, it was found that there was not currently a usable \gls{cml} implementation with appropriate performance available at this time.}

\subsection{`Faked' message passing in shared memory}

\begin{anfxnote}
Roughly, most of the message passing involved here is largely, in effect, just handing around pointers to memory locations.  It would seem that the message passing itself places some overhead in the way of that.  Could there be some way to fake the message passing, so that to the program's writer it looks like an actual message passing implementation, but in reality it is just doing normal updates on mutable memory?  This \emph{might} enable the best of both worlds -- programming the algorithms according to their theory, but running in a highly efficient fashion `under-the-hood'.
\end{anfxnote}

\fxwarning*{\ }{None of the below were investigated further, due to a lack of time, but they are obvious next steps to look at.}

\begin{anfxnote}
It is not too clear how to achieve this (if it is indeed possible), but Rust would appear to be a good language to target for it.  Rust is fairly high-performance by default and enables quite a lot of low-level memory manipulation.  Moreover, its move semantics pretty much fit exactly to the concepts used here, and, on the face of it, its macro system would appear to be a convenient way to abstract over many of the details and provide a message passing façade, while actually doing efficient operations behind that.
\end{anfxnote}

\fxwarning*{Look up}{Extended Dataflow Actors?  Or is it Extended Dataflow Architecture?}

\fxerror*{Not entirely sure BSP is relevant here}{See also the Bulk Synchronous Parallel approach.}

% It looks like the C++ \texttt{mess} library is intended to be exactly this sort of thing:  \url{https://github.com/LouisCharlesC/mess}.  The developer seems to say that the user gets to write their program in a message-passing fashion, but mess does some clever meta-programming so that there ends up being zero overhead in the end.  Also, absolutely \emph{must} address Halide, and explain why it wasn't pursued here.  Otherwise, that'll be the elephant in the room.

% https://scholar.google.co.nz/scholar?q=related:Z8GZl-HQcSkJ:scholar.google.com/&scioq=A+Static+Mapping+System+for+Logically+Shared+Memory+Parallel+Programs&hl=en&as_sdt=0,5&inst=15360723290749679499

% http://citeseerx.ist.psu.edu/viewdoc/download?doi=10.1.1.50.8739&rep=rep1&type=pdf

% https://link.springer.com/chapter/10.1007/BFb0057916

% https://link.springer.com/chapter/10.1007/3-540-63697-8_82

\subsection{Other hardware}
This work focused on CPU-based systems, largely excluding other hardware.  If used well, all three of \glspl{gpu}, \glspl{fpga} and \glspl{dsp} have potential to perform vastly more computations per second, suggesting that a high-performance message passing-based system would do well to use them.  In each case, however, they do not work in quite the same way as CPUs, meaning that programming them is not necessarily straightforward, especially when trying to achieve a programming style that differs to their default.%  As described in \fxwarning*{insert the appropriate cross-reference}{the literature review chapter}, fast \gls{gpu} implementations of \gls{bp} have been created, suggesting that message passing on a \gls{gpu} is entirely plausible.

It would appear worthwhile to investigate how to implement a high-performance message passing system atop these hardware alternatives, due to the potential for many more computations per second.  Even better would be to create a heterogeneous system which can take full advantage of the strengths of each hardware type, while overcoming its weaknesses.  Precisely how to achieve efficient implementations on them is unknown.  The fact that OpenCL \fxerror[inline]{[ref]} (and, to some extent at least, OpenACC \fxerror[inline]{[ref]}) can be compiled from the same base code to different devices makes it an obvious starting point.   \fxerror{Note to self}{There has already been at least one publication on implementing \glspl{actor} in OpenCL \fxerror[inline]{[ref]}, suggesting it is possible -- though how well synchronous message passing will work as compared to asynchronous remains to be seen.}

\begin{anfxnote}
The use of Full/Empty bits looks highly promising for making message passing more efficient.  It doesn't seem to have any real support in mainstream/commodity hardware, however.  There were Intel's TGX instructions, but they have been taken out of processors again.
\end{anfxnote}

% \fxwarning{For future reference}{See also \url{https://hastlayer.com/} -- they seem to say that they do relevant stuff on FPGAs.  Interestingly, they also do unums/posits, apparently.}

\subsection{Universal Numbers}
\begin{anfxnote}
The \gls{cps} work kinda ignored non-integer numbers, which in real practice is quite a glaring omission.  Investigating modelling IEEE-754 floating point (and/or the fixed-point version), as well as Gufstafson's unums/posits, would be another good step to take.
\end{anfxnote}

%%%%%%%%%%%%%%%%%%%%%%%%%%%%%%%%%%%%%%%%%%%%%%%%%%%%%%%%%%%%%%%%%%%%%%%%%%%%%%%%%

\subsection{\nameref{chap:tsp}}
% One-way multiset unification occurs frequently in \gls{cps}, with unification being used in every rule presented above.  \fxerror*{Isn't this what Yezhou's paper \cite{Liu2021} covers?}{An efficient algorithm to perform this task would be highly beneficial for creating useful simulations of systems (we are not aware of an efficient algorithm in the case of multisets).}  For example, our simulations of the \gls{tsp} algorithm written in functional programming languages (see \cref{sec:tsp:simulation}) regularly simply iterate over all relevant objects in the system, even though frequently most will be of little use in a given function call, and so the simulations could benefit from improved unification in practice.

% We would like to further develop the capacity to simulate \gls{cps}, in particular developing more advanced techniques for translating \gls{cps} rules to efficient parallel code.  Work down this avenue has not begun as yet, however.

One-way multiset unification occurs frequently in \gls{cps}, with unification being used in every rule presented above.  An efficient algorithm to perform this task would be highly beneficial for creating useful simulations of systems.  For example, the simulations of the \gls{tsp} algorithm written in functional programming languages (see \vref{sec:tsp:simulation}) regularly simply iterate over all relevant objects in the system, even though frequently most will be of little use in a given function call, and so the simulations could benefit from improved unification in practice.  At the time \cref{chap:tsp} was completed, no efficient algorithm for this task was known, but more recently \citeauthor{Liu2021} published \citetitle{Liu2021} \cite{Liu2021}.  The \gls{tsp} simulations should be revisited in light of this development.

%%%%%%%%%%%%%%%%%%%%%%%%%%%%%%%%%%%%%%%%%%%%%%%%%%%%%%%%%%%%%%%%%%%%%%%%%%%%%%%%%

\subsection{\nameref{chap:median}}

The median filter algorithm investigated was pretty much the most basic one possible.  The experiment should be extended to cover more complicated algorithms that perform better at reconstructing the image.  These may be better suited to a message-passing approach compared to a na\"ive one than the basic method.

Future work should explore \gls{cml} as applied to other computer vision algorithms, especially those which can naturally be characterised in terms of message passing, or which use more complex data types.  Other \gls{cml} implementations beyond the one used here should be tested too, as it is not clear how much the problems experienced by this \gls{cml} program may stem from the implementation of the \gls{cml} library used.

%%%%%%%%%%%%%%%%%%%%%%%%%%%%%%%%%%%%%%%%%%%%%%%%%%%%%%%%%%%%%%%%%%%%%%%%%%%%%%%%%

\subsection{\nameref{chap:nmp}}
% The next step in this work is to adapt the system to the purpose of Belief Propagation Stereo Matching, using \gls{nmp} precepts.  This will be presented in Part Two.  Furthermore, work has begun on implementing a close approximation of the asynchronous system as a framework in a standard programming languages, to explore the effectiveness of this approach in modern computer systems.  We plan to implement Belief Propagation Stereo Matching (see e.g. \cite{Blake2011,Felzenszwalb2011,JianSun2003}) atop this as a proof-of-concept.

One aspect the systems presented in \cref{chap:nmp} lack is that both the size and shape of the grid involved, as well as the communication topology between neighbours, are permanently fixed at the time of system initialisation.  In most cases, this is unneeded, but the greater flexibility could be of use when implementing certain algorithms.

Furthermore, at present, it is implicitly assumed that every \gls{pe} remains active throughout the entirety of the system's evolution until it has sent and received all of its scheduled messages.  Within the context of \gls{cps} this is largely irrelevant, but permitting \glspl{pe} to deactivate at appropriate points could save processing power in other circumstances with bounded parallelism.  Complicating this is ensuring that those \glspl{pe} which do remain active can continue messaging as needed despite one or more neighbours deactivating.

% We also have yet to examine the systems with respect to communication complexity measures such as those found in \cite{Juayong2020}.  The precise results presented there are not directly applicable to this work, given the use of different P~systems models, but the underlying concepts appear directly relevant.

%%%%%%%%%%%%%%%%%%%%%%%%%%%%%%%%%%%%%%%%%%%%%%%%%%%%%%%%%%%%%%%%%%%%%%%%%%%%%%%%%