% The below was taken from https://tex.stackexchange.com/a/562932.  It is used in the glossary.tex file, but placed here because I'm not sure putting it in the glossary file won't b0rk things.  It was originally called 'newdefineabbreviation', but I changed it because that was irritatingly long.
% #1 - reference e.g. api
% #2 - Short e.g. API
% #3 - Full name e.g. Application Programming Interface
% #4 - Description
\newcommand{\newdefacr}[4]
{
    % Glossary entry
    \newglossaryentry{#1-glossary}
    {
        text={#2},
        long={#3},
        name={\glsentrylong{#1-glossary} (\glsentrytext{#1-glossary})},
        description={#4},
        sort={#3}
    }

    % Acronym
    \newglossaryentry{#1}
    {
        type=\acronymtype,
        name={\glsentrytext{#1-glossary}}, % Short
        description={\glsentrylong{#1-glossary}}, % Full name
        first={\glsentryname{#1-glossary}\glsadd{#1-glossary}},
        see=[Glossary:]{#1-glossary}, % Reference to corresponding glossary entry
        sort={#2}
    }
}

%%%%  These ones are used in the NMP stuff %%%%

\newcommand*{\cpvv}[3]{\cpfuncnn{v}{#1}{#2}{#3}}
\newcommand*{\cpvvbf}[3]{\cpfuncnn{\mathbf{v}}{#1}{#2}{#3}}

\newcommand*{\cpvw}[3]{\cpfuncnn{w}{#1}{#2}{#3}}
\newcommand*{\cpvwbf}[3]{\cpfuncnn{\mathbf{w}}{#1}{#2}{#3}}

\newcommand*{\cpvq}[2]{\cpfuncn{v}{#1}{#2}}
\newcommand*{\cpvqbf}[2]{\cpfuncn{\mathbf{v}}{#1}{#2}}

\newcommand*{\cpvqw}[2]{\cpfuncn{w}{#1}{#2}}

%%%%%%%%%%%%%%%%%%%%%%%%%%%%%%%%%%%%%%%%%%%%%%%%%%%%%

%%%%%%%%%%%%% Theorem declarations %%%%%%%%%%%%%%%%%%

\newtheorem{theorem}{Theorem}[chapter]
% \newcommand{\theoremautorefname}{Theorem}
\newtheorem{proposition}[theorem]{Proposition}
% \newcommand{\propositionautorefname}{Proposition}
\newtheorem{lemma}[theorem]{Lemma}
% \newcommand{\lemmaautorefname}{Lemma}
\newtheorem{conjecture}[theorem]{Conjecture}
% \newcommand{\conjectureautorefname}{Conjecture}
% \begin{anfxerror}
% Need to remember to come back to this one and set it appropriately - I don't think it should be following the standard newtheorem style
% \end{anfxerror}
\newtheorem{remark}[theorem]{Remark}
% \newcommand{\remarkautorefname}{Remark}

%%%%%%%%%%%%%%%%%%%%%%%%%%%%%%%%%%%%%%%%%%%%%%%%%%%%%


%%%%%%%%%%%%%%%%  Random other shorthands  %%%%%%%%%%%%%%%%%%%%%%%
\newcommand{\etc}{\textit{et cetera.}}

\newcommand{\fsharp}{F\nolinebreak\hspace{-.05em}\raisebox{.3ex}{\tiny{\textbf{\#}}}}

\newcommand{\bigoh}[1]{O(\ensuremath{#1})}

% \crefname{cprulesetfloat}{ruleset}{rulesets}
% \crefname{cpobjectsfloat}{objects group}{objects groups}