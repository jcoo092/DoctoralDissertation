\begin{abstract}
    
    \Gls{mc}, also known as \gls{ps}, is a field of theoretical computer science initially inspired by the principles of the interactions of chemicals within, and their movements through, the membranes of biological cells.  An important property of most types of \gls{ps} is that they are inherently concurrent, with the contents of every membrane or equivalent evolving simultaneously, and that they have an unbounded space capacity, empowering them to solve traditionally computationally challenging problems in computer science quickly.  \gls{cps} is a type of \gls{ps} that takes a high-level approach to modelling problems in \gls{mc}, allowing for more straightforward solutions while retaining the advantages for time complexity.
    
    This dissertation applies \gls{cps}' the capacity for large-scale concurrency to pre-existing problems in computer science, including the Travelling Salesman Problem, the Graph Colouring Problem, and image median filtering.  It also explores the pre-existing concept termed here ``neighbourhood message passing'', which involves separate logical processing elements communicating with their neighbours via messaging, from a new angle.  The critical differentiator between neighbourhood message passing and similar concepts is the requirement for a processing element to compute outgoing messages to neighbours based on the messages received from every neighbour \emph{except} the intended recipient.
    
    Novel \gls{cps} solutions to the Travelling Salesman Problem and Graph Colouring Problem, with time complexity linear to the number of nodes in the graph, along with a constant-time solution to median filtering, are presented and analysed here.  Computer implementations of solutions are also experimented with to validate said solutions and explore the efficacy of different implementation methods.
    
    Three variants of neighbourhood message passing are also explored.  Firstly, the traditional \gls{gs} style used in belief propagation, then an asynchronous approach which is closer to the underlying concept but seemingly heretofore unexplored, and finally an intermediate \gls{ls} form that naturally arises from studying the previous two.  Detailed analysis and experiments of these variants show that the \gls{gs} and \gls{ls} styles compute identical results, but the \gls{ls} one is generally approximately 5-13\% faster in practice, while the asynchronous variant is typically around another 10\% faster again yet computes almost-identical results.
    
\end{abstract}
