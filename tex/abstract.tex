\begin{abstract}
    
% \Gls{mc}, also known as \gls{ps}, is a field of theoretical computer science initially inspired by the principles of the interactions of chemicals within, and their movements through, the membranes of biological cells.  An important property of most types of \gls{ps} is that they are inherently concurrent, with the contents of every membrane evolving simultaneously, and that they have an unbounded space capacity, empowering them to solve traditional computationally challenging problems quickly.  \gls{cps} is a type of \gls{ps} that takes a high-level approach to modelling problems in \gls{mc}, allowing for more straightforward solutions while retaining the advantages for time complexity.

\Gls{mc}, also known as \gls{ps}, is a field of theoretical computer science initially inspired by the principles of the interactions of chemicals within, and their movements through, the membranes of biological cells.  Important properties of most types of \gls{ps} are that they
\begin{inparaenum}[a)]
\item are inherently concurrent, with the contents of every membrane evolving simultaneously; and,
\item that they have an unbounded space capacity, empowering them to solve traditional computationally challenging problems quickly.
\end{inparaenum}
\gls{cps} is a type of \gls{ps} that takes a high-level approach to modelling problems in \gls{mc}, allowing for more straightforward solutions while retaining the advantages for time complexity.
    
This dissertation applies \gls{cps}' capacity for large-scale concurrency to established problems in computer science, including the \glsentrylong{tsp-glossary}, the \glsentrylong{gcp-glossary}, and image \gls{medianfilter}ing.  It also explores from a new angle the pre-existing concept termed here ``\glsxtrlong{nmp}'', which involves separate logical \glsxtrlongpl{pe} communicating with their neighbours via messaging.  A critical difference, explored in some depth in this dissertation, between \glsxtrlong{nmp} and similar concepts is the requirement for a \glsxtrlong{pe} to compute outgoing messages to neighbours based on the messages received from every neighbour \emph{except} the intended recipient.
    
Novel \gls{cps} solutions to the \glsentrylong{tsp-glossary} and \glsentrylong{gcp-glossary}, with time complexity linear to the number of nodes in the graph, along with a constant-time solution to median filtering, are presented.  Experiments with computer implementations of solutions validate those solutions and are used to explore the efficacy of different implementation methods.

Three variants of \glsxtrlong{nmp} are also explored.  Firstly, the traditional \gls{gs} style used in \glsxtrlong{bp}, then an asynchronous approach which is closer to the underlying concept but seemingly so far unexplored, and finally an intermediate \gls{ls} form that naturally arises from studying the previous two.  Analysis of and experiments with these variants show that the \gls{gs} and \gls{ls} styles compute identical results, but the \gls{ls} one is approximately 5-13\% faster in general practice, while the asynchronous variant is typically around another 10\% faster again yet computes almost-identical results.
    
\end{abstract}
