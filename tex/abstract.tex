% \begin{abstract}
% Something about how there's a group of stereo matching algorithms that can be regarded as `message passing-based', and how they can be represented in \gls{cps}, as well as with \gls{cml}.

% This dissertation presents both \gls{cps} representations and \gls{cml}-based implementations of four stereo matching algorithms.
% \end{abstract}

% \chapter*{Abstract}
% % Something about how there's a group of stereo matching algorithms that can be regarded as `message passing-based', and how they can be represented in \gls{cps}, as well as with \gls{cml}.

% This dissertation presents both \gls{cps} representations and \gls{cml}-based implementations of message-passing-based stereo matching algorithms.

% Could also look at doing something where I `fake' message passing on a shared-memory system using a backing array or two.  As in, while the program is written with a message passing basis, where each disparity map pixel does its own thing and then sends out messages to its neighbours, but in reality all that is happening is that there is a big array behind it all, and each message passing step is just transformed behind-the-scenes into array updates.  Thus, one could program with a message passing style, but still retain the relatively efficient array-update style in the background.

\begin{abstract}
    Belief Propagation represents as a graph individual processing elements communicating via message passing to perform some greater computation.  BP is only guaranteed to converge for trees, but Loopy Belief Propagation has proven to be useful in practice.  Implementations of BP seemingly just use a basic synchronous approach to modelling it, where every message for a given iteration of message passing is computed ``simultaneously'' --- all computations for every node in the graph is computed before anyone moves onto another iteration.  No evidence can be found demonstrating that anyone has actually attempted in the past to model or program BP in an asynchronous fashion.  This dissertation explores that.
\end{abstract}
