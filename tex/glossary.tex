% \makeglossaries

% Acronyms (sorted in alphabetical order by their label, rather than their full definition)

%% A-M
\newacronym{bp}{BP}{Belief~Propagation}
% \newacronym{cml}{CML}{Concurrent~ML}
\newacronym{cp}{CP}{Concurrent~Propagation}
\newacronym{csp}{CSP}{Communicating Sequential Processes}
\newacronym{dpsm}{DPSM}{Dynamic~Programming Stereo~Matching}
\newacronym{dsp}{DSP}{Digital Signal Processor}
\newacronym{fpga}{FPGA}{Field-Programmable Gate Array}
\newacronym{fps}{FPS}{Frames per Second}
\newacronym{gpu}{GPU}{Graphics Processing Unit}
\newacronym{gpgpu}{GPGPU}{General-purpose GPU}
\newacronym{lbp}{LBP}{Loopy Belief~Propagation}
\newacronym{mpbsm}{MPBSM}{Message Passing-Based Stereo~Matching}
\newacronym{mpi}{MPI}{Message Passing Interface}
\newacronym{mrf}{MRF}{Markov Random Field}

%% N-Z
\newacronym{ndcsm}{NDCSM}{Noise-Driven Concurrent Stereo~Matching}
\newacronym{nmp}{NMP}{neighbourhood message passing}
\newacronym{prox}{proxel}{Processing Element}
\newacronym{sad}{SAD}{Sum of Absolute Differences}
\newacronym{skps}{SKP}{Simple Kernel P~systems}
% \newacronym{sm}{SM}{Stereo~Matching}
\newacronym{ssd}{SSD}{Sum of Squared Differences}

% Combo glossary & acronyms

% A-M
\newdefacr{cml}{CML}{Concurrent~ML}{A programming style introduced by John Reppy.  Channels}
\newdefacr{hcp}{HCP}{Hamiltonian Cycle Problem}{Closely related to the \acrlong{hcp}.  In this problem, the Hamiltonian Path must also be a cycle, i.e. it ends back at the origin node.}
\newdefacr{hpp}{HPP}{Hamiltonian Path Problem}{The problem of determining whether, for a given graph, there exists a traversal of the graph in which each node is visited exactly once.}

% N - Z
\newdefacr{sgm}{SGM}{Semi-global~Matching}{A Stereo Matching algorithm introduced by Hirschmüller that, as the name suggests, sits somewhere between traditional global and local Stereo Matching Algorithms.  This provides advantages in that retains much of the speed advantage of local algorithms, while also receiving much of the benefit derived from taking into consideration a larger part of the total input images.}
\newdefacr{tsp}{TSP}{Travelling Salesman Problem}{An extension of the \acrfull{hcp}.  Here, the edges between nodes have weights, and the goal of the problem is to find a minimum weight Hamiltonian cycle on the graph.}

% Glossary Entries
% \newglossaryentry{}{name={},description={}}

%% A-M
\newglossaryentry{actor}{name={Actor},description={A model of message-passing-based concurrent programming originally developed  chiefly by Carl Hewitt.  Its defining characteristics are arguably that it uses asynchronous messaging, whereby the sender and receiver do not need to coordinate or synchronise at all; and that instead of using channels or similar as a go-between, Actors send messages directly to each other, which necessitates `knowing' (i.e. holding a reference to) the intended recipient.  Notable examples of implementations of Actors are found in the programming languages Erlang and Pony, and the Scala library Akka}}
\newglossaryentry{blas}{name={BLAS},description={Basic Linear Algebra Subroutines}}
\newglossaryentry{clps}{name={Cell-like P~systems},description={The original \gls{ps} variant, based on the operation of chemicals in cells}}
\newglossaryentry{cps}{name={cP~systems},description={A variant of P~systems created by Nicolescu}}
\newglossaryentry{disparity}{name={disparity},description={The shift, measured as a number of pixels, of a point in one stereo image to its location in the other image.  When combined with information about the cameras derived from calibration, the measured disparity is used to estimate the distance from the cameras to the point in the scene}.  Disparity is often also referred to as `parallax'.}
\newglossaryentry{disparitymap}{name={disparity map},description={An output array/image (often just using a single channel of 8- or 16-bit integers) which encodes the final estimated disparities for every pixel in the input images.  The range of values in a disparity map itself is restricted to range of possible disparities used in the \gls{sm} algorithm.  It is common, however, to adjust the disparity map in some fashion to make differences in disparity estimates more visible through techniques such as histogram equalisation.  Most published figures of disparity maps will have had such an adjustment applied to make them more useful.  Disparity maps are sometimes also referred to as `digital parallax maps.'}}
\newglossaryentry{lapack}{name={LAPACK},description={Linear Algebra Package (???)}}
\newglossaryentry{mc}{name={Membrane Computing},description={A computational model inspired by the functioning of biological systems, specifically the interactions of chemicals inside the membranes of a biological cell.  The terms Membrane Computing and P systems are frequently used interchangeably}}
\newglossaryentry{mecosim}{name={MeCoSim}, description={Explain MeCoSim}}

%% N-Z
\newglossaryentry{ps}{name={P~systems},description={See \glsentrytext{mc}}}
\newglossaryentry{plingua}{name={P-Lingua}, description={A plain-text markup language used to represent P~systems specifications in computer files readable by both humans and computers.}}
\newglossaryentry{sm}{name={Stereo~Matching},description={A family of methods to match points from different images of the same scene to estimate the distance from the capturing cameras to objects in the scene.  See \autoref{subsec:smgeneral}}}
\newglossaryentry{snps}{name={Spiking Neural P~systems},description={Based on the workings of the brain}}
\newglossaryentry{tlps}{name={Tissue-like P~systems},description={Based on the transmission of chemicals between cells via channels in biological tissue}}